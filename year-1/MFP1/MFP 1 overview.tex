\documentclass{article}
\usepackage{amsmath}
\usepackage[version=4]{mhchem}
\usepackage{parskip}
\usepackage{siunitx}
\usepackage[margin=1in]{geometry}
\usepackage{fancyhdr}
\usepackage{amssymb}
\usepackage{centernot}
\pagestyle{fancy}
\lhead{Willoughby Seago}
\rhead{MFP 1}% put the date here in the form dd/mm/yyyy
\cfoot{Page \thepage}
\renewcommand{\headrulewidth}{0.4pt}
\renewcommand{\footrulewidth}{0.4pt}

\author{Willoughby Seago}
\title{Maths For Physics 1 Overview}
\date{September 2018}

\newcommand{\notesVersion}{1.0}
\newcommand{\notesDate}{04/01/2021}


\begin{document}
\maketitle
This is an overview of important topics I created for the \textit{maths for physics 1} course from the University of Edinburgh as part of the first year of the theoretical physics degree.
When I took this course in the 2018/19 academic year it was taught by Dr Kristel Torokoff\footnote{\url{https://www.ph.ed.ac.uk/people/kristel-torokoff}}.
These notes are based on the lectures delivered as part of this course, and the notes provided as part of this course.
The content within is correct to the best of my knowledge but if you find a mistake or just disagree with something or think it could be improved please let me know.

These notes were produced using \LaTeX\footnote{\url{https://www.latex-project.org/}}.

This is version \notesVersion~of these notes, which is up to date as of \notesDate.
\begin{flushright}
    Willoughby Seago
    
    s1824487@ed.ac.uk
\end{flushright}
\clearpage
\section*{Algebra}

\begin{itemize}
\item Power rules:
\[a^ba^c=a^{b+c}~\text{and}~(a^b)^c=a^{bc}\]
\item Log rules:
\[\ln(ab)=\ln a+\ln b~\text{and}~\ln a^b=b\ln a\]
\item Quadratics:
\[x^2+ax+c=0\implies\left(x+\frac{a}{2}\right)^2-\frac{a^2}{4}+c=0\]
\[ax^2+bx+c=0\implies x=\frac{-b\pm\sqrt{b^2-4ac}}{2a}\]
\item Factor theorem (FT) - If \(P(x)\) has \((x-a)\) as a factor then \(P(a)=0\)
\item Fundamental theorem of algebra (FTA) - If \(P(x)\) is a polynomial of \(n^{\text{th}}\) degree then \(P(x)\) has \(n\) roots
\item Partial fractions
\[\frac{A}{(x+a)(x+b)}=\frac{B}{x+a}+\frac{C}{x+b}\implies A=B(x+b)+C(x+a)\]
Equate coefficients or set \(x=-a,-b\) and solve for \(B\) and \(C\)
\item Sketching graphs:
\begin{enumerate}
\item Lable axis
\item Where does it cross the x axis? \((f(x)=0)\)
\item Where does it cross the y axis? \((f(0))\)
\item Where are the minima/maxima? \((f'(x)=0)\)
\item What is the nature of these points? \((f''(x)<0\implies \text{maxima},~f''(x)=0\implies \text{point of inflection}\text{ and }f''(x)>0\implies \text{minima})\)
\item Are there any vertical asymptotes? Is anything divided by a value that can be 0?
\item Are there any horizontal asymptotes? What happens as \(x\to\pm\infty\)?
\end{enumerate}
\item Conic sections:
\begin{enumerate}
\item Ellipse
\[\left(\frac{x-x_0}{a}\right)^2+\left(\frac{y-y_0}{b}\right)^2=1\]
This is an ellipse centre \((x_0,y_0)\) height \(2b\) and width \(2a\). If \(a=b\) then it is a circle radius \(a\)
\item Parabola
\[ax^2+bx+c=y \text{ or } ay^2+by+c=x\]
The former will give a north or south facing parabola and the latter will give a east or west facing parabola.
\item Hyperbola
\[\left(\frac{x-x_0}{a}\right)^2-\left(\frac{y-y_0}{b}\right)^2=1 \text{ or } \left(\frac{y-y_0}{a}\right)^2-\left(\frac{x-x_0}{b}\right)^2=1\]
This will give a hyperbola centre \((x_0,y_0)\) The former will give a east/west facing hyperbola and the latter will give a north/south facing hyperbola.
\end{enumerate}
\end{itemize}

\section*{Trigonometry}

\begin{itemize}
\item Must know sine and cosine of common angles:
\begin{center}
\begin{tabular}{|c|c|c|c|}\hline
Angle, \(\vartheta\) (rad) & \(\sin\vartheta\) & \(\cos\vartheta\) & \(\tan\vartheta\)\\[5 pt] \hline
0 & 0 & 1 & 0\\[5 pt] \hline
\(\pi\) & 0 & -1 & 0 \\[5 pt] \hline
\(\frac{\pi}{2}\) & 1 & 0 & Undefined \\[5 pt] \hline
\(\frac{\pi}{3}\) & \(\frac{\sqrt{3}}{2}\) & \(\frac{1}{2}\) & \(\sqrt{3}\) \\[5 pt] \hline
\(\frac{\pi}{4}\) & \(\frac{\sqrt{2}}{2}\) & \(\frac{\sqrt{2}}{2}\) & 1 \\[5 pt] \hline
\(\frac{\pi}{6}\) & \(\frac{1}{2}\) & \(\frac{\sqrt{3}}{2}\) & \(\frac{\sqrt{3}}{3}\) \\[5 pt] \hline
\end{tabular}
\end{center}
\item Must know common trig identities:
\begin{enumerate}
\item \(\sin(a+b)=\sin a \cos b +\sin b \cos a\)
\item \(\cos(a+b)=\cos a \cos b -\sin a \sin b\)
\item \(\sin^2x+\cos^2x=1\)
\item \(\tan x=\frac{\sin x}{\cos x}\)
\end{enumerate}
\item If \(\sin(ax+b)=c\) where \(a,b\text{ and }c\) are constants then \(ax+b=\arcsin(c)+2n\pi\) where \(n\in\mathbb{Z}\). Remember that \(\arcsin c\) has two principal values.
\item If \(a\cos\varphi+b\sin\varphi=A\sin(\varphi+\vartheta)\text{ or }A\cos(\varphi+\vartheta)\) then expand the RHS using addition formulae and solve for \(A\) and \(\varphi\)
\end{itemize}

\section*{Complex numbers}

\begin{itemize}
\item \(i^2\triangleq-1\)
\item If \(z\) is a complex number then \(z=x+iy=re^{i\vartheta}=r(\cos\vartheta+i\sin\vartheta)\) Where \(r=|z|\) and \(\vartheta=\arg(z)\)
\item \(x=r\cos\vartheta,~y=r\sin\vartheta,~r=\sqrt{x^2+y^2}, \text{ and } \vartheta=\arccos\frac{x}{r}=\arcsin\frac{y}{r}\)
\item The complex conjugate of \(z\) is defined as \(\bar{z}\triangleq x-iy=re^{-i\vartheta}\)
\item De Moivre's theorem:
\[z^n=r^ne^{in\vartheta}=r^n(\cos n\vartheta+i\sin n\vartheta)\]
\item To write the \(n^{\text{th}}\) power of a trig function as multiple angles expand \((\cos\vartheta+i\sin\vartheta)^n\) and take the real part for cosine, the imaginary part for sine and the real part over the imaginary part for tangent.
\item To write multiple angles as powers you need to know:
\[\left(z+z^{-1}\right)^n=(2\cos\vartheta)^n,~\left(z^n+z^{-n}\right)=2\cos n\vartheta,~\left(z-z^{-1}\right)^n=(2\sin\vartheta)^n\text{ and }\left(z^n-z^{-n}\right)=2\sin n\vartheta\]
Then expand and simplify \((z+z^{-1})^n\) collect powers of equal size together and replace with trig functions
\end{itemize}

\section*{Power series expansions}

\begin{itemize}
\item Must know standard series:
\begin{enumerate}
\item \(\sin x=x-\frac{x^3}{3!}+\frac{x^5}{5!}+\cdots+(-1)^n\frac{x^{2n+1}}{(2n+1)!}+\cdots~n\in\{0, \mathbb{N}\}\wedge x\in\mathbb{R}\)
\item \(\cos x=1-\frac{x^2}{2!}+\frac{x^4}{4!}+\cdots+(-1)^n\frac{x^{2n}}{(2n)!}+\cdots~n\in\{0, \mathbb{N}\}\wedge x\in\mathbb{R}\)
\item \(e^x=1+x+\frac{x^2}{2!}+\frac{x^3}{3!}+\cdots+\frac{x^n}{n!}+\cdots~n\in\{0, \mathbb{N}\}\wedge x\in\mathbb{R}\)
\item \((1+x)^p=1+px+\frac{p(p-1)}{2!}x^2+\frac{p(p-1)(p-2)}{3!}x^3+\cdots+\frac{\prod_{r=0}^{n}(p-r)}{n!}x^n+\cdots~n\in\{0,\mathbb{N}\}\wedge |x|<1\wedge x\in\mathbb{R}\)
\item \(\ln(1+x)=x-\frac{x^2}{2}+\frac{x^3}{3}+\frac{x^4}{4}+\cdots+(-1)^{n+1}\frac{x^n}{n}+\cdots~n\in\mathbb{N}\wedge x\in\mathbb{R}\)
\end{enumerate}
\item To find expansions of other functions there are two methods:
\begin{enumerate}
\item Combine the above standard series taking care that \(x\) must be small.
\item Use the general formula for \(f(x)\) about \(x_0\):\\
\(f(x)=f(x_0)+f'(x_0)(x-x_0)+\frac{f''(x_0)}{2!}(x-x_0)^2+\frac{f'''(x_0)}{3!}(x-x_0)^3+\cdots+\frac{f^{(n)}(x_0)}{n!}(x-x_0)^n+\cdots\)\\ \(n\in\{0,\mathbb{N}\}\wedge x,\,x_0\in\mathbb{R}\)
\end{enumerate}
\item If \(x\) isn't small and takes some value \(\approx x_0\) it is possible to use the standard series expansions by replacing \(x\) with \(x_0+x-x_0\) and then rearranging to get \(c+f(x-x_0)\) where \(c\) is a constant and \(f\) is a function such that it is possible to use the standard series to find the expansion of it. Then find the series expansion of \(f(x-x_0)\)
\end{itemize}

\section*{Limits}
\begin{itemize}
\item Iff \(\displaystyle{\lim_{x\to a_+}f(x)=\lim_{x\to a_-}f(x)\ne\pm\infty}\) then the function \(f\) is continuous at point \(a\)
\item If \(\displaystyle{\lim_{x\to a}\frac{f(x)}{g(x)}}\) is of an indeterminate form then \(\displaystyle{\lim_{x\to a}\frac{f(x)}{g(x)}=\lim_{x\to a}\frac{f'(x)}{g'(x)}=\lim_{x\to a}\frac{f''(x)}{g''(x)}=\lim_{x\to a}\frac{f^{(n)}(x)}{g^{(n)}(x)}}\)
\end{itemize}

\section*{Calculus}

\begin{itemize}
\item The limit definition for the derivative of \(y=f(x)\) is:
\[\frac{\text{Change in y}}{\text{change in x}}=\frac{\Delta y}{\Delta x}=\lim_{\Delta x \to 0}\frac{f(x+\Delta x)-f(x)}{\Delta x}=\frac{df}{dx}=\frac{dy}{dx}=f'(x)=y'\]
\item For the limit to exist it must be the same when approached from either side, ie:
\[\lim_{\Delta x \to 0}\frac{f(x+\Delta x)-f(x)}{\Delta x}=\lim_{\Delta x \to 0}\frac{f(x)-f(x-\Delta x)}{\Delta x}\]
\item Genral remarks on derivatives:
\begin{itemize}
\item If \(f'(x)\) exists in \([a,b]\implies f(x)\) is continuous in \([a,b]\)
\item If \(f(x)\) is continuous in \([a,b]\centernot \implies f'(x)\) exists in \([a,b]\)
\item \(f'(x)\) represents rate of change
\item Higher order derivatives are represented as:
\[\frac{d^ny}{dx^n}=f^{(n)}(x)=\lim_{\Delta x \to 0}\frac{f^{(n-1)}(x+\Delta x)-f^{(n-1)}(x)}{\Delta x}\]
\end{itemize}
\item Must know basic derivatives:
\begin{center}
\begin{tabular}{|c|c|}\hline
Function \(f(x)\) & Derivative \(f'(x)\)\\ \hline
\(x^n\) & \(nx^{n-1}\)\\ \hline
\(\sin x\) & \(\cos x\)\\ \hline
\(\cos x\) & \(-\sin x\)\\ \hline
\(e^x\) & \(e^x\)\\ \hline
\(\ln x\) & \(\frac{1}{x}\)\\ \hline
\end{tabular}
\end{center}
\item Must know general properties of derivatives:
\begin{itemize}

\item Linearity:		\(f(x)=g(x)+h(x)\implies f'(x)=g'(x)+h'(x)\)
\item Constants:		\(f(x)=cg(x)\implies f'(x)=cg'(x)\)
\item Product rule:	\(f(x)=g(x)h(x)\implies f'(x)=g'(x)h(x)+g(x)h'(x)\)
\item Chain rule:		\(f(x)=f(g(x))\implies f'(x)=\frac{df}{dx}=\frac{df(g(x))}{dx}=\frac{df(g(x))}{dg(x)}\cdot\frac{dg(x)}{dx}\)
\item Quotient rule:	\(f(x)=\frac{g(x)}{h(x)}\implies f'(x)=\frac{g'(x)h(x)-g(x)h'(x)}{[h(x)]^2}\)
\item Reciprocal rule:	\(\frac{dy}{dx}=\frac{1}{\frac{dx}{dy}}\)
\end{itemize}
\item To differentiate an inverse function set it equal to something and then rearrange to get rid of inverses. Next differentiate and rearrange to have it in terms of the variable that you started with.
\item To differentiate something of the form \(a^x\) rearrange it to \(e^{x\ln a}\) and differentiate
\item Curve sketching:
\begin{enumerate}
\item Check for symmetry. Is it odd, even or asymmetric?
\item Where does it cross the \(x\) axis? Where is \(f(x)=0\)?
\item Where does it cross the \(y\) axis? Where is \(f(0)\)?
\item Check for vertical asymptotes. What happens as \(x \to \pm \infty\)?
\item Check for horizontal asymptotes. What happens as the divisor \(\to 0\)?
\item Find stationary points. Where does \(f'(x)=0\)?
\item What is the nature of stationary points? What is the value of \(f''(x)\) at the stationary points?
\end{enumerate}
\item Fundamental theorem of calculus:	If \(f(x)\) is integrable over \([a.b]\) then:
\[\int_a^bf(x)\,dx=[F(x)]_a^b=F(b)-F(a)\]
This is a deffinite or Riemann integral.
\item An indefinate integral has no limits and only the only constraint is that \(\frac{dF}{dx}=f\) This means that \(\int f(x)\,dx=F(x)+c\) where \(c\) is a constant of integration.
\item General properties of integrals:
\begin{itemize}
\item Linearity: \(\int f(x)+g(x)\,dx\equiv\int f(x)\,dx+\int g(x)\,dx\)
\item If \(c\) is a constant then \(\int cf(x)\,dx\equiv c\int f(x)\,dx\)
\item \(\int_a^bf(x)\,dx\equiv-\int_b^af(x)\,dx\)
\item If \(a<b<c\) then \(\int_a^cf(x)\,dx\equiv\int_a^bf(x)\,dx+\int_b^cf(x)\,dx\)
\item If \(f(x)\) is even about \(a\) and the interval is even about \(a\) then:
\[\int_{a-b}^{a+b}f(x)\,dx\equiv2\int_0^{a+b}f(x)\,dx\equiv2\int_{a-b}^0f(x)\,dx\]
\item If \(f(x)\) is odd about \(a\) and the interval is even about \(a\) then:
\[\int_{a-b}^{a+b}f(x)\,dx\equiv0\]
\end{itemize}
\item Must know basic integrals:
\begin{center}
\begin{tabular}{|c|c|}\hline
Function \(f(x)\) & Integral \(\int f(x)\,dx\)\\ \hline
\(x^n\) & \(\frac{x^{n+1}}{n+1}+c\)\\ \hline
\(\sin x\) & \(-\cos x +c\)\\ \hline
\(\cos x\) & \(\sin x +c\)\\ \hline
\(e^x\) & \(e^x +c\)\\ \hline
\(\frac{1}{x}\) & \(\ln x +c\)\\ \hline
\end{tabular}
\end{center}
\item Methods of integration:
\begin{enumerate}
\item Algebra - simplify to a form that can be integrated
\item Substitution - Replace a function with a different variable. Remember to change limits and replace \(dx\)
\item Integration by parts:
\[\int u(x)v'(x)\,dx\equiv u(x)v(x)-\int u'(x)v(x)\,dx\]
\end{enumerate}
\end{itemize}
\end{document}
