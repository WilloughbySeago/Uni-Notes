\documentclass{article}
\usepackage{parskip}
\usepackage{siunitx}
\usepackage[margin=1.5 in]{geometry}
\usepackage{fancyhdr}
\usepackage{esvect}
\usepackage{amsmath}
\usepackage{amssymb}
\pagestyle{fancy}
\lhead{Willoughby Seago}
\rhead{Physics equation list}
\cfoot{Page \thepage}
\renewcommand{\headrulewidth}{0.4pt}
\renewcommand{\footrulewidth}{0.4pt}
\title{Physics 1A Equations}
\author{Willoughby Seago}
\date{September 2018}

\newcommand{\notesVersion}{1.0}
\newcommand{\notesDate}{05/01/2021}

\begin{document}
\maketitle
This is a list of equations for the \textit{physics 1A} course from the University of Edinburgh as part of the first year of the theoretical physics degree.
When I took this course in the 2018/19 academic year it was taught by Dr Ross Galloway\footnote{\url{https://www.ph.ed.ac.uk/people/ross-galloway}}, Dr Will Hossack\footnote{\url{https://www.ph.ed.ac.uk/people/will-hossack}}, and Dr John Loveday\footnote{\url{https://www.ph.ed.ac.uk/people/john-loveday}}.
These notes are based on the lectures delivered as part of this course, and the notes provided as part of this course.
The content within is correct to the best of my knowledge but if you find a mistake or just disagree with something or think it could be improved please let me know.

These notes were produced using \LaTeX\footnote{\url{https://www.latex-project.org/}}.

This is version \notesVersion~of these notes, which is up to date as of \notesDate.
\begin{flushright}
    Willoughby Seago
    
    s1824487@ed.ac.uk
\end{flushright}
\clearpage
\section*{Constants}
\begin{center}
\begin{tabular}{|c|c|c|}
\hline
Quantity & Symbol & value\\
\hline
Speed of light in a vacuum & \(c\) & \num{2.998e8} ms\(^{-1}\)\\
Permittivity of free space & \(\varepsilon_0\) & \num{8.854e-12} Fm\(^{-1}\)\\
Coulomb's law constant & \(k\) & \(\frac{1}{4\pi\varepsilon_0}=\num{8.99e9}\) Nm\(^2\)C\(^{-2}\)\\
Boltzmann constant & \(k_B\) & \num{1.3806e23} m\(^2\) kg s\(^{-2}\) K\(^{-1}\)\\
Planck constant & \(h\) & \num{6.626e-34}\\
\hline
\end{tabular}
\end{center}
\section*{Vectors}

\(\vv{A}=x\hat{\imath}+y\hat{\jmath}+z\hat{k}\) is a vector. The length of \(\vv{A}\) is given by \(|\vv{A}|=A=\sqrt{x^2+y^2+z^2}\)

The scalar (dot) product between \(\vv{A}\) and \(\vv{B}\) which join at an angle \(\vartheta\) is given by:
\[\vv{A}\cdot\vv{B}=AB\cos\vartheta\]

The vector (cross) product between \(\vv{A}\) and \(\vv{B}\) which join at an angle \(\vartheta\) is given by:
\[\vv{A}\times\vv{B}=AB\hat{n}\sin\vartheta\Rightarrow|\vv{A}\times\vv{B}|=AB\sin\vartheta\]

\section*{Kinematics}

\[\frac{d\vv{x}}{dt}=\vv{v} \hspace{1 cm} \frac{d^2\vv{x}}{dt^2}=\frac{d\vv{v}}{dt}=\vv{a}\]

\[v=v_0+at \hspace{1 cm} x-x_0=v_0t+\frac{1}{2}at^2 \hspace{1 cm} (x-x_0)=\frac{1}{2}(v_0+v)t \hspace{1 cm} v^2=v_0^2+2a(x-x_0)\]

\section*{Projectiles}

For a projectile launched on horizontal ground at velocity \(v_0\) at angle \(\vartheta\) above the horizontal:

\begin{center}
\begin{tabular}{c c}
~&~\\[0.05 cm]
Time of flight & \(t_f=\frac{2v_0\sin\vartheta}{g}\)\\ [0.25 cm]
Range & \(R=\frac{2v_0^2\sin\vartheta\cos\vartheta}{g}=\frac{v_0^2\sin2\vartheta}{g}\)\\[0.25 cm]
Trajectory equation & \(y=x\tan\vartheta-\frac{gx^2\sec^2\vartheta}{2v_0^2}\)
\end{tabular}
\end{center}

\section*{Circular motion}

\[a=\frac{v^2}{r}\Rightarrow F_C=\frac{mv^2}{r}\]
Where \(F_C\) is the centripetal force

\section*{Relativity}

For a point \(P\) in reference frames \(A\) and \(B\) the notation \(\vv{r_{PA}}\) and \(\vv{r_{PB}}\) is used to give the position vector of \(P\) in reference frame \(A\) and \(B\) respectively. The position of reference frame \(B\) in reference frame \(A\) is \(\vv{r_{BA}}\):
\[\vv{r_{PA}}=\vv{r_{PB}}+\vv{r_{BA}}\]
Differentiate w.r.t time:
\[\vv{v_{PA}}=\vv{v_{PB}}+\vv{v_{BA}}\]
Differentiate w.r.t time:
\[(\vv{a_{PA}}=\vv{a_{PB}}+\vv{a_{BA}}\]
This assumes that time is simple (ie the same in all refrence frames)\\

\section*{Newton's laws of motion}

1\(^\text{st}\) Law - A body at rest will stay at rest and a body moving at a constant velocity will continue moving at that constant velocity unless a force acts upon the body

2\(^\text{nd}\) Law - An object mass \(m\) acted on by force \(\vv{F}\) will have acceleration \(\vv{a}\) given by \(\vv{F}=m\vv{a}\)

3\(^\text{rd}\) Law - An object that has a force applied to it will apply an equal and opposite force in the opposite direction: \(\vv{F_{AB}}=-\vv{F_{BA}}\)

\section*{Gravity}

The force on an object gravitational mass \(m_1\) due to gravity is \(\vv{F_G}\). If \(\hat{y}\) is a unit vector in the direction of the local vertical and g is the magnitude of the local acceleration due to gravity then the force is given by:
\[\vv{F_G}=-m_1g\hat{y}\]
If the object is near an object of gravitational mass \(m_2\) and the seperation is \(r\) then the gravitational force between them is:
\[F_G=\frac{Gm_1m_2}{r^2}\]

\section*{Friction}

For a stationary object the frictional force \(F_F\) is equal to the applied force \(F_A\). If the normal contact force acting on the object is \(F_N\) and the coefficient of static friction is \(\mu_s\)then \(F_F\) is given as:
\[F_A=F_F\le\mu_sF_N\]
For the same object now moving the coefficient of kinetic friction is \(\mu_k\) and \(F_F\) is given by:
\[F_F=\mu_kF_N\]
Nb \(\mu_s>\mu_k\)

\section*{Air resistance}

The drag force on an object is \(F_D\). If \(\rho\) is the density of the fluid, \(A\) is the projected area of the object, \(C_D\) is the drag coefficient for the object and \(v\) is the velocity of the object then \(F_D\) is:
\[F_D=\frac{1}{2}\rho AC_Dv^2\]

\section*{Linear restoring force}

For an object displacement \(x\) the linear restoring force \(F\) acting on the object is:
\[F\propto -x\Rightarrow F=-kx\]

This comes from the general formula for SHM and Newton's 2\(^\text{nd}\) law:
\[F=ma=m\frac{d^2x}{dt^2}=-kx\]

\section*{Electrostatic force}

For two charges \(q_1\) and \(q_2\) seperated a distance \(r\) the electrostatic force \(F_E\) between them is:
\[F_E=\frac{k|q_1q_2|}{r^2}\]
Where \(k\) is a constant of proportionality equal to \(\frac{1}{4\pi\varepsilon_0}=\num{8.99e9}\) Nm\(^2\)C\(^{-2}\)

In vector form \(\hat{r}_{12}\) is a unit vector from \(q_1\) in the direction of \(q_2\). Then the electrostatic force on \(q_1\) due to \(q_2\) is \(\vv{F_{12}}\) and is given by:
\[\vv{F_{12}}=-\frac{kq_1q_2}{r^2}\hat{r}_{12}\]
The total force acting on \(q_1\) due to N charges \(q_N\)is \(F_1\) and it is given by:
\[F_1=F_{12}+F_{13}+F_{14}+\cdots=\sum_{n=2}^{N} F_{1n}\]

\section*{Work, energy and power}

Work done \(W\) by constant force \(\vv{F}\) moving an object displacement \(\vv{d}\) is defined as:
\[W\triangleq\vv{F}\cdot\vv{d}\]
If \(\vv{F}\) is not constant and the path isn't a straight line then work is defined as:
\[W\triangleq\int_{\text{start}}^{\text{finnish}}\vv{F}\cdot\vv{dr}\]
Where \(\vv{dr}\) is an infinitesimal step along the path.

In 1-D with varying force \(F(x)\) the work done is:
\[W=\int_{\text{start}}^{\text{finnish}}F(x)\,dx\]

Power \(P\) is defined as the rate that is work done. This means that average power \(P_{av}\) over time \(\Delta t\) is given by:
\[P_{av}=\frac{W}{\Delta t}\]
And instantaneous power is given by:
\[P=\frac{dW}{dt}\]
For an object moving at constant velocity \(\vv{v}\) the power is:
\[P=\vv{F}\cdot\vv{v}\]
The work done is equal to the change in kinetic energy:
\[W=K_f-K_i\]
Kintetic energy of an object mass \(m\) travelling at constant velocity \(v\) is given by:
\[K=\frac{1}{2}mv^2\]
For a conservative force the change in potential energy \(\Delta U\) is given by:
\[\Delta U=U_f-U_i=-W\]
In 1-D:
\[F=-\frac{dU}{dx}\]
Gravitational potential energy:
\[U=\frac{Gm_1m_2}{r}=mgh\]
If the total amount of energy, \(Q\) in a system doesn't change with respect to time \((\text{ie}~\frac{dQ}{dt}=0)\) then we can say that all of the force acting on the system are conservative. If energy is conserved this means that energy before = energy after:
\[K_i+U_i=K_f+U_f\]
It is necessary and sufficient to be able to write net force \(F\) in the following form to show energy is conserved:
\[F=-\frac{dU(x)}{dx}\]

\section*{Centre of mass (com)}
If the position vector of the com is \(\vv{r_{com}}\) and the system is made of \(n\) particles mass \(m_n\) displacement \(\vv{r_n}\) then:
\[\vv{r_{com}}=\frac{1}{M}\sum_{i=1}^{n}\vv{r_i}m_i\]
Where \(M=\sum_{i=1}^{n}m_i\). It is possible to do this calculation for each component of \(\vv{r}\) separately.
By taking the first derivative w.r.t time we get:
\[\vv{v_{com}}=\frac{1}{M}\sum_{i=1}^{n}\vv{v_i}m_i\]
Taking the second derivative w.r.t time this time gives:
\[\vv{a_{com}}=\frac{1}{M}\sum_{i=1}^{n}\vv{a_i}m_i\]
Rearranging this gives:
\[M\vv{a_{com}}=\sum_{i=1}^{n}\vv{a_i}m_i\]
This is just Newton's 2\(^{\text{nd}}\) law.

\section*{Linear momentum}

The linear momentum of a system is \(\vv{p}\). For a single particle \(\vv{p}=mv\). For a system of n particles it is defined as
\[\vv{p}=M\vv{v_{com}}=\sum_{i=1}^{n}\vv{v_i}m_i\]
Differentiating this we get:
\[\vv{F}=\frac{d\vv{p}}{dt}=\frac{d(m\vv{v})}{dt}=m\vv{a}\]
This is equivalent to Newton's 2\(^{\text{nd}}\) law provided \(m\) is constant.
The change in momentum in time t can be expressed as below:
\[\vv{\Delta p}=\int_0^tF(t)\,dt=\vv{p}(t)-\vv{p}(0)\]
From this the average force \(\vv{F_{av}}\) to produce the required linear momentum change \(\\vv{\Delta p}\)in time \(\Delta t\) is
\[\vv{F_{av}}\Delta t=\vv{\Delta p}\]
If linear momentum is conserved then:
\[\frac{d\vv{P_{\text{tot}}}}{dt}=0\]
Impulse is the change in momentum during a collision.

\section*{Angular momentum}

The values used in angular momentum are defined such that they have linear counterparts so the same equations can be used. For this table \(r\) is the radius of the circle traveled.
\begin{center}
\begin{tabular}{|c|c|c|c|c|} \hline
Linear quantity & Symbol & Angular quantity &  symbol & Relation\\ \hline
Position & \(x\) & Angle & \(\vartheta\) & \(x=r\vartheta\)\\
Velocity & \(\vv{v}\) & Angular velocity & \(\vv{\omega}\) & \(v=r\omega\)\\
Acceleration & \(\vv{a}\) & Angular acceleration & \(\vv{\alpha}\) & Radial: \(\vv{a_t}=r\frac{d\vv{\omega}}{dt}=r\vv{\alpha}\) \\
~&~&~&~&Tangential: \(\vv{a_r}=\frac{\vv{v}^2}{r}=r\vv{\omega}^2\)\\
Inertial mass & \(m\) & Moment of inertia & \(I\) & Depends on the body being rotated\\
Linear momentum & \(\vv{p}\) & Angular momentum & \(\vv{L}\) & \(\vv{L}=\vv{r}\times\vv{p}=m(\vv{r}\times\vv{v})\)\\
Force & \(\vv{F}\) & Torque & \(\vv{\tau}\) & \(\vv{\tau}=\vv{r}\times\vv{F}\)\\ \hline
\end{tabular}
\end{center}

Moment of inertia for specific shapes where the mass of the shape is \(M\) and the shapes have constant density:
\begin{center}
\begin{tabular}{|c|c|}\hline
Shape & Moment of inertia (kg m\(^2\))\\ \hline
Collection of \(n\) particles mass \(m_i\) distance \(r_i\) from the axis of rotation & \(\sum_{i=1}^nm_ir_i^2\)\\
Ring radius \(R\) about its centre & \(MR^2\)\\
Thin rod length \(L\) about its end & \(\frac{1}{3}ML^2\)\\
Thin rod length \(L\) about its centre & \(\frac{1}{12}ML^2\)\\
Thin disk radius \(R\) about its centre & \(\frac{1}{2}MR^2\)\\
Thin cylinlrical shell radius \(R\) about the axis along its length & \(MR^2\)\\
Solid cylinder radius \(R\) about the axis along its length & \(\frac12MR^2\)\\
Thin sperical shell radius \(R\) & \(\frac23MR^2\)\\
Solid sphere radius \(R\) & \(\frac25MR^2\)\\
\hline
\end{tabular}
\end{center}
Parallel axis theorem - For an object rotated around its center of mass its moment of inertia is \(I_{com}\). It has mass \(M\). If it is rotated about point \(P\) such that \(P\) is distance r perpendicular from the centre of mass. If the two axis of rotation are parallel and its moment of inertia about \(P\) is \(I\) then:
\[I=I_{com}+Mr^2\]
The rotational kinetic energy is:
\[k_{rot}=\frac{1}{2}I\omega^2\]
This means that the total rotational velocity is:
\[K_{tot}=K_{lin}+K_{rot}=\frac{1}{2}mv^2+\frac{1}{2}I\omega^2\]
Angular momentum rules:
\[|\vv{L}|=I\omega\]
For external torque \(\vv{\tau_{ext}}\) acting on a system:
\[\vv{\tau_{ext}}=\frac{d\vv{L}}{dt}\]
This is Newton's 2\(^\text{nd}\) law for rotation. It implies and is implied by all of the below:
\[\tau = I\alpha = I\frac{d\omega}{dt} = I\frac{d^2\vartheta}{dt^2}\]
If \(\vv{\tau_{ext}}=0\) then angular momentum is conserved.

\section*{Simple harmonic motion}

The force associated with SHM is linear and restoring:
\[F=-kx\]
The associated potential energy is given by:
\[U(x)=U(0)+\frac12kx^2\]
The force and potential energy are linked by:
\[F(x)=-\frac{dU(x)}{dx}\]
Since \(x=0\) is the equilibrium point \(F(0)=0\implies \frac{dU(x)}{dx}|_{x=0}=0\). This means that \(U(x)\) has a turning point at \(x=0\). It is also true that \(\frac{d^2U(x)}{dx^2}=-\frac{dF(x)}{dx}=k\). 

The key equation of SHM is:
\[ma=m\frac{d^2x}{dt^2}=m\ddot x=F(x)=-kx\]
We write this as:
\[\ddot x=-\omega^2x\]
Where \(omega\) is the angular frequency with units s\(^{-1}\). Depending on the system in question \(\omega\) can take different values which can be expressed as a function of constants related to the system in question.
\begin{center}
\begin{tabular}{|c|c|}\hline
System & \(omega\) \\ \hline
Mass spring system (mass \(m\), spring constant \(k\))& \(\omega=\sqrt{\frac km}\) \\ \hline
Pendulum (length \(L\), acceleration due to gravity \(g\)) & \(\omega=\sqrt{\frac gL}\) \\ \hline
\end{tabular}
\end{center}
There are two solutions \(x(t)\) to the SHM equation \(\left(\ddot x=-\omega^2x\right)\). It is also possible to find \(\dot{x}(t)\) and \(\ddot{x}(t)\). \(x_m\) is the maximum value that \(x\) can take, \(t\) is the time in seconds from the start and \(\varphi\) is the phase difference in radians.
\begin{center}
\begin{tabular}{|c||c|}\hline
\(x(t)=x_m \cos(\omega t+\varphi)\) & \(x(t)=x_m \sin(\omega t+\varphi)\) \\
\(\dot x(t)=-\omega x_m \sin(\omega t+\varphi)\) & \(\dot x(t)=\omega x_m\cos(\omega t+\varphi)\) \\
\(\ddot x(t)=-\omega^2 x_m \cos(\omega t+\varphi)\) & \(\ddot x(t)=-\omega^2 x_m\sin(\omega t+\varphi)\)\\ \hline
\end{tabular}
\end{center}
Since this is the solution to a second order differential equation it is also true that \(x(t)=A\sin(\omega t+\varphi)+B\cos(\omega t+\varphi)\) is a solution. These solutions are equivalent as they all just depend on the value of \(\varphi\) since sine and cosine are linked by \(\sin\vartheta=\cos(\vartheta+\frac\pi2)\) and \(\cos\vartheta=\sin(\vartheta+\frac\pi2)\)

\(x_m\) is called the amplitude. Since both cine and cosine are limited to between 1 and -1 \(x\) is limited to between \(x_m\) and \(-x_m\). The time period, \(T\) of the system is the time taken for one cycle to be completed. The frequency \(f\) is the number of cycles per second. These values are linked by:
\[f=\frac1T=\frac{\omega}{2\pi}\]

Another way of visualising SHM is an object moving around a circle radius \(x_m\) where the object has angular velocity \(\omega\) and starts at angular displacement \(\vartheta_0=\varphi\). At time \(t\) the angular displacement is \(\omega t+\varphi\) and the \(x\) coordinate is the projection of the radial vector onto the x axis \(x=x_m\cos(\omega t+\varphi)\)

If \(x\) is linear displacement then \(\dot x(t)=v(t)\) and \(\ddot x(t)=a(t)\).

It is possible to calculate \(\varphi\) and \(x_m\) from initial displacement \(x(0)\), initial velocity \(v(0)\) and \(\omega\):
\[\tan\varphi=-\frac{v(0)}{\omega x(0)}\qquad x_m^2=[x(0)]^2+\frac{[v(0)]^2}{\omega^2}\]

For a pendulum \(x=\vartheta\) and \(\ddot x=\ddot \vartheta\) A pendulum follows the SHM equation iff \(\vartheta\) is small as the small angle approximation \(\sin\vartheta=\vartheta\) is made.

In general a system follows SHM if its equation of motion can be written as:
\[\ddot x=-\omega^2x\]
Where
\[\omega=\sqrt{\frac{\text{measure of stiffness of the system}}{\text{measure of inertia of the system}}}\]
Energy in the system undergoing SHM is present in two forms, kinetic and potential. The total energy is conserved:
\[E=K+U=\frac12m\dot x^2+\frac12kx\]
At maximum displacement \((x_m)\) the potential energy is at a maximum \(\left(U(t)=\frac12kx_m^2=E\right)\) and \(K=0\). At equilibrium the kinetic energy is a maximum \(\left(K=\frac12mx_m^2\omega^2=E\right)\) and \(U=0\).

Iff, \(E=K+U=\frac12Sx^2+\frac12I\dot x^2=\)constant, then it can be shown that \(\omega=\sqrt{\frac SI}\)

If there is driving:
\[F_{\text{driving}}=F_D\cos(\omega_Dt)\]
\[m\ddot x=-kx+F_D\cos(\omega_Dt)\]
\[x_m=\frac{F_D}{m(\omega^2-\omega_D^2)}\]
\[\lim_{\omega_D\to\omega}x_m=\infty\]
I there is damping
\[F_{\text{damping}}=-b\dot x\]
\[m\ddot x=-kx-b\dot x\]
\[x_m(t)=x_m(0)e^{-\frac{\gamma t}{2}}\]
Where \(\gamma=\frac bm\)

If there is driving and damping:
\[m \ddot x=-kx+F_D\cos(\omega_Dt)-b\dot x\]
This is SHM frequency \(\omega_D\) with phase different to that of the driving force and a large but not infinite amplitude.
Combining all of this into one equation:
\[x(t)=x_m\cos(\omega_Dt+\varphi)\]
Where \[x_m=\frac{F_D}{m\sqrt{(\omega_D^2-\omega^2)^2+\gamma^2\omega_D^2}}\]
and \[\tan\varphi=\frac{\gamma\omega_D}{\omega^2-\omega_D^2}\]
\end{document}
