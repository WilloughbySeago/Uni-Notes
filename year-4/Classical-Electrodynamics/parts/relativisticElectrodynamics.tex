\part{Relativistic Electrodynamics}
\chapter{Relativistic Electrodynamic Quantities}
\section{Four Current}
Recall that the continuity equation for charge density/current is
\begin{equation}
    \diffp{\rho}{t} + \div\vv{\rho} = 0.
\end{equation}
The right hand side of this is a Lorentz scalar.
We want to write this in a way that makes it relativistic, so the entire equation is invariant under transformations.
First we notice that we have time and space derivatives, which in relativity we should combine into a single derivative, \(\partial_\mu\).
This is a covariant vector and we want to combine it with an object to give a Lorentz scalar, so said object should be contravariant.
Call this object \(J^\mu = (J^0, \vv{J})\).
We then look to define \(J^\mu\) such that
\begin{equation}
    0 = \partial_\mu J^\mu = \frac{1}{c}\diffp{J^0}{t} + \div \vv{J}.
\end{equation}
From this we see that we should interpret \(J^0\) as \(\rho c\) and \(\vv{J}\) is indeed the current density.
We call \(J^\mu\) the \defineindex{four-current}, and it follows the \defineindex{relativistic continuity equation}
\begin{equation}
    \partial_\mu J^\mu = 0.
\end{equation}

We know that \(J^\mu\) is contravariant, so it transforms as
\begin{equation}
    J'^\mu = \tensor{\Lambda}{^\mu_\nu} J^\nu.
\end{equation}
For a standard Lorentz transformation we then have
\begin{equation}
    \begin{pmatrix}
        \rho' c\\ J'^1\\ J'^2\\ J'^3
    \end{pmatrix}
    =
    \begin{pmatrix}
        \gamma & -\gamma\beta & 0 & 0\\
        -\gamma\beta & \gamma & 0 & 0\\
        0 & 0 & 1 & 0\\
        0 & 0 & 0 & 1
    \end{pmatrix}
    \begin{pmatrix}
        \rho c\\ J^1\\ J^2\\ J^3
    \end{pmatrix}
    =
    \begin{pmatrix}
        \gamma\rho c - \gamma\beta J^1\\
        \gamma J^1 - \gamma\beta \rho c\\
        J^2\\
        J^3
    \end{pmatrix}
    .
\end{equation}
Notice that this clearly shows that \(\rho\) is \emph{not} a scalar.
We can interpret the fact that \(\rho\) changes between frames by understanding that \(\rho\) is a density, and hence changes when the volume element changes, which happens in special relativity due to Lorentz contraction, in particular \(\dl{^3x} \to \dl{^3x}/\gamma\).
This means that if there is no current then we have \(\rho' = \gamma \rho\).
Of course, in the limit of \(v \ll c\) we have \(\gamma \approx 1\) and \(\beta/c = v/c^2 \approx 0\) so we recover \(\rho \approx \rho'\).

\subsection{Feynman's Wire}
Consider a neutral wire of cross sectional area \(A\).
We can consider the wire to be made of two charge densities, one positive, \(\rho_+\), and one negative, \(\rho_-\), such that the total charge density, \(\rho = \rho_+ + \rho_-\), is zero.
For a metal we think of \(\rho_+\) as the ions and \(\rho_-\) as the sea of delocalised electrons.
In order for this to hold we must have that \(\rho_{\pm} = \pm \rho_0\) for some constant density \(\rho_0\).
When there is a current in the wire this results in the negative charges moving with velocity \(\vv{v}\), giving a current density \(\vv{J} = \rho_-\vv{v}\).

\begin{figure}
    \tikzsetnextfilename{feynman-wire}
    \begin{tikzpicture}
        \draw (0, 0) -- (7, 0) arc (-20:20:5) coordinate (A) -- ++ (-7, 0);
        \draw[xscale=-1] (-7, 0) arc (-20:20:5);
        \draw[xscale=-1] (0, 0) arc (-20:20:5);
        \path (7, 0) -- (A) node[midway] (B) {\(A\)};
        \node at ($(B) - (4.5, 0)$) {\(\rho_-\)};
        \draw[->] ($(B) - (4.3, 0)$) -- ++ (1, 0) node[right] {\(\vv{v}\)};
        \node at (3.3, 0.8) {\(\rho_+\)};
        \draw[->] ($(B) + (1.4, 0)$) -- ++ (1, 0) node[right] {\(\vh{z}\)};
        \draw[->] ($(B) + (1.4, 0)$) -- ++ (-0.4, -0.4) node[left] {\(\vh{\varphi}\)};
        \draw[->] ($(B) + (1.4, 0)$) -- ++ (0, 1) node[above] {\(\vh{r}\)};
    \end{tikzpicture}
    \caption[Feynman's Wire]{Feynman's wire with cross sectional area \(A\) carrying current \(\vv{J} = \rho_-\vv{v}\).}
\end{figure}

We can compute the magnetic field outside of the wire using Amp\`ere's law,
\begin{equation}
    \curl\vv{B} = \frac{1}{c}\vv{J},
\end{equation}
since \(\diffp{\vv{E}}/{t} = 0\) in this scenario.
In order to be useful in this situation we need the integral form of Amp\`ere's law.
For this we introduce an Amp\`erian loop encircling the wire at a distance \(r\), greater than the radius of the wire, and we integrate Amp\`ere's law over a surface bounded by this loop.
The left hand side then becomes
\begin{equation}
    \int_S (\curl\vv{B}) \cdot \dl{\vv{S}} = \oint \vv{B} \cdot \dl{\vv{r}}
\end{equation}
through an application of Stoke's theorem, and the right hand side becomes
\begin{equation}
    \frac{1}{c}\int_S \vv{J} \cdot \dl{\vv{S}} = \frac{1}{c}\rho_-vA,
\end{equation}
where we have used the fact that \(\vv{J}\) is \(\rho_-\vv{v}\) inside the wire and zero outside the wire.

By symmetry we have that \(\vv{B}\) depends only on \(r\), the distance from the centre of the wire.
The right hand rule tells us that \(\vv{B}\) will be in the \(\vh{\varphi}\) direction.
So, \(\vv{B}(\vv{r}) = B(r)\vh{\varphi}\).
Hence we have
\begin{equation}
    \oint \vv{B} \cdot \dl{\vv{r}} = 2\pi r B(r).
\end{equation}
Equating these terms and solving for \(\vv{B}\) we have that
\begin{equation}
    \vv{B}(r) = \frac{\rho_-vA}{2\pi r c}\vh{\varphi}.
\end{equation}
Note that inside the wire the only thing that changes is we replace \(A\) with \(\pi r^2\).
For a neutral wire we have that \(\rho_+ = \rho_0\) and \(\rho_- = -\rho_0\), which allows us to write the magnetic field as
\begin{equation}
    \vv{B} = -\frac{\rho_0vA}{2\pi r c}\vh{\varphi}.
\end{equation}

Suppose we place a small test charge, \(q\), at a distance \(r\) from the wire, such that the test charge also moves with velocity \(\vv{v}\).
The force on this test charge will be
\begin{equation}
    \vv{F} = \frac{q}{c}\vv{v} \times \vv{B} = \frac{q\rho_0A}{2\pi r}\frac{v^2}{c^2}\vh{r}.
\end{equation}
Here we have used the fact that \(\vv{v} = v\vh{z}\) and \(\vh{z} \times \vh{\varphi} = -\vh{r}\).
The result is that the test charge will accelerate radially away from the wire, at least initially.

Suppose now, that we transform to a frame in which the negative charges are stationary.
In this frame the positive charges now move with velocity \(-\vv{v}\).
In the first frame there was a net force on the charge, so we expect there to be a force in this frame also.
However, now that \(q\) is stationary in this frame the magnetic field cannot be the source of this force.
Instead the force is due to the electric field in this frame.
To see this we have to transform the current into our frame.

In the first frame we had
\begin{equation}
    J^\mu = (0, \rho_-\vh{z}).
\end{equation}
Transforming to the new frame with a boost along the \(z\)-axis we get that
\begin{equation}
    c\rho' = \gamma(c\rho - \beta J^z) = -\beta\rho_-v \implies \rho' = \gamma\beta^2\rho_0.
\end{equation}
So, despite being neutral in the first frame we see that the wire is charged in the new frame.
This is because densities are not Lorentz invariant, since volumes are subject to Lorentz contraction.
The result is that there is an electric field in this frame given by
\begin{equation}
    \vv{E}' = \frac{\rho_0 \gamma A}{2\pi r}\frac{v^2}{c^2}\vh{r}.
\end{equation}
This is just the standard electric field for a charged rod,
\begin{equation}
    \vv{E} = \frac{\rho A}{2\pi r}\vh{r},
\end{equation}
which can be derived from Gauss' law.
The force on the test charge is then
\begin{equation}
    \vv{F}' = q\vv{E}' = \gamma\frac{q\rho_0A}{2\pi r}\frac{v^2}{c^2}\vh{r}.
\end{equation}
Note that this force is the same as in the first frame, but with an extra factor of \(\gamma\), since we are using the force here as defined by \(\vv{F} = \diff{\vv{p}}/{t}\), rather than \(\vv{F} = \diff{\vv{p}}/{\tau}\), so the force also changes between frames due to time dilation.
The resulting motion of the test charge will of course be the same regardless of the frame we perform the calculations in.

This example shows the rather remarkable fact that we can turn electric fields into magnetic fields and vice versa by transforming to different frames.
It should be noted that there is still a magnetic field in the second frame, it just has no effect on the particle since it is stationary.
This idea will shortly lead to us combining the electric and magnetic fields into a single object.

\section{Electromagnetic Potentials}
\epigraph{Lorentz and Lorenz. They're different guys. Not that it matters, they're both dead.}{Donal O'Connell}
Recall that in the Lorenz gauge we have
\begin{equation}
    \left( \frac{1}{c^2}\diffp[2]{}{t} - \laplacian \right)\varphi = \rho, \qqand \left( \frac{1}{c^2}\diffp[2]{}{t} - \laplacian \right)\vv{A} = \frac{1}{c}\vv{J}.
\end{equation}
These are simply \cref{eqn:wave equation for scalar potential lorenz gauge,eqn:wave equation for vector potential lorenz gauge}.
We can write these more compactly using the d'Alembertian, \(\partial^2\):
\begin{equation}
    \dalembertian \varphi = \rho = \frac{1}{c}J^0, \qqand \dalembertian \vv{A} = \frac{1}{c}\vv{J} = \frac{1}{c}J^i\ve{i}.
\end{equation}
This suggests that we should define a new quantity \(A^\mu = (\varphi, \vv{A})\), such that
\begin{equation}\label{eqn:wave operator on potential gives current}
    \dalembertian A^\mu = \frac{1}{c}J^\mu.
\end{equation}
Since the right hand side is a contravariant Lorentz vector and the d'Alembertian is a scalar operator the quotient theorem means that \(A^\mu\) is a contravariant Lorentz vector also.
We call \(A^\mu\) the \defineindex{four-potential}, \define{gauge potential}\index{gauge potential|see{four-potential}}, or \define{gauge field}\index{gauge field|see{four-potential}}.

\subsection{Gauge Transformations}
Consider the Lorenz gauge condition,
\begin{equation}
    \frac{1}{c}\diffp{\varphi}{t} + \div\vv{A} = 0.
\end{equation}
We can rewrite this in relativistic notation as
\begin{equation}
    \partial_\mu A^\mu = 0,
\end{equation}
which shows that this gauge condition is Lorentz invariant.
This makes this gauge condition a good choice for our use.
Compare this to the Coulomb gauge condition, \(\div\vv{A} = 0\), which is \emph{not} Lorentz invariant, and so isn't of much use when doing relativistic electrodynamics.

Consider the generic gauge transformations
\begin{equation}
    \varphi \to \varphi' = \varphi - \frac{1}{c}\diffp{\chi}{t}, \qqand \vv{A} \to \vv{A}' = \vv{A} + \grad\chi.
\end{equation}
We can write these in terms of the four-potential as
\begin{equation}
    A^0 \to A'^0 = A^0 - \partial_0\chi, \qqand A^i \to A'^i = A^i + \partial_i\chi.
\end{equation}
The mixture of up and down indices is not good here, we can lower the time indices freely and the spatial indices at the cost of a minus sign giving
\begin{equation*}
    A_0 \to A'_0 = A_0 - \partial_0\chi, \qand -A_i \to -A'_i = -A_i + \partial_i\chi \implies A_i \to A'_i = A_i - \partial_i\chi.
\end{equation*}
We can then combine these into a single Lorentz invariant statement:
\begin{equation}
    A_\mu \to A'_\mu - \partial_\mu\chi.
\end{equation}

\chapter{The Field Tensor}
\section{Electromagnetic Fields}
The electric field is given in terms of the potentials by
\begin{equation}
    \vv{E} = -\frac{1}{c}\diffp{\vv{A}}{t} - \grad\varphi.
\end{equation}
We can write this in terms of components as
\begin{equation}
    E^i = -\frac{1}{c}\diffp{A^i}{t} - \partial_i\varphi = -\partial_0A^i - \partial_iA^0.
\end{equation}
Again, we have mixed up and down indices which we should get rid of:
\begin{equation}
    E^i = -\partial^0A^i + \partial^iA^0.
\end{equation}

The magnetic field is given by
\begin{equation}
    \vv{B} = \curl \vv{A}.
\end{equation}
The components of \(\vv{B}\) are then given by
\begin{align}
    B^1 &= \partial_2A^3 - \partial_3A^2 = -\partial^2A^3 + \partial^3A^2,\\
    B^2 &= \partial_3A^1 - \partial_1A^3 = -\partial^3A^1 + \partial^1A^3,\\
    B^3 &= \partial_1A^2 - \partial_2A^1 = -\partial^1A^2 + \partial^2A^1.\\
\end{align}
Notice how similar the components of both \(\vv{E}\) and \(\vv{B}\) are when written in terms of \(A^\mu\).
Following this similarity we define the quantity
\begin{equation}
    F^{\mu\nu} \coloneqq \partial^\mu A^\mu - \partial^\nu A^\mu,
\end{equation}
which we call the \defineindex{electromagnetic field strength tensor}.
First, notice that \(F^{\mu\nu}\) is an antisymmetric rank 2 tensor, which as given above transforms contravariantly.
Since it is antisymmetric it has at most 6 independent, non-zero components, which just so happen to correspond to the components of \(\vv{E}\) and \(\vv{B}\).
We can easily see from the definition that
\begin{equation}
    F^{i0} = -F^{0i} = E^i.
\end{equation}
A bit more work shows us that the components of \(\vv{B}\) appear in \(F\) through
\begin{align}
    \varepsilon^{ijk}B^k &= \varepsilon^{ijk}(\curl\vv{A})^k\\
    &= \varepsilon^{ijk}\varepsilon^{klm}\partial_lA^m\\
    &= -\varepsilon^{ijk}\varepsilon^{klm}\partial^lA^m\\
    &= -(\delta^{il}\delta^{jm} - \delta^{im}\delta^{jl})\partial^lA^m\\
    &= -\partial^iA^j + \partial^jA^i\\
    &= -F^{ij}.
\end{align}
Since \(F\) is antisymmetric its diagonal vanishes.
Writing out \(F\) in terms of \(E^i\) and \(B^i\) we have
\begin{equation}
    F^{\mu\nu} = 
    \begin{pmatrix}
        0 & -E^1 & -E^2 & -E^3\\
        E^1 & 0 & -B^3 & B^2\\
        E^2 & B^3 & 0 & -B^1\\
        E^3 & -B^2 & B^1 & 0
    \end{pmatrix}
    .
\end{equation}
We can lower indices on \(F^{\mu\nu}\) giving
\begin{align}
    \tensor{F}{_\mu^\nu} &= 
    \begin{pmatrix}
        0 & -E_x & -E_y & -E_z\\
        -E_x & 0 & B_z & -B_y\\
        -E_y & -B_z & 0 & B_x\\
        -E_z & B_y & -B_x & 0
    \end{pmatrix}
    ,\\
    \tensor{F}{^\mu_\nu} &= 
    \begin{pmatrix}
        0 & E_x & E_y & E_z\\
        E_x & 0 & B_z & -B_y\\
        E_y & -B_z & 0 & B_x\\
        E_z & B_y & -B_x & 0
    \end{pmatrix}
    ,\\
    F_{\mu\nu} &=
    \begin{pmatrix}
        0 & E_x & E_y & E_z\\
        -E_x & 0 & -B_z & B_y\\
        -E_y & B_z & 0 & -B_x\\
        -E_z & -B_y & B_x & 0
    \end{pmatrix}
    .
\end{align}
A handy trick for filling in the indices on the \(B\) components of the field strength tensor is that the indices follow sudoku like rules.
For example consider \(F^{21}\).
This cannot be \(B^1\) since \(F^{01} = -E^1\), and it cannot be \(B^2\) since \(F^{20} = E^2\).
So we are left with \(B^3\).
For the sign we still have to consider \(F^{ij} = -\varepsilon^{ijk}B^k\).

\section{Gauge Invariance}
The field strength tensor inherits gauge invariance from the fields.
Consider the gauge transformation \(A_\mu \to A'_\mu - \partial_\chi\), we have
\begin{align}
    F_{\mu\nu} \to F'_{\mu\nu} &= \partial_\mu(A_\nu - \partial_\nu\chi) - \partial_\nu(A_\mu - \partial_\mu\chi)\\
    &= \partial_\mu A_\nu - \partial_\nu A_\mu - \partial_\mu\partial_\nu\chi + \partial_\nu\partial_\mu\chi\\
    &= F_{\mu\nu},
\end{align}
assuming that \(\chi\) is sufficiently smooth for partial derivatives to commute.

\section{Inhomogeneous Maxwell Equations}
Consider the divergence of \(F^{\mu\nu}\):
\begin{equation}
    \partial_{\mu}F^{\mu\nu} = \partial_\mu(\partial^\mu A^\nu - \partial^\nu A^\mu) = \dalembertian A^\nu - \partial^\nu \partial_\mu A^\mu.
\end{equation}
The final term contains \(\partial_\mu A^\mu\), which vanishes in the Lorenz gauge, so in the Lorenz gauge we have that
\begin{equation}
    \partial_\mu F^{\mu\nu} = \dalembertian A^\nu = \frac{1}{c}J^\nu.
\end{equation}
This last equality is \cref{eqn:wave operator on potential gives current}, which is the defining property of \(A^\mu\).

Now, notice that the left hand side is gauge invariant, since \(F^{\mu\nu}\) is gauge invariant.
The right hand side is also gauge invariant, since the gauge transformations effect only the potentials, not the sources, which have physical meaning.
Whenever we have a start and end point that are gauge invariant it doesn't matter if we made a gauge choice in between, and so we have in \emph{all} gauges that
\begin{equation}
    \partial_\mu F^{\mu\nu} = \frac{1}{c}J^\nu.
\end{equation}

This equation immediately implies conservation of current.
Recall that in relativistic notation the continuity equation is \(\partial_\nu J^\nu = 0\), so we take the divergence of \(\partial_\mu F^{\mu\nu}\) to give us
\begin{equation}
    \partial_\nu \partial_\mu F^{\mu\nu} = \frac{1}{c}\partial_\nu J^\nu.
\end{equation}
Now, the left hand side of this is something symmetric in \(\mu\) and \(\nu\), \(\partial_\nu \partial_\mu\), and something antisymmetric in \(\mu\) and \(\nu\), \(F^{\mu\nu}\).
The product therefore vanishes.
This is a general rule, that the product of symmetric and antisymmetric components in the same indices vanishes.
In this case we can prove it by first commuting the derivatives, then swapping the indices on the field tensor giving a negative sign, and then relabelling \(\mu \leftrightarrow \nu\):
\begin{equation}
    \partial_\nu \partial_\mu F^{\mu\nu} = \partial_\mu \partial_\nu F^{\mu\nu} = -\partial_\mu \partial_\nu F^{\nu\mu} \stackrel{\mu \leftrightarrow \nu}{=} -\partial_\nu \partial_\mu F^{\mu\nu}.
\end{equation}
Finally, the only thing equal to its negative is \(0\) so we must have
\begin{equation}
    0 = \partial_\nu \partial_\mu F^{\mu\nu} = \frac{1}{c}\partial_\nu J^\nu \implies \partial_\nu J^\nu = 0.
\end{equation}
So, \(F^{\mu\nu}\) automatically encodes the conservation laws for charge and current.

Now consider again the divergence \(\partial_\mu F^{\mu\nu} = J^\nu/c\).
In order to recover our pre-relativistic-notation equations we should treat time and space separately.
First, consider the case of \(\nu = 0\).
We then have \(J^0/c = \rho\), giving
\begin{equation}
    \partial_\mu F^{\mu0} = \partial_0F^{00} + \partial_iF^{i0} = \partial_iF^{i0} = \partial_iE^i = \div\vv{E}.
\end{equation}
Here we have used the fact that \(F^{00} = 0\) and \(F^{i0} = E^i\).
Combining these results we have
\begin{equation}
    \div\vv{E} = \rho,
\end{equation}
which is Gauss' law.

This time consider \(\nu = i\).
We then have
\begin{align}
    \partial_iF^{\mu i} &= \partial_0 F^{0i} + \partial_jF^{ji}\\
    &= -\partial_0 F^{i0} - \partial_jF^{ij}\\
    &= -\partial_0 E^i + \partial_j\varepsilon^{ijk}B^k\\
    &= -\frac{1}{c}\diffp{E^i}{t} + (\curl\vv{B})^i.
\end{align}
Here we have used antisymmetry: \(F^{\mu\nu} = -F^{\nu\mu}\), and identified the components \(F^{i0} = E^i\) and \(F^{ij} = -\varepsilon^{ijk}B^k\).
We also have \(\partial_iF^{\mu i} = J^i/c\).
Combining these results and rearranging for \(\curl\vv{B}\) we get
\begin{equation}
    \curl\vv{B} = \frac{1}{c}\vv{J} + \frac{1}{c}\diffp{\vv{E}}{t}.
\end{equation}
This is Amp\`ere's law.

\section{Homogeneous Maxwell Equations}
When we introduced the potentials we did so as the solutions to the two homogeneous (source free) Maxwell equations, \(\div\vv{B} = 0\) and \(\curl\vv{E} + c^{-1}\diffp{\vv{B}}/{t} = \vv{0}\).
We therefore look for an equation for which \(F^{\mu\nu}\) is a solution causing the equation to vanish identically.

If we consider the term \(\partial_\mu F^{\mu\nu}\) appearing in the previous section to be the divergence then the fact that both divergences and curls appear in Maxwell's equations suggests that we should look for some curl-like term.
Since the curl has components \(\varepsilon^{ijk}\partial_j V^k\) we will try something similar, say \(\varepsilon^{\mu\nu\alpha\beta}\partial_\mu F_{\alpha\beta}\).
The question is, does this vanish?

Before we answer this question we introduce a new quantity, called the \defineindex{dual field strength tensor}, and defined by
\begin{equation}
    F^*_{\mu\nu} \coloneqq -\frac{1}{2}\varepsilon_{\mu\nu\alpha\beta}F^{\alpha\beta}.
\end{equation}
\begin{ntn}{}{}
    There is not a consistent notation for this quantity.
    Some sources call it \(F^*_{\mu\nu}\), like we have, others use \(G_{\mu\nu}\), or \(\tilde{F}_{\mu\nu}\).
    
    Note that the star is nothing to do with complex conjugates, the field strength tensor is a real quantity.
    Instead we can think of the star as relating to the Hodge star operator, since in the language of differential forms the potential is a one-form, \(A\), the field strength is a two-form, \(F = \dl{A}\), and the dual field strength is its Hodge dual, \(\star F\).
\end{ntn}
We can easily derive the components of the dual field tensor in terms of the potential:
\begin{align}
    F^*_{\mu\nu} &= -\frac{1}{2}\varepsilon_{\mu\nu\alpha\beta}F^{\alpha\beta}\\
    &= -\frac{1}{2}\varepsilon_{\mu\nu\alpha\beta}(\partial^\alpha A^\beta - \partial^\beta A^\alpha)\\
    &= -\frac{1}{2}\varepsilon_{\mu\nu\alpha\beta}\partial^\alpha A^\beta + \frac{1}{2}\varepsilon_{\mu\nu\alpha\beta} \partial^\beta A^\alpha\\
    &= -\frac{1}{2}\varepsilon_{\mu\nu\alpha\beta}\partial^\alpha A^\beta - \frac{1}{2}\varepsilon_{\mu\nu\beta\alpha} \partial^\beta A^\alpha\\
    &= -\varepsilon_{\mu\nu\alpha\beta}\partial^\alpha A^\beta
\end{align}
where in the last step we relabel indices in the second term, \(\alpha \leftrightarrow \beta\), and in the step before that we used the antisymmetry of the Levi-Civita symbol.

Now consider the divergence of the dual field tensor:
\begin{align}
    \partial_\mu F^{*\mu\nu} &= -\frac{1}{2}\partial_\mu \varepsilon^{\mu\nu\alpha\beta}F_{\alpha\beta}\\
    &= -\frac{1}{2}\varepsilon^{\mu\nu\alpha\beta}\partial_\mu F_{\alpha\beta}\\
    &= -\frac{1}{2}\varepsilon^{\mu\nu\alpha\beta}\partial_\mu (\partial_\alpha A_\beta - \partial_\beta A_\alpha)\\
    &= -\frac{1}{2}\varepsilon^{\mu\nu\alpha\beta}\partial_\mu\partial_\alpha A_\beta + \frac{1}{2}\varepsilon^{\mu\nu\alpha\beta}\partial_\mu\partial_\beta A_\alpha\\
    &= 0.
\end{align}
This last equality holds since the partial derivatives are symmetric and the Levi-Civita symbol is antisymmetric, and so their product vanishes.
Hence,
\begin{equation}
    \partial_\mu F^{*\mu\nu} = -\frac{1}{2}\varepsilon^{\mu\nu\alpha\beta}\partial_\mu F_{\alpha\beta} = 0.
\end{equation}

We can fairly easily work out what the components of \(F^*\) are in terms of the electric and magnetic fields.
First, consider the case of \(F^{*i0}\):
\begin{equation}
    F^{*i0} = -\frac{1}{2}\varepsilon^{i0\alpha\beta}F_{\alpha\beta}.
\end{equation}
The non-vanishing terms of this cannot have \(\alpha = 0\) or \(\beta = 0\), so we have that \(\alpha\) and \(\beta\) are spatial indices, and so
\begin{equation}
    F^{*i0} = -\frac{1}{2}\varepsilon^{i0jk}F_{jk}.
\end{equation}
now consider the Levi-Civita symbol.
Swapping the first two indices gives us a negative sign,
\begin{equation}
    F^{*i0} = \frac{1}{2}\varepsilon^{0ijk}F_{jk}.
\end{equation}
Suppose \(ijk = 123\).
Then we have \(\varepsilon^{0123} = -1\), following our convention of \(\varepsilon_{0123} \coloneqq 1\).
This means that \(\varepsilon^{0123} = -\varepsilon^{123}\).
The Levi-Civita symbol is entirely defined by a single value and requiring that it is totally antisymmetric.
This allows us to conclude from this that \(\varepsilon^{0ijk} = -\varepsilon^{ijk}\).
Hence,
\begin{align}
    F^{*0i} &= -\frac{1}{2}\varepsilon^{ijk}F_{jk}\\
    &= \frac{1}{2}\varepsilon^{ijk}\varepsilon^{jkl}B^l,
\end{align}
where we have used the fact that \(F^{jk} = -\varepsilon^{jkl}B^l\).
We now have two options, we can apply an identity for the product of two Levi-Civita symbols, first using the antisymmetry to get the indices to line up with the usual form for the identity:
\begin{align}
    F^{*0i} &= -\frac{1}{2}\varepsilon^{ijk}\varepsilon^{kjl}B^l\\
    &= -\frac{1}{2}(\delta^{ij}\delta^{jl} - \delta^{il}\delta^{jj})B^l\\
    &= -\frac{1}{2}(\delta^{il} - 3\delta^{il})B^l\\
    &= \delta^{il}B^l\\
    &= B^l.
\end{align}
We could also have gotten to the same result by considering \(\varepsilon^{ijk}\varepsilon^{jkl}\) with \(i = 1\), which means that for a non-vanishing term we must have either \(j = 2\) and \(k = 3\), or \(j = 3\) and \(k = 2\).
Either way a non-vanishing result then requires that \(l = 1\).
Doing the same with \(i = 2, 3\) we see that \(i = l\) and the values of \(j\) and \(k\) aren't important for non-vanishing terms and so \(\varepsilon^{ijk}\varepsilon^{jkl} = 2\delta^{il}\).

now consider the components \(F^{*ij}\):
\begin{equation}
    F^{*ij} = -\frac{1}{2}\varepsilon^{ij\alpha\beta}F_{\alpha\beta}.
\end{equation}
In order for this not to vanish either \(\alpha = 0\) or \(\beta = 0\).
Suppose \(\beta = 0\), we can then swap \(\alpha\) and \(\beta\) in both \(\varepsilon^{ij\alpha\beta}\) and \(F_{\alpha\beta}\).
Each swap gives a minus sign, and there are two swaps, so there is no change.
We can then relabel \(\alpha \leftrightarrow \beta\).
Therefore we have two non-zero terms, one with \(\alpha = 0\) and one with \(\beta = 0\), but with the previously mentioned index reshuffling these two terms are actually the same, so we just cancel the factor of \(1/2\).
This gives
\begin{equation}
    F^{*ij} = -\varepsilon^{ij0\beta}F_{0\beta}.
\end{equation}
In order for this not to vanish \(\beta \ne 0\), and so \(\beta\) can be replaced with a spatial index:
\begin{equation}
    F^{*ij} = -\varepsilon^{ij0k}F_{0k}.
\end{equation}
Swapping \(0\) and \(j\) on the Levi-Civita gives a minus sign, which then cancels if we further swap \(0\) and \(i\), that is \(\varepsilon^{ij0k} = -\varepsilon^{i0jk} = \varepsilon^{0ijk} = -\varepsilon^{ijk}\), with the last equality following from the previous logic about four and three index Levi-Civita symbols.
Hence,
\begin{equation}
    F^{*ij} = \varepsilon^{ijk}F_{0k}.
\end{equation}
Now raise the two indices on \(F_{0k}\).
We can raise the \(0\) for free, and raiding the \(k\) introduces a minus sign:
\begin{equation}
    F^{*ij} = -\varepsilon^{ijk}F^{0k}.
\end{equation}
Now, we know \(F^{k0} = E^k\), and \(F^{0k} = -F^{k0}\), so
\begin{equation}
    F^{*ij} = \varepsilon^{ijk}E^k.
\end{equation}

Comparing the components of the two field tensors we have
\begin{alignat}{3}
    F^{*i0} &= B^i, \qquad  & F^{*ij} &= \varepsilon^{ijk}E^k,\\
    F^{i0} &= E^i, \qquad & F^{*ij} &= -\varepsilon^{ijk}B^k.
\end{alignat}
Notice that we can obtain the dual field tensor from the field tensor, \(F \to F^*\), via \(\vv{E} \to \vv{B}\) and \(\vv{B} \to -\vv{E}\).
This is called \defineindex{electromagnetic duality}.
We can view it as a result of Maxwell's equations in a vacuum not distinguishing between the electric and magnetic fields, up to a sign, since we have
\begin{alignat}{3}
    \div\vv{E} &= 0, \qquad & \div\vv{B} &= 0,\\
    \curl\vv{E} &= -\frac{1}{c}\diffp{\vv{B}}{t}, \qquad & \curl\vv{B} &= \frac{1}{c}\diffp{\vv{E}}{t}.
\end{alignat}

\begin{app}{Complex Electromagnetic Field}{}
    One way to make the electromagnetic duality more obvious is to define a complex electromagnetic field
    \begin{equation}
        \symcal{E} \coloneqq \vv{E} + i\vv{B}.
    \end{equation}
    Then the vacuum Maxwell equations can be written compactly as two equations:
    \begin{equation}
        \div\symcal{E} = 0, \qqand \curl \vv{\symcal{E}} = \frac{i}{c}\diffp{\symcal{E}}{t}.
    \end{equation}
    Electromagnetic duality is then expressed by multiplying this quantity by \(-i\):
    \begin{equation}
        -i\symcal{E} = \vv{B} - i\vv{E}.
    \end{equation}
    This is the same result as replacing \(\vv{E}\) with \(\vv{B}\) and \(\vv{B}\) with \(-\vv{E}\).
\end{app}

We can now finally recover the homogenous Maxwell equations.
To do so we consider the divergence of the dual field tensor, which is known to vanish, \(\partial_\mu F^{*\mu\nu} = 0\).
First, consider the case of \(\nu = 0\):
\begin{align}
    0 &= \partial_\mu F^{*\mu0}\\
    &= \partial_0 F^{*00} + \partial_i F^{*i0}\\
    &= \partial_i F^{*i0}\\
    &= \partial_i B^i\\
    &= \div\vv{B}.
\end{align}
Second, consider the case of \(\nu = i\):
\begin{align}
    0 &= \partial_\mu F^{*\mu i}\\
    &= \partial_0 F^{*0i} + \partial_jF^{*ji}\\
    &= -\partial_0 F^{*i0} - \partial_j F^{*ij}\\
    &= -\partial_0 B^i - \partial_j \varepsilon^{ijk}E^k\\
    &= -\frac{1}{c}\diffp{B^i}{t} - (\curl\vv{E})^{i},
\end{align}
which gives Faraday's law:
\begin{equation}
    \curl\vv{E} = -\frac{1}{c}\diffp{\vv{B}}{t}.
\end{equation}

To summarise, the electromagnetic field tensor allows us to write the four Maxwell equations in vector calculus notation as two equations in relativistic notation:
\begin{alignat}{3}
    \partial_\mu F^{\mu\nu} &= \frac{1}{c} J^\nu \qquad && \left\{
    \begin{array}{l}
        \displaystyle \div\vv{E} = \rho,\\
        \displaystyle \curl\vv{B} = \frac{1}{c}\vv{J} + \frac{1}{c}\diffp{\vv{E}}{t},
    \end{array}
    \right.\\
    \partial_\mu F^{*\mu\nu} &= 0 \qquad && \left\{
    \begin{array}{l}
        \displaystyle \div\vv{B} = 0,\\
        \displaystyle \curl\vv{E} = -\frac{1}{c}\diffp{\vv{B}}{t}.
    \end{array}
    \right.
\end{alignat}
Notice that in the absence of sources electromagnetic duality is more obvious:
\begin{equation}
    \partial_\mu F^{\mu\nu} = 0 = \partial_\mu F^{*\mu\nu}.
\end{equation}

\section{Lorentz Transform of the Field Tensor}
The electric and magnetic field are \emph{not} Lorentz vectors, this is because they are really just a neat packaging of components of the electromagnetic field tensor, appropriate for non-relativistic electrodynamics.
The field tensor is a rank 2 contravariant tensor, and so transforms as
\begin{equation}
    F^{\mu\nu} \to F'^{\mu\nu} = \tensor{\Lambda}{^\mu_\alpha} \tensor{\Lambda}{^\nu_\beta} F^{\alpha\beta}.
\end{equation}
Consider a standard Lorentz transformation
\begin{equation}
    \tensor{\Lambda}{^\mu_\nu} = 
    \begin{pmatrix}
        \gamma & -\gamma\beta & 0 & 0\\
        -\gamma\beta & \gamma & 0 & 0\\
        0 & 0 & 1 & 0\\
        0 & 0 & 0 & 1
    \end{pmatrix}
    .
\end{equation}

Consider how the component \(E^1 = F^{10}\) transforms under this Lorentz transformation, in particular we can use the fact that the zeroth and first rows of the standard Lorentz transform are non-zero only in the zeroth and first column.
Hence,
\begin{align}
    E^1 = F^{10} \to F'^{10} &= \tensor{\Lambda}{^1_\alpha} \tensor{\Lambda}{^0_\beta}F^{\alpha\beta}\\
    &= \tensor{\Lambda}{^1_0}\tensor{\Lambda}{^0_0} F^{00} + \tensor{\Lambda}{^1_0}\tensor{\Lambda}{^0_1} F^{01} + \tensor{\Lambda}{^1_1}\tensor{\Lambda}{^0_0} F^{10} + \tensor{\Lambda}{^1_1}\tensor{\Lambda}{^0_1} F^{11}.\notag
\end{align}
We can now use the fact that \(F^{00} = F^{11} = 0\) to further simplify this giving
\begin{align}
    E^1 = F^{10} \to F'^{10} &= \tensor{\Lambda}{^1_0}\tensor{\Lambda}{^0_1} F^{01} + \tensor{\Lambda}{^1_1}\tensor{\Lambda}{^0_0} F^{10}\\
    &= -\tensor{\Lambda}{^1_0} \tensor{\Lambda}{^0_1} F^{10} + \tensor{\Lambda}{^1_1}\tensor{\Lambda}{^0_0}F^{10}\\
    &= \gamma^2(1 - \beta^2)F^{10}\\
    &= E^1,
\end{align}
since \(\gamma^2 = 1/(1 - \beta^2)\).

Now consider \(E^2 = F^{20}\), the transformation coefficient \(\tensor{\Lambda}{^2_\alpha}\) is only non-zero if \(\alpha = 2\), in which case \(\tensor{\Lambda}{^2_2} = 1\) so we have
\begin{align}
    E^2 = F^{20} \to F'^{20} &= \tensor{\Lambda}{^2_\alpha} \tensor{\Lambda}{^0_\beta}F^{\alpha\beta}\\
    &= \tensor{\Lambda}{^2_2}\tensor{\Lambda}{^0_\beta} F^{2\beta}\\
    &= \tensor{\Lambda}{^0_\beta} F^{2\beta}\\
    &= \tensor{\Lambda}{^0_0}F^{20} + \tensor{\Lambda}{^0_1}F^{21}\\
    &= \tensor{\Lambda}{^0_0}E^2 - \tensor{\Lambda}{^0_1}B^3\\
    &= \gamma E^2 - \gamma\beta B^3.
\end{align}
Similarly
\begin{equation}
    E^3 = F^{03} \to F'^{03} = \gamma (E^3 + \beta B^3).
\end{equation}

We can derive the transformation of the magnetic field in the same way, but it is easier to use electromagnetic duality, replacing \(E^i \to B^i\) and \(B^i \to -E^i\), giving
\begin{align}
    B^1 &\to B^1,\\
    B^2 &\to \gamma(B^2 + \beta E^3),\\
    B^3 &\to \gamma(B^3 - \beta E^2).
\end{align}

\subsection{Point Charge}
Consider a point charge, \(q\).
We know that in the rest frame of this charge, \(S'\), the electric field is
\begin{equation}
    \vv{E}'(\vv{r}') = \frac{q}{4\pi r'^3} \vv{r}'
\end{equation}
and the magnetic field is \(\vv{B}'(\vv{r}') = \vv{0}\).
Suppose that in a frame stationary with respect to some observer this particle has velocity \(\vv{v}\).
We want to calculate the values that the observer would measure for the electric and magnetic fields.
We are free to set up the frames such that \(\vv{v} = v\vh{x}\) and at time \(t = 0\) we have the origins of the two frames coincide, this allows us to use standard Lorentz transforms, with the slight change that if the particle has velocity \(\vv{v}\), then the \(S\) frame has velocity \(-\vv{v}\) in frame \(S'\), and so we replace \(\beta\) with \(-\beta\) in our transformations.
For simplicity we only consider the fields at time \(t = 0\).

There are two steps required to calculate the fields the observer measures:
\begin{enumerate}
    \item First, we must Lorentz transform the fields, so \(\vv{E}'(\vv{r}') \to \vv{E}(\vv{r}')\) and \(\vv{B}'(\vv{r}') \to \vv{B}(\vv{r}')\).
    \item Second, we must perform a coordinate transformation to relate positions in the particles rest frame, \(S'\), to positions in the frame at rest relative to the observer, \(S\), so \(\vv{r}' \to \vv{r}\).
\end{enumerate}

First we take the Lorentz transform of the fields.
We'll start with the electric field and come back to the magnetic field later.
We have already seen how the electric field transforms under a standard Lorentz transformation and so we can use this, we just need to be careful to swap \(\beta\) for \(-\beta\).
This gives the components
\begin{align}
    E^1 &= E'^1,\\
    E^2 &= \gamma(E'^2 + \beta B'^3) = \gamma E'^2,\\
    E^3 &= \gamma(E'^3 - \beta B'^2) = \gamma E'^3.
\end{align}
Here we have used the fact that \(B'^i = 0\) for all \(i\).
From this we can fairly easily see that the electric field at position \(\vv{r}'\), as measured in the rest frame of the particle, is
\begin{equation}
    \vv{E}(\vv{r}') = \frac{q}{4\pi r'^3}(x', \gamma y', \gamma z').
\end{equation}

We now need to make a coordinate transformation to express this same value in un-primed coordinates.
This is simple enough for a standard Lorentz transformation, using the fact that we are choosing to have \(t = 0\), we simply have \(x' = \gamma x\), \(y' = y\), and \(z' = z\).
Now consider the value \(r'^2\):
\begin{align}
    r'^2 &= x'^2 + y'^2 + z'^2\\
    &= \gamma^2 x^2 + y^2 + z^2\\
    &= \gamma^2(x^2 + y^2 + z^2) + (1 - \gamma^2)(y^2 + z^2).
\end{align}
To see why this last equality holds just expand it out and it quickly reduces to the previous line.
We can now use
\begin{equation}
    1 - \gamma^2 = 1 - \frac{1}{1 - \beta^2} = \frac{1 - \beta^2}{1 - \beta^2} - \frac{1}{1 - \beta^2} = \frac{1 - \beta^2 - 1}{1 - \beta^2} = -\frac{\beta^2}{1 - \beta^2} = -\beta^2\gamma^2.
\end{equation}
This gives
\begin{align}
    r'^2 &= \gamma^2(x^2 + y^2 + z^2) - \gamma^2\beta^2(y^2 + z^2)\\
    &= \gamma^2r^2 - \gamma^2\beta^2(y^2 + z^2)\\
    &= \gamma^2r^2\left( 1 - \beta^2\frac{y^2 + z^2}{r^2} \right).
\end{align}

\begin{figure}
    \tikzsetnextfilename{point-charge-geometry}
    \begin{tikzpicture}
        \path (3, 3, 2) coordinate (C) -- (3, 0, 0) coordinate (A) -- (0, 0, 0) coordinate (B);
        \path pic (theta) [draw=highlight, ultra thick, fill=highlight!50, angle eccentricity=1.25, ""] {angle};
        \node at ($(theta) + (0, 0.1)$) {\(\vartheta\)};
        \path pic [draw=highlight, ultra thick, fill=highlight!50, angle radius=0.4 cm] {right angle=B--A--C};
        \draw[->] (0, 0, 0) -- (3, 3, 2) node[midway, above left] {\(r^2\)} node[above right] {\(\vv{r}\)};
        \draw[dashed] (3, 3, 2) -- (3, 0, 0) node[midway, right] {\(\sqrt{y^2 + z^2}\)} -- (0, 0, 0);
        \draw[very thick, ->] (0, 0) -- (4, 0) node[right] {\(x\)};
        \draw[very thick, ->] (0, 0) -- (0, 4) node[above] {\(y\)};
        \draw[very thick, ->] (0, 0, 0) -- (0, 0, 4) node[below left] {\(z\)};
    \end{tikzpicture}
    \caption{Coordinates for describing a point charge.}
    \label{fig:point charge geometry}
\end{figure}

Now consider the angle \(\vartheta\) defined to be the angle between the position vector, \(\vv{r}\), and the \(x\)-axis.
This is shown in \cref{fig:point charge geometry}.
We can see that the side opposite \(\vartheta\) is \(\sqrt{y^2 + z^2}\), and the hypotenuse is \(r^2\).
Hence,
\begin{equation}
    \sin^2\vartheta = \frac{y^2 + z^2}{r^2}.
\end{equation}
This allows us to write \(r'^2\) in terms of \(\vartheta\):
\begin{equation}
    r'^2  = \gamma^2r^2(1 - \beta^2\sin^2\vartheta).
\end{equation}
For the electric field we actually need the value \(r'^3 = (r'^2)^{3/2}\):
\begin{equation}
    r'^3 = \gamma^3r^3(1 - \beta^2\sin^2\vartheta)^{3/2}.
\end{equation}
The electric field as measured by the observer at position \(\vv{r}\) in their frame is
\begin{equation}
    \vv{E}(\vv{r}) = \frac{q}{4\pi} \frac{1}{\gamma^3r^3(1 - \beta^2\sin^2\vartheta)^{3/2}}(\gamma x, \gamma y, \gamma z) = \frac{q}{4\pi} \frac{1}{\gamma^2r^3(1 - \beta^2\sin^2\vartheta)}\vv{r}.
\end{equation}
Note that the factors of \(\gamma\) for \(x\) comes from the coordinate transformation and the factors of \(\gamma\) for \(y\) and \(z\) come from transforming the field.

Notice that the electric field is still radial, meaning along the \(\vv{r}\) direction.
However, it is no longer spherically symmetric.
For example, if \(\vartheta = 0\) then
\begin{equation}
    \vv{E}(\vv{r}) = \frac{1}{\gamma^2}\frac{q}{4\pi r^3}\vv{r},
\end{equation}
which, for \(\vv{v} \ne \vv{0}\), is lower than the static case since \(\gamma \ge 1\).
On the other hand, for \(\vartheta = \pi/2\) we get a factor of
\begin{equation}
    \gamma^2(1 - \beta^2)^{3/2} = \gamma^2 \gamma^{-3} = \gamma^{-1}.
\end{equation}
So, the electric field is
\begin{equation}
    \vv{E}(\vv{r}) = \gamma\frac{q}{4\pi r^3}\vv{r},
\end{equation}
which is generally greater than in the static case.
The electric field is shown in \cref{fig:moving point charge}.

\begin{figure}
    \tikzsetnextfilename{point-charge-moving-field}
    \begin{tikzpicture}
        \foreach \a in {0, 30, ..., 330} {
            \draw[->] (0, 0) -- (\a:0.7);
            \draw (\a:0.65) -- (\a:1.2);
        }
        \node[fill=positive red!50, draw=positive red, very thick, text=positive red, circle, inner sep=0pt] {\(+\)};
        \begin{scope}[xshift=2.7cm]
            \foreach \a in {0, 40, 70, 90, 110, 140, 180, 220, 250, 270, 290, 320} {
                \draw[->] (0, 0) -- (\a:0.7);
                \draw (\a:0.65) -- (\a:1.2);
            }
            \node[fill=positive red!50, draw=positive red, very thick, text=positive red, circle, inner sep=0pt] {\(+\)};
        \end{scope}
        \begin{scope}[xshift=5.4cm]
            \foreach \a in {0, 50, 80, 90, 100, 130, 180, 230, 260, 270, 280, 310} {
                \draw[->] (0, 0) -- (\a:0.7);
                \draw (\a:0.65) -- (\a:1.2);
            }
            \node[fill=positive red!50, draw=positive red, very thick, text=positive red, circle, inner sep=0pt] {\(+\)};
        \end{scope}
        \begin{scope}[xshift=7.3cm]
            \foreach \a in {90, 85, 95, 270, 265, 275} {
                \draw[->] (0, 0) -- (\a:0.7);
                \draw (\a:0.65) -- (\a:1.2);
            }
            \node[fill=positive red!50, draw=positive red, very thick, text=positive red, circle, inner sep=0pt] {\(+\)};
        \end{scope}
    \end{tikzpicture}
    \caption[Electric field of a moving point charge]{The electric field of a moving point charge. The velocity of the point charge increases to the right, and the velocity is to the right. The left most point charge is stationary. As velocity increases the field becomes stronger above and below the charge and weaker in front and behind. In the extreme case of \(v = c\) we get a shock wave, much like a sonic boom.}
    \label{fig:moving point charge}
\end{figure}

We can also compute the magnetic field in the observers frame, since in their frame the charge is moving we expect that the magnetic field will not vanish.
To do so we use the fact that \(\vv{B}' = \vv{0}\).
This allows us to use the relation
\begin{equation}
    \vv{B}' = -\vv{\beta} \times \vv{E}',
\end{equation}
where \(\vv{\beta} = \vv{v}/c\).
This relation holds for a standard Lorentz transformation whenever \(\vv{B} = \vv{0}\).
This is easily shown for a standard Lorentz transformation, simply compute the cross product and show that we recover the known transformation laws for the components of \(\vv{B}\).
In our case we are actually transforming from the primed frame to the un-primed frame, and we have \(-\vv{\beta}\), instead of \(\vv{\beta}\), so
\begin{equation}
    \vv{B} = \vv{\beta} \times \vv{E}.
\end{equation}
We know that \(\vv{E} \propto \vv{r}\), and \(\vv{\beta} \propto \vv{v}\), and so in full we have
\begin{equation}
    \vv{B}(\vv{r}) = \vv{\beta} \times \vv{E} = \frac{q}{4\pi r^2} \frac{1}{\gamma^2(1 - \beta^2\sin^2\vartheta)^{3/2}}\frac{1}{c}\vv{v} \times \vv{r}.
\end{equation}

\section{Field Invariants}
Given a field strength tensor and its dual there are two invariants we can form, \(F^{\mu\nu}F_{\mu\nu}\) and \(F^{\mu\nu}F^*_{\mu\nu}\).
All other combinations, such as \(F^{*\mu\nu}F_{*\mu\nu}\), can be written in terms of these two invariants.
We can evaluate these two invariants fairly easily in terms of the electric and magnetic fields.
First, consider \(F^{\mu\nu}F_{\mu\nu}\):
\begin{align}
    F^{\mu\nu}F_{\mu\nu} &= F^{i0}F_{i0} + F^{0i}F_{0i} + F^{ij}F_{ij}\\
    &= 2F^{i0}F_{i0} + F^{ij}F_{ij}\\
    &= -2F^{i0}F^{i0} + F^{ij}F^{ij}\\
    &= -2E^iE^i + \varepsilon^{ijk}B^k\varepsilon^{ijl}B^l\\
    &= -2\vv{E}^2 + 2\delta^{kl}B^kB^l\\
    &= -2\vv{E}^2 + 2\vv{B}^2.
\end{align}
Here we have used antisymmetry, \(F^{0i}F_{0i} = -F^{i0}F_{0i} = F^{i0}F_{i0}\), raising spatial indices at the cost of a minus sign, \(F^{i0}F_{i0} = -F^{i0}\tensor{F}{^i_0} = -F^{i0}F^{i0}\) and \(F^{ij}F_{ij} = -F^{ij}\tensor{F}{^i_j} = F^{ij}F^{ij}\), the relations \(F^{i0} = E^i\) and \(F^{ij} = -\varepsilon^{ijk}B^k\), and the identity \(\varepsilon^{ijk}\varepsilon^{ijl} = 2\delta^{kl}\).

Similarly we find that
\begin{align}
    F^{\mu\nu}F^*_{\mu\nu} &= F^{i0}F^*_{i0} + F^{0i}F^*_{0i} + F^{ij}F^*_{ij}\\
    &= 2F^{i0}F^*_{i0} + F^{ij}F^*_{ij}\\
    &= -2F^{i0}F^{*i0} + F^{ij}F^{*ij}\\
    &= -2E^iB^i - \varepsilon^{ijk}B^k\varepsilon^{ijl}E^{l}\\
    &= -2\vv{E} \cdot \vv{B} - 2 \delta^{kl}B^kE^l\\
    &= -4\vv{E} \cdot \vv{B}.
\end{align}

These invariants can be used to shortcut having to do full calculations.
For example, if \(\vv{E} \cdot \vv{B} = 0\) in some frame, as is the case for electromagnetic waves, then \(\vv{E} \cdot \vv{B} = 0\) in all frames.
If \(\vv{E}^2 = \vv{B}^2\) in some frame then \(\vv{E}^2 = \vv{B}^2\) in all frames, again this is true for electromagnetic waves.
Further, if \(E < B\) in some frame then \(E > B\) in all frames, and vice versa.

We can also use these invariants to guide which transformations we might consider.
For example, often we want to make a transformation such that one field vanishes in the new frame.
If \(F^{\mu\nu}F_{\mu\nu} > 0\) then this implies that \(B < E\), and so we might look for a frame such that \(\vv{B}\) vanishes.
On the other hand if \(F^{\mu\nu}F_{\mu\nu} < 0\) then \(E < B\) and so we look for a frame in which \(\vv{E}\) vanishes.

\section{Binachi Identity}
The \defineindex{Binachi identity}, is
\begin{equation}
    \partial_\rho F_{\mu\nu} + \partial_\mu F_{\nu\rho} + \partial_\nu F_{\rho\mu} = 0.
\end{equation}
This is easily shown by expanding \(F_{\mu\nu}\) in terms of potentials and using the symmetry of partial derivatives:
\begin{align}
    \partial_\rho F_{\mu\nu} &+ \partial_\mu F_{\nu\rho} + \partial_\nu F_{\rho\mu}\\
    &= \partial_\rho(\partial_\mu A_\nu - \partial_\nu A_\mu) + \partial_\mu (\partial_\nu A_\rho - \partial_\rho A_\nu) + \partial_\nu (\partial_\rho A_\mu - \partial_\mu A_\rho)\\
    &= \textcolor{highlight}{\partial_\rho \partial_\mu A_\nu} - \textcolor{my purple}{\partial_\rho \partial_\nu A_\mu} + \textcolor{my orange}{\partial_\mu \partial_\nu A_\rho} - \textcolor{highlight}{\partial_\mu \partial_\rho A_\nu} + \textcolor{my purple}{\partial_\nu \partial_\rho A_\mu} - \textcolor{my orange}{\partial_\nu \partial_\mu A_\rho}\\
    &= 0.
\end{align}
If we consider the Binachi identity multiplied by \(\varepsilon^{\mu\nu\rho\sigma}\) we find that this implies \(\partial_\mu F^{*ms\nu} = 0\), which is what we started looking for to define \(F^{*\mu\nu}\) in the first place, so we can replace this condition with the Binachi identity if we wish.

\begin{app}{Binachi Identity}{}
    In general relativity there is a similar identity also known as the Binachi identity for the Riemann curvature tensor:
    \begin{equation}
        \nabla_\varepsilon \tensor{R}{^\alpha_{\beta\gamma\delta}} + \nabla_\gamma \tensor{R}{^\alpha_{\beta\delta\varepsilon}} + \nabla_\delta \tensor{R}{^\alpha_{\beta\varepsilon\gamma}} = 0.
    \end{equation}
    Here \(\nabla_\mu\) is a covariant derivative, which reduces to the partial derivative \(\partial_\mu\) in a locally inertial frame.
    The Riemann tensor is defined by
    \begin{equation}
        \tensor{R}{^\alpha_{\beta\gamma\delta}} = \partial_\gamma \tensor{\Gamma}{^\alpha_{\delta\beta}} - \partial_\delta \tensor{\Gamma}{^\alpha_{\gamma\beta}} + \tensor{\Gamma}{^\alpha_{\gamma\mu}}\tensor{\Gamma}{^\mu_{\beta\delta}} - \tensor{\Gamma}{^\alpha_{\delta\mu}}\tensor{\Gamma}{^\mu_{\beta\gamma}}.
    \end{equation}
    Here \(\tensor{\Gamma}{^\alpha_{\beta\gamma}}\) are the affine connections.
    In a locally inertial frame this reduces to
    \begin{equation}
        \tensor{R}{^\alpha_{\beta\gamma\delta}} = \partial_\gamma \tensor{\Gamma}{^\alpha_{\delta\beta}} - \partial_\delta \tensor{\Gamma}{^\alpha_{\gamma\beta}}.
    \end{equation}
    Notice then that if we replace \(\tensor{R}{^\alpha_{\beta\gamma\delta}} \to F_{\alpha\beta}\) and \(\tensor{\Gamma}{^\mu_{\alpha\sigma}} \to A_\alpha\) we get the definition of the field strength tensor.
    
    These two Binachi identities are related by the fact that we can express both the Riemann tensor and the field strength tensor as exterior derivatives.
    For the field strength tensor we have \(F = \dl{A}\), and then we can use the identity \(\symup{d}^2 = 0\) giving \(\dl{F} = \dl{^2A} = 0\).
    For the Riemann tensor the Riemann tensor can be expressed as the exterior derivative of the covariant derivative of a vector field.
\end{app}

\chapter{Energy and Momentum}
\section{Energy--Momentum Tensor}
The \defineindex{energy--momentum tensor}, also called the \define{stress--energy tensor}\index{stress--energy tensor|see{energy--momentum tensor}} is a central object of study in electrodynamics, as well as in wider relativity, including general relativity, where it appears as the source term in Einstein's field equations.
For the purposes of this course we will simply define it and then probe its properties.
One can also derive it as a logical quantity to study based on a few assumptions, or it is possible to derive it entirely from symmetry arguments.

\begin{dfn}{Energy--Momentum Tensor}{}
    The energy--momentum tensor, \(T\), is a rank 2 Lorentz tensor with components
    \begin{equation}
        T_{\mu\nu} \coloneqq \tensor{F}{_\mu^\alpha}\tensor{F}{_{\alpha\nu}} + \frac{1}{4}\eta_{\mu\nu}\tensor{F}{^{\alpha\beta}}\tensor{F}{_{\alpha\beta}}.
    \end{equation}
\end{dfn}
We defined the energy--momentum tensor here in terms of covariant components, but we could equally define it in terms of the contravariant components
\begin{equation}
    T^{\mu\nu} = \tensor{F}{^{\mu\alpha}}\tensor{F}{_\alpha^\nu} + \frac{1}{4}\eta^{\mu\nu}\tensor{F}{^{\alpha\beta}}\tensor{F}{_{\alpha\beta}}.
\end{equation}
Now that we have defined the energy--momentum tensor we will investigate some of its properties.

\subsection{Symmetry}
The energy--momentum tensor is symmetric, that is \(T^{\mu\nu} = T^{\nu\mu}\).
Clearly the second term is symmetric, since the only appearance of \(\mu\) and \(\nu\) is in the metric, \(\eta^{\mu\nu}\), which is symmetric.
For the first term we have
\begin{equation}
    \tensor{F}{_\mu^\alpha} \tensor{F}{_{\alpha\nu}} \to \tensor{F}{_\nu^\alpha}\tensor{F}{_{\alpha\mu}} = \tensor{F}{_{\alpha\mu}}\tensor{F}{_\nu^\alpha} = -\tensor{F}{_{\mu\alpha}}\tensor{F}{_\nu^\alpha} = -\tensor{F}{_\mu^\alpha}\tensor{F}{_{\nu\alpha}} = \tensor{F}{_\mu^\alpha}\tensor{F}{_{\alpha\nu}}.
\end{equation}
Here we have used the antisymmetry of the field tensor, \(F_{\mu\nu} = -F_{\nu\mu}\), as well as the freedom we have to raise an summed index, here \(\alpha\), as long as we lower its pair.
So the first term is also symmetric, and therefore the energy--momentum tensor is symmetric.

\subsection{Trace}
The energy--momentum tensor is traceless, that is \(\tensor{T}{^\mu_\mu} = 0\).
This is simple enough to show, first note that we can write the energy--momentum tensor in a mixed form as
\begin{equation}
    \tensor{T}{^\mu_\nu} = \tensor{F}{^{\mu\alpha}} \tensor{F}{_{\alpha\nu}} + \frac{1}{4}\tensor{\eta}{^\mu_\nu}\tensor{F}{^{\alpha\beta}}\tensor{F}{_{\alpha\beta}}.
\end{equation}
Recall that \(\tensor{\eta}{^\mu_\nu} = \tensor{\delta}{^\mu_\nu}\), and that \(\tensor{\delta}{^\mu_\mu} = 4\) in 4 dimensions.
Hence, the trace of the energy--momentum tensor is
\begin{align}
    \tensor{T}{^\mu_\mu} &= \tensor{F}{^{\mu\alpha}}\tensor{F}{_{\alpha\mu}} + \frac{1}{4}\tensor{\delta}{^\mu_\mu}\tensor{F}{^{\alpha\beta}}\tensor{F}{_{\alpha\beta}}\\
    &= \tensor{F}{^{\mu\alpha}}\tensor{F}{_{\alpha\mu}} + \frac{1}{4}\cdot 4 \tensor{F}{^{\alpha\beta}}\tensor{F}{_{\alpha\beta}}\\
    &= \tensor{F}{^{\mu\alpha}}\tensor{F}{_{\alpha\mu}} + \tensor{F}{^{\alpha\beta}}\tensor{F}{_{\alpha\beta}}\\
    &= -\tensor{F}{^{\mu\alpha}}\tensor{F}{_{\mu\alpha}} + \tensor{F}{^{\alpha\beta}}\tensor{F}{_{\alpha\beta}}\\
    &= 0.
\end{align}
In the last step we can simply relabel \(\alpha \to \mu\) and \(\beta \to \alpha\) in the last term and we see it is the same as the first but with the opposite sign.

\subsection{Components}
We can compute the components of the energy--momentum tensor in terms of the electric and magnetic fields.
This will allow us to prescribe interpretations to these terms.
We will start with the time-time terms, so \(\mu = \nu = 0\).
We have that \(\eta^{00} = 1\), and so
\begin{align}
    T^{00} &= \tensor{F}{^{0\alpha}}\tensor{F}{_\alpha^0} + \frac{1}{4} \tensor{F}{^{\alpha\beta}}\tensor{F}{_{\alpha\beta}}\\
    &= \tensor{F}{^{00}}\tensor{F}{_0^0} + \tensor{F}{^{0i}}\tensor{F}{_i^0} - \frac{1}{2}(\vv{E}^2 - \vv{B}^2)\\
    &= -\tensor{F}{^{0i}}\tensor{F}{^{i0}} - \frac{1}{2}(\vv{E}^2 - \vv{B}^2)\\
    &= \tensor{F}{^{i0}}\tensor{F}{^{i0}} - \frac{1}{2}(\vv{E}^2 - \vv{B}^2)\\
    &= E^iE^i - \frac{1}{2}(\vv{E}^2 - \vv{B}^2)\\
    &= \frac{1}{2}(\vv{E}^2 + \vv{B}^2)\\
    &= u.
\end{align}
Here \(u\) is the energy density of the electromagnetic field.
We have used here the antisymmetry of the electromagnetic field tensor, which implies \(F^{00} = 0\), the fact that raising a spatial index gives a minus sign, the component \(F^{i0} = E^i\), and the field tensor invariant
\begin{equation}
    F^{\mu\nu}F_{\mu\nu} = 2(\vv{B}^2 - \vv{E}^2).
\end{equation}

From this we can see that the interpretation of the time-time component of the energy--momentum tensor is as the energy density of the electromagnetic field.

Next consider the time-space component, so \(\nu = 0\) and \(\mu = i\), or equivalently \(\mu = 0\) and \(\nu = i\) since the energy--momentum tensor is symmetric.
We have
\begin{align}
    T^{i0} &= \tensor{F}{^{i\alpha}}\tensor{F}{_\alpha^0} + \frac{1}{4}\eta^{i0} \tensor{F}{^{\alpha\beta}}\tensor{F}{_{\alpha\beta}}\\
    &= \tensor{F}{^{i0}}\tensor{F}{_0^0} + \tensor{F}{^{ij}}\tensor{F}{_j^0}\\
    &= \tensor{F}{^{i0}}\tensor{F}{^{00}} - \tensor{F}{^{ij}}\tensor{F}{^{j0}}\\
    &= \varepsilon^{ijk}B^kE^j\\
    &= (\vv{E}\times\vv{B})^{i}\\
    &= \frac{1}{c}S^i.
\end{align}
Here \(\vv{S} \coloneqq c\vv{E}\times\vv{B}\) is the Poynting vector.
As well as the previously mentioned properties, we have used the fact that \(\eta\) is diagonal, and the components \(F^{ij} = -\varepsilon^{ijk}B^k\).

From this we can see that the interpretation of the time-space components of the energy--momentum tensor is as the energy density flux.

Finally consider the space-space components, so \(\mu = i\) and \(\nu = j\).
We then have
\begin{align}
    T^{ij} &= \tensor{F}{^{i\alpha}}\tensor{F}{_\alpha^j} + \frac{1}{4}\eta^{ij} \tensor{F}{^{\alpha\beta}}\tensor{F}{_{\alpha\beta}}\\
    &= \tensor{F}{^{i0}}\tensor{F}{_0^j} + \tensor{F}{^{ik}}\tensor{F}{_k^j} + \frac{1}{2}\delta^{ij}(\vv{E}^2 - \vv{B}^2)\\
    &= \tensor{F}{^{i0}}\tensor{F}{^{0j}} - \tensor{F}{^{ik}}\tensor{F}{^{kj}} + \frac{1}{2}\delta^{ij}(\vv{E}^2 - \vv{B}^2)\\
    &= -\tensor{F}{^{i0}}\tensor{F}{^{j0}} - \varepsilon^{ikl}B^l\varepsilon^{kjm}B^m + \frac{1}{2}\delta^{ij}(\vv{E}^2 - \vv{B}^2).
\end{align}
As well as the previously mentioned properties, we have used the fact that \(\eta^{ij} = -\delta^{ij}\).
We now use
\begin{equation}
    \varepsilon^{ikl}\varepsilon^{jkm} = \varepsilon^{ilk}\varepsilon^{kjm} = \delta^{ij}\delta^{lm} - \delta^{im}\delta^{lj}
\end{equation}
and so 
\begin{equation}
    \varepsilon^{ikl}\varepsilon^{jkm}B^lB^m = \vv{B}^2 - B^iB^j.
\end{equation}
Hence,
\begin{align}
    T^{ij} &= -E^iE^j - \vv{B}^{2} - B^iB^j \frac{1}{2}\delta^{ij}(\vv{E}^2 - \vv{B}^2)\\
    &= -\left( E^iE^j + B^iB^j - \frac{1}{2}\delta^{ij}(\vv{E}^2 + \vv{B}^2) \right)\\
    &= - [\text{Maxwell Stress Tensor}]^{ij}.
\end{align}
Here we define the \defineindex{Maxwell stress tensor} as the tensor with components
\begin{equation}
    \sigma_{ij} = E^iE^j + B^iB^j - \frac{1}{2}\delta^{ij}(\vv{E}^2 + \v{B}^2).
\end{equation}
We can think of this as the flux of three--momentum, which also gives us an interpretation of the space-space components of the energy--momentum tensor, and also explains the alternative name of the stress--energy tensor.

\subsection{Divergence}
The divergence of the energy--momentum tensor is fairly easy to compute:
\begin{align}
    \partial_\mu T^{\mu\nu} &= \partial_\mu(\tensor{F}{^{\mu\alpha}} \tensor{F}{_\alpha^\nu}) + \frac{1}{4}\eta^{\mu\nu} \partial_\mu (\tensor{F}{^{\alpha\beta}}\tensor{F}{_{\alpha\beta}})\\
    &= (\partial_\mu\tensor{F}{^{\mu\alpha}})\tensor{F}{_\alpha^\nu} + \tensor{F}{^{\alpha\mu}} \partial_\mu \tensor{F}{_\alpha^\nu} + \frac{1}{4} (\partial^\nu \tensor{F}{^{\alpha\beta}}) \tensor{F}{_{\alpha\beta}} + \frac{1}{4} \tensor{F}{^{\alpha\beta}} \partial^\nu \tensor{F}{_{\alpha\beta}}.
\end{align}
So far this is just the product rule.
In the first term we can identify \(\partial_\mu F^{\mu\alpha} = J^\alpha/c\) as the inhomogeneous Maxwell equation.
In the second term we can lower the \(\alpha\) and \(\mu\) on the first field tensor and raise the \(\mu\) on the derivative and the \(\alpha\)  on the second field tensor, so the second term becomes \(F_{\alpha\mu}\partial^\mu F^{\alpha\nu}\).
In the final term if we lower the \(\alpha\) and \(\beta\) on the first field tensor and raise them on the second then we see that the third and final terms are equal.
Hence,
\begin{align}
    \partial_\mu T^{\mu\nu} &= \frac{1}{c}J^\alpha\tensor{F}{_\alpha^\nu} + \tensor{F}{_{\alpha\mu}}\partial^\mu\tensor{F}{^{\alpha\nu}} + \frac{1}{2}\tensor{F}{_{\alpha\beta}}\partial^\nu \tensor{F}{^{\alpha\beta}}\\
    &= \frac{1}{c}J_\alpha\tensor{F}{^{\alpha\nu}} + \tensor{F}{_{\alpha\mu}}\partial^\mu\tensor{F}{^{\alpha\nu}} + \frac{1}{2}\tensor{F}{_{\alpha\beta}}\partial^\nu \tensor{F}{^{\alpha\beta}}\label{eqn:div energy momentum}
\end{align}
Now consider the final term of this.
We can use the Binachi identity, \(\partial^\nu F^{\alpha\beta} + \partial^\alpha F^{\beta\nu} + \partial^\beta F^{\nu\alpha} = 0\), to rewrite the derivative as \(\partial^\nu F^{\alpha\beta} = -\partial^\alpha F^{\beta\nu} - \partial^\beta F^{\nu\alpha}\).
The final term is therefore
\begin{equation}
    \frac{1}{2}F_{\alpha\beta} \partial^\nu F^{\alpha\beta} = -\frac{1}{2}(F_{\alpha\beta}\partial^\alpha F^{\beta\nu} - F_{\alpha\beta} \partial^\beta F^{\nu\alpha}).
\end{equation}
If we swap \(\alpha \to \mu\) and \(\beta \to \alpha\) in the first term and \(\beta \to \mu\) in the second term then we see both terms here are the same and we are left with \(-F_{\mu\alpha}\partial^\mu F^{\alpha\nu}\), which cancels with the second term in \cref{eqn:div energy momentum}.
Hence,
\begin{equation}
    \partial_\mu T^{\mu\nu} = \frac{1}{c}J_\alpha F^{\alpha\nu}.
\end{equation}

From this we see that the divergence is non-zero when there is a current flow, and hence a flow fo energy/momentum.

Consider the time component of the divergence of the energy--momentum tensor, so \(\nu = 0\).
On the one side we have
\begin{align}
    \partial_\mu T^{\mu0} &= \partial_0T^{00} + \partial_iT^{i0}\\
    &= \frac{1}{c}\diffp{u}{t} + \frac{1}{c}\div\vv{S}.
\end{align}
On the other side we have
\begin{align}
    \frac{1}{c}J_\mu F^{\mu0} = \frac{1}{c}J_0F^{00} + \frac{1}{c}J_iF^{i0} = \frac{1}{c}J_iF^{i0} = -\frac{1}{c}J^iF^{i0} = -\frac{1}{c}J^iE^i = -\frac{1}{c}\vv{J}\cdot\vv{E}.
\end{align}
We can interpret this as the power density, and we have
\begin{equation}
    \diffp{u}{t} + \div\vv{S} = -\vv{J}\cdot\vv{E}.
\end{equation}

Consider instead the space components of the divergence of the energy--momentum tensor, so \(\nu = i\).
We have
\begin{align}
    \frac{1}{c}J_\mu F^{\mu i} &= \frac{1}{c}(J_0F^{0i} + J_jF^{ji})\\
    &= \frac{1}{c}(-J_0F^{i0} - J_jF^{ij})\\
    &= \frac{1}{c}(-J^0F^{i0} + J^jF^{ij})\\
    &= \frac{1}{c}(-J^0E^i - J^j\varepsilon^{ijk}B^k)\\
    &= \frac{1}{c}(-\rho cE^i - (\vv{J}\times\vv{B})^i)\\
    &= -\left( \rho E^i + \frac{1}{c}(\vv{J} \times \vv{B})^i \right)\\
    &= -[\text{Lorentz Force Density}]^i.
\end{align}
So we can interpret this as the force.

This allows us to interpret \(\partial_\mu T^{\mu\nu} = J_\mu F^{\mu\nu}/c\) as a conservation law.
In particular,
\begin{equation}
    \diffp{p_{\symup{em}}^i}{t} + \partial_j T^{ij} = -\diffp{p_{\symup{mech}}^i}{t},
\end{equation}
where \(p_{\symup{em}}^i\) is the three-momentum associated with the electromagnetic fields and \(p_{\symup{mech}}^i\) is the three-momentum associated with the movement of matter, whose rate of change is the force.

\section{Relativistic Point Particle}
Consider a point particle.
We posit that each particle has two associated Lorentz scalars, its mass, \(m\), and its charge, \(q\).
Further the current associated with such a particle is given by \(j^\mu = qu^\mu\), where \(u^\mu = \diff{x^\mu}/{\tau}\) is the four-velocity of the particle.
We therefore have
\begin{equation}
    \frac{1}{c}F^{\mu\nu}j_\nu = \frac{q}{c}F^{\mu\nu}u_\nu.
\end{equation}
The equation of motion is then
\begin{equation}
    \diff{p^\mu}{\tau} = m\diff{u^\mu}{\tau} = m a^\mu = \frac{q}{c}F^{\mu\nu}u_\nu.
\end{equation}

We know that \(u \cdot u = c^2\) is constant and so \(u \cdot a = 0\).
Hence we have
\begin{equation}
    0 = m u\cdot a = \frac{q}{c}u_\mu F^{\mu\nu}u_\nu.
\end{equation}
This is guaranteed to be true by the fact that \(u_\mu u_\nu\) is symmetric in \(\mu\) and \(\nu\) whereas \(F^{\mu\nu}\) is antisymmetric in \(\mu\) and \(\nu\).
This is a good sign that we are going in the correct direction.

Recall that \(u^\mu = (\gamma c, \gamma\vv{v})\), \(\diff{}/{\tau} = \gamma\diff{}/{t}\), and \(p^\mu = (E/c, \vv{p})\).
Consider the \(\mu = 0\) component, we have
\begin{multline}
    \diff{p^0}{\tau} = \frac{\gamma}{c} \diff{E}{t} = \frac{q}{c}F^{0\nu}u_\nu = -\frac{q}{c}F^{\nu0}u_\nu\\
    = -\frac{q}{c}F^{i0}u_i = \frac{q}{c}F^{i0}u^i = \frac{q}{c}E^iu^i = \frac{q}{c}\vv{E}\cdot\vv{v}.
\end{multline}
Now consider the \(\mu = i\) components,
\begin{align}
    \diff{p^i}{\tau} &= \gamma\diff{p^i}{t}\\
    &= \frac{q}{c}F^{i\nu}u_\nu\\
    &= \frac{q}{c}(F^{i0}u_0 + F^{ij}u_j)\\
    &= \frac{q}{c}(E^iu_0 + \varepsilon^{ijk}B^k\gamma v^j)\\
    &= \frac{q}{v}(\gamma cE^i + \gamma(\vv{v}\times\vv{B})^i)\\
    &= q\gamma E^i + \frac{q}{c}\gamma(\vv{v}\times \vv{B})^i.
\end{align}
Hence,
\begin{equation}
    \diffp{\vv{p}}{t} = q\vv{E} + \frac{q}{c}(\vv{v} \times \vv{B}).
\end{equation}
This is exactly the Lorentz force law for a charged particle in an electromagnetic field.
This shows us that we are considering the correct current with \(j_\nu = qu_\nu\).