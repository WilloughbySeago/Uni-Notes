\chapter{Introduction}
\section{Content}
Classical electrodynamics is the study of relativistic electromagnetism.
The classical refers to the fact that this course does \emph{not} cover quantum electrodynamics, QED.
However, the content covered in this course should leave a student well prepared to start studying QED and other quantum field theories.

Broadly there are four sections to this course, we start with the fundamentals, reviewing non-relativistic electrodynamics, and special relativity, before combining them in the next section, relativistic electrodynamics.
We then study action principles, an incredibly important area of study if going on to do any field theory.
We then have a short section on Green's functions, before turning to our final topic, radiation.

This course assumes prior knowledge of the following subjects:
\begin{itemize}
    \item Maxwell's equations. (\textasteriskcentered{})
    \item The scalar and vector potentials \(\varphi\) and \(\vv{A}\). (\textasteriskcentered{} \textsection{})
    \item Conservation laws, such as  conservation of charge and energy. (\textasteriskcentered{} \textsection{})
    \item The Poynting vector, which is proportional to \(\vv{E} \times \vv{B}\). (\textasteriskcentered{})
    \item Gauge transformations, such as \(\vv{A} \to \vv{A} + \grad\chi\) and \(\varphi \to \varphi - \partial_t\chi\). (\textasteriskcentered{} \textparagraph{} \textbardbl{})
    \item Relativistic notation, such as \(x^\mu = (ct, \vv{x})\), \(x_\mu = (ct, -\vv{x})\), \(\partial_\mu = \left( \tfrac{1}{c}\partial_t, \grad \right)\), and \(\partial^\mu = \left( \tfrac{1}{c}\partial_t, -\grad \right)\). (\textdagger{} \textparagraph{} \textbardbl{})
    \item Proper time, \(\tau\). (\textdagger{} \textparagraph{})
    \item Lagrangian dynamics. (\textsection)
    \item Green's functions. (\# \textblank)
    \item Vector calculus. (\textlnot)
    \item Tensors and index notation. (\textlnot{} \textexclamdown{} \textparagraph{})
\end{itemize}
These topics are covered in the following courses, all of which have notes which can be found at \url{https://github.com/WilloughbySeago/Uni-Notes}:
\begin{itemize}
    \item[\textasteriskcentered] Electromagnetism.
    \item[\textdagger] Relativity, Nuclear, and Particle Physics (in particular the relativity bit).
    \item[\textsection] Lagrangian Dynamics.
    \item[\textparagraph] General Relativity.
    \item[\textbardbl] Quantum Theory.
    \item[\#] Fourier Analysis and Statistics (in particular the Fourier analysis bit).
    \item[\textblank] Methods of Mathematical Physics.
    \item[\textlnot] Dynamics and Vector Calculus (in particular the vector calculus part).
    \item[\textexclamdown] Methods of Theoretical Physics.
\end{itemize}

\section{Conventions and Units}
For most of the course we work in Heaviside--Lorentz units, which we will introduce in the first section.
These are the same units as used in the Quantum Theory course.
Equations \emph{not} in Heaviside--Lorentz units will be flagged with a label giving the appropriate units, so for example an equation in SI units will have an accompanying \SIpic, or an equation in Gaussian units will have \gaussianpic.

We follow the particle physics convention of using the metric signature \(({+}{-}{-}{-})\), meaning the Minkowski metric is given by \(\eta_{\mu\nu} = \eta^{\mu\nu} = \diag(+1, -1, -1, -1)\).
This matches all other courses on relativity, including Relativity Nuclear and Particle Physics, Quantum Theory, and General Relativity.

\part{Fundamentals and Review}
\chapter{Electrodynamics}
\section{Maxwell's Equations}
Recall that in SI units \defineindex{Maxwell's equations} are
\begin{align}
    \div\vv{E} &= \frac{\rho}{\varepsilon_0},\SIunits\\
    \div\vv{B} &= 0,\SIunits\\
    \curl\vv{E} &= -\diffp{\vv{B}}{t},\SIunits\\
    \curl\vv{B} &= \mu_0\vv{J} + \varepsilon_0\mu_0\diffp{\vv{E}}{t}.\SIunits
\end{align}
The first equation is called Gauss' law, the third Faraday's law and the fourth either Amp\`ere's law or the Amp\`ere--Maxwell law.
Notice the presence of the two constants \(\varepsilon_0 = \qty{8.854e-12}{\farad\per\metre}\) and \(\mu_0 = \qty{1.257e-6}{\henry\per\metre}\).

Apart from when explicitly stated otherwise the Einstein summation convention is in place.
This means that we sum over repeated indices.
We follow the convention of Latin indices, such as \(i\) and \(j\), running over \(1\), \(2\), and \(3\), and Greek indices, such as \(\mu\) and \(\nu\), running over \(0\), \(1\), \(2\), and \(3\).

From these, using various vector calculus identities, such as the fact that \(\curl(\curl\vv{V}) = \grad(\div\vv{V}) - \laplacian\vv{V}\), we can derive the wave equations
\begin{equation}
    \varepsilon_0\mu_0 \diffp[2]{\vv{E}}{t} - \laplacian\vv{E} = 0, \qqand \varepsilon_0\mu_0 \diffp[2]{\vv{B}}{t} - \laplacian\vv{B} = 0.
\end{equation}
Identifying electromagnetic waves as light we see that the speed of light is given by
\begin{equation}
    c = \frac{1}{\sqrt{\varepsilon_0\mu_0}}.
\end{equation}

In order to see the effect of the electric and magnetic fields on matter we need the \defineindex{Lorentz force law}:
\begin{equation}
    \vv{f} = \rho\vv{E} + \vv{J} \times \vv{B}. \SIunits
\end{equation}
This gives the force density, \(\vv{f}\), which is the force per unit mass, on a charge density, \(\rho\), and current density, \(\vv{J}\).
In order to use this we need to know how matter reacts to a force.
For now we can simply consider the non-relativistic response given by Newton's second law
\begin{equation}
    \vv{F} = m\vv{a},
\end{equation}
where
\begin{equation}
    \vv{F} = \int_V \vv{f} \dd{^3x}.
\end{equation}

\subsection{Lorentz--Heaviside Units}
Notice that the electric and magnetic fields have different dimensions in SI units.
The can be seen from, for example Faraday's law, which shows that
\begin{equation}
    [\curl\vv{E}] = \frac{[\vv{E}]}{\dimension{L}} = \left[ \diffp{\vv{B}}{t} \right] = \frac{[\vv{B}]}{\dimension{T}}.
\end{equation}
Rearranging this we have that
\begin{equation}
    [\vv{B}] = [\vv{E}] \dimension{L}^{-1}\dimension{T}.
\end{equation}

In order to combine the electric field and magnetic field into a single object to allow us to study their connection it is best to use a unit system where the electric and magnetic field have the same dimensions.
This can be done by slightly redefining the fields and sources to absorb some factors of \(\varepsilon_0\) and \(\mu_0\):
\begin{equation}
    \vv{E} \to \frac{1}{\sqrt{\varepsilon_0}}\vv{E}, \qquad \vv{B} \to \sqrt{\mu_0}\vv{B}, \qquad \rho \to \sqrt{\varepsilon_0}\rho, \qqand \vv{J} \to \sqrt{\varepsilon_0}\vv{J}.
\end{equation}
This defines a system of units called \defineindex{Lorentz--Heaviside units}.
Alternatively to scaling the charge and current densities we could scale the charge, \(q \to \sqrt{\varepsilon_0}q\), which then carries on through to the charge and current densities.

Maxwell's equations take a slightly different form in these units.
To see this we take the SI versions and make the above changes:
\begin{alignat}{4}
    \div\vv{E} = \frac{\rho}{\varepsilon_0} &\to \div \frac{1}{\sqrt{\varepsilon_0}}\vv{E} = \frac{\sqrt{\varepsilon_0}\rho}{\varepsilon_0} \implies \div\vv{E} = \rho,\\
    \div\vv{B} = 0 &\to \div \sqrt{\mu_0}\vv{B} \implies \div\vv{B} = 0,\\
    \curl\vv{E} = -\diffp{\vv{B}}{t} &\to \curl\frac{1}{\sqrt{\varepsilon_0}}\vv{E} = -\diffp{}{t}\sqrt{\mu_0}\vv{B}\notag\\
    &\implies \curl\vv{E} = -\sqrt{\varepsilon_0\mu_0}\diffp{\vv{B}}{t} = -\frac{1}{c}\diffp{\vv{B}}{t},\\
    \curl\vv{B} = \mu_0\vv{J} + \mu_0\varepsilon_0\diffp{\vv{E}}{t} &\to \curl\sqrt{\mu_0}\vv{B} = \mu_0\sqrt{\varepsilon_0}\vv{J} + \varepsilon_0\mu_0\diffp{}{t}\sqrt{\varepsilon_0}\vv{E}\notag\\
    &\implies \curl\vv{B} = \sqrt{\mu_0\varepsilon_0}\vv{J} + \sqrt{\mu_0\varepsilon_0}\diffp{\vv{E}}{t} = \frac{1}{c}\vv{J} + \frac{1}{c}\diffp{\vv{E}}{t}.
\end{alignat}
Another way to derive these is to start with the form of the SI equations but dropping all factors of \(\varepsilon_0\) and \(\mu_0\) and inserting factors of \(c\) to enforce the requirement that \(\vv{E}\) and \(\vv{B}\) have the same units.
For example, Faraday's law is
\begin{equation}
    \curl\vv{E} = -x\diffp{\vv{B}}{t}
\end{equation}
where \(x\) is some power of \(c\) to be determined.
Looking at the units and requiring \([\vv{E}] = [\vv{B}]\) we have
\begin{equation}
    [\vv{E}] \dimension{L}^{-1} = [x] [\vv{B}] \dimension{T}^{-1} \implies [x] = \dimension{L}^{-1}\dimension{T},
\end{equation}
so we take \(x = 1/c\) giving us the same result as before.

Either way we work it out we find that in Lorentz--Heaviside units Maxwell's equations are
\begin{align}
    \div\vv{E} &= \rho,\\
    \div\vv{B} &= 0,\\
    \curl\vv{E} &= -\frac{1}{c}\diffp{\vv{B}}{t},\\
    \curl\vv{B} &= \frac{1}{c}\vv{J} + \frac{1}{c}\diffp{\vv{E}}{t}.
\end{align}
The Lorentz force law also changes in these units.
First notice that the factor of \(\varepsilon_0\) in \(\rho \to \sqrt{\varepsilon_0}\) and \(\vv{E} \to \vv{E}/\sqrt{\varepsilon_0}\) cancel, so only the second term changes, picking up a factor of \(\sqrt{\varepsilon_0}\) from \(\vv{J}\) and \(\sqrt{\mu_0}\) from \(\vv{B}\) giving
\begin{equation}
    \vv{f} = \rho\vv{E} + \frac{1}{c}\vv{J}\times\vv{B}.
\end{equation}
Again, we can check that \(\vv{E}\) and \(\vv{B}\) have the same dimensions by using the fact that for a charge of fixed shape moving at velocity \(\vv{v}\) we have \(\vv{J} = \rho\vv{v}\), and so \(\vv{J}/c\) and \(\rho\) have the same units, meaning \(\vv{E}\) and \(\vv{B}\) must also have the same units.

\subsubsection{Gaussian Units}
There is a third set of units also in common use in electromagnetism, known as \defineindex{Gaussian units}.
These are related to Lorentz--Heaviside units by introducing a factor of \(4\pi\) to the charge: \(q \to 4\pi q\).
This introduces factors of \(4\pi\) in Maxwell's equations but removes them from solutions.
For this reason these units are often preferred for practical use with electromagnetism where the solutions are of interest, rather than in theoretical use where the equations themselves are the interesting thing.
In Gaussian units Maxwell's equations are
\begin{align}
    \div\vv{E} &= 4\pi\rho,\gaussian\\
    \div\vv{B} &= 0,\gaussian\\
    \curl\vv{E} &= -\frac{1}{c}\diffp{\vv{B}}{t},\gaussian\\
    \curl\vv{B} &= \frac{4\pi}{c}\vv{J} + \frac{1}{c}\diffp{\vv{E}}{t}.\gaussian
\end{align}
From now on we will work almost exclusively with Lorentz--Heaviside units.

\section{Conservation Laws}
\subsection{Current Conservation}
\begin{lma}{}{}
    For a smooth vector field \(\vv{V} \colon \reals^3 \to \reals^3\) we have
    \begin{equation}
        \div(\curl\vv{V}) = 0.
    \end{equation}
    \begin{proof}
        In index notation \((\curl\vv{V})_i = \varepsilon_{ijk}\partial_jV_k\).
        Hence,
        \begin{equation}
            \div(\curl\vv{V}) = \partial_i\varepsilon_{ijk}\partial_jV_k.
        \end{equation}
        Notice that \(\partial_i\partial_j\) is symmetric in \(i\) and \(j\) whereas \(\varepsilon_{ijk}\) is antisymmetric so this vanishes.
    \end{proof}
\end{lma}

Using this identity we can take any equation for the curl of a field and reduce it to zero.
Doing so for Faraday's law we have
\begin{equation}
    0 = \div(\curl\vv{E}) = -\frac{1}{c}\div\diffp{\vv{B}}{t} = -\frac{1}{c}\diffp{}{t}(\div\vv{B}) = 0.
\end{equation}
This is consistent, but doesn't give us any new information.

To get new information we have to apply this identity to Amp\`ere's law:
\begin{equation}
    0 = \div(\curl\vv{E}) = \frac{1}{c}\div\vv{J} + \frac{1}{c}\diffp{}{t}(\div\vv{E}) = \frac{1}{c}\div\vv{J} + \frac{1}{c}\diffp{\rho}{t}.
\end{equation}
We can identify this as the \defineindex{continuity equation}.
What it says is that not only is charge conserved globally, meaning that the total charge is constant, but it is conserved locally, meaning that to change the charge at a point we have to have a current of charge flow into or out of that point.
That is, if \(\rho\) is to decrease at a point, so \(\partial_t\rho < 0\), we must have that \(\div\vv{J} > 0\), which is to say there is a current coming out of that point.

\subsection{Conservation of Energy}
The equation for conservation of energy is slightly more complicated because we have to consider different forms of energy.
We will consider energy densities, which can then be integrated over a volume to get the energy.
Start by considering the mechanical work, \(w\), done by the Lorentz force.
For an infinitesimal displacement, \(\dl{\vv{r}}\), the work done is
\begin{equation}
    \dl{w} = \vv{f} \cdot \dl{\vv{r}}.
\end{equation}
Using \(\vv{v} = \diff{\vv{r}}/{t}\) we have \(\dl{\vv{r}} = \vv{v}\dd{t}\).
Hence, \(\dl{w} = \vv{f} \cdot \vv{v} \dd{t}\).
We now write the current density as \(\vv{J} = \rho\vv{v}\), from which it is clear that \(\vv{J} \times \vv{B}\) is perpendicular to the velocity, and so its dot product with the velocity vanishes:
\begin{equation}
    \dl{w} = \vv{f} \cdot \vv{v} \dd{t} = \left( \rho\vv{E} + \frac{1}{c}\vv{J} \times \vv{B} \right) \cdot \vv{v} \dd{t} = \rho\vv{E} \cdot \vv{v} \dd{t} = \vv{E} \cdot \vv{J} \dd{t}.
\end{equation}
From this we can identify the rate at which work is done:
\begin{equation}
    \diffp{w}{t} = \vv{E} \cdot \vv{J}.
\end{equation}

The problem with this is that it involves the current density, but we want a statement about the energy stored in the fields.
We can use Amp\`ere's law to eliminate \(\vv{J}\):
\begin{equation}
    \vv{J} = c\curl\vv{B} - \diffp{\vv{E}}{t}.
\end{equation}
We then have
\begin{equation}
    \vv{E} \cdot \vv{J} = c\vv{E} \cdot (\curl \vv{B}) - \vv{E} \cdot \diffp{\vv{E}}{t}.
\end{equation}
We now use the product rule to identify
\begin{equation}
    \diffp{}{t} \vv{E}^2 = 2\vv{E} \cdot \diffp{\vv{E}}{t}
\end{equation}
and so
\begin{equation}
    \vv{E} \cdot \vv{J} = c\vv{E} \cdot (\curl \vv{B}) - \frac{1}{2}\diffp{}{t}\vv{E}^2.
\end{equation}
\begin{lma}{}{}
    For smooth vector fields \(\vv{a}, \vv{b} \colon \reals^3 \to \reals^3\) we have
    \begin{equation}
        \div(\vv{a} \times \vv{b}) = \vv{a} \cdot (\curl \vv{b}) - \vv{b} \cdot (\curl \vv{a}).
    \end{equation}
    \begin{proof}
        Consider the left hand side in index notation:
        \begin{align}
            \div(\vv{a} \times \vv{b}) &= \partial_i(\vv{a}\times\vv{b})_i\\
            &= \partial_i\varepsilon_{ijk}a_jb_k\\
            &= \varepsilon_{ijk}b_k\partial_ia_j + \varepsilon_{ijk}a_j\partial_ib_k\\
            &= b_k\varepsilon_{kij}\partial_ia_j - a_j\varepsilon_{jik}\partial_ib_k\\
            &= b_k(\curl \vv{a})_k - a_k(\curl \vv{b})_j\\
            &= \vv{b} \cdot (\curl \vv{a}) - \vv{a} \cdot (\curl \vv{b}).\qedhere
        \end{align}
    \end{proof}
\end{lma}
Using this identity we have
\begin{equation}
    \div(\vv{E} \times \vv{B}) = \vv{B} \cdot (\curl \vv{E}) - \vv{E} \cdot (\curl \vv{B}) \implies c\vv{E} \cdot (\curl\vv{B}) = -c\div(\vv{E} \times \vv{B}) - c\vv{B} \cdot (\curl \vv{E}).
\end{equation}
Using Faraday's law we can replace \(\curl\vv{E}\) to get
\begin{equation}
    c\vv{E}\cdot(\curl\vv{B}) = -c\div(\vv{E} \times \vv{B}) - \vv{B}\cdot \diffp{\vv{B}}{t}.
\end{equation}
Again, using the product rule trick we can write the last term as half the time derivative of \(\vv{B}^2\) giving
\begin{equation}
    c\vv{E} \cdot (\curl\vv{B}) = -c\div(\vv{E} \times \vv{B}) - \frac{1}{2}\diffp{}{t}\vv{B}^2.
\end{equation}
Hence, we have
\begin{equation}
    \vv{E} \cdot \vv{J} = -c\div(\vv{E} \times \vv{B}) - \frac{1}{2}\diffp{}{t}(\vv{E}^2 + \vv{B}^2).
\end{equation}
We now define two quantities, first, the \defineindex{electromagnetic energy density},
\begin{align}
    u &\coloneqq \frac{1}{2}(\vv{E}^2 + \vv{B}^2),\\
    u &\coloneqq \frac{1}{2}\left( \varepsilon_0\vv{E}^2 + \frac{1}{\mu_0}\vv{B}^2 \right).\SIunits
\end{align}
Second, the \defineindex{Poynting vector},
\begin{align}
    \vv{S} &\coloneqq c\vv{E} \times \vv{B},\\
    \vv{S} &\coloneqq \frac{1}{\mu_0}\vv{E}\times \vv{B}.\SIunits
\end{align}
We then have that the power density is
\begin{equation}
    \diffp{w}{t} = \diffp{u}{t} + \div\vv{S}.
\end{equation}
For the case where an electromagnetic field does no work we can identify this as the continuity equation
\begin{equation}
    \diffp{u}{t} + \div\vv{S} = 0.
\end{equation}
This means that the only way for the energy stored in a field at a point to change is for energy to flow in or out, which allows us to identify \(\vv{S}\) as the energy flux.

For the case where an electromagnetic field does do work we get \defineindex{Poynting's theorem}
\begin{equation}
    \diffp{}{t}(u + w) + \div\vv{S} = 0.
\end{equation}
This says that for an electromagnetic process to increase the mechanical energy, by doing work, there must be a corresponding decrease in the electromagnetic energy, and vice versa.

\subsection{Conservation of Momentum}
\begin{lma}{}{lma:conservation of momentum identity}
    Given a smooth vector field, \(\vv{V} \colon \reals^3 \to \reals^3\) we have
    \begin{equation}
        [(\div\vv{V})\vv{V} - \vv{V} \times (\curl \vv{V})]_i = \partial_j\left( V_iV_j - \frac{1}{2}V^2 \right).
    \end{equation}
    \begin{proof}
        We start by writing the left hand side in index notation:
        \begin{align}
            [(\div\vv{V})\vv{V} - \vv{V} \times (\curl \vv{V})]_i &= (\partial_jV_j)V_i - \varepsilon_{ijk}V_j(\curl\vv{V})_k\\
            &= (\partial_jV_j)V_i - \varepsilon_{ijk}\varepsilon_{klm}V_j\partial_lV_m\\
            &= V_i\partial_jV_j - (\delta_{il}\delta_{jm} - \delta_{im}\delta_{jl})V_j\partial_lV_m\\
            &= \textcolor{highlight}{V_i\partial_jV_j} - \textcolor{my orange}{V_j\partial_iV_j} + \textcolor{my purple}{V_j\partial_jV_i}.
        \end{align}
        Now notice that
        \begin{equation}
            \partial_j(V_iV_j) = \textcolor{highlight}{V_i\partial_jV_j} + \textcolor{my purple}{V_j\partial_jV_i}
        \end{equation}
        and
        \begin{equation}
            \partial_iV^2 = \partial_i(V_jV_j) = 2\textcolor{my orange}{V_j\partial_iV_j}.
        \end{equation}
        We can change the index on this partial derivative using \(\delta_{ij}\):
        \begin{equation}
            \partial_iV^2 = \delta_{ij}\partial_jV^2.
        \end{equation}
        Hence, we have
        \begin{equation}
            [(\div\vv{V})\vv{V} - \vv{V} \times (\curl \vv{V})]_i = \partial_j\left( V_iV_j - \frac{1}{2} \delta_{ij} V^2 \right).
        \end{equation}
    \end{proof}
\end{lma}

Consider the Lorentz force law
\begin{equation}
    \vv{f} = \rho\vv{E} + \frac{1}{c}\vv{J}\times \vv{B}.
\end{equation}
We can use Maxwell's equations to eliminate the sources.
First, Gauss' law gives \(\rho = \div\vv{E}\), and Amp\`ere's law gives
\begin{equation}
    \frac{1}{c}\vv{J} = \curl\vv{B} - \frac{1}{c}\diffp{\vv{E}}{t}.
\end{equation}
Inserting this the Lorentz force law becomes
\begin{equation}
    \vv{f} = (\div\vv{E})\vv{E} + \left[ \curl\vv{B} - \frac{1}{c}\diffp{\vv{E}}{t} \right] \times \vv{B}.
\end{equation}
Now consider the product rule applied to \(\vv{E} \times \vv{B}\):
\begin{equation}
    \diffp{}{t}(\vv{E} \times \vv{B}) = \diffp{\vv{E}}{t}\times\vv{B} + \vv{E} \times \diffp{\vv{B}}{t} \implies -\frac{1}{c}\diffp{\vv{E}}{t} \times \vv{B} = \frac{1}{c} \vv{E} \times \diffp{\vv{B}}{t} - \diffp{}{t}\left( \frac{1}{c}\vv{E} \times \vv{B} \right).
\end{equation}
We can further eliminate the time derivative of \(\vv{B}\) using Faraday's law:
\begin{equation}
    \diffp{\vv{B}}{t} = -c\curl\vv{E}.
\end{equation}
Hence
\begin{equation}
    -\frac{1}{c}\diffp{\vv{E}}{t}\times\vv{B} = -\vv{E} \times (\curl\vv{E}) - \diffp{}{t}\left( \frac{1}{c} \vv{E} \times \vv{B} \right).
\end{equation}
Putting this into the Lorentz force law, and using the fact that the cross product is antisymmetric we get
\begin{equation}
    \vv{f} = (\div\vv{E})\vv{E} - \vv{E} \times (\curl\vv{E}) - \vv{B} \times (\curl \vv{B}) - \diffp{}{t}\left( \frac{1}{c}\vv{E} \times \vv{B} \right).
\end{equation}
Finally we can use the fact that \(\div\vv{B} = 0\) to insert a term proportional to \(\vv{B}\) to make the equation more symmetric in \(\vv{E}\) and \(\vv{B}\).
We also identify \(\vv{E} \times \vv{B} = \vv{S}/c\) giving
\begin{equation}
    \vv{f} = [(\div\vv{E})\vv{E} - \vv{E} \times (\curl \vv{E})] + [(\div\vv{B})\vv{B} - \vv{B} \times (\curl \vv{B})] - \frac{1}{c^2}\diffp{\vv{S}}{t}.
\end{equation}

We now use Newton's second law to identify that the rate of change of momentum is given by
\begin{equation}
    \diffp{\vv{P}}{t} = \vv{F} = \int_V \vv{f} \dd{^3x}.
\end{equation}
Considering just a single component of this equation we have
\begin{align}
    \diffp{P_i}{t} &= \int_V f_i \dd{^3x}\\
    &= \int_V \bigg( [(\div\vv{E})\vv{E} - \vv{E} \times (\curl \vv{E})]_i\notag\\
    &\qquad+ [(\div\vv{B})\vv{B} - \vv{B} \times (\curl \vv{B})]_i - \frac{1}{c^2}\diffp{S_i}{t} \bigg) \dd{^3x}.
\end{align}
We can now apply the identity from \cref{lma:conservation of momentum identity} giving
\begin{equation}
    \diffp{P_i}{t} = \int_V \partial_j \left[ \left( E_iE_j - \frac{1}{2}\delta_{ij}E^2 \right) + \left( B_iB_j - \frac{1}{2}\delta_{ij}B^2 \right) \right] \dd{^3x} - \frac{1}{c^2}\int_V \diffp{S_i}{t} \dd{^3x}.
\end{equation}

Now consider the divergence theorem:
\begin{equation}
    \int_V \div\vv{U} \dd{^3x} = \int_{A} \vv{U} \cdot \dl{\vv{A}} \iff \int_V \partial_jU_j \dd{^3x} = \int_{A} U_j \dd{A_j}.
\end{equation}
Here \(V\) is some closed volume with boundary \(A\).
Using this index form of the divergence theorem we get
\begin{equation}\label{eqn:rate of change of momentum}
    \diffp{P_i}{t} = \int_A \left[ \left( E_iE_j - \frac{1}{2}\delta_{ij} \right) + \left( B_iB_j - \frac{1}{2}\delta_{ij}B^2 \right) \right] \dd{A_j} - \frac{1}{c^2} \diff{}{t} \int_V S_i \dd{^3x}.
\end{equation}

Now consider the continuity equation
\begin{equation}
    0 = \div\vv{J} + \diffp{\rho}{t}.
\end{equation}
Integrating this over some volume, \(V\), with boundary \(A\) we get
\begin{align}
    0 &= \int_V \div\vv{J} \dd{^3x} + \int_V \diffp{\rho}{t} \dd{^3x}\\
    &= \int_A \vv{J} \cdot \dd{\vv{A}} + \int_V \diffp{\rho}{t} \dd{^3x}\\
    &= \int_A J_j \cdot \dd{A_j} + \int_V \diffp{\rho}{t} \dd{^3x}.
\end{align}
Now compare this to \cref{eqn:rate of change of momentum} and we can identify this as an integral form of a continuity equation.
Notice, however, that there are two indices, \(i\) and \(j\), and so we have a tensor equation.
In particular we can identify the momentum flux tensor, \(T_{ij}\), as
\begin{equation}
    T_{ij} \coloneqq  -\left[ E_iE_j + B_iB_j - \frac{1}{2}\delta_{ij}(E^2 + B^2) \right]
\end{equation}
and the electromagnetic momentum density
\begin{equation}
    p_i \coloneqq \frac{1}{c^2} S_i.
\end{equation}
The interpretation of \(T_{ij}\) is the flux of momentum along the \(i\) direction in the \(j\) direction.
We then have the continuity equation
\begin{equation}
    \partial_jT_{ij} + \diffp{p_i}{t} = 0,
\end{equation}
or in a more familiar form
\begin{equation}
    \div\underline{\symbfup{T}} + \diffp{\vv{p}}{t} = \vv{0}
\end{equation}
where we use upright bold underlined for second rank tensors and define the divergence of such a tensor to be the vector with components \([\div\underline{\symbfup{T}}]_i \coloneqq \partial_j T_ij \).


\section{Electromagnetic Potentials}
\subsection{Static Case}
We always have that
\begin{equation}
    \div\vv{B} = 0.
\end{equation}
It can be shown that if this is the case then there exists some \defineindex{vector potential}, \(\vv{A}\), such that
\begin{equation}
    \vv{B} = \curl\vv{A}.
\end{equation}
Thus if we can find \(\vv{A}\) we will have found \(\vv{B}\).
Inserting this into Faraday's law, setting time derivatives to zero for the static case, we have
\begin{equation}
    \curl\vv{B} = \curl(\curl\vv{A}) = \frac{1}{c}\vv{J}.
\end{equation}
\begin{lma}{}{}
    Given a smooth vector field \(\vv{V} \colon \reals^3 \to \reals^3\) we have
    \begin{equation}
        \laplacian\vv{V} = \grad(\div\vv{V}) - \curl(\curl\vv{V}).
    \end{equation}
    \begin{proof}
        Consider the right hand side in index notation:
        \begin{align}
            [\grad(\div\vv{V}) - \curl(\curl\vv{V})]_i &= \partial_i(\div\vv{V}) - \varepsilon_{ijk}\partial_j(\curl\vv{V})_k\\
            &= \partial_i\partial_jV_j - \varepsilon_{ijk}\partial_j\varepsilon_{klm}\partial_lV_m.
        \end{align}
        Now we use the identity
        \begin{equation}
            \varepsilon_{ijk}\varepsilon_{klm} = \delta_{il}\delta_{jm} - \delta_{im}\delta_{jl}.
        \end{equation}
        This gives
        \begin{align}
            [\grad(\div\vv{V}) - \curl(\curl\vv{V})]_i &= \partial_i\partial_jV_j - (\delta_{il}\delta_{jm} - \delta_{im}\delta_{jl})\partial_j\partial_lV_m\\
            &= \partial_i\partial_jV_j - \partial_j\partial_iV_j + \partial_j\partial_j V_i\\
            &= \partial_j\partial_j V_i\\
            &= [\laplacian \vv{V}]_i.\qedhere
        \end{align}
    \end{proof}
\end{lma}
Using this identity we have
\begin{equation}
    \curl\vv{B} = \curl(\curl\vv{A}) = \grad(\div\vv{A}) - \laplacian\vv{A} = \frac{1}{c}\vv{J}.
\end{equation}
This gives us an equation that can, in theory, be solved for the vector potential given the current density.

In the static case Faraday's law becomes \(\curl\vv{E} = \vv{0}\).
It can be shown that whenever the curl vanishes there is a \defineindex{scalar potential}, \(\varphi\), such that \(\vv{E} = -\grad\varphi\).
So if we find \(\varphi\) we have solved Faraday's law.
Substituting this definition into Gauss' law we have
\begin{equation}
    \div\vv{E} = \div(-\grad\varphi) = \rho \implies \laplacian\varphi = -\rho.
\end{equation}
This gives us an equation that we can solve for \(\varphi\) given a charge density, \(\rho\).

\subsection{Dynamic Case}
The dynamic case is similar to the static case but the non-vanishing time derivatives make things a bit more complicated.
The process is the same, we take the homogeneous Maxwell equations, that is the equations without sources, and we introduce potentials to solve these.
We then take the inhomogeneous Maxwell equations, with sources, and we derive equations for the potentials in terms of the sources, which we can then, in theory, solve for the potentials.

Since we still have \(\div\vv{B} = 0\) we still have some vector potential, \(\vv{A}\), such that \(\vv{B} = \curl\vv{A}\).
However, the equation for this vector potential will be different since we now have a time derivative term in Amp\`ere's law.

Next consider Faraday's law:
\begin{equation}
    \curl\vv{E} = -\frac{1}{c}\diffp{\vv{B}}{t} \ne \vv{0}.
\end{equation}
This means that there does \emph{not} exist a scalar potential, \(\varphi\), such that \(\vv{E} = -\grad\varphi\).
Instead, we can insert the vector potential giving
\begin{equation}
    \curl\vv{E} = -\frac{1}{c}\diffp{}{t}(\curl\vv{A}) \implies \curl\left( \vv{E} + \frac{1}{c}\diffp{\vv{A}}{t} \right) = 0.
\end{equation}
So there \emph{is} a scalar potential, \(\varphi\), such that
\begin{equation}
    \vv{E} + \frac{1}{c} \diffp{\vv{A}}{t} = -\grad\varphi \implies \vv{E} = -\grad\varphi - \frac{1}{c}\diffp{\vv{A}}{t}.
\end{equation}

We have now solved the two homogenous Maxwell equations, given that we can find the two potentials, \(\varphi\) and \(\vv{A}\).
In order to find equations for these we consider the inhomogeneous equations.
First, since \(\div\vv{E} = \rho\) we have
\begin{equation}\label{eqn:scalar potential general}
    \div\vv{E} = \div\left( -\grad\varphi - \frac{1}{c} \diffp{\vv{A}}{t} \right) \implies \laplacian\varphi + \frac{1}{c}\diffp{}{t}\div\vv{A} = -\rho.
\end{equation}
This equation si complicated, but in theory solvable.
Similarly starting with Amp\`ere's law we have
\begin{equation}
    \curl\vv{B}= \curl(\curl \vv{A}) = \frac{1}{c}\vv{J} + \frac{1}{c}\diffp{\vv{E}}{t}.
\end{equation}
Using our identity for the curl of the curl again we have
\begin{equation}
    \grad(\div\vv{A}) - \laplacian\vv{A} = \frac{1}{c}\vv{J} + \frac{1}{c}\diffp{\vv{E}}{t}.
\end{equation}
Rearranging and substituting in the potentials for \(\vv{E}\) we get
\begin{equation}\label{eqn:vector potential general}
    \laplacian\vv{A} - \frac{1}{c^2}\diffp[2]{\vv{A}}{t} - \grad\left( \div\vv{A} + \frac{1}{c}\diffp{\varphi}{t} \right) = -\frac{1}{c}\vv{J}.
\end{equation}
This is again complicated but solvable.
Fortunately we have a certain freedom in the definitions of the potentials which allows us to simplify these equations.

\section{Gauge Transformations}
Suppose we replace the vector potential, \(\vv{A}\), with another vector potential, \(\vv{A}'\), according to
\begin{equation}
    \vv{A} \to \vv{A}' = \vv{A} + \grad\chi,
\end{equation}
for some arbitrary smooth scalar field \(\chi\).
Then we have
\begin{equation}
    \curl\vv{A}' = \curl\vv{A} + \curl(\grad\chi).
\end{equation}
\begin{lma}{}{}
    For a smooth scalar field, \(f\colon\reals^3 \to \reals\), we have
    \begin{equation}
        \curl(\grad f) = 0.
    \end{equation}
    \begin{proof}
        Consider this equation in index notation:
        \begin{equation}
            [\curl(\grad f)]_i = \varepsilon_{ijk}\partial_j(\grad f)_k = \varepsilon_{ijk}\partial_j\partial_kf.
        \end{equation}
        Notice that \(\partial_j\partial_k\) is symmetric in \(j\) and \(k\), whereas \(\varepsilon_{ijk}\) is antisymmetric.
        Hence this vanishes proving our lemma.
    \end{proof}
\end{lma}
Using this we have that
\begin{equation}
    \curl\vv{A}' = \curl\vv{A} + \curl(\grad\chi) = \curl\vv{A} = \vv{B}.
\end{equation}
So the important thing, which is the \(\vv{B}\) field, is unchanged by this transformation.

Now, consider what happens if we change the scalar potential from \(\varphi\) to \(\varphi'\) according to
\begin{equation}
    \varphi \to \varphi' = \varphi - \frac{1}{c}\diffp{\chi}{t}.
\end{equation}
We then have
\begin{equation}
    -\grad\varphi' - \frac{1}{c}\diffp{\vv{A}'}{t} = -\grad\varphi + \frac{1}{c}\grad\diffp{\chi}{t} - \frac{1}{c}\diffp{\vv{A}}{t} - \frac{1}{c}\diffp{}{t}\grad\chi = -\grad\varphi - \frac{1}{c}\diffp{\vv{A}}{t} = \vv{E}.
\end{equation}
So, again, the important thing, now \(\vv{E}\), is not changed by this transformation.

We call these transformations \define{gauge transformations}\index{gauge transformation}, and the fact that \(\vv{E}\) and \(\vv{B}\) don't change under these transformations is known as \defineindex{gauge invariance}.
Essentially what we have is a freedom in choice, both pairs, \((\varphi, \vv{A})\) and \((\varphi', \vv{A}')\), describe the electromagnetic fields equally well.
We exploit this freedom to simplify equations.

\subsection{Lorenz Gauge}
Suppose that we choose to define \(\chi\) such that
\begin{equation}
    \div\vv{A} + \frac{1}{c}\diffp{\varphi}{t} = 0.
\end{equation}
This is called the \defineindex{Lorenz gauge}.
If this is the case then we have
\begin{equation}
    \div\vv{A} = -\frac{1}{c}\diffp{\varphi}{t},
\end{equation}
which when substituted into \cref{eqn:scalar potential general} gives
\begin{equation}\label{eqn:wave equation for scalar potential lorenz gauge}
    \laplacian\varphi + \frac{1}{c}\diffp{}{t}\left( -\frac{1}{c}\diffp{\varphi}{t} \right) = \laplacian\varphi - \frac{1}{c^2}\diffp[2]{\varphi}{t} = -\rho.
\end{equation}
Similarly inserting this in \cref{eqn:vector potential general} the third term vanishes giving
\begin{equation}\label{eqn:wave equation for vector potential lorenz gauge}
    \laplacian\vv{A} - \frac{1}{c^2}\diffp[2]{\vv{A}}{t} = -\frac{1}{c}\vv{J}.
\end{equation}
These are wave equations with sources which can be solved much more easily than before we imposed the Lorenz gauge condition.

Notice that in the static case these wave equations reduce to Poisson's equation:
\begin{equation}
    \laplacian\varphi = -\rho, \qqand \laplacian\vv{A} = -\frac{1}{c}\vv{J}.
\end{equation}

\subsubsection{Imposing the Lorenz Gauge}
Is it always possible to impose the Lorenz gauge condition?
Suppose we have potentials \((\varphi, \vv{A})\) such that
\begin{equation}
    \div\vv{A} + \frac{1}{c}\diffp{\varphi}{t} = \psi \ne 0.
\end{equation}
We then have
\begin{equation}
    \div(\vv{A}' - \grad\chi) + \frac{1}{c}\diffp{}{t}\left( \varphi' + \frac{1}{c}\diffp{\chi}{t} \right) = \psi.
\end{equation}
Hence, we have
\begin{equation}
    \div\vv{A}' + \frac{1}{c}\diffp{\varphi'}{t} = \psi + \laplacian\chi - \frac{1}{c^2}\diffp{\chi}{t}.
\end{equation}
So, in order to impose the Lorenz gauge condition on \((\varphi', \vv{A}')\) we need to find \(\chi\) such that the right hand side vanishes.
This can be done easily by choosing to have \(\chi\) satisfy
\begin{equation}
    \laplacian\chi - \frac{1}{c^2}\diffp{\chi}{t} = -\psi,
\end{equation}
which is again a wave equation with a source and can be solved so we can find potentials satisfying the Lorenz gauge conditions.


\chapter{Relativity}
\epigraph{I'm a great believer in having a simple life. Somehow I ended up doing physics. Obviously something went wrong.}{Donal O'Connell}
Special relativity is based on two postulates:
\begin{itemize}
    \item The speed of light is the same in all inertial frames.
    \item The laws of physics take the same mathematical form in all inertial frames.
\end{itemize}
It is the second of these which we shall focus on.
In special relativity we transform between inertial reference frames with Lorentz transformations.
This means we should look for laws of physics which are invariant under Lorentz transformations.
Consider rotations.
We know that \(\vv{x} \cdot \vv{y}\) is invariant under rotation, therefore any equation that we see made entirely of scalar products is clearly invariant under rotations.
We will now introduce a special relativity equivalent of this notation which makes it immediately clear when an equation is Lorentz invariant.
We will be motivated by two examples.

First, we ask if the current conservation law,
\begin{equation}
    \div\vv{J} + \diffp{\varphi}{t} = 0,
\end{equation}
is Lorentz invariant.
Clearly it is rotationally invariant, since it is built from dot products and scalars, since time is unaffected by rotations.
It is not immediately clear but it is also invariant under Lorentz rotations.

Second, we ask if the Lorentz force law,
\begin{equation}
    \vv{f} = \rho\vv{E} + \frac{1}{c}\vv{J}\times\vv{B},
\end{equation}
is Lorentz invariant.
Again, it is clearly invariant under rotations since the left and right hand sides are both vectors and therefore vary in the same way under rotations.
We say that this equation is \defineindex{covariant} under rotations.

\section{Lorentz Transformations}
\epigraph{If people are confused we're going to have an even more confusing thing in a moment. Which is \emph{not} my fault.}{Donal O'Connell}
Lengths and time differences aren't preserved by Lorentz transformations, one of the first things that one learns about special relativity is the existence of time dilation and length contractions.
Instead the invariant spacetime interval between two events is conserved.
In order to define this we need to combine the position, \(\vv{x} = (x, y, z)\), and the time, \(t\), into a single object, known as the position \defineindex{four-vector}:
\begin{equation}
    x^\mu \coloneqq (ct, x, y, z) = (ct, \vv{x}) = (x^0, x^1, x^2, x^3) = (x^0, x^i).
\end{equation}

The space time interval between and event, \(x^\mu\), and the origin, \((0, 0, 0, 0)\), is defined to be
\begin{equation}
    s^2 \coloneqq (ct)^2 - x^2 - y^2 - z^2 = 
    \begin{pmatrix}
        ct & x & y & z
    \end{pmatrix}
    \begin{pmatrix}
        1 & & & \\
        & -1 & & \\
        & & -1 & \\
        & & & -1
    \end{pmatrix}
    \begin{pmatrix}
        ct\\ x\\ y\\ z
    \end{pmatrix}
    .
\end{equation}
We use this to define the \defineindex{Minkowski metric}
\begin{equation}
    \eta_{\mu\nu} \coloneqq 
    \begin{pmatrix}
        1 & & & \\
        & -1 & & \\
        & & -1 & \\
        & & & -1
    \end{pmatrix}
    .
\end{equation}
We then define the scalar product according to
\begin{equation}
    x\cdot x \coloneqq x^\mu\eta_{\mu\nu}x^\nu = x_\mu x^\mu
\end{equation}
where we define
\begin{equation}
    x_\mu \coloneqq \eta_{\mu\nu}x^\nu.
\end{equation}
From this we see that
\begin{equation}
    x_\mu = (ct, -x, -y, -z) = (ct, -\vv{x}) = (x^0, -x^1, -x^2, -x^3) = (x^0, -x^i).
\end{equation}
So \(x_0 = x^0\) and \(x_i = -x^i\).
We call vectors with upper indices \defineindex{contravariant} and vectors with lower indices \defineindex{covariant}.

We define the \defineindex{Kronecker delta} to be the identity matrix
\begin{equation}
    \tensor{\delta}{^\mu_\nu} \coloneqq 
    \begin{pmatrix}
        1 & & & \\
        & 1 & & \\
        & & 1 & \\
        & & & 1
    \end{pmatrix}
    = \ident_4,
\end{equation}
and the \defineindex{inverse metric} to be
\begin{equation}
    \eta^{\mu\nu} = 
    \begin{pmatrix}
        1 & & & \\
        & -1 & & \\
        & & -1 & \\
        & & & -1
    \end{pmatrix}
    .
\end{equation}
Notice that numerically this is the same as \(\eta_{\mu\nu}\), but it plays a different role in calculations, and the numerical equivalence only holds in a flat spacetime.
The inverse metric is of course defined such that
\begin{equation}
    \eta_{\mu\nu}\eta^{\nu\rho} = \tensor{\delta}{^\rho_\mu} \iff \eta^{-1}\eta = \ident_4.
\end{equation}

Suppose we have two frames, \(S\) and \(S'\), such that the origin of \(S'\) is travelling along the \(x\)-axis of \(S\) at constant speed \(v\) and at time \(t = 0\) the origins of the frames coincide.
This is called the \define{standard configuration}\index{standard!configuration}, and the transformation between these frames is the \define{standard Lorentz transformation}\index{standard!Lorentz transformation}, and is given by
\begin{align}
    ct' = x'^0 &= \gamma(x^0 - \beta x^1),\\
    x'^1 &= \gamma(x^1 - \beta x^0),\\
    x'^2 &= x^2,\\
    x'^3 &= x^3.
\end{align}
We can write this more compactly as
\begin{equation}
    x'^\mu = \tensor{\Lambda}{^\mu_\nu}x^\nu
\end{equation}
where \(\tensor{\Lambda}{^\mu_\nu}\) is the matrix
\begin{equation}
    \tensor{\Lambda}{^\mu_\nu} =
    \begin{pmatrix}
        \gamma & -\gamma\beta & 0 & 0\\
        -\gamma\beta & \gamma & 0 & 0\\
        0 & 0 & 1 & 0\\
        0 & 0 & 0 & 1
    \end{pmatrix}
    .
\end{equation}

We want the scalar product, \(x \cdot x \coloneqq x^\mu \eta_{\mu\nu} x^\nu\) to be Lorentz invariant, meaning that \(x'^\mu \eta_{\mu\nu} x'^\nu = x^\mu \eta_{\mu\nu} x^\nu\).
Using the Lorentz transformation for \(x'^\mu\) this gives
\begin{align}
    x'^\mu\eta_{\mu\nu}x'^\nu &= (\tensor{\Lambda}{^\mu_\rho}x^\rho)\eta_{\mu\nu}(\tensor{\Lambda}{^\nu_\sigma}x^\sigma)\\
    &= x^\rho(\tensor{\Lambda}{^\mu_\rho}\eta_{\mu\nu}\tensor{\Lambda}{^\nu_\sigma})x^\sigma\\
    &= x^\rho\eta_{\rho\sigma}x^\sigma.
\end{align}
For this to hold we must have that
\begin{equation}
    \tensor{\Lambda}{^\mu_\rho} \eta_{\mu\nu} \tensor{\Lambda}{^\nu_\sigma} = \eta_{\rho\sigma} \iff \Lambda^\trans \eta \Lambda = \eta.
\end{equation}
This property defines Lorentz transformations.

Now that we know how a contravariant vector, \(x^\mu\), transforms we can work out how a covariant vector transforms, since we have \(x_\mu = \eta_{\mu\nu}x^\nu\).
We therefore have
\begin{equation}
    x'_\mu = \eta_{\mu\nu}x'^\nu = \eta_{\mu\nu}\tensor{\Lambda}{^\nu_\rho}x^\rho = \eta_{\mu\nu}\tensor{\Lambda}{^\nu_\rho} \eta^{\rho\sigma}x_\sigma.
\end{equation}
Hence, the transformation law for a covariant vector is
\begin{equation}
    x'_\mu = \tensor{\Lambda}{_\mu^\nu}x_\nu
\end{equation}
where we define
\begin{equation}
    \tensor{\Lambda}{_\mu^\nu} \coloneqq \eta_{\mu\sigma}\tensor{\Lambda}{^\sigma_\rho}\eta^{\rho\nu}.
\end{equation}
We can find out how \(\tensor{\Lambda}{^\mu_\nu}\) and \(\tensor{\Lambda}{_\mu^\nu}\) are related by considering the invariant spacetime interval:
\begin{equation}
    x_\mu x^\mu = x'_\mu x'^\mu = (\tensor{\Lambda}{_\mu^\rho} x_\rho)(\tensor{\Lambda}{^\mu_\sigma}x^\sigma) = (\tensor{\Lambda}{_\mu^\rho}\tensor{\Lambda}{^\mu_\sigma}) x_\rho x^\sigma.
\end{equation}
In order for this to hold we must have that
\begin{equation}
    \tensor{\Lambda}{_\mu^\rho}\tensor{\Lambda}{^\mu_\sigma} = \tensor{\delta}{^\rho_\sigma}.
\end{equation}
From this we see that \(\tensor{\Lambda}{_\mu^\rho}\) is the transpose of the inverse of \(\tensor{\Lambda}{^\mu_\rho}\).

We can readily extend these definitions to tensors.
First, and simplest, we have scalars, \(f\), which are the same in all frames, that is
\begin{equation}
    f' = f.
\end{equation}
Then we have contravariant vectors, \(v^\mu\), and covariant vectors, \(v_\mu\), which transform as
\begin{equation}
    v'^\mu = \tensor{\Lambda}{^\mu_\nu}v^\nu, \qqand v'_\mu = \tensor{\Lambda}{_\mu^\nu}v_\nu.
\end{equation}
For second rank tensors there are three options, contravariant, \(T^{\mu\nu}\), covariant, \(T_{\mu\nu}\), and mixed, \(\tensor{T}{^\mu_\nu}\).
These transform according to
\begin{align}
    T'^{\mu\nu} &= \tensor{\Lambda}{^\mu_\rho}\tensor{\Lambda}{^\nu_\sigma} T^{\rho\sigma},\\
    T'_{\mu\nu} &= \tensor{\Lambda}{_\mu^\rho}\tensor{\Lambda}{_\nu^\sigma} T_{\rho\sigma},\\
    {T'^\mu}_\nu &= \tensor{\Lambda}{^\mu_\rho}\tensor{\Lambda}{_\nu^\sigma} \tensor{T}{^\rho_\sigma}.
\end{align}

\section{Derivatives}
Consider the following derivative:
\begin{equation}
    \diffp{}{x^\mu}x^\mu = \diffp{}{x^0}x^0 + \diffp{}{x^1}x^1 + \diffp{}{x^2}x^2 + \diffp{}{x^3} x^3 = 4.
\end{equation}
Clearly this is a scalar, which means that \(\diffp{}/{x^\mu}\) must transform as a covariant vector.
We therefore define
\begin{equation}
    \partial_\mu \coloneqq \diffp{}{x^\mu} = \left( \diffp{}{x^0}, \diffp{}{x^1}, \diffp{}{x^2}, \diffp{}{x^3} \right) = (\partial_0, \partial_1, \partial_2, \partial_3) = \left( \frac{1}{c}\diffp{}{t}, \grad \right).
\end{equation}
We can then also define a contravariant derivative
\begin{equation}
    \partial^\mu = \eta^{\mu\nu}\partial_\nu = \left( \frac{1}{c} \diffp{}{t}, -\grad \right).
\end{equation}

Notice that
\begin{equation}
    \diffp{x'^\mu}{x^\nu} = \diffp{}{x^\nu} \tensor{\Lambda}{^\mu_\sigma}x^\sigma = \tensor{\Lambda}{^\mu_\nu}.
\end{equation}

We can build a scalar operator as the scalar product of derivative operators:
\begin{equation}
    \dalembertian \coloneqq \partial_\mu\partial^\mu = \eta^{\mu\nu}\partial_\mu\partial_\nu = \frac{1}{c^2}\diffp[2]{}{t} - \laplacian.
\end{equation}
This same operator is sometimes denoted \(\square\) or \(\square^2\), and is a four-dimensional analogue of the Laplacian.
Notice that this allows us to compactly write the wave equation as
\begin{equation}
    \dalembertian f = 0.
\end{equation}
In particular this also shows that the wave equation is unchanged under Lorentz transformations.

\section{Lorentz Group}
Consider two contravariant vectors, \(x\) and \(y\).
We know that \(x^\trans \eta y\) is invariant, so
\begin{equation}
    x'^\trans \eta y' = (\Lambda x)^\trans \eta \Lambda y = x^\trans \Lambda^\trans \eta \Lambda y = x\eta y
\end{equation}
from which we see that
\begin{equation}
    \Lambda^\trans \eta \Lambda = \eta.
\end{equation}
In fact we can take this as the defining condition of a Lorentz transformation, and define the \defineindex{Lorentz group} to be
\begin{equation}
    \orthogonal(1, 3) \coloneqq \{\Lambda \in \matrices{4}{\reals} \mid \Lambda^\trans \eta \Lambda = \eta\}.
\end{equation}

Applying the determinant to this defining condition we have
\begin{equation}
    \det \eta = \det(\Lambda^\trans \eta \Lambda) = \det(\Lambda^\trans) \det(\eta) \det(\Lambda).
\end{equation}
Using the fact that \(\det \eta = -1 \ne 0\) and \(\det M^\trans = \det M\) for all matrices \(M\) we have that \((\det\Lambda)^2 = 1\), which means that \(\det\Lambda = \pm 1\).
We define a \defineindex{proper Lorentz transform} to be one with \(\det \Lambda = +1\), and an improper Lorentz transform has \(\det\Lambda = -1\).
The proper Lorentz transformations also form a group:
\begin{equation}
    \specialOrthogonal(1, 3) \coloneqq \{\Lambda \in \matrices{4}{\reals} \mid \Lambda^\trans \eta \Lambda = \eta \text{ and } \det \Lambda = +1\}.
\end{equation}

Note that the Lorentz group and proper Lorentz group contain the rotation groups and proper rotation groups, \(\orthogonal(3)\) and \(\specialOrthogonal(3)\), respectively as subgroups, leaving the time unaffected.

\section{Special Tensors}
Consider the \defineindex{Levi--Civita symbol} in four dimensions.
We can define a contravariant tensor, \(\varepsilon^{\mu\nu\rho\sigma}\).
Applying the transformation law for a rank four tensor we have
\begin{equation}
    \varepsilon'^{\mu\nu\rho\sigma} = \tensor{\Lambda}{^\mu_\alpha} \tensor{\Lambda}{^\nu_\beta} \tensor{\Lambda}{^\rho_\gamma} \tensor{\Lambda}{^\sigma_\delta} \varepsilon^{\alpha\beta\gamma\delta}.
\end{equation}
We can now use a property of the determinant,
\begin{equation}
    \tensor{\Lambda}{^\mu_\alpha} \tensor{\Lambda}{^\nu_\beta} \tensor{\Lambda}{^\rho_\gamma} \tensor{\Lambda}{^\sigma_\delta} \varepsilon^{\alpha\beta\gamma\delta} = \varepsilon^{\mu\nu\rho\sigma} \det\Lambda.
\end{equation}
If \(\Lambda\) is a proper Lorentz transformation we therefore have
\begin{equation}
    \varepsilon'^{\mu\nu\rho\sigma} = \varepsilon^{\mu\nu\rho\sigma},
\end{equation}
so the Levi--Civita symbol is invariant under Lorentz transforms.
If instead \(\Lambda\) is an improper Lorentz transformation then we pick up a minus sign,
\begin{equation}
    \varepsilon'^{\mu\nu\rho\sigma} = -\varepsilon^{\mu\nu\rho\sigma},
\end{equation}
so we see that the Levi--Civita symbol is really a pseudo tensor.

We choose to define \(\varepsilon_{0123} = +1\).
Notice that this then means that \(\varepsilon^{0123} = -1\), since we can raise the \(0\) for free and pick up a minus sign raising each of \(1\), \(2\), and \(3\).

More care is needed when finding the value of various index combinations for the Levi--Civita symbol in four dimensions since it is no longer sufficient to consider cyclic permutations, anti-cyclic permutations, and repeating indices like in three dimensions.
For example, \(1230\) is a cyclic permutation of \(0123\) but it is an odd permutation, since we need 3 swaps, \(1230 \to 1203 \to 1023 \to 0123\), to get back to \(0123\).
Hence \(\varepsilon_{1230} = -1\).
Of course, if an index is repeated we still get zero like in three dimensions.

One of the main uses of the Levi--Civita symbol is in defining the \defineindex{volume element}, which in three dimensions is
\begin{equation}
    \dl{^3x} \coloneqq \frac{1}{3!}\varepsilon_{ijk}\dd{x^i}\dd{x^j}\dd{x^k} = \dl{x}\dd{y}\dd{z},
\end{equation}
and in four-dimensions is
\begin{align}
    \dl{^4x} \coloneqq \frac{1}{4!}\varepsilon_{\mu\nu\rho\sigma} \dd{x^\mu}\dd{x^\nu}\dd{x^\rho}\dd{x^\sigma} = c\dd{t}\dd{x}\dd{y}\dd{z}.
\end{align}
It should be noted that the three dimensional volume form is \emph{not} Lorentz invariant.
Choose the \(x\) direction to be in the direction of motion and length contraction gives us \(\dl{x} \to \dl{x}/\gamma\), so
\begin{equation}
    \dl{^3x} \to \frac{1}{\gamma}\dl{^3x}.
\end{equation}

Another common special tensor is the \defineindex{Kronecker delta}, \(\tensor{\delta}{^\mu_\nu}\), which is defined to be 1 if \(\mu = \nu\) and 0 otherwise.
Consider how this transforms, given that it is a mixed tensor:
\begin{equation}
    \tensor{\delta}{^\mu_\nu} = \tensor{\Lambda}{^\mu_\rho}\tensor{\Lambda}{_\nu^\sigma}\tensor{\delta}{^\rho_\sigma} = \tensor{\Lambda}{^\mu_\rho}\tensor{\Lambda}{_\nu^\rho} = \tensor{\delta}{^\mu_\nu},
\end{equation}
where we have used the fact that the contravariant and covariant Lorentz transformations are inverses.
So \(\tensor{\delta}{^\mu_\nu}\) is a tensor and is invariant under Lorentz transformations.

\chapter{Dynamics}
\section{Proper Time}
So far we have considered single points in spacetime.
Now consider the path of some particle.
We can specify the path as \((t, x(t), y(t), z(t))\), so given a time, \(t\), in a given frame we can work out where the particle is.
Given a path like this we can find a scalar value that everyone will agree upon, called the \defineindex{proper time}, \(\tau\).
This is defined by
\begin{equation}
    \dl{\tau}^2 \coloneqq \frac{1}{c^2}\dd{s^2} = \dl{t}^2 - \frac{1}{c^2}\dl{\vv{x}}^2.
\end{equation}
We can then use the fact that \(x\) is parametrised as a function of \(t\) to write \(\dl{x}^i\) as
\begin{equation}
    \dl{x}^i = \dl{x^i(t)} = \diff{x^i}{t}\dd{t} = v^i\dd{t},
\end{equation}
where we define the (coordinate) velocity to be \(\vv{v} \coloneqq \diff{\vv{x}}/{t}\).
We then have that
\begin{equation}
    \dl{\tau}^2 = \dl{t}^2 - \frac{1}{c^2}\dd{x}^i\dd{x}^i = \left( 1 - \frac{\vv{v}^2}{c^2} \right) \dd{t}^2.
\end{equation}
Choosing to take the positive square root, so that \(\tau\) increases as \(t\) increases, we get
\begin{equation}
    \dl{\tau} = \frac{1}{\gamma}\dd{t}, \qqwhere \gamma \coloneqq \frac{1}{\sqrt{1 - \vv{v}^2/c^2}}.
\end{equation}
Dividing through by \(\dl{t}\) we get and ordinary differential equation for \(\tau\):
\begin{equation}
    \diff{\tau}{t} = \gamma \implies \tau = \int \gamma \dd{t}.
\end{equation}
In general the velocity, and hence \(\gamma\), will be a function of time, and this integral may be non-trivial.

We interpret the proper time as the time elapsed along the path in the frame of the particle, since in this frame \(\dl{x^i} = 0\) and so \(\dl{s}^2 = c^2\dl{t}^2 = c^2\dl{\tau}^2\), with \(\gamma = 1\) in this case.
The nice thing about the proper time is that while the coordinate time, \(t\), is frame dependent the proper time is not and all observers will agree on its value.
This makes it much more useful for computing values.
For this reason we prefer to use the proper time to parametrise our path, \((t(\tau), x(\tau), y(\tau), z(\tau))\), where we interpret \(x(\tau)\) as \(x(t(\tau))\), and similarly for \(y\) and \(z\).
We find the function \(t(\tau)\) by simply inverting whatever equation we have found for \(\tau\) by integrating \(\gamma\) with respect to \(t\).

Compare this process to considering a circle.
We could in theory give a point on the circle as \((x, y(x))\), but this is awkward to work with.
It is much better to find a third value, say the angle from the \(x\)-axis, \(\vartheta\), and parametrise the circle using this, so \((x(\vartheta), y(\vartheta))\).
This gives a much more natural interpretation to quantities like \(\diff{\vv{r}}/{\vartheta} = (x'(\vartheta), y'(\vartheta))\), as the velocity, rather than \(\diff{\vv{r}}/{x} = (1, y'(x))\) with the old parametrisation.
We are used to using the time to parametrise the position in Newtonian mechanics, it is only in relativity when time and space are treated on a more equal footing where this starts to fail.

\section{Proper Velocity and Acceleration}
\subsection{Proper Velocity}
Now that we have parametrised the path of a particle in times of the proper time it makes sense to define the \defineindex{proper velocity}, \(u^\mu\), as the proper time derivative of the position:
\begin{equation}
    u^\mu \coloneqq \diff{x^\mu}{\tau} = \diff{t}{\tau}\diff{x^\mu}{t} = \gamma\diff{x^\mu}{t} = \gamma v^\mu.
\end{equation}
Here we are defining the \defineindex{coordinate velocity}, \(v^\mu\), as the coordinate time derivative of the position.
These two velocities are therefore related by
\begin{equation}
    u^\mu = \gamma v^\mu = (\gamma c, \gamma\vv{v}).
\end{equation}
Note that the proper velocity, unlike the coordinate velocity, is the derivative of a four-vector by a scalar quantity, and hence it is a four-vector.

Going back to the circle example we see that the proper velocity is like \(\diff{\vv{x}}/{\vartheta}\) and the coordinate velocity is like \(\diff{\vv{x}}/{x}\).

Consider the square of the proper velocity:
\begin{equation}
    u^2 = u \cdot u = \gamma^2(c^2 - \vv{v}^2) = \frac{1}{1 - \vv{v}^2/c^2}(c^2 - \vv{v}^2) = c^2.
\end{equation}
We see that thee proper velocity is always normalised to \(c\).
Another way to see this is to consider the definition of the proper time:
\begin{equation}
    u^2 = \diff{x^\mu}{\tau}\diff{x_\mu}{\tau} = \left( \diff{s}{\tau} \right)^2 = c^2,
\end{equation}
since \(\dl{\tau}^2 = \dl{s}^2/c^2\).
A third way to see this is to consider the proper velocity in the rest frame of the particle, in which case \(u^\mu = (c, \vv{0})\), since \(\gamma = 1\) in this frame, and we then trivially have \(u^2 = c^2\).
This then holds in any frame since \(u^2\) is a Lorentz scalar.

\subsection{Proper Acceleration}
We can keep going taking higher order derivatives.
We define the \defineindex{proper acceleration}, \(a^\mu\), to be
\begin{equation}
    a^\mu \coloneqq \diff{u^\mu}{\tau} = \diff[2]{x^\mu}{\tau}.
\end{equation}
This is a again a four-vector, since it is the derivative of a four-vector with respect to a scalar\footnote{Note that this fails in general relativity, since the transformation coefficients are not constant there and so derivatives of four-vectors aren't necessarily four vectors.}.

Now consider the following:
\begin{equation}
    \diff{}{\tau}(u \cdot u) = 2u\cdot\diff{u}{\tau} = 2u\cdot a, \qqand \diff{}{\tau}(u \cdot u) = \diff{}{\tau}c^2 = 0.
\end{equation}
This shows that the proper acceleration is always orthogonal to the proper acceleration.
This is a bit like circular motion, where we have centripetal acceleration towards the centre and the velocity is tangential.
The way we interpret this is that there are only really three degrees of freedom here and the time component of the proper acceleration doesn't really mean much, its just chosen to make \(u\) and \(a\) perpendicular.

\section{Momentum}
We make the assumption that each particle has an associated Lorentz scalar mass.
That is the mass is the same in all frames.
This is different to the more old fashioned idea of \enquote{relativistic mass} which scales with \(\gamma\).
We further assume that this is constant, which basically means we don't consider cases where particles decay or combine to form other particles.
We want to define a relativistic analogue of the momentum.
It seems sensible that this should be done by replacing the coordinate velocity in three dimensions with the proper velocity in four dimensions:
\begin{equation}
    p^\mu \coloneqq mu^\mu = m\diff{x^\mu}{\tau} = (\gamma mc, \gamma m\vv{v}) = (p^0, \vv{p}).
\end{equation}
Note that \(\vv{p} \coloneqq \gamma m\vv{v} \ne m \vv{v}\), this last term is the non-relativistic momentum.

For the interpretation of the first term consider the case when \(v \ll c\).
We can Taylor expand \(\gamma\) giving
\begin{equation}
    \gamma = (1 - v^2/c^2)^{-1/2} \approx 1 + \frac{1}{2}\frac{v^2}{c^2}.
\end{equation}
We then have that
\begin{equation}
    p^0 = \gamma mc = \frac{1}{c} \gamma mc^2 \approx \frac{1}{c}\left( 1 + \frac{1}{2}\frac{v^2}{c^2} \right)mc^2 = \frac{mc^2}{c} + \frac{1}{2}mv^2.
\end{equation}
We can identify the first term as the rest energy and the second as the non-relativistic kinetic energy, this suggests we interpret \(p^0\) as the energy, \(E/c\):
\begin{equation}
    p^\mu = (E/c, \vv{p}).
\end{equation}
Notice that keeping terms only to first order in \(v/c\) we also recover the non-relativistic momentum, \(\vv{p} = \gamma m\vv{v} \approx m\vv{v}\).

Now consider the square of the four-momentum:
\begin{equation}
    p^2 = p\cdot p = \frac{E^2}{c^2} - \vv{p}^2.
\end{equation}
In the rest frame of the particle \(\vv{p} = \vv{0}\) and \(\gamma = 1\) so we have \(p^\mu = (mc, \vv{0})\) giving \(p^2 = m^2c^2\).
This must hold in all frames since \(p^2\) is a scalar and so after a bit of rearranging we have that
\begin{equation}
    E^2 = \vv{p}^2c^2 + m^2c^4.
\end{equation}
This is of course the famous energy-momentum relation, which reduces to the even more famous mass-energy equivalence \(E = mc^2\) in the rest frame of the particle.
Notice that for a massless particle, such as a photon, we have \(E^2 = p^2c^2\), and \(p^2 = 0\), in which case we can write
\begin{equation}
    p^\mu = \left( \frac{E}{c}, \frac{E}{c}\vh{n} \right),
\end{equation}
where \(\vh{n}\) is a unit vector, since this gives
\begin{equation}
    p^2 = \frac{E^2}{c^2} - \frac{E^2}{c^2}\ve{i}\cdot\vh{n} = 0.
\end{equation}

\subsection{Force}
Newton's second law in the non-relativistic case is most generally written as
\begin{equation}
    \vv{f} = \diff{\vv{p}}{t}.
\end{equation}
The obvious generalisation to special relativity is to define the force to be
\begin{equation}
    f^\mu \coloneqq \diff{p^\mu}{\tau} = m\diff{u^\mu}{\tau} = ma^\mu.
\end{equation}
Since \(f^\mu\) is just proportional to the acceleration, assuming constant mass, we have that \(f\cdot u = 0\), so the force must be perpendicular to the velocity.
This reduces the degrees of freedom so there is no need for an interpretation of \(f^0\).