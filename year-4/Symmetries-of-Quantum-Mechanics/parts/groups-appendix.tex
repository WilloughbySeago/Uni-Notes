\chapter{Groups}
\section{Finite Groups}

\begin{dfn}{}{}
    The \defineindex{trivial group} is the group containing only the identity, \(\{e\}\).
    It is the only group of order 1.
    
    
    \begin{multicols}{2}
        \begin{itemize}
            \item Order 1.
            \item Rank 1.
            \item Cyclic.
            \item Abelian.
        \end{itemize}
    \end{multicols}
    The trivial group is isomorphic to \(\integers_1\), \(S_1\), and \(\specialOrthogonal(1)\).
\end{dfn}

\begin{dfn}{}{}
    \(\integers_2\) is the cyclic group of order 2, see \cref{dfn:cyclic group}.
    It is the only group of order 2.
    
    
    \begin{multicols}{2}
        \begin{itemize}
            \item Order 2.
            \item Rank 1.
            \item Cyclic.
            \item Abelian.
        \end{itemize}
    \end{multicols}
    \(\integers_2\) is isomorphic to \(S_2\).
\end{dfn}

\begin{dfn}{}{}
    \(\integers_3\) is the cyclic group of order 3, see \cref{dfn:cyclic group}.
    It is the only group of order 3.
    
    \begin{multicols}{2}
        \begin{itemize}
            \item Order 3.
            \item Rank 1.
            \item Cyclic.
            \item Abelian.
        \end{itemize}
    \end{multicols}
\end{dfn}

\begin{dfn}{}{}
    \(\integers_4\) is the cyclic group of order 4, see \cref{dfn:cyclic group}.
    It is one of two groups of order 4.
    
    \begin{multicols}{2}
        \begin{itemize}
            \item  Order 4.
            \item Rank 1.
            \item Cyclic.
            \item Abelian.
        \end{itemize}
    \end{multicols}
\end{dfn}

\begin{dfn}{}{}
    \(\integers_2\times\integers_2\) is the \defineindex{Klein \textit{Vierergruppe}}.
    It is one of two groups of order 4.
    It is a direct product of two copies of \(\integers_2\).
    
    \begin{multicols}{2}
        \begin{itemize}
            \item Order 4.
            \item Rank 2.
            \item Abelian.
        \end{itemize}
    \end{multicols}
\end{dfn}

\begin{dfn}{}{}
    \(S_3\) is the permutation group on 3 elements, see \cref{dfn:permutation group}.
    
    \begin{multicols}{2}
        \begin{itemize}
            \item Order 6.
            \item Rank 2.
            \item Non-Abelian.
        \end{itemize}
    \end{multicols}
\end{dfn}

\begin{dfn}{}{}
    The \defineindex{quaternion group}, \(Q\)\index{Q@\(Q\), quaternion group}, has the group presentation
    \begin{equation}
        Q = \presentation{-e, i, j, k}{(-e)^2 = e, i^2 = j^2 = k^2 = ijk = e}.
    \end{equation}
    
    \begin{multicols}{2}
        \begin{itemize}
            \item Order 8.
            \item Rank 2.
            \item Non-Abelian.
        \end{itemize}
    \end{multicols}
    The Pauli matrices provide a two-dimensional complex representation by the correspondence \((-e, i, j, k) \to (-\ident, \sigma_1, \sigma_2, \sigma_3)\).
\end{dfn}

\subsection{Other Finite Groups}
\begin{dfn}{}{dfn:cyclic group}
    The \defineindex{cyclic group} of order \(n\), denoted \(\integers_n\), is given by the presentation
    \begin{equation}
        \integers_n = \presentation{a}{a^n = e}.
    \end{equation}
    Identifying \(a = \e^{2i\pi/n}\) and the operation as multiplication we get a group formed from the \(n\)th roots of unity.
    Identifying \(a = 1\) and the operation as addition modulo \(n\) we get a group formed from \(\{0, \dotsc, n-1\}\).
    
    \begin{multicols}{2}
        \begin{itemize}
            \item Order \(n\).
            \item Rank 1.
            \item Cyclic.
            \item Abelian.
        \end{itemize}
    \end{multicols}
    All finite cyclic groups are isomorphic to \(\integers_n\) for some \(n\).
\end{dfn}

\begin{dfn}{}{dfn:permutation group}
    The \defineindex{permutation group} on \(n\) objects is the group of all permutations (bijections) of \(\{1, \dotsc, n\}\), with function composition as the group operation.
    
    \begin{multicols}{2}
        \begin{itemize}
            \item Order \(n!\).
            \item Rank 2.
            \item Non-Abelian (\(n > 2\)).
        \end{itemize}
    \end{multicols}
    \(S_1\) and \(S_0\) are isomorphic to the trivial group.
    
    \(S_2\) is isomorphic to \(\integers_2\).
\end{dfn}

\section{Discrete Groups}

\begin{dfn}{}{}
    The integers, \(\integers\), under addition.
    
    \begin{multicols}{2}
        \begin{itemize}
            \item Rank 1.
            \item Cyclic.
            \item Abelian.
        \end{itemize}
    \end{multicols}
\end{dfn}

\begin{dfn}{}{}
    The rational numbers, \(\rationals\), under addition.
    
    \begin{multicols}{2}
        \begin{itemize}
            \item Abelian.
        \end{itemize}
    \end{multicols}
\end{dfn}

\begin{dfn}{}{}
    The nonzero rational numbers, \(\rationals^*\), under multiplication.
    
    \begin{multicols}{2}
        \begin{itemize}
            \item Abelian.
        \end{itemize}
    \end{multicols}
\end{dfn}

\section{Continuous Groups}

\subsection{Scalars}

\begin{dfn}{}{}
    The real numbers, \(\reals\), under addition.
    
    \begin{multicols}{2}
        \begin{itemize}
            \item Abelian.
        \end{itemize}
    \end{multicols}
    \((\reals, +)\) is isomorphic to \((\reals_{>0}, \cdot)\).
\end{dfn}

\begin{dfn}{}{}
    The nonzero real numbers, \(\reals^*\), under multiplication.
    
    \begin{multicols}{2}
        \begin{itemize}
            \item Abelian.
        \end{itemize}
    \end{multicols}
\end{dfn}
    
\begin{dfn}{}{}
    The complex numbers, \(\complex\), under addition.
    
    \begin{multicols}{2}
        \begin{itemize}
            \item Abelian.
        \end{itemize}
    \end{multicols}
\end{dfn}

\begin{dfn}{}{}
    The nonzero complex numbers, \(\complex^*\), under multiplication.
    
    \begin{multicols}{2}
        \begin{itemize}
            \item Abelian.
        \end{itemize}
    \end{multicols}
\end{dfn}
    
\subsection{Matrices}

\begin{dfn}{}{}
    The \defineindex{general linear group}\index{GL(n, F)@\(\generalLinear(n, \field)\), general linear group}
    \begin{equation}
        \generalLinear(n, \field) = \left\{ M \in \matrices{n}{\field} \mid \det M \ne 0 \right\}.
    \end{equation}
    If \(V\) is a vector space of dimension \(n\) over \(\field\) then this group is also denoted \(\generalLinear(V)\).
    If \(\field\) is obvious from context then this group is denoted \(\generalLinear(n)\).
    
    \begin{multicols}{2}
        \begin{itemize}
            \item Non-Abelian (\(n > 1\)).
        \end{itemize}
    \end{multicols}
\end{dfn}

\begin{dfn}{}{}
    The \defineindex{special linear group}\index{SL(n, F)@\(\specialLinear(n, \field)\), special linear group}
    \begin{equation}
        \specialLinear(n, \field) = \{ M \in \matrices{n}{\field} \mid \det M = 1 \}.
    \end{equation}
    If \(V\) is a vector space of dimension \(n\) over \(\field\) then this group is also denoted \(\specialLinear(V)\).
    If \(\field\) is obvious from context then this group is denoted \(\specialLinear(n)\).
    
    \begin{multicols}{2}
        \begin{itemize}
            \item Non-Abelian (\(n > 1\)).
        \end{itemize}
    \end{multicols}
    
    \(\specialLinear(n, \field)\) is a subgroup of \(\generalLinear(n, \field)\).
\end{dfn}

\begin{dfn}{}{}
    The \defineindex{orthogonal group}\index{O(n)@\(\orthogonal(n)\), orthogonal group}
    \begin{equation}
        \orthogonal(n) = \{ O \in \matrices{n}{\reals} \mid O^\trans O = OO^\trans = \ident \}.
    \end{equation}
    
    \begin{multicols}{2}
        \begin{itemize}
            \item Non-Abelian (\(n > 1\)).
        \end{itemize}
    \end{multicols}

    \(\orthogonal(n)\) is a subgroup of \(\generalLinear(n, \reals)\).
    
    \(\orthogonal(n)\) is the group of distance preserving transformations of Euclidean space which leave the origin invariant.
    
    \(\orthogonal(n)\) is the group of rotations and inversions of \(\reals^n\).
\end{dfn}

\begin{dfn}{}{}
    The \defineindex{special orthogonal group}\index{SO(n)@\(\specialOrthogonal(n)\), special orthogonal group}
    \begin{equation}
        \specialOrthogonal(n) = \{ O \in \matrices{n}{\reals} \mid O^\trans O = OO^\trans = \ident \text{ and } \det O = 1 \}.
    \end{equation}
    
    \begin{multicols}{2}
        \begin{itemize}
            \item Non-Abelian (\(n > 1\)).
        \end{itemize}
    \end{multicols}
    
    \(\specialOrthogonal(n)\) is a subgroup of \(\orthogonal(n)\) and \(\specialLinear(n, \reals)\).
    
    \(\specialOrthogonal(n)\) is the group of rotations of \(\reals^n\).
    
    \(\specialOrthogonal(2)\) is isomorphic to \(\unitary(1)\) and the circle group, \(\mathbb{T} = \{z \in \complex \mid \abs{z} = 1\}\) under multiplication.
\end{dfn}

\begin{dfn}{}{}
    The \defineindex{unitary group}\index{U(n)@\(\unitary(n)\), unitary group}
    \begin{equation}
        \unitary(n) = \{ U \in \matrices{n}{\complex} \mid U^\hermit U = UU^\hermit = \ident \}.
    \end{equation}
    
    \begin{multicols}{2}
        \begin{itemize}
            \item Non-Abelian (\(n > 1\)).
        \end{itemize}
    \end{multicols}
    
    \(\unitary(n)\) is a subgroup of \(\generalLinear(n, \complex)\).
    
    \(\unitary(n)\) is the group which preserves the standard inner product on \(\complex^n\).
    
    \(\unitary(1)\) is isomorphic to \(\specialOrthogonal(2)\) and the circle group, \(\mathbb{T} = \{z \in \complex \mid \abs{z} = 1\}\) under multiplication.
\end{dfn}

\begin{dfn}{}{}
    The \defineindex{special unitary group}\index{SU(n)@\(\specialUnitary(n)\), special unitary group}
    \begin{equation}
        \specialUnitary(n) = \{ U \in \matrices{n}{\complex} \mid U^\hermit U = UU^\hermit = \ident \text{ and } \det U = \ident \}.
    \end{equation}
    
    \begin{multicols}{2}
        \begin{itemize}
            \item Non-Abelian (\(n > 1\)).
        \end{itemize}
    \end{multicols}

    \(\specialUnitary(n)\) is a subgroup of \(\unitary(n)\) and \(\specialLinear(n, \complex)\).
\end{dfn}

\begin{dfn}{}{}
    The \defineindex{isometries of Euclidean space}
    \begin{equation}
        \ISO(n) = \orthogonal(n) \ltimes \reals^n
    \end{equation}
    where \((R, \vv{a}) (R', \vv{a'}) \coloneqq (RR', \vv{a} + R\vv{a'})\).
    \begin{multicols}{2}
        \begin{itemize}
            \item Non-Abelian
        \end{itemize}
    \end{multicols}
    \(\ISO(n)\) is the group of distance preserving transformations of Euclidean space.
    
    \(\ISO(n)\) is the group of rotations, reflections, and translations of \(\reals^n\).
    
    \(\ISO(n)\) has both \(\orthogonal(n)\) and \(\reals^n\) as normal subgroups.
\end{dfn}

\begin{dfn}{}{}
    The \defineindex{Lorentz group}, \(\orthogonal(1, 3)\)\index{O(1, 3)@\(\orthogonal(1, 3)\), Lorentz group}, is the group of all Lorentz transformations of Minkowski space.
    \begin{multicols}{2}
        \begin{itemize}
            \item Non-Abelian.
        \end{itemize}
    \end{multicols}

    \(\orthogonal(1, 3)\) is the group that preserves the quadratic form \((t, x, y, z) \mapsto t^2 - x^2 - y^2 - z^2\).
    
    \(\specialOrthogonal^+(1, 3)\) is the subgroup of \(\orthogonal(1, 3)\) which preserves the orientation of space (S for special, that is unit determinant) and direction of time (that's what the \(+\) represents).
\end{dfn}

\begin{dfn}{}{}
    The \defineindex{Poincar\'e group} is the group of all isometries of Minkowski space, sometimes denoted \(\isometry(1, 3)\)\index{ISO(1, 3)@\(\isometry(1, 3)\), Poincar\'e group}.
    That is, it is the group of all Lorentz transformations and translations.
    \begin{multicols}{2}
        \begin{itemize}
            \item Non-Abelian.
        \end{itemize}
    \end{multicols}
    
    The Poincar\'e group can be identified as the semidirect product \(\isometry(1, 3) = \reals^{1,3} \rtimes  \orthogonal(1, 3)\) where \(\reals^{1,3}\) is the group of spacetime translations of Minkowski space and \(\orthogonal(1, 3)\) is the Lorentz group.
\end{dfn}
