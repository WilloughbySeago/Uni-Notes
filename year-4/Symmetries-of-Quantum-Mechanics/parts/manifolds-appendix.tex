\chapter{Manifolds}\label{app:manifold}
\begin{rmk}
    The notes in this section are repeated from the notes for general relativity so for more details look at those notes.
\end{rmk}
The theory of Lie groups starts by defining them as manifolds with a group structure, or groups with a manifold structure.
In order to make this rigorous we need to define a manifold.
We will define a real manifold for simplicity and the definition of a complex manifold is nearly identical replacing \(\reals\) with \(\complex\) and requiring maps be holomorphic instead of smooth.

I go into far more detail here than necessary.
I also make statements like \enquote{\(\category{Top}\) is the category of topological spaces with continuous functions as morphisms and homeomorphisms as isomorphisms} without explanation, these sorts of statement aren't important if you don't understand them.

\section{Manifolds}
Manifolds are the stage for the maths of GR, and in fact most physics.
We start with a rough definition of a manifold which lacks a few details which we will fill out later.
For our purposes this rough definition is sufficient and the full definition is for completeness.
\begin{dfn}{Manifold}{}
    A \defineindex{manifold} is a space where we can locally introduce Cartesian coordinates such that different choices of local coordinates will give compatible descriptions and by defining sets of local coordinates covering the whole space we have complete information about the space.
\end{dfn}

A manifold is parametrised continuously and differentiably as an \(n\)-tuple of real numbers, \((x^1, \dotsc, x^n) \in \reals^n\), which are the coordinates of the point.
We can connect two points with a curve parametrised by some \(\lambda \in \reals\) such that derivatives of the coordinates along this path exist, that is \(\diff{x^i}/{\lambda}\) exists for all \(i = 1, \dotsc, n\).

By this definition it should be clear that Euclidean space, where non-relativistic mechanics takes place, is a manifold.
However, not all manifolds are as nice as Euclidean space, for example Euclidean space has a metric (way to measure distance), which not all manifolds do.
Locally all manifolds behave like a subset of Euclidean space due to their differentiable nature.

\subsection{Topology}
\begin{dfn}{Topological Space}{}
    A \define{topological space}\index{topological!space}, \((X, \mathcal{T})\), is a set, \(X\), and a collection of subsets of \(X\), \(\mathcal{T} \subset \mathcal{P}(X)\), called the \defineindex{topology}, such that
    \begin{itemize}
        \item \(X \in \mathcal{T}\),
        \item \(\emptysetAlt \in \mathcal{T}\),
        \item \(\mathcal{T}\) is closed under unions, that is if \(\{U_i\}_{i \in I} \subseteq \mathcal{T}\) then\vspace{-1ex}
        \begin{equation}
            \bigcup_{i\in I} U_i \in \mathcal{T},
        \end{equation}\vspace{-4ex}
        \item \(\mathcal{T}\) is closed under finite intersections, that is if \(\{U_i, \dotsc, U_N\} \subseteq \mathcal{T}\) then\vspace{-4ex}
        \begin{equation}
            \bigcap_{i = 1}^{N} U_i \in \mathcal{T}.
        \end{equation}
    \end{itemize}\vspace{-1ex}
    We call the elements of \(\mathcal{T}\) \define{open sets}\index{open set}.
\end{dfn}

It is common to refer to \(X\) as a topological space and then specify \(\mathcal{T}\) separately, or just leave \(\mathcal{T}\) implicit, much in the same way that we might define a group \(G\) and operation \(\cdot\), when more formally it might be more correct to call \((G, \cdot)\) and \(G\) the underlying set.

Topological spaces can be quite abstract, for example the following topology makes \((\{a, b, c, d\}, \mathcal{T})\) a topological space:
\begin{equation}
    \mathcal{T} = \{\emptysetAlt, \{a, b, c, d\}, \{a, b\}, \{c, d\}\}.
\end{equation}
We only deal with more familiar topological spaces.
We can define topologies through the notion of a basis.

\begin{dfn}{Topological Basis}{}
    A \define{basis}\index{topological!basis} for a topology on some set \(X\) is a collection of sets, \(\mathcal{B} \subseteq \mathcal{P}(X)\) such that
    \begin{itemize}
        \item the elements of \(\mathcal{B}\) cover \(X\), that is for all \(x \in X\) there exists some \(B \in \mathcal{B}\) such that \(x \in B\).
        \item given \(B_1, B_2 \in \mathcal{B}\) for all \(x \in B_1 \cap B_2\) there exists some \(B \in \mathcal{B}\) such that \(x \in B\) and \(B_3 \subseteq B_1 \cap B_2\).
    \end{itemize}
    The topology generated by a basis, \(\mathcal{B}\), is then defined to be the set of all subsets \(U \subseteq X\) such that\vspace{-0.5ex}
    \begin{equation}
        U = \bigcup_{B_i \in B \subseteq \mathcal{B}} B_i.
    \end{equation}\vspace{-0.5ex}
    That is all elements of the topology are unions of basis sets.
\end{dfn}

Using this definition we can define the \defineindex{standard topology} on \(\reals^n\) as the topology generated by the basis of open balls in \(\reals^n\), where by an open ball we mean
\begin{equation}
    B = \{\vv{r} \in \reals^n \mid \abs{\vv{r} - \vv{a}}^{n} \text{ for some } \vv{a} \in \reals^n\}
\end{equation}
with \(\abs{\vv{r}} = \sum_i x_i^2\) being the usual Euclidean metric.
This is not the only topology on \(\reals^n\) but it is the only one we care about.

Let \(X\) and \(Y\) be topological spaces with topologies \(\mathcal{T}_X\) and \(\mathcal{T}_Y\) respectively.
Then we can define a function, \(f \colon X \to Y\).
The \defineindex{preimage} of some \(S \subseteq Y\) under this function is the set
\begin{equation}
    f^*(S) = f^{-1}(S) \coloneqq \{a \in X \mid f(a) \in S\} \subseteq X.
\end{equation}
The notation \(f^{-1}\) suggests that this is related to the inverse of the function \(f\), and indeed if the inverse exists then it is, as the preimage is the image of the inverse restricted to \(S\).

The function \(f \colon X \to Y\) is \defineindex{continuous} if all preimages of open sets of \(Y\) are open sets of \(X\), that is if \(f^*(U_Y) \in \mathcal{T}_X\) for all \(U_Y \in \mathcal{T}_Y\).
With the standard topology for \(\reals^n\) this definition coincides with the normal \(\varepsilon\)-\(\delta\) definition of continuity from analysis.

An invertible (and hence bijective) continuous function between topological spaces is called a \defineindex{homeomorphism}.
If there exist homeomorphisms between two topological spaces we say those spaces are \defineindex{homeomorphic}.
\(\category{Top}\) is the category of topological spaces with continuous functions as morphisms and homeomorphisms as isomorphisms.

\subsection{Charts}
In order to be able to define differentiation in topological spaces we need them to look locally like \(\reals^n\).
By \enquote{look like} we mean we want there to be a homeomorphism between neighbourhoods of a topological space and \(\reals^n\).
Homeomorphisms preserve topological properties so this allows us to translate things into Euclidean terms, perform calculations, and then convert the results back to the topological space.
This isn't possible for all topological spaces but is a defining property of manifolds.

To better define the notion of \enquote{looks like} we define charts.

\begin{dfn}{Chart}{}
    Given a topological space, \((X, \mathcal{T})\), a \defineindex{chart}, \(C\), is an ordered pair \(C = (U, \varphi)\), where \(U \in \mathcal{T}\) and \(\varphi\) is a homeomorphism
    \begin{equation}
        \varphi \colon U \to \mathop{\mathrm{im}}\varphi \subseteq \reals^n.
    \end{equation}
    We call \(n \in \integers_{>0}\) the \defineindex{dimension} of \(U\), it is independent of \(\varphi\).
\end{dfn}

Simply put a chart associates with each neighbourhood in a topological space an open subset of \(\reals^n\) and gives us a way to move between \(U\) and this open subset while preserving key topological properties.

\subsection{Topological Manifolds}
For our first definition of a manifold we will need a few preliminary definitions.
\begin{dfn}{Locally Euclidean}{}
    A topological space, \(X\), is \defineindex{locally Euclidean} if there exists \(n \in \integers_{>0}\) such that for every point, \(p \in X\) there is a chart, \(C = (U, \varphi)\), such that \(p \in U\).
\end{dfn}

That is \(X\) must be covered by the open sets \(\{U_i\}_{i\in I}\) and for each open set in this covering there must be an associated chart, \(C_i = (U_i, \varphi_i)\).

\begin{dfn}{Hausdorff}{}
    Let \(X\) be a topological space.
    Then \(x, y \in X\) are separated by neighbourhoods if there exists some open sets \(U \ni x\) and \(V \ni y\) such that \(U \cap V = \emptysetAlt\).
    
    \(X\) is a \defineindex{Hausdorff space} if all distinct points in \(X\) are pairwise neighbourhood separable.
\end{dfn}
Roughly put the condition of being Hausdorff means for any two points we can find sufficiently small neighbourhoods around the two points such that the two points aren't both in the neighbourhoods.
Take for example, \(\reals^2\), the plane with the standard topology.
Since the standard topology here is the topology generated by the open discs we can draw two discs around any two points such that the discs don't overlap and so \(\reals^2\) is Hausdorff.

\begin{dfn}{Topological Manifold}{}
    A \define{topological manifold}\index{topological!manifold}\index{manifold!topological} is a Hausdorff space, \(M\), which is locally Euclidean.
\end{dfn}

It is common to include extra conditions, such as being \defineindex{second-countable}, which means that the topology is generated by a countable basis.

A topological manifold is the simplest manifold, and all manifolds are topological manifolds.
However, topological manifolds don't have the required structure for our purposes, since there is still no sense of differentiation on general topological manifolds.
We will need a more specific type of manifold, which we will develop in the next few sections.

\subsection{Local Coordinates}
Charts allow us to rigorously define the notion local coordinates.
Intuitively local coordinates are a way to parametrise a neighbourhood with tuples in \(\reals^n\), formally local coordinates are defined as follows.

\begin{dfn}{Local Coordinates}{}
    Let \(M\) be a topological manifold and \(U\) an open subset of \(M\) such that \((U, \varphi)\) is a chart.
    The coordinates of some point \(p \in U\) are the Cartesian coordinates, \(\vv{x}_p\), given by \(\varphi(p) \in \reals^n\).
\end{dfn}

In general charts are not unique and it is important that the physics we do doesn't depend on the choice of coordinate system, since coordinates are a man made construction and don't actually reflect nature, they just let us apply maths.
This means we need charts to be compatible in the following sense.

\begin{dfn}{Compatible Charts}{}
    Let \(M\) be a topological manifold with charts \(C_1 = (U_1, \varphi_1)\) and \(C_2 = (U_2, \varphi_2)\).
    If \(U_1 \cap U_2 \ne \emptysetAlt\) we can define the \defineindex{transition maps} to be
    \begin{align}
        \varphi_1 \circ \varphi_2^{-1} \colon \varphi_2(U_1 \cap U_2) &\to \varphi_1(U_1 \cap U_2),\\
        \varphi_2 \circ \varphi_1^{-1} \colon \varphi_1(U_1 \cap U_2) &\to \varphi_2(U_1 \cap U_2).
    \end{align}
    We say that \(C_1\) and \(C_2\) are \define{compatible}\index{compatible!charts} if either the transition maps are homeomorphisms or \(U_1 \cap U_2 = \emptysetAlt\).
    
    We say that \(C_1\) and \(C_2\) are \defineindex{smoothly compatible} if they are compatible and the transitions functions, if defined, are smooth.
\end{dfn}

We can think of the two charts as two separate sets of local coordinates and the transition maps as coordinate transformations, that is
\begin{align}
    (\varphi_1 \circ \varphi_2^{-1})(\vv{x}_p^{(2)}) &= \varphi_1(p) = \vv{x}_p^{(1)},\\
    (\varphi_2 \circ \varphi_1^{-1})(\vv{x}_p^{(1)}) &= \varphi_1(p) = \vv{x}_p^{(2)}.
\end{align}

We can use local coordinates to express functions on some manifold in terms of local coordinates.

\begin{dfn}{Local Function}{}
    Given a continuous function, \(f \colon M \to \reals\) for some topological manifold, \(M\), and a chart, \(C = (U, \varphi)\), we can associate the restriction of \(f\) to \(U\) with the function \(f_U = f \circ \varphi^{-1}\) so
    \begin{equation}
        f_U \colon \varphi(U) \subseteq \reals^n \to \reals
    \end{equation}
    is defined by
    \begin{equation}
        f(p) = f_U(\vv{x}_p)
    \end{equation}
    for \(p \in U\).
\end{dfn}

This is useful since we can deal with functions \(\reals^n \to \reals\) using the normal tools of vector calculus.

For compatible charts we can represent the function in two different ways in terms of local coordinates.
namely by the functions \(f_{U_i}\) with \(i = 1, 2\).
On the intersection \(U_1 \cap U_2\) we define \(f = f_1 \circ \varphi_1 = f_2 \circ \varphi_2\) so rearranging this final equality \(f_2 = f_1 \circ \varphi_1 \circ \varphi_2^{-1}\) and \(f_1 = f_2 \circ \varphi_2 \circ \varphi_1^{-1}\).
We can think of these as the same function with a change of variables, so \(f_1(\vv{x}_p^{(1)}) = f_2(\vv{x}_p^{(2)})\).

\subsection{Smooth Manifold}
\begin{dfn}{Smooth Atlas}{}
    A \define{smooth atlas}\index{smooth!atlas}, \(\mathcal{A}(M)\), for a topological manifold, \(M\), is a family of charts, \(\{C_a = (U_a, \varphi_a)\}\), which cover \(M\), so \(p \in U_a\) for some \(a\) for all \(p \in M\), such that the charts are all mutually smoothly compatible.
    
    Two smooth atlases, \(\mathcal{A}_1(M)\) and \(\mathcal{A}_2(M)\), are said to be \define{compatible}\index{compatible!atlases} if all of the charts of \(\mathcal{A}_1\) are compatible with all of the charts of \(\mathcal{A}_2\).
\end{dfn}

We can think of charts as mapping the manifold and then the name atlas is naturally a collection of maps.
Compatibility defines an equivalence relation on the set of all atlases for a given manifold.

\begin{dfn}{Smooth Manifold}{}
    A \define{smooth structure}\index{smooth!structure} on a topological manifold, \(M\), is an equivalence class of smooth atlases:
    \begin{equation}
        \mathcal{S}(M) = [\mathcal{A}(M)],
    \end{equation}
    that is \(\mathcal{S}(M)\) is the set of all atlases compatible with \(\mathcal{A}(M)\).
    
    A \define{smooth manifold}\index{smooth!manifold}\index{manifold!smooth} is a topological manifold, \(M\), equipped with a smooth structure, \(\mathcal{S}\).
\end{dfn}

\begin{dfn}{Smooth Functions and Maps}{}\
    A function, \(f \colon M \to \reals\), on a smooth manifold, \(M\), is said to be \define{smooth}\index{smooth!function} if all of its local coordinate representatives, 
    \begin{equation}
        f_a = f \circ \varphi_a^{-1} \colon \varphi(U_a) \to \reals,
    \end{equation}
    are smooth.
    
    Let \(M\) and \(V\) be smooth manifolds of dimensions \(m\) and \(n\) with charts \((U_a, \varphi_a)\) and \((V_b, \psi_b)\) respectively.
    Then the map \(\mu \colon M \to N\) is \define{smooth}\index{smooth!map} if all of its local coordinate representatives,
    \begin{equation}
        \mu_{ab} = \psi_b \circ \mu \circ \varphi_a^{-1} \colon \varphi_a(U_a) \subseteq \reals^m \to \psi_b(V_b) \subseteq \reals^n
    \end{equation}
    are smooth.
    Smooth invertible maps between manifolds are also called \define{diffeomorphisms}\index{diffeomorphism}.
\end{dfn}

It is possible to differentiate smooth functions and maps by first differentiating the local coordinates and then mapping the result back to the manifolds.

\(\category{SmoothMan}\) is the category of smooth manifolds with smooth maps as homomorphisms and diffeomorphisms as isomorphisms.

Smooth manifolds are the most common type of manifolds, and are the type we will be using, but there are other types.
For example, we can define \(C^k\)-differentiable manifolds by replacing the requirement that the transition functions be smooth (\(C^{\infty}\)) with requirements for transition functions to be \(C^k\)-differentiable.
We can define real analytic manifolds by requiring that transition functions are real analytic functions.
We can define complex smooth manifolds by replacing \(\reals\) with \(\complex\) and requiring that transition functions be holomorphic.
The list of possible manifold types goes on.

\subsection{Examples of Manifolds}
Now that we've had quite an abstract introduction to manifolds it helps to discuss a few examples.
The first and most obvious example is Euclidean space itself, which has an atlas consisting or a single chart, \((\reals^n, \mathrm{id})\), where \(\mathrm{id}\) is the identity function.

Our next example is the circle, \(S^1\).
This is a one-dimensional manifold.
It takes at least two charts to form an atlas.
One chart can be given by defining \(\vartheta = 0\) to be the top of the circle, and to wrap around to \(2\pi\).
The definition of a chart requires that the image of an open set is open in \(\reals\), in this case this means we can't include \(0\) or \(2\pi\) in the neighbour hood, so our neighbour hood is \((0, 2\pi)\).
We can introduce another chart that defines \(\vartheta' = 0\) to be the right most point of the circle and similarly wraps around to \(2\pi\), this again maps to the neighbourhood \((0, 2\pi)\) and covers the point excluded in the first chart.

The circle is fairly obviously a manifold since it is usually viewed embedded in Euclidean space.
This isn't necessary for all manifolds however.
A more abstract manifold is \(\specialOrthogonal(3)\), the group of rotations in three dimensions.
This is continuously parametrised by three Euler angles, and so is a three-dimensional manifold.
The Lorentz group, \(\orthogonal(1, 3)\), is a three dimensional manifold parametrised by the three components of the velocity of the frame.

For a system of \(N\) particles their phase space, consisting of their positions and momenta, is a \(6N\)-dimensional manifold.

Given an equation with two variables, \(x\) and \(y\), we can define the set of all \((x, y)\) to be a manifold and any particular solution to the equation is a curve in this manifold.

Vector spaces are manifolds, in particular real finite-dimensional vector spaces.
To see this we simply construct a basis and then parametrise the vectors by the coefficients of the basis vectors.

\section{Tangent Spaces}
One of the main changes when moving from discussing vectors in Euclidean space to vectors in some manifold is that it becomes important where the vector is on the manifold.
In Euclidean geometry we usually think of vectors as being at the origin, but we can also move them around freely, say to add them by combining them tip to tail.
This doesn't work on a general manifold as we will see later.

Let \(M\) be a smooth manifold.
For each point, \(p \in M\), we define a \defineindex{tangent space} at \(p\), \(T_pM\), and we can treat the vectors in this tangent space as Euclidean vectors.
A tangent vector is a vector in the tangent space.
The formal definition of a tangent space is a bit abstract:
\begin{dfn}{Tangent Space}{}
    Let \(M\) be a smooth manifold and \(f \colon M \to \reals\) a smooth function on \(M\).
    The set of all smooth functions on \(M\), denoted \(\End(M)\) or \(C^{\infty}(M)\), is a real associative algebra under point wise addition and products, that is for \(f, g \in \End(M)\) we define \((f + g)(p) = f(p) + g(p)\) and \(fg(p) = f(p)g(p)\).
    
    A \defineindex{derivation} at \(p \in M\) is a linear map, \(D \colon \End(M) \to \reals\) satisfying the Leibniz identity,
    \begin{equation}
        D(fg) = D(f)g(p) + f(p)D(g)
    \end{equation}
    for all \(f, g \in \End(M)\).
    Note the similarity to the product rule.
    
    Define addition and scalar multiplication of derivations at \(p\) by
    \begin{equation}
        (D_1 + D_2)(f) = D_1(f) + D_2(f), \qand (\lambda D)(f) = \lambda D(f)
    \end{equation}
    for derivations at \(p\) \(D_1\), \(D_2\), and \(D\), and some \(\lambda \in \reals\).
    This makes the space of derivations a real vector space and this is the space we define as the tangent space, \(T_pM\).
    
    We can then define the \defineindex{tangent bundle}, \(TM\), as the disjoint union of all tangent spaces,
    \begin{equation}
        TM \coloneqq \bigsqcup_{p \in M} T_pM = \{(p, x) \mid p \in M, x \in T_pM\}.
    \end{equation}
    That is we combine all vectors in any tangent space, \(T_pM\), into a single set and tag each one with the point \(p\), to keep track of \(p\), since this is important.
\end{dfn}

Given this definition we can define a vector field as an assignment of a tangent vector to each point in the manifold.
More formally we define a vector field, \(V \colon M \to TM\) such that \(\pi \circ V\) is the identity mapping on \(M\) where \(\pi \colon TM \to M\) is defined to be the projection \(\pi(p, x) = p\).
