\chapter{Algebra}
In this appendix we define the notion of \emph{an} algebra and build our way to a definition of a Lie algebra independent from any definition of a Lie group.

\begin{dfn}{Algebra}{}
    An \defineindex{algebra} is a vector space, \(A\), over a field, \(\field\), with a bilinear binary operation \(\cdot A \times A \to A\).
    To be bilinear means that this operation satisfies the following:
    \begin{itemize}
        \item \define{Right distributivity}: For all \(x, y, z \in A\) we have \((x + y) z = xz + yz\).
        \item \define{Left distributivity}: For all \(x, y, z \in A\) we have \(z(x + y) = zx + zy\).
        \item \define{Compatibility with scalar multiplication}: For all \(x, y \in A\) and \(a, b \in \field\) we ahve \((ax)(by) = (ab)(xy)\).
    \end{itemize}
    
    If \(x(yz) = (xy)z\) for all \(x, y, z \in A\) then we say that \(A\) is an \defineindex{associative algebra}.
\end{dfn}
When it is important to express the field we call \(A\) a \(\field\)-algebra.

\begin{exm}{}{}
    \begin{itemize}
        \item The complex numbers, \(\complex\), can be viewed as an associative \(\reals\)-algebra with the vector space \(\reals^2\) with the product of complex numbers as the bilinear operation.
        \item Three dimensional Euclidean space, \(\reals^3\), can be viewed as a non-associative \(\reals\)-algebra with the cross product as the bilinear operation.
        \item The quaternions, \(\quaternions\), can be viewed as an associative \(\reals\)-algebra with the vector space \(\reals^4\) with quaternion multiplication as the bilinear operation.
    \end{itemize}
\end{exm}

\begin{dfn}{\(\field\)-Algebra Morphisms}{}
    Given two \(\field\)-algebras, \(A\) and \(B\), a \define{\(\field\)-algebra homomorphism}\index{homomorphism!of algebras} is a \(\field\)-linear map, \(\varphi\colon A \to B\) such that
    \begin{equation}
        f(a\vv{x} + b\vv{y}) = af(\vv{x}) + bf(\vv{y}), \qqand f(\vv{x}\vv{y}) = f(\vv{x})f(\vv{y})
    \end{equation}
    for all \(a, b \in \field\) and \(\vv{x}, \vv{y} \in A\).
    
    A \define{\(\field\)-algebra isomorphism}\index{isomorphism!of algebras} is a bijective \(\field\)-algebra homomorphism.
\end{dfn}

\begin{dfn}{Lie Algebra}{}
    A \defineindex{Lie algebra} is a vector space, \(\lie{g}\), over some field, \(\field\), equipped with a non-associative binary operation endowing \(\lie{g}\) with the structure of an algebra.
    This operation, \(\liebracket{-}{-} \colon \lie{g}\times\lie{g} \to \lie{g}\), called the \defineindex{Lie bracket}, must also satisfy the following
    \begin{itemize}
        \item \define{Bilinearity}: For all \(x, y, z \in \lie{g}\) and \(a, b \in \field\)
        \begin{equation}
            \liebracket{ax + by}{z} = a\liebracket{x}{z} + b\liebracket{y}{z}, \qqand \liebracket{z}{ax + by} = a\liebracket{z}{x} + b\liebracket{z}{y}.
        \end{equation}
        \item \define{Alternativity}: For all \(x \in \lie{g}\)
        \begin{equation}
            \liebracket{x}{x} = 0.
        \end{equation}
        \item The \define{Jacobi identity}: For all \(x, y, z \in \lie{g}\)
        \begin{equation}
            \liebracket{x}{\liebracket{y}{z}} + \liebracket{y}{\liebracket{z}{x}} + \liebracket{z}{\liebracket{x}{y}} = 0.
        \end{equation}
    \end{itemize}
\end{dfn}
It is common, particularly in physics where we typically take \(\field\) to be \(\reals\) or \(\complex\), to require anticommutativity, \(\liebracket{x}{y} = -\liebracket{y}{x}\), instead of alternativity.
For fields with characteristics other than 2 this then implies alternativity, however for fields with characteristic 2 this isn't the case so this is a slightly weaker condition, but the distinction isn't important for us.

Notice that the left hand side of the Jacobi identity, while it looks complex, is really a sum over cyclic permutations of \(\liebracket{x}{\liebracket{y}{z}}\).

Consider the vector space of square \(n\times n\) matrices.
A Lie bracket defined on this vector space is the commutator, \(\liebracket{A}{B} \coloneqq AB - BA\), this is the prototypical example of a Lie bracket and the reason that the Lie bracket is written in the way it is.

Consider the vector space \(\reals^3\).
A Lie bracket defined on this vector space is the cross product, \(\liebracket{\vv{v}}{\vv{u}} \coloneqq \vv{v} \times \vv{u}\).

Let \(A\) be a commutative \(\reals\)-algebra, that is \(\vv{x}\vv{y} = \vv{y}\vv{x}\) for all \(\vv{x}, \vv{y} \in A\).
We define a derivation, \(D \colon A \to A\) to be a \(\reals\)-linear map (\(D(a\vv{x}) = aD(\vv{x})\) for all \(a \in \reals\) and \(\vv{x} \in A\)) such that \(D\) satisfies the Leibniz law:
\begin{equation}
    D(\vv{x}\vv{y}) = \vv{x}D(\vv{y}) + D(\vv{x})\vv{y}.
\end{equation}
The space of derivations, \(\mathop{\mathrm{Der}}(A)\), is a Lie algebra with the Lie bracket given by \(\liebracket{D_1}{D_2}(\vv{x}) = D_1(D_2(\vv{x})) - D_2(D_1(\vv{x}))\).

\begin{dfn}{Lie Algebra Morphisms}{}
    Given two Lie algebras, \(\lie{g}\) and \(\lie{h}\), over the same base field, \(\field\), a \define{Lie algebra homomorphism}\index{homomorphism!of Lie algebras} is an \(\field\)-algebra homomorphism, \(\varphi\colon \lie{g} \to \lie{h}\), which preserves the Lie bracket, so
    \begin{equation}
        \varphi(\liebracket{x}{y}) = \liebracket{\varphi(x)}{\varphi(y)}.
    \end{equation}
    If this is bijective then it is an \define{isomorphism of Lie algebras}\index{isomorphism!of Lie algebras}.
\end{dfn}

\section{Connection to Lie Groups}
Recall that we can think of a vector field on a smooth manifold as derivations.
Any group, \(G\), which acts smoothly on a manifold acts on the vector fields.
Further, the vector space of vector fields fixed by the group action is closed under the Lie bracket of derivations and so forms a Lie algebra.
Consider the case where \(G\) is a Lie group, so is itself a manifold, and acts smoothly on itself by left translation, \(g\action h = L_g(h) = gh\).
Then the space of left invariant vector fields, that is vector fields satisfying \(L_{g^*}X_h = X_{gh}\) for all \(h \in G\) with \(L_{g^*}\) being the differential of \(L_g\), is a Lie algebra under te Lie bracket of vector fields.
We can extend any tangent vector at the identity, that is an element of \(T_eG\), to a left invariant vector field by left translating the tangent vector to other points on the manifold.
In particular the left invariant extension of some \(v \in T_eG\) is defined by \(\tensor{v}{^\wedge_g} \coloneqq L_{g^*}v\).
This identifies \(T_eG\) with the space of left invariant vector fields, which makes \(T_eG\) a Lie algebra.

We usually denote the Lie algebra \(T_eG\) by the lowercase fraktur letter \(\lie{g}\).
The Lie bracket on \(\lie{g}\) is defined to be \(\liebracket{v}{w} \coloneqq \liebracket{v^{\wedge}}{w^{\wedge}}_e\), where \(\liebracket{v^{\wedge}}{w^{\wedge}}_e\) is the Lie bracket of left invariant vector fields at the identity.