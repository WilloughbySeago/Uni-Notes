\chapter{Integrals}
        \section{Gaussian Integrals}\label{app:gaussian integral}
        \subsection{Standard Gaussian Integral}
        The \defineindex{Gaussian integral} is
        \begin{equation}
            I = \int_{-\infty}^{\infty} \e^{-ax^2} \dd{x}
        \end{equation}
        for some positive \(a\).
        The simplest method to evaluate this is to instead compute \(I^2\):
        \begin{align}
            I^2 &= \int_{-\infty}^{\infty} \e^{-ax^2} \dd{x} \int_{-\infty}^{\infty} \e^{-ay^2} \dd{y}\\
            &= \int_{-\infty}^{\infty} \int_{-\infty}^{\infty} \e^{-a(x^2 + y^2)} \dd{x} \dd{y}\\
            &= \int_{0}^{\infty} \e^{-ar^2}\dd{r} \int_{0}^{2\pi} \dd{\vartheta}\\
            &= 2\pi \int_{0}^{\infty} r\e^{-ar^2}\dd{r}.
        \end{align}
        We can then tackle this integral by a change of variables to \(u = r^2\), meaning \(\dl{u} = 2r\dl{r}\) and
        \begin{equation}
            I^2 = 2\pi\frac{1}{2}\int_{0}^{\infty} \e^{-au} \dd{u} = -\frac{2\pi}{2a}[\e^{-au}]_{0}^{\infty} = -\frac{\pi}{a}(0 - 1) = \frac{\pi}{a}.
        \end{equation}
        Hence,
        \begin{equation}
            I = \int_{0}^{\infty} \e^{-ax^2} \dd{x} = \sqrt{\frac{\pi}{a}}.
        \end{equation}
        
        \subsection{Gaussians with a Polynomial Factor}
        Now consider integrals of the form
        \begin{equation}
            I_n = \int_{-\infty}^{\infty} x^n \e^{-ax^2} \dd{x}
        \end{equation}
        for \(n \in \naturals\).
        If \(n\) is odd then the integrand is odd and the integral is zero.
        If \(n\) is even then we can compute the integral by differentiation under the integral sign.
        We will do so here for the simple case of \(n = 2\).
        Higher order terms then follow by repeating the same method.
        \begin{align}
            I_2 &= \int_{-\infty}^{\infty} x^2\e^{-ax^2} \dd{x}\\
            &= -\int_{-\infty}^{\infty} \diffp*{\e^{-ax^2}}{a}\dd{x}\\
            &= -\diff*{}{a} \int_{-\infty}^{\infty} \e^{-ax^2}\dd{x}\\
            &= -\diff*{}{a} \sqrt{\frac{\pi}{a}}\\
            &= \frac{1}{2}\sqrt{\frac{\pi}{a^3}}.
        \end{align}
        Similarly by differentiating twice with respect to \(a\) we can show that
        \begin{equation}
            I_4 = \int_{-\infty}^{\infty} x^4\e^{-ax^2}\dd{x} = \frac{3}{4}\sqrt{\frac{\pi}{a^{5}}}
        \end{equation}
        
        \subsection{Quadratic Exponent}
        Consider the integral
        \begin{equation}
            I = \int_{-\infty}^{\infty} \e^{-ax^2 + bx + c} \dd{x}.
        \end{equation}
        Completing the square this becomes
        \begin{equation}
            I = \int_{-\infty}^{\infty} \e^{-a(x - b/2a)^2 + b^2/4a + c}\dd{x}.
        \end{equation}
        Using a change of variables to \(u = x - b/2a\) this becomes a standard Gaussian times a constant exponential and we find that
        \begin{equation}
            I = \e^{-b^2/4a + c} \sqrt{\frac{\pi}{a}}.
        \end{equation}
        This is valid also for complex \(b\) and \(c\) except that formally then we need to consider a contour integral but it reduces to the same answer.
        
        One useful case is
        \begin{equation}
            \int_{-\infty}^{\infty} \e^{-ax^2 + ikx} \dd{x} = \sqrt{\frac{\pi}{a}}\e^{-k^2/4a}.
        \end{equation}
        This shows that the Fourier transform of a Gaussian is another Gaussian.
        
        \section{Fresnel Integrals}
        The simplest Fresnel integral is
        \begin{equation}
            I = \int_{-\infty}^{\infty} \e^{iax^2} \dd{x}
        \end{equation}
        It can be shown using a wedge contour that 
        \begin{equation}
            I = \sqrt{\frac{\pi}{-ia}} = \sqrt{\frac{i\pi}{a}} = \e^{i\pi/4}\sqrt{\frac{\pi}{a}},
        \end{equation}
        where we make a branch cut from 0 to infinity along the negative real axis.
        
        For negative \(a\) we get a similar result if we take the complex conjugate.
        In general this result holds for all complex \(a\) with \(\Im(a) > 0\).