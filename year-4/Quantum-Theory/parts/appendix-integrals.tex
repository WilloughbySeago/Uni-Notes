\chapter{Integrals}
        \section{Gaussian Integrals}\label{app:gaussian integral}
        \subsection{Standard Gaussian Integral}
        The \defineindex{Gaussian integral} is
        \begin{equation}
            I = \int_{-\infty}^{\infty} \e^{-ax^2} \dd{x}
        \end{equation}
        for some positive \(a\).
        The simplest method to evaluate this is to instead compute \(I^2\):
        \begin{align}
            I^2 &= \int_{-\infty}^{\infty} \e^{-ax^2} \dd{x} \int_{-\infty}^{\infty} \e^{-ay^2} \dd{y}\\
            &= \int_{-\infty}^{\infty} \int_{-\infty}^{\infty} \e^{-a(x^2 + y^2)} \dd{x} \dd{y}\\
            &= \int_{0}^{\infty} \e^{-ar^2}\dd{r} \int_{0}^{2\pi} \dd{\vartheta}\\
            &= 2\pi \int_{0}^{\infty} r\e^{-ar^2}\dd{r}.
        \end{align}
        We can then tackle this integral by a change of variables to \(u = r^2\), meaning \(\dl{u} = 2r\dl{r}\) and
        \begin{equation}
            I^2 = 2\pi\frac{1}{2}\int_{0}^{\infty} \e^{-au} \dd{u} = -\frac{2\pi}{2a}[\e^{-au}]_{0}^{\infty} = -\frac{\pi}{a}(0 - 1) = \frac{\pi}{a}.
        \end{equation}
        Hence,
        \begin{equation}
            I = \int_{0}^{\infty} \e^{-ax^2} \dd{x} = \sqrt{\frac{\pi}{a}}.
        \end{equation}
        
        \subsection{Gaussians with a Polynomial Factor}
        Now consider integrals of the form
        \begin{equation}
            I_n = \int_{-\infty}^{\infty} x^n \e^{-ax^2} \dd{x}
        \end{equation}
        for \(n \in \naturals\).
        If \(n\) is odd then the integrand is odd and the integral is zero.
        If \(n\) is even then we can compute the integral by differentiation under the integral sign.
        We will do so here for the simple case of \(n = 2\).
        Higher order terms then follow by repeating the same method.
        \begin{align}
            I_2 &= \int_{-\infty}^{\infty} x^2\e^{-ax^2} \dd{x}\\
            &= -\int_{-\infty}^{\infty} \diffp*{\e^{-ax^2}}{a}\dd{x}\\
            &= -\diff*{}{a} \int_{-\infty}^{\infty} \e^{-ax^2}\dd{x}\\
            &= -\diff*{}{a} \sqrt{\frac{\pi}{a}}\\
            &= \frac{1}{2}\sqrt{\frac{\pi}{a^3}}.
        \end{align}
        Similarly by differentiating twice with respect to \(a\) we can show that
        \begin{equation}
            I_4 = \int_{-\infty}^{\infty} x^4\e^{-ax^2}\dd{x} = \frac{3}{4}\sqrt{\frac{\pi}{a^{5}}}
        \end{equation}
        
        \subsection{Quadratic Exponent}
        Consider the integral
        \begin{equation}
            I = \int_{-\infty}^{\infty} \e^{-ax^2 + bx + c} \dd{x}.
        \end{equation}
        Completing the square this becomes
        \begin{equation}
            I = \int_{-\infty}^{\infty} \e^{-a(x - b/2a)^2 + b^2/4a + c}\dd{x}.
        \end{equation}
        Using a change of variables to \(u = x - b/2a\) this becomes a standard Gaussian times a constant exponential and we find that
        \begin{equation}
            I = \e^{-b^2/4a + c} \sqrt{\frac{\pi}{a}}.
        \end{equation}
        This is valid also for complex \(b\) and \(c\) except that formally then we need to consider a contour integral but it reduces to the same answer.
        
        One useful case is
        \begin{equation}
            \int_{-\infty}^{\infty} \e^{-ax^2 + ikx} \dd{x} = \sqrt{\frac{\pi}{a}}\e^{-k^2/4a}.
        \end{equation}
        This shows that the Fourier transform of a Gaussian is another Gaussian.
        
        \section{Fresnel Integrals}
        The simplest Fresnel integral is
        \begin{equation}
            I = \int_{-\infty}^{\infty} \e^{iax^2} \dd{x}
        \end{equation}
        It can be shown using a wedge contour that 
        \begin{equation}
            I = \sqrt{\frac{\pi}{-ia}} = \sqrt{\frac{i\pi}{a}} = \e^{i\pi/4}\sqrt{\frac{\pi}{a}},
        \end{equation}
        where we make a branch cut from 0 to infinity along the negative real axis.
        
        For negative \(a\) we get a similar result if we take the complex conjugate.
        In general this result holds for all complex \(a\) with \(\Im(a) > 0\).
        
        \section{Feynman's Scattering Integral}\label{sec:feynman's scattering integral}
        Consider the integral
        \begin{equation}
            F(\alpha, \beta) = \int_0^T \frac{1}{[t(T - t)]^{3/2}} \exp\left[ -\frac{\alpha}{T - t} - \frac{\beta}{t} \right].
        \end{equation}
        To evaluate this we start with the simpler integral
        \begin{equation}
            I(a, b) = \int_{0}^{\infty} \exp\left[ -\frac{a}{x^2} - bx^2 \right] \dd{x}
        \end{equation}
        for \(a, b \in\reals_{>0}\).
        Making the substituting \(\xi = x\sqrt{b/a}\) we have \(\dl{\xi} = \sqrt{b/a}\dd{x}\), and the limits are unchanged.
        We therefore have
        \begin{align}
            I(a, b) &= \sqrt{\frac{a}{b}}\int_{0}^{\infty} \exp\left[ -\frac{a}{\frac{a}{b}\xi^2} - b\frac{a}{b}\xi^2 \right] \dd{\xi}\\
            &= \sqrt{\frac{a}{b}} \int_{0}^{\infty} \exp\left[ -\frac{b}{\xi^2} - a\xi^2 \right]\dd{\xi}\\
            &= \sqrt{\frac{a}{b}} I(b, a).
        \end{align}
    
        If instead we make the substitution \(y = 1/x\) then we have \(\dd{y} = -\dl{x}/x^2 = -y^2\dl{x}\) so \(\dl{x} = -\dl{y}/y^2\).
        The limits change from \((0, \infty)\) to \((\infty, 0)\).
        Hence
        \begin{align}
            I(a, b) &= -\int_{\infty}^{0} \frac{1}{y^2} \exp\left[ -ay^2 - \frac{b}{y^2} \right] \dd{y}\\
            &= \int_{0}^{\infty} \frac{1}{y^2} \exp\left[ -ay^2 - \frac{b}{y^2} \right] \dd{y}\\
            &= -\diffp*{I(b, a)}{b}.
        \end{align}
        
        Combining these two results we have
        \begin{equation}\label{eqn:I' = -sqrt(a/b) I}
            \diffp*{I(b, a)}{b} = -\sqrt{\frac{a}{b}} I(b, a).
        \end{equation}
        Notice now that
        \begin{equation}
            \diffp{}{b}I(0, a)\e^{-2\sqrt{ab}} = -\sqrt{\frac{a}{b}}I(0, a)
        \end{equation}
        and since this completely encapsulates the behaviour of \(b\) we know that
        \begin{equation}
            I(b, a) = I(0, a)\e^{-2\sqrt{ab}}.
        \end{equation}
        Putting this into \cref{eqn:I' = -sqrt(a/b) I} we have
        \begin{equation}
            \diffp*{I(b, a)}{b} = \diffp*{I(0, a)\e^{-2\sqrt{ab}}}{b} = \sqrt{\frac{a}{b}}I(b, a).
        \end{equation}
        Integrating this gives a Gaussian integral in \(\sqrt{b}\), which then gives us the solution
        \begin{equation}
            I(0, a) = \frac{1}{2}\sqrt{\frac{\pi}{a}},
        \end{equation}
        which gives the full result
        \begin{equation}
            I(a, b) = \frac{1}{2}\sqrt{\frac{\pi}{b}} \e^{-2\sqrt{ab}}.
        \end{equation}
        
        We can use this result to evaluate \(F(\alpha, \beta)\) with \(\alpha, \beta \in \reals_{>0}\).
        First let \(x = \sqrt{(T - t)/t}\) and so
        \begin{equation}
            \frac{a}{x^2} + bx^2 = a\frac{T}{T - t} + b\frac{T}{t} - a - b,
        \end{equation}
        and
        \begin{equation}
            \diff{x}{t} = -\frac{T}{t}\sqrt{\frac{T - t}{t}} \frac{1}{t(T - t)} = -\frac{T}{2} \frac{1}{t^{3/2}\sqrt{T - t}}.
        \end{equation}
        Making this substitution in the definition of \(I(a, b)\) we have
        \begin{equation}
            I(a, b) = \frac{1}{2}T\e^{a + b} \int_0^T \frac{1}{t^{3/2}\sqrt{T - t}} \exp\left[ -a\frac{T}{T - t} - b\frac{T}{t} \right] \dd{t}.
        \end{equation}
        We see that
        \begin{align}
            F(\alpha, \beta) &= -\diffp{}{\alpha} \int_0^T \frac{1}{t^{3/2}\sqrt{T - t}} \exp\left[ -\frac{\alpha}{T - t} - \frac{\beta}{t} \right]\\
            &= -\diffp{}{\alpha} \left[ \frac{2}{T}I\left( \frac{\alpha}{T}, \frac{\beta}{T} \right) \exp\left( -\frac{\alpha}{T} - \frac{\beta}{T} \right) \right].
        \end{align}
        Using the result for \(I(a, b)\) we get
        \begin{align}
            F(\alpha, \beta) &= -\diffp{}{\alpha} \sqrt{\frac{\pi}{\beta T}} \exp\left[ \frac{1}{T}(\sqrt{\alpha} + \sqrt{\beta})^2 \right]\\
            &= \frac{1}{T}\sqrt{\frac{\pi}{T}} \left( \frac{1}{\sqrt{\alpha}} + \frac{1}{\sqrt{\beta}} \right) \exp\left[ \frac{1}{T}(\sqrt{\alpha} + \sqrt{\beta})^2 \right].
        \end{align}
        
        This result also holds for \(\alpha, \beta \in \complex\) so long as \(\Re(\alpha), \Re(\beta) > 0\).
        It also holds for purely imaginary \(\alpha\) and \(\beta\) by analytic continuation.
        