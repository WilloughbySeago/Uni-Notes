\chapter{Formal Definitions}
    \section{Hilbert Space}
    A \defineindex{Hilbert space} is an inner product space which is complete with respect to the induced metric.
    This is quite a lot so we'll break it down.
    For simplicity we will restrict ourselves to the idea of a complex Hilbert space, although all of this applies to a real vector space, we just ignore any complex conjugation.
    
    \section{Vector Space}
    A complex \defineindex{vector space} is a set of vectors, \(V\), with two operations defined:
    \begin{equation}
        +\colon V\times V \to V, \qqand \cdot\colon \complex\times V \to V.
    \end{equation}
    These operations, referred to as addition and scalar multiplication respectively, satisfy the following for all \(\vv{u}, \vv{v}, \vv{w}\in V\) and \(\alpha, \beta\in\complex\):
    \begin{itemize}
        \item Vector addition is associative: \(\vv{u} + (\vv{v} + \vv{w}) = (\vv{u} + \vv{v}) + \vv{w}\).
        \item Vector addition is commutative: \(\vv{u} + \vv{v} = \vv{v} + \vv{u}\).
        \item Identity element for addition: there exists \(\vv{0} \in V\) such that \(\vv{v} + \vv{0} = \vv{v}\).
        \item Inverse of vector addition: there exists \(-\vv{v}\in V\) such that \(\vv{v} + (-\vv{v}) = \vv{0}\).
        \item Compatibility of scalar and field multiplication: \(\alpha(\beta\vv{v}) = (\alpha\beta)\vv{v}\).
        \item Identity of scalar multiplication: \(1\vv{v} = \vv{v}\).
        \item Distributivity of scalar multiplication over vector addition: \(\alpha(\vv{u} + \vv{v}) = \alpha\vv{u} + \alpha\vv{v}\).
        \item Distributivity of scalar addition: \((\alpha + \beta)\vv{v} = \alpha\vv{v} + \beta\vv{v}\).
    \end{itemize}

    \subsection{Dual Space}
    The \defineindex{dual space} of a vector space \(V\) is the space of linear forms on \(V\), that is maps \(V \to \complex\), assuming a complex vector space.
    We denote this space by \(V^*\).
    
    \subsection{Inner Product Space}
    An \defineindex{inner product space} is a vector space, \(V\), equipped with an inner product
    \begin{equation}
        \langle \cdot, \cdot \rangle \colon V \times V \to \complex
    \end{equation}
    which satisfies the following for all \(\vv{u}, \vv{v}, \vv{w} \in V\) and \(\alpha \in \complex\) with \(\vv{0}\) being the zero vector:
    \begin{itemize}
        \item \(\langle \vv{0}, \vv{v} \rangle = \langle \vv{v}, \vv{0} \rangle = 0\).
        \item \(\langle \vv{u} + \vv{v}, \vv{w} \rangle = \langle \vv{u}, \vv{w} \rangle + \langle \vv{v}, \vv{w} \rangle\).
        \item \(\langle \alpha\vv{u}, \vv{v} \rangle = \alpha^*\langle \vv{u}, \vv{v} \rangle\) and \(\langle \vv{u}, \alpha\vv{v} \rangle = \alpha\langle \vv{u}, \vv{v} \rangle\).
        \item \(\langle \vv{u}, \vv{v} \rangle = \langle \vv{v}, \vv{u} \rangle^*\).
    \end{itemize}
    
    \subsection{Induced Metric}
    Given an inner product space, \(V\), we define the \defineindex{metric} induced by the inner product to be a function
    \begin{align}
        \norm{\cdot} \colon V &\to \complex,\\
        \norm{\vv{v}} &\mapsto \sqrt{\langle \vv{v}, \vv{v}\rangle}.
    \end{align}
    The statement that this is a metric requires that this is \defineindex{positive definite}.
    That is, \(\norm{\vv{v}} = 0\) if and only if \(\vv{v} = \vv{0}\).
    
    \subsection{Completeness}
    An inner product space, \(V\), is complete if any series of vectors,
    \begin{equation}
        \sum_{n=1}^{\infty} \vv{v_i}
    \end{equation}
    converges absolutely with respect to the induced metric, that is
    \begin{equation}
        \sum_{n=1}^{\infty} \norm{\vv{v_i}}
    \end{equation}
    converges, then the series
    \begin{equation}
        \sum_{n=1}^{\infty} \vv{v_i}
    \end{equation}
    converges to some value in \(V\).
    
    \subsection{Hilbert Space}
    A \defineindex{Hilbert space} is an inner product space which is complete with respect to the induced metric.
    
    We now see that this simply means that a Hilbert space is a set of vectors with addition and multiplication and an inner product, all satisfying the properties we are used and a condition on convergence of series.
    This final condition is rarely invoked directly in physics, instead we just do the usual physicist thing of assuming that things converge nicely.
    The completeness relation just sometimes allows us to offload proving convergence in the Hilbert space to proving absolute convergence in \(\reals\), which is often easier.