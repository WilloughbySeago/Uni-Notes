\chapter{Newton's Laws and Galilean Transformations}
    \section{Newton's Laws}
    Newton's laws should be familiar, so we state them here in their standard form and expand upon the details later:
    \begin{itemize}
    \item \defineindex{Newton's first law} (N1)\glossary[acronym]{N1}{Newton's first law} states that a body remains in a state of rest or uniform motion in a straight line unless acted on by an external force.
    
    \item \defineindex{Newton's second law} (N2)\glossary[acronym]{N2}{Newton's second law} states that the rate of change of linear momentum of a body is proportional to the magnitude of the force acting upon the body and is in the direction of the force.
    
    \item \defineindex{Newton's third law} (N3)\glossary[acronym]{N3}{Newton's third law} states that for every action there is an equal and opposite reaction.
    
    \item \defineindex{Newton's law of gravitation} (NG)\glossary[acronym]{NG}{Newton's law of gravitation} states that the force between two massive bodies due to gravity is proportional to the product of their masses and inversely proportional to the square of the distance between them.
    Also the force always acts to bring the bodies closer together.
    \end{itemize}
    
    In these we take \enquote{a body} to be a point particle.
    Uniform motion means that the velocity is constant.
    This means that the direction of motion doesn't change so, for example, circular motion at a constant speed is \emph{not} uniform motion.
    
    We can regard N1 as a special case of N2 with the force equal to zero.
    We can \emph{define} force using N2 by choosing appropriate units such that the constant of proportionality is equal to 1 and hence
    \begin{equation}
        \vv{F} \coloneqq \dot{\vv{p}}
    \end{equation}
    where we use the dot notation \(\dot{f} \coloneqq \diff{f}/{t}\) for some differentiable quantity, \(f\), and time \(t\).
    Here \(\vv{p}\) is the linear momentum which, for Newtonian mechanics, is defined to be
    \begin{equation}
        \vv{p} \coloneqq m\vv{v} = n\dot{\vv{r}}
    \end{equation}
    where \(\vv{v}\) is the velocity vector and \(\vv{r}\) the position vector.
    
    We often deal with the special case where \(m\) is constant.
    In this case \(\dot{\vv{p}} = m\dot{v} = m\ddot{\vv{r}} = m\vv{a}\).
    Unless otherwise specified we will usually assume that mass is a constant.
    
    It seems like N2 defies everyday experience where when we stop pushing an object it stops moving.
    This is simply because we fail to account for the force of friction opposing the motion.
    
    We can write NG as
    \begin{equation}
        \vv{F} = -\frac{GMm}{r^2}\vh{r}
    \end{equation}
    where \(M\) and \(m\) are the masses of the two bodies, \(r\) is their separation, \(\vh{r}\) is a unit vector pointing between the bodies, to the body upon which the force acts, and \(G = \qty{6.6743}{\meter\cubed\per\kilogram\per\squared\second}\) is the universal \defineindex{gravitational constant}.
    
    As a final remark for this section: we have seen two different ways that mass appears here.
    In N2 mass is related to the inertia of the body, that is its ability to resist changes in its velocity.
    In NG mass is a source of a force.
    We will discus this when we consider GR.
    
    \section{Frame of Reference}
    In order to have equations of motion we generally need a coordinate system and a way to specify time.
    This is what we call a frame of reference.
    More specifically we can consider some Cartesian coordinate system, \((x, y, z)\), and time measured by some ideal clock, \(t\).
    
    If N1 holds in a frame we call it an \defineindex{inertial frame}.
    
    We are most interested in the case of two inertial frames moving relative to each other.
    To aid us in our analysis we define the \defineindex{standard configuration} in which one frame, \(S'\), is moving at a speed \(V\) in a straight line relative to another frame, \(S\).
    We choose a right handed Cartesian coordinate system such that the motion of \(S'\) is along the \(x\)-axis of \(S\) and all axes align.
    We define time such that \(t = t' = 0\) when the origins of the two frames coincide.
    Unless we state otherwise we will assume frames are in the standard configuration.
    
    \section{Galilean Transformations}
    An \defineindex{event}, for example a light flashing, has a position, \((x, y, z)\), in a given frame as well as occurring at some specified time, \(t\).
    In Newtonian mechanics the existence of \defineindex{universal time} is axiomatic.
    That is once two clocks are synchronised they will agree on all time measurements.
    
    At this point we mention that all of this assumes the existence of an intelligent observer capable of making ideal measurements of position and time.
    
    \begin{figure}
        \tikzsetnextfilename{relative-frame-motion}
        \begin{tikzpicture}
            \tikzset{axis/.style={highlight, very thick, ->, text=black}}
            \draw[axis] (0, 0, 0) -- (3, 0, 0) node[right] {\(x\)};
            \draw[axis] (0, 0, 0) -- (0, 3, 0) node[above] {\(y\)};
            \draw[axis] (0, 0, 0) -- (0, 0, 3) node[below left] {\(z\)};
            \node at (1.5, 4) {\(S\)};
            \draw[axis] (5, 0, 0) -- (8, 0, 0) node[right] {\(x'\)};
            \draw[axis] (5, 0, 0) -- (5, 3, 0) node[above] {\(y'\)};
            \draw[axis] (5, 0, 0) -- (5, 0, 3) node[below left] {\(z'\)};
            \node at (6.5, 4) {\(S'\)};
            \draw[gray, very thick, ->, text=black] (7, 4) -- (8, 4) node[right] {\(\vv{V}\)};
            \draw[gray, dashed] (6, 2, 0) -- (6, 0, 0);
            \fill[highlight] (6, 2, 0) circle [radius=0.1];
            \draw[|<->|, gray, text=black] (5, 1.5, 0) -- (6, 1.5, 0) node[midway, above] {\(x'\)};
            \draw[|<->|, gray, text=black] (0, 1.5, 0) -- (5, 1.5, 0) node[midway, above] {\(Vt\)};
            \draw[|<->|, gray, text=black] (0, 1, 0) -- (6, 1, 0) node[midway, above] {\(x\)};
        \end{tikzpicture}
        \caption{Two frames, \(S\) and \(S'\), in the standard configuration with \(S'\) moving at speed \(V\) relative to \(S\). Also shown is an event, here a dot, and how this relates to various lengths in the diagram.}
        \label{fig:relative motion of frames}
    \end{figure}
    
    Consider \cref{fig:relative motion of frames} which shows two frames in the standard configuration with frame \(S'\) moving at speed \(V\) relative to \(S\).
    Using the lengths \(x\), \(x'\), and \(Vt\) shown in the diagram we can readily see that \(x = x' + Vt\).
    Combining this with the fact that the other axes are perpendicular to the motion and at one point coincided we get the \defineindex{Galilean transformation} for position and time:
    \begin{equation}
        x' = x - Vt, \qquad y' = y, \qquad z' = z, \qqand t' = t.
    \end{equation}
    Equivalently taking \(\vv{V} = (V, 0, 0)\), \(\vv{r} = (x, y, z)\), and \(\vv{r'} = (x', y', z')\) we have
    \begin{equation}
        \vv{r'} = \vv{r} - \vv{V}t, \qqand t' = t.
    \end{equation}

    Note that in this derivation we have made the assumption that space is isotropic, that is the laws of physics are the same no matter where we are.
    If we combine this idea with the speed of light being constant we can derive all of SR so it is clearly a very powerful concept.
    
    We can simply differentiate these equations to get the relevant transformation for velocities:
    \begin{equation}
        \dot{x}' = \dot{x} - V, \qquad \dot{y}' = \dot{y}, \qqand \dot{z}' = \dot{z},
    \end{equation}
    which implies
    \begin{equation}
        u_x' = u_x - V, \qquad u_y' = u_y, \qqand u_z' = u_s
    \end{equation}
    or equivalently
    \begin{equation}
        \vv{u'} = \vv{u} - \vv{V}
    \end{equation}
    where \(\vv{u} = (u_x, u_y, u_z) = (\dot{x}, \dot{y}, \dot{z})\) and similarly \(\vv{u'} = (u_x', u_y', u_z') = (\dot{x}', \dot{y}', \dot{z}')\).
    
    From this we see that the velocity as measured relative to the origin of a given frame is frame dependent.
    However, suppose we have two objects, moving at speeds \(\vv{u}\) and \(\vv{v}\) in frame \(S\) which are \(\vv{u'}\) and \(\vv{v'}\) in frame \(S'\).
    Then the relative velocity of these two objects in \(S'\) is
    \begin{equation}
        \vv{v'} - \vv{u'} = \vv{v} - \vv{V} - (\vv{u} - \vv{V}) = \vv{v} - \vv{u}
    \end{equation}
    which is the relative velocity in frame \(S\).
    This means that relative velocities are not dependent on the frame.
    An alternative way of putting this is that relative velocities are Galilean \defineindex{invariant}, meaning they don't change under a Galilean transformation.
    This is a powerful concept and invariants under various transformations are incredibly important in many areas of physics.
    
    \subsection{Invariance of Newton's Laws}
    Consider some force \(\vv{F}\) and a particle of constant mass, \(m\).
    Then in some inertial frame, \(S\), we have
    \begin{equation}
        F_x = m\diff{v_x}{t} = m\diff[2]{x}{t}
    \end{equation}
    and in another initial frame, \(S'\), we have
    \begin{equation}
        F_x' = m\diff{v_x'}{t} = m\diff*{(v_x - V)}{t} = m\diff{v_x}{t} = F_x
    \end{equation}
    where we assume that \(V\) is constant.
    Similarly we can show that \(F_y = F_y'\) and \(F_z = F_z'\).
    This shows that \(\vv{F'} = \vv{F}\) for inertial frames and hence \(\vv{a'} = \vv{a}\).
    What this means is that N2 is Galilean invariant, by which we mean its form is unaffected by a Galilean transformation.
    
    \subsection{Newtonian Relativity Principle}
    The \defineindex{Newtonian principle of relativity} states that 
    \begin{displayquote}
        The laws of Newtonian mechanics are invariant under Galilean transformations and have the same form in all inertial frames of reference.
    \end{displayquote}
    This is basically a fancy way of stating that all inertial frames are equivalent and there is no experiment we can do to distinguish between an inertial frame at rest and an inertial frame that is moving relative to some other inertial frame without making measurements in this external frame.
    
    \section{Accelerating Frames}
    Consider the setup in \cref{fig:acceleration of frames} which shows two frames, \(S\) and \(S'\), where \(S'\) is accelerating along the \(x\)-axis of \(S\) at some acceleration \(A\).
    
    \begin{figure}
        \tikzsetnextfilename{relative-frame-motion}
        \begin{tikzpicture}
            \tikzset{axis/.style={highlight, very thick, ->, text=black}}
            \draw[axis] (0, 0, 0) -- (3, 0, 0) node[right] {\(x\)};
            \draw[axis] (0, 0, 0) -- (0, 3, 0) node[above] {\(y\)};
            \draw[axis] (0, 0, 0) -- (0, 0, 3) node[below left] {\(z\)};
            \node at (1.5, 4) {\(S\)};
            \draw[axis] (5, 0, 0) -- (8, 0, 0) node[right] {\(x'\)};
            \draw[axis] (5, 0, 0) -- (5, 3, 0) node[above] {\(y'\)};
            \draw[axis] (5, 0, 0) -- (5, 0, 3) node[below left] {\(z'\)};
            \node at (6.5, 4) {\(S'\)};
            \draw[gray, very thick, ->, text=black] (7, 4) -- (8, 4) node[right] {\(\vv{A}\)};
            \draw[gray, dashed] (6, 2, 0) -- (6, 0, 0);
            \fill[highlight] (6, 2, 0) circle [radius=0.1];
            \draw[|<->|, gray, text=black] (5, 1.5, 0) -- (6, 1.5, 0) node[midway, above] {\(x'\)};
            \draw[|<->|, gray, text=black] (0, 1.5, 0) -- (5, 1.5, 0) node[midway, above] {\(X\)};
            \draw[|<->|, gray, text=black] (0, 1, 0) -- (6, 1, 0) node[midway, above] {\(x\)};
        \end{tikzpicture}
        \caption{Two frames, \(S\) and \(S'\), in the standard configuration with \(S'\) moving at speed \(V\) relative to \(S\). Also shown is an event, here a dot, and how this relates to various lengths in the diagram.}
        \label{fig:acceleration of frames}
    \end{figure}

    The same logic we used to derive the Galilean transformations gives us
    \begin{equation}
        \vv{r'} = \vv{r} - \vv{X}, \qqand t' = t.
    \end{equation}
    where \(\vv{X}\) is the position of the frame \(S'\) in frame \(S\).
    Differentiating we get
    \begin{equation}
        \vv{u'} = \vv{u} - \dot{\vv{X}} = \vv{u} - \vv{V}
    \end{equation}
    where we define \(\vv{V} \coloneqq \dot{\vv{X}}\) as the instantaneous velocity of \(S'\) with respect to \(S\).
    Suppose that \(\vv{A}\) is constant.
    Then basic mechanics tells us that
    \begin{equation}
        V = V_0 + At
    \end{equation}
    where \(V_0\) is the speed of frame \(S'\) in \(S\) at \(t = 0\).
    Substituting this into our equation we have
    \begin{equation}
        \vv{u'} = \vv{u} - \vv{V} = \vv{u} - \vv{V_0} - \vv{A}t
    \end{equation}
    differentiating gives us
    \begin{equation}
        \vv{a'} = \vv{a} - \vv{A}.
    \end{equation}
    Multiplying through by the mass, \(m\), of some particle we have
    \begin{equation}
        m\vv{a'} = m\vv{a} - m\vv{A}.
    \end{equation}
    The left hand side of this is the force acting on the particle to produce the acceleration observed in frame \(S'\).
    That is
    \begin{equation}
        \vv{F'} = m\vv{a} - m\vv{A}.
    \end{equation}
    We see that the force is composed of two parts.
    The expected \(m\vv{a}\) and an extra term, \(m\vv{A}\), which we call a \defineindex{fictitious force} or \defineindex{inertial force} due to the fact that it exists only due to the acceleration of the frame.
    
    This \enquote{fictitious} force can have very real effects.
    For example consider going around a corner in a car.
    It feels like you are pushed back into your seat.
    This is a fictitious force due to the acceleration needed to change direction.
    
    \begin{exm}{Pendulum on a Train}{exm:pendulum on a train}
        Consider a simple pendulum hanging vertically downward on a stationary train.
        The Newtonian principle of relativity states that if the train is instead moving at constant velocity then the pendulum will again hang vertically downward.
        The interesting case is when the train is accelerating with constant acceleration, \(\vv{A}\).
        These three cases are shown in \cref{fig:pendulum on a train}.
        
        For the accelerating case we first work in the frame of the ground, in which the train is accelerating.
        Resolving the horizontal and vertical forces gives us
        \begin{equation}
            T\sin\vartheta = mA, \qqand T\cos\vartheta - mg = 0
        \end{equation}
        where \(m\) is the mass of the pendulum and \(g\) the magnitude of the acceleration due to gravity.
        The left hand side of these equations is the force and the right hand side the resulting acceleration.
        
        Working now in the frame of the train and accounting for the resulting fictitious force, \(m\vv{A}\), we can again resolve horizontally and vertically to get
        \begin{equation}
            T\sin\vartheta - mA, \qqand T\cos\vartheta - mg = 0.
        \end{equation}
        Again we interpret the left hand side as the force and the right hand side as acceleration.
        
        Notice that the equations are the same in both cases, but that the interpretation differs in whether the acceleration of the train is a force or an acceleration.
    \end{exm}
    \begin{figure}
        \tikzsetnextfilename{pendulum-on-train}
        \begin{tikzpicture}
            \fill[pattern=north east lines] (-1, 4) rectangle (1, 4.5);
            \draw[thick] (-1, 4) -- (1, 4);
            \draw[very thick] (0, 0) -- (0, 4);
            \fill (0, 0) circle [radius=0.3];
            \draw[ultra thick, highlight, ->, text=black] (0, 0.3) -- (0, 2) node[left] {\(\vv{T}\)};
            \draw[ultra thick, highlight, ->] (0, -0.3) -- (0, -2) node[left] {\(\vv{F_g}\)};
            
            \begin{scope}[xshift=3cm]
                \fill[pattern=north east lines] (-1, 4) rectangle (1, 4.5);
                \draw[thick] (-1, 4) -- (1, 4);
                \draw[very thick] (0, 0) -- (0, 4);
                \fill (0, 0) circle [radius=0.3];
                \draw[ultra thick, highlight, ->, text=black] (0, 0.3) -- (0, 2) node[left] {\(\vv{T}\)};
                \draw[ultra thick, highlight, ->] (0, -0.3) -- (0, -2) node[left] {\(\vv{F_g}\)};
                \draw[highlight, ->, text=black] (-0.75, 5) -- (0.75, 5) node[midway, above] {\(\vv{V}\)};
            \end{scope}
            
            \begin{scope}[xshift=6cm]
                \fill[pattern=north east lines] (-1, 4) rectangle (1, 4.5);
                \draw[thick] (-1, 4) -- (1, 4);
                \draw[very thick] (0, 4) -- ($(0, 4) + (-100:4)$);
                \fill ($(0, 4) + (-100:4)$) circle [radius=0.3];
                \draw[ultra thick, highlight, ->, text=black] ($(0, 4) + (-100:3.7)$) -- ++ (80:2) node[left, pos=0.9] {\(\vv{T}\)};
                \draw[ultra thick, highlight, ->] ($(0, 4) + (-100:4) - (0, 0.3)$) -- ++(0, -2) node[left] {\(\vv{F_g}\)};
                \draw[highlight, ->, text=black] (-0.75, 5) -- (0.75, 5) node[midway, above] {\(\vv{A}\)};
                \draw[dashed] (0, 4) -- (0, 2.5);
                \begin{scope}
                    \clip (0, 4) -- ($(0, 4) + (-100:4)$) -- (0, 2.5) -- cycle;
                    \draw[highlight] (0, 4) circle [radius = 1.4];
                \end{scope}
                \node at ($(0, 4) + (-95:1.6)$) {\(\vartheta\)};
            \end{scope}
        \end{tikzpicture}
        \caption{A pendulum on a train at rest (left), constant velocity (middle) and accelerating (left).}
        \label{fig:pendulum on a train}
    \end{figure}
    
    \subsubsection{A Brief Note on Gravity}
    In \cref{exm:pendulum on a train} we were in no doubt that the gravity acted downwards.
    However, notice that we could interpret the fictitious force as the same as gravity acting in a slightly different direction.
    This is because both the fictitious force, \(mA\), and the force due to gravity, \(mg\), scale with the mass.
    Equivalently, both forces satisfy Galileo's principle which states that all bodies, irrespective of their mass, have the same acceleration due to gravity.
    If gravity and fictitious forces give rise to the same physical effects can we distinguish them?
    It turns out that the answer is no, and not only this, but we can actually treat gravity as the result of acceleration.
    This was one of the things that lead Einstein to develop GR.
    
    \chapter{Two Body Systems}
    In this chapter we consider two body systems.
    That is systems of two particles with masses \(m_1\) and \(m_2\) and positions \(\vv{r_1}\) and \(\vv{r_2}\).
    We define \(\vv{f}\) to be the force on body one due to body two, and by N3 \(-\vv{f}\) is the force on body two due to body one.
    We also consider the external forces \(\vv{F_1}\) and \(\vv{F_2}\) acting on bodies one and two respectively.
    
    \section{Conservation of Momentum}
    Newton's second law gives us the equations of motion
    \begin{align}
        \vv{F_1} + \vv{f} &= m_1\ddot{\vv{r}}_{\vv{1}}\label{eqn:two body eom 1}\\
        \vv{F_2} - \vv{f} &= m_2\ddot{\vv{r}}_{\vv{2}}\label{eqn:two body eom 2}
    \end{align}
    Adding together these equations we get
    \begin{equation}
        \vv{F_1} + \vv{F_2} = m_1\ddot{\vv{r}}_{\vv{1}} + m_2\ddot{\vv{r}}_{\vv{2}} = m_1\dot{\vv{v_1}} + m_2\dot{\vv{v_2}}.
    \end{equation}
    In the absence of external forces, i.e. if \(\vv{F_1} = \vv{F_2} = \vv{0}\), then this reduces to
    \begin{equation}
        m_1\dot{\vv{v_1}} + m_2\dot{\vv{v_2}} = 0.
    \end{equation}
    Integrating with respect to time and recognising the definition of linear momentum we have
    \begin{equation}
        m_1\vv{v_1} + m_2\vv{v_2} = \vv{p_1} + \vv{p_2} = \text{constant.}
    \end{equation}
    From this we conclude that in the absence of external forces the total linear momentum is conserved.
    Importantly this is independent of the interactions between the particles since the \(\vv{f}\)s cancel out early on.
    This suggests that we should consider the system as a whole.
    
    \section{Conservation of Energy}
    The \defineindex{work done} by a force \(\vv{F}\) acting on a particle to move it through an infinitesimal displacement \(\dl{\vv{r}}\) is \(\dl{W} = \vv{F} \cdot \dl{\vv{r}}\).
    The work done to move the particle along a path from \(A\) to \(B\) is then
    \begin{equation}
        W_{AB} = \int_A^B \vv{F}\cdot\dl{\vv{r}}
    \end{equation}
    where the integral is over the path.
    Evaluating this integral we have
    \begin{align}
        W_{AB} &= \int_A^B \vv{F} \cdot \dl{\vv{r}}\\
        &= \int_A^B \dot{\vv{p}} \cdot \dl{\vv{r}}\\
        &= m\int_A^B \dot{\vv{v}} \cdot \dl{\vv{r}}\\
        &= m\int_A^B \dot{\vv{v}} \cdot \diff{\vv{r}}{t}\dd{t}\\
        &= m\int_A^B \dot{\vv{v}}\cdot\vv{v} \dd{t}\\
        &= \frac{m}{2}\int_A^B \diff*{(v^2)}{t} \dd{t}\\
        &= \frac{1}{2}mv_B^2  - \frac{1}{2}mv_A^2\\
        &= T_B - T_A.
    \end{align}
    Here we have used the product rule to simplify the integrand:
    \begin{equation}
        \diff*{(v^2)}{t} = \diff*{(\vv{v}\cdot\vv{v})}{t} = \vv{v}\cdot\dot{\vv{v}} + \dot{\vv{v}}\cdot\vv{v} = 2\dot{\vv{v}}\cdot\vv{v} \implies \dot{\vv{v}}\cdot\vv{v} = \frac{1}{2}\diff*{(v^2)}{t}.
    \end{equation}
    We identify \(T\) as the \defineindex{kinetic energy} with subscripts \(A\) and \(B\) on \(T\) and \(v\) denoting the kinetic energy or velocity at the relevant location.
    
    \subsection{Collisions}
    An \defineindex{elastic collision} between two bodies is one where the total kinetic energy is conserved.
    That is
    \begin{equation}
        \frac{1}{2}m_1v_{1i}^2 + \frac{1}{2}m_2v_{2i}^2 = \frac{1}{2}m_1v_{1f}^2 + \frac{1}{2}m_2v_{2f}^2
    \end{equation}
    with subscript \(i\) and \(f\) denoting initial and final quantities from before and after the collision respectively.
    
    Far more realistic in most cases are \defineindex{inelastic collisions} where the total kinetic energy decreases, so that
    \begin{equation}
        \frac{1}{2}m_1v_{1i}^2 + \frac{1}{2}m_2v_{2i}^2 > \frac{1}{2}m_1v_{1f}^2 + \frac{1}{2}m_2v_{2f}^2.
    \end{equation}
    Since the total energy is still conserved this energy must be transformed into another form, often as sound or heat, or it is dissipated in deforming the bodies.
    
    \section{Centre of Mass Coordinates}
    We mentioned earlier that it can be advantageous to treat a two body system as a whole.
    Centre of mass coordinates are one way of doing this.
    The centre of mass is located at \(\vv{R}\) which is such that
    \begin{equation}
        \vv{R} \coloneqq \frac{m_1\vv{r_1} + m_2\vv{r_2}}{m_1 + m_2}.
    \end{equation}
    We also define \(\vv{r}\) as the position of body one from body 2:
    \begin{equation}
        \vv{r} \coloneqq \vv{r_1} - \vv{r_2}.
    \end{equation}
    See \cref{fig:centre of mass coord}.
    
    \begin{figure}
        \tikzsetnextfilename{centre-of-mass-coord}
        \begin{tikzpicture}
            \fill (3, 1) circle [radius = 0.1] node[below right] {\(m_2\)};
            \fill (2, 4) circle [radius = 0.1] node[above] {\(m_1\)};
            \draw[->, very thick, highlight, text=black] ($(3, 1)!0.05cm!(2, 4)$) -- ($(2, 4)!0.05cm!(3, 1)$) node[midway, right] {\(\vv{r}\)};
            \draw[<->, very thick, highlight, text=black] ($(2, 4)!0.05cm!(0, 0)$) -- (0, 0) node[pos=0.5, left] {\(\vv{r_1}\)} -- ($(3, 1)!0.05cm!(0, 0)$) node[pos=0.5, below] {\(\vv{r_2}\)};
            \node[left] at (0, 0) {\(O\)};
            \draw[->, very thick, highlight, text=black] (0, 0) -- ($(3, 1)!0.3!(2, 4)$) node[midway, above] {\(\vv{R}\)};
        \end{tikzpicture}
        \caption{Two particles in the centre of mass frame. Drawn as if \(m_2 > m_1\) but this needn't be the case.}
        \label{fig:centre of mass coord}
    \end{figure}
    
    Let \(M = m_1 + m_2\) be the total mass of the system.
    Then the centre of mass lies on a line between the two particles a fraction of \(m_2/M\) along from particle 1, or \(m_1/M\) along from particle 2.
    Using this, and the fact that \(\vv{r}\) points along this line, from particle 2 to 1, we have
    \begin{equation}
        \vv{r_1} = \vv{R} + \frac{m_2}{M}\vv{r}, \qqand \vv{r_2} = \vv{R} - \frac{m_1}{M}\vv{r}.
    \end{equation}
    We also define the \defineindex{reduced mass} of the system to be
    \begin{equation}
        \mu \coloneqq \frac{m_1m_2}{m_1 + m_2}.
    \end{equation}
    
    Using these we can work out \cref{eqn:two body eom 1,eqn:two body eom 2} in centre of mass coordinates in the absence of external forces.
    Substituting in the centre of mass and relative position vectors we get
    \begin{align}
        \vv{f} = m_1\ddot{\vv{r}}_{\vv{1}} &= m_1\left( \ddot{\vv{R}} + \frac{m_2}{M}\ddot{\vv{r}} \right) = m_1\ddot{\vv{R}} + \mu\ddot{\vv{r}},\\
        -\vv{f} = m_2\ddot{\vv{r}}_{\vv{2}} &= m_2\left( \ddot{\vv{R}} - \frac{m_1}{M}\ddot{\vv{r}} \right) = m_2\ddot{\vv{R}} + \mu\ddot{\vv{r}}.
    \end{align}
    Adding these together we get
    \begin{equation}
        \vv{0} = m_1\ddot{\vv{R}} + \mu\ddot{\vv{r}} + m_2\ddot{\vv{R}} - \mu\ddot{\vv{r}} = M\ddot{\vv{R}}.
    \end{equation}
    Hence \(\ddot{\vv{R}} = \vv{0}\), which means that the centre of mass moves at a constant velocity, \(\dot{\vv{R}}\).
    Subtracting the second from the first with \(\ddot{\vv{R}} = \vv{0}\) we have
    \begin{equation}
        \vv{f} = \mu\ddot{\vv{r}}.
    \end{equation}
    
    These two equations are very useful.
    The first describes the motion of the centre of mass, and is independent of any interactions between the particles.
    The second describes the relative motion of the particles, and is independent of the motion of the centre of mass.
    
    \section{Scattering}
    \subsection{Centre of Mass and Lab Frames}
    There are two forms of scattering experiment that physicists normally carry out.
    The first has a fixed target hit by a beam of particles.
    The second has two beams collide.
    To make the mathematical treatment of both cases the same we can transform between frames.
    
    We typically consider the two following frames:
    \begin{itemize}
        \item The \define{\LAB{} frame}\index{LAB frame@\LAB{} frame}, in which we take particle two to be at rest acting as a target before the experiment.
        \item The \defineindex{centre of mass frame}\index{CM frame@\CM{} frame|see{centre of mass frame}} (\CM)\glossary[acronym]{CM}{centre of mass}, in which the centre of mass is at rest and at the origin, that is \(\vv{R}^* = \vv{0}\).
    \end{itemize}
    
    \begin{ntn}{\CM{} Quantities}{}
        We denote quantities in the \CM{} frame by an asterisk, \(^*\).
        So, for example \(\vv{r_1}^*\) is the position of particle 1 in the \CM{} frame.
    \end{ntn}
    
    The \CM{} frame is a special case of \define{centre of momentum frames}\index{centre of momentum frame}, also known as \define{zero momentum frames}\index{zero momentum frame} which are frames where the total momentum is zero.
    In these frames the centre of mass is still stationary, but it may not be at the origin.
    
    We will assume there are no external forces present so both the \LAB{} and \CM{} frames are inertial.
    In the \CM{} frame the position vectors are
    \begin{equation}
        \vv{r_1}^* = \vv{r_1} - \vv{R} = \frac{m_2}{M}\vv{r} = \frac{\mu}{m_1}\vv{r}, \qqand \vv{r_2}^* = \vv{r_2} - \vv{R} = -\frac{m_1}{M}\vv{r} = -\frac{\mu}{m_2}\vv{r}.
    \end{equation}
    The separation, \(\vv{r}\), and relative velocity, \(\dot{\vv{r}}\), are Galilean invariant as they are always measured between the two particles and so don't depend on the frame.
    Hence
    \begin{equation}
        \vv{r}^* = \vv{r_1}^* - \vv{r_2}^* = \vv{r_1} - \vv{r_2} = \vv{r}.
    \end{equation}
    In the \CM{} frame the momenta of the two particles are equal and opposite, that is
    \begin{equation}
        m_1\dot{\vv{r}}_{\vv{1}}^* = -m_2\dot{\vv{r}}_{\vv{2}}^* = \mu\dot{\vv{r}} \eqqcolon \vv{p}^*.
    \end{equation}
    This shows that the total momentum in the \CM{} frame is zero, and hence it is also a centre of momentum frame.
    
    In any other frame, including, but not limited to, the \LAB{} frame, the centre of mass moves with velocity \(\dot{\vv{R}}\) and the particles have velocities
    \begin{equation}
        \dot{\vv{r}}_{\vv{1}} = \dot{\vv{R}} + \dot{\vv{r}}_{\vv{1}}^*, \qqand \dot{\vv{r}}_{\vv{2}} = \dot{\vv{R}} + \dot{\vv{r}}_{\vv{2}}^*.
    \end{equation}
    The momenta of the particles will be
    \begin{equation}
        \vv{p_1} = m_1\dot{\vv{r}}_{\vv{r}} = m_1\dot{\vv{R}} + \vv{p}^*, \qqand \vv{p_2} = m_2\dot{\vv{r}}_{\vv{2}} = m_2\dot{\vv{R}} - \vv{p}^*
    \end{equation}
    
    In the \LAB{} frame where particle 2 is initially at rest and hence \(\vv{p_2} = \vv{0}\) the centre of mass moves with velocity \(\vv{R} = \vv{p}^*/m_2\).
    Therefore the \LAB{} frame in the \CM{} frame moves in the opposite direction with velocity \(\dot{\vv{r}}_{\vv{2}}^* = -\vv{p}^*/m_2\).
    
    Since both the \LAB{} and \CM{} frames are inertial we can move between them with Galilean transformations.
    
    \section{Elastic Collisions}
    In the absence of external forces the total momentum in any inertial frame must be conserved.
    We make a further simplification and assume \define{elastic collisions}\index{elastic collision} in which the total kinetic energy in an inertial frame is conserved.
    In reality kinetic energy is lost in most collisions, usually as sound or heat, but it can still be a reasonable approximation, for interactions involving small, potentially subatomic, particles collisions are far more likely to be elastic, simply because there are fewer ways for them to lose kinetic energy.
    
    There are typically two ways in which scattering collisions take place.
    We can either collide two beams of particles or one beam of particles and a target.
    These are equivalent through a change of frame.
    
    We start by considering an elastic collision in the \LAB{} frame.
    In this frame one particle is stationary, we refer to this as the target particle, and another particle comes in and collides with it.
    We denote the momentum of the incoming particle in the \LAB{} frame by \(\vv{p_1}\).
    The momentum of the target particle is zero.
    After the collision we denote the momenta of the incoming and target particle by \(\vv{q_1}\) and \(\vv{q_2}\) respectively.
    We denote the angle through which the incoming and target particles are scattered by \(\vartheta\) and \(\psi\) respectively.
    Note that when we speak of \enquote{before} and \enquote{after} what we really mean is sufficiently far away that the interaction between the two particles is negligible.
    See \cref{fig:LAB scattering}.
    
    \begin{figure}
        \tikzsetnextfilename{LAB-scattering}
        \begin{tikzpicture}
            \coordinate (particle 2) at (0, 0);
            \coordinate (q1) at (3, 2);
            \coordinate (q2) at (2, -3);
            \coordinate (A) at (1, 0);
            \draw pic [draw, "\(\vartheta\)", angle radius=0.85cm] {angle=A--particle 2--q1};
            \draw pic [draw, double, "\(\psi\)", angle radius=0.7cm] {angle=q2--particle 2--A};
            \draw[very thick, ->] (-3, 0) -- (-0.1, 0) node [midway, above] {\(\vv{p_1}\)};
            \draw[very thick, ->] (particle 2) -- (q1) node [midway, above left] {\(\vv{q_1}\)};
            \draw[very thick, ->] (particle 2) -- (q2) node [midway, below left] {\(\vv{q_2}\)};
            \draw[lightgray] (particle 2) -- ++ (1, 0);
            \fill[highlight] (-3, 0) circle [radius = 0.1cm];
            \fill[highlight] (particle 2) circle [radius = 0.1cm];
        \end{tikzpicture}
        \caption{Elastic collision of particles in the \protect\LAB{} frame.}
        \label{fig:LAB scattering}
    \end{figure}
    
    In the \CM{} frame on the other hand both particles have equal and opposite momenta before the collision, that is the momentum of particle one is \(\vv{p}^*\) and the momentum of particle two is \(-\vv{p}^*\).
    Conservation of momentum means that the net momentum must be zero after the collision also and so the momenta must be equal and opposite afterwards, we'll denote the momentum of particle 1 after the collision by \(\vv{q}^*\).
    See \cref{fig:CM scattering}.
    
    \begin{figure}
        \tikzsetnextfilename{CM-scattering}
        \begin{tikzpicture}
            \draw[very thick, ->] (-3, 0) -- (0, 0) node[midway, above] {\(\vv{p}^*\)};
            \draw[very thick, ->] (3, 0) coordinate (particle 1) -- (0, 0) coordinate (O) node[midway, above] {\(-\vv{p}\)};
            \draw[very thick, ->] (0, 0) -- (2, 3) coordinate (particle 2) node[midway, above left] {\(\vv{q}^*\)};
            \draw[very thick, ->] (0, 0) -- (-2, -3) node[midway, above left] {\(-\vv{q}^*\)};
            \draw pic [draw, "\(\vartheta^*\)", angle radius=0.75cm] {angle=particle 1--O--particle 2};
            \fill[highlight] (-3, 0) circle [radius = 0.1cm];
            \fill[highlight] (3, 0) circle [radius = 0.1cm];
        \end{tikzpicture}
        \caption{Elastic collision of particles in the \protect\CM{} frame.}
        \label{fig:CM scattering}
    \end{figure}
    
    We can transform between the two frames with Galilean transformations.
    Before the collision we have
    \begin{gather}
        \vv{p_1} = m_1\dot{\vv{r}}_{\vv{1}} = m_1(\dot{\vv{r}}_{\vv{1}}^* + \dot{\vv{R}}) = \hphantom{-} \vv{p}^* + m_1\dot{\vv{R}},\\
        \vv{p_2} = m_2\dot{\vv{r}}_{\vv{2}} = m_2(\dot{\vv{r}}_{\vv{2}}^* + \dot{\vv{R}}) = - \vv{p}^* + m_2\dot{\vv{R}}.
    \end{gather}
    Similarly after the collision we have
    \begin{gather}
        \vv{q_1} = \hphantom{-}\vv{q}^* + m_1\vv{R},\\
        \vv{q_2} = -\vv{q}^* + m_2\dot{\vv{R}}.
    \end{gather}
    However, \(\vv{q_2} = \vv{0}\) in the \LAB{} frame and so \(\vv{p}^* = m_2\dot{\vv{R}}\).
    Therefore
    \begin{align}
        \vv{p_1} &= \vv{p}^*\left( 1 + \frac{m_1}{m_2} \right),\\
        \vv{p_2} &= \vv{q}^* + \vv{p}^* \frac{m_1}{m_2},\\
        \vv{q_2} &= \vv{p}^* - \vv{q}^*.
    \end{align}
    
    Conservation of momentum also gives us \(\vv{p_1} = \vv{q_1} + \vv{q_2}\).
    We can combine these relations into a single vector diagram, see \cref{fig:scattering vector diagram}.
    We can use this diagram to find various relations between quantities.
    For example we can show that
    \begin{equation}
        2\psi = \pi - \vartheta^*, \qqand \tan\vartheta = \frac{\sin\vartheta^*}{\cos\vartheta^* + m_1/m_2}.
    \end{equation}
    This last relation in the special case of equal masses gives
    \begin{equation}
        \tan\vartheta = \frac{\sin\vartheta^*}{\cos\vartheta^* + 1} = \tan\left( \frac{\vartheta^*}{2} \right) \implies \vartheta = \frac{\vartheta^*}{2}.
    \end{equation}
    From this it follows that \(\psi + \vartheta = \pi/2\), that is the paths of scattered particles of equal mass are at right angles.
    
    \begin{figure}
        \tikzsetnextfilename{scattering-vector-diagram}
        \begin{tikzpicture}
            \coordinate (O) at (0, 0);
            \coordinate (top) at (3, 3);
            \coordinate (right) at (5, 0);
            \coordinate (mid) at (2, 0);
            \draw pic [draw, "\(\vartheta\)", angle radius=0.75cm] {angle=mid--O--top};
            \draw pic [draw, "\(\vartheta^*\)", angle radius=0.75cm] {angle=right--mid--top};
            \draw pic [draw, "\(\psi\)", angle radius=0.75cm] {angle=top--right--mid};
            \draw[very thick, highlight, ->] (0, -0.5) -- (5, -0.5) node [midway, below] {\(\vv{p_1}\)};
            \draw[very thick, tetrad green, ->] (O) -- (mid) node[pos=0.7, above] {\(\frac{m_1}{m_2}\vv{p}^*\)};
            \draw[very thick, tetrad blue, ->] (mid) -- (right) node[midway, above] {\(\vv{p}^*\)};
            \draw[very thick, darker, ->] (O) -- (top) node[midway, left] {\(\vv{q_1}\)};
            \draw[very thick, colder, ->] (top) -- (right) node[midway, right] {\(\vv{q_2}\)};
            \draw[very thick, tetrad purple, ->] (mid) -- (top) node[midway, left] {\(\vv{q}^*\)};
        \end{tikzpicture}
        \caption{A vector diagram showing the relationships between the various momenta and angles in the \protect\LAB{} and \protect\CM{} frames.}
        \label{fig:scattering vector diagram}
    \end{figure}
    
    \chapter{Particle Scattering}
    \section{Cross Sections}
    Cross sections allow us to explore interactions between particles.
    We can think of them as a generalisation of the cross section of macroscopic objects, the shape and position of which tells us whether the two objects will collide.
    The difference is that particles don't really collide, they interact by various mechanisms, all of which we can take into account with these generalised cross sections.
    
    \subsection{Differential Cross Sections}
    For simplicity we will consider a stationary target and an incoming beam of particles, that is we will work in the \LAB{} frame.
    We characterise the intensity of the incoming beam with the \defineindex{incident flux}, \(f\), which is defined as the number of particles crossing unit area normal to the beam direction per unit time.
    
    The \defineindex{impact parameter}, \(b\), is the closest two particles would come if there is no interaction between them.
    See \cref{fig:scattering impact parameter}.
    We take the \(z\)-axis to be the beam direction and the \(x\) and \(y\)-axes to be perpendicular to the beam.
    The scattering angle, \(\vartheta\), is the polar angle along which the deflected beam travels after it is sufficiently far from the target such that the interaction between the two is negligible.
    Due to the symmetry about the \(z\) axis the incident particles move in a plane of constant azimuthal angle, \(\varphi\).
    
    \begin{figure}
        \tikzsetnextfilename{scattering}
        \begin{tikzpicture}
            \draw[lightgray] (0, 0) -- (6, 0);
            \draw[lightgray] (0, -1) -- (4, -1);
            \draw[lightgray] (4, 0) -- (6, 2);
            \draw[<->, lightgray] (2, 0) -- (2, -1) node[midway, left, gray] {\(b\)};
            \draw[colder, ultra thick, ->] (0, 0) .. controls (4, 0) .. (6, 2);
            \path (6, 2) coordinate (C) -- (4, 0) coordinate (B) -- (6, 0) coordinate (A) pic [draw, "\(\vartheta\)", angle radius=0.7cm, lightgray, text=gray] {angle};
            \fill[highlight] (0, 0) circle [radius = 0.2cm];
            \fill[highlight] (4, -1) circle [radius = 0.2cm];
        \end{tikzpicture}
        \caption{Scattering process showing the impact parameter, \(b\), and deflection angle, \(\vartheta\).}
        \label{fig:scattering impact parameter}
    \end{figure}
    
    In a scattering experiment we have detectors of a finite size.
    We therefore need to consider the number of particles scattered into an area.
    We consider the area that has polar angle between azimuthal angle between \(\varphi\) and \(\varphi + \dl{\varphi}\).
    If we change \(b\) slightly, say by \(\dl{b}\), then this will change the scattering angle, \(\vartheta\), by a small amount, \(\dl{\vartheta}\).
    We can interpret these small intervals of polar coordinates as defining an area, \(\dl{\sigma}\).
    See \cref{fig:d sigma}.
    Changing \(b\) also changes the \defineindex{scattering rate}, \(w\), by some small amount, \(\dl{w}\), given by
    \begin{equation}
        \dl{w} = f\dd{\sigma} = fb\dd{\varphi}\abs{\dl{b}}.
    \end{equation}
    Note that we take the absolute value of \(\dl{b}\) since it is possible that \(b\) may decrease.
    From this we can see that the scattering rate depends on the incident flux, \(f\), and the angular size of the detector, \(\dl{\Omega}\).
    We normalise this out by dividing by \(f\) and \(\dl{\Omega}\).
    
    \begin{figure}
        \tikzsetnextfilename{d-sigma-def}
        \begin{tikzpicture}
            \fill[highlight!20] (110:3) -- (110:2.5) arc(110:70:2.5) -- (70:3) arc(70:110:3);
            \node at (0, 2.75) {\(\dl{\sigma}\)};
            \draw (110:3) -- (0, 0) -- (70:3);
            \draw (110:3) arc(110:70:3);
            \draw (110:2.5) arc(110:70:2.5);
            \draw[shift=(-20:0.2), |<->|] (0, 0) -- (70:2.5) node[midway, right] {\(b\)};
            \draw[|<->|] (110:3.2) arc(110:70:3.2) node[midway, above] {\(b\dd{\varphi}\)};
            \draw[|<->|] (113:3) -- (113:2.5) node[midway, left] {\(\dl{b}\)};
            \path (110:3) coordinate (C) -- (0, 0) coordinate (B) -- (70:3) coordinate (A) pic [draw, "\(\dl{\varphi}\)", angle radius = 1.2cm] {angle};
        \end{tikzpicture}
        \caption{The cross section, \(\dl{\sigma}\). Seen as if looking along the beam direction.}
        \label{fig:d sigma}
    \end{figure}
    
    The angular size, \(\dl{\Omega}\), is a \defineindex{solid angle}.
    The solid angle subtended at the origin by an element of area \(\dl{A}\) at distance \(L\) is given by \(\dl{\Omega} = \dl{A}/L^2\).
    In spherical coordinates we have
    \begin{equation}
        \dl{A} = (L\sin\vartheta \dd{\vartheta})(L\dd{\varphi})
    \end{equation}
    which gives
    \begin{equation}
        \dl{\Omega} = \sin\vartheta\dd{\vartheta}\dd{\varphi}.
    \end{equation}
    Solid angles are dimensionless, they are measured in steradians, which are to spheres as radians are to circles.
    The total solid angle subtended by an entire sphere is
    \begin{equation}
        \int_{S^2} \int_0^{\pi} \sin\vartheta\dd{\vartheta} \int_0^{2\pi} \dl{\varphi} = 4\pi.
    \end{equation}
    Here \(S^2\) is the unit sphere.
    
    It follows that the number of particles scattered in the direction defined by \(\vartheta\) and \(\varphi\) per second, per unit flux, per unit angular size is
    \begin{equation}
        \frac{\dl{w}}{f\dd{\Omega}} = \diff{\sigma}{\Omega} = \frac{b}{\sin\vartheta}\abs*{\diff{b}{\vartheta}}.
    \end{equation}
    We call the quantity \(\diff{\sigma}/{\Omega}\) the \defineindex{differential cross section}.
    It depends only on the relationship between the impact parameter, \(b\), and the scattering angle, \(\vartheta\).
    This in turn is determined by the nature of the interaction between the particles.
    
    \subsection{Total Cross Section}
    Integrating over all scattering directions we get
    \begin{equation}
        \sigma \coloneqq \int \diff{\sigma}{\Omega}\dd{\Omega}.
    \end{equation}
    This is the \defineindex{cross section} for the region over which we integrate.
    If we integrate over the entire sphere then this is the \defineindex{total cross section}.
    
    \section{Conservation Laws}
    So far we have considered the interaction in the \LAB{} frame, in which the target particle remains stationary throughout.
    If we want to consider the motion in the \CM{} frame then we need to know how the target particle moves after the interaction.
    The exact details of course depend on the interaction, but there are some general principles we can use.
    We can use conservation of angular momentum and kinetic energy conservation.
    Note that kinetic energy is conserved only when there is no interaction, that is at large distances.
    When the particles are close some of the kinetic energy will be transferred to potential energy.
    This is simply energy conservation when potential energy is zero.
    
    In general the particles in the beam will change momentum and this must be balanced by the recoil of the target particle.
    In the limit of a significantly more massive target particle this recoil velocity is small.
    The kinetic energy goes with the square of the velocity and therefore the kinetic energy taken away by the target is small and we consider the collision to be elastic.
    
    Inelastic collisions are also possible.
    In these kinetic energy is lost.
    After the probe and target particles are sufficiently separated there can be no energy stored as potential energy.
    So where has this kinetic energy gone?
    The answer is that the particles must be changed in some way.
    Possibly by excitation into a more energetic state or, in more extreme cases, the particles may break up after the interaction.
    We will see later that \emph{all} inelastic collisions result in violation of conservation of mass.
    
    \section{Hard Sphere Scattering in the \protect\CM{} Frame}
    Consider two hard, smooth, spheres of radius \(R\), such as billiard balls, in the \CM{} frame.
    By \enquote{hard} here we mean that the surface cannot be deformed and therefore no kinetic energy is lost to deforming the spheres.
    By \enquote{smooth} we mean there is no friction and so kinetic energy is not lost due to friction.
    These conditions mean we can consider elastic collisions.
    
    In order to find the differential cross section in the \CM{} frame, \(\diff{\sigma^*}/{\Omega^*}\), we need to find a relationship between \(b\) and \(\vartheta^*\).
    Notice that the spheres collide only if \(b < 2R\).
    See \cref{fig:hard sphere scattering}, from this we see that \(b = 2R\sin\alpha^*\).
    We need to relate \(\alpha^*\) to \(\vartheta^*\) in order to find the cross section.
    To do this we need to consider the nature of the interaction.
    In this simple case we have a force acting perpendicular to the surfaces, along the radial direction, at the point of contact.
    This means that there is no component of force in the tangential direction and hence there is no momentum change along this direction.
    This means that \(p^*\sin\alpha^* = q^*\sin\beta^*\).
    For an elastic collision \(p^* = q^*\) and hence \(\alpha^* = \beta^*\).
    It then follows that \(2\alpha^* + \vartheta^* = \pi\) and hence
    \begin{equation}
        b = 2R\sin\left( \frac{\pi - \vartheta^*}{2} \right) = 2R\cos\frac{\vartheta^*}{2}.
    \end{equation}
    Hence
    \begin{equation}
        \diff{\sigma^*}{\Omega^*} = \frac{b}{\sin\vartheta^*} \abs*{\diff{b}{\vartheta^*}} = R^2.
    \end{equation}
    
    \begin{figure}
        \tikzsetnextfilename{hard-sphere-scattering}
        \begin{tikzpicture}
            \draw[highlight, fill=highlight!20, very thick] (0, 0) circle [radius=1cm];
            \draw[highlight, fill=highlight!20, very thick] (-60:2) circle [radius=1cm];
            \path (1, 0) coordinate (A) -- (0, 0) coordinate (B) -- (2, 3) coordinate (C) pic [draw, "\(\vartheta^*\)", angle radius=0.7cm] {angle};
            \coordinate (D) at (120:3);
            \path pic[draw, "\(\beta^*\)", angle radius=0.8cm] {angle=C--B--D};
            \coordinate (E) at (-1, 0);
            \path pic[draw, "\(\alpha^*\)", angle radius=0.7cm] {angle=D--B--E};
            \draw[very thick, ->] (-3, 0) -- (0, 0) node[above, pos=0.3] {\(\vv{p}^*\)};
            \draw[very thick, ->] (0, 0) -- (2, 3) node[pos=0.7, left] {\(\vv{q}^*\)};
            \draw[very thick, <-] (-60:2) -- ++ (3, 0) node[pos=0.7, below] {\(-\vv{p}^*\)};
            \draw[very thick, ->] (-60:2) -- ++ (-2, -3) node[pos=0.7, right] {\(-\vv{q}^*\)};
            \draw[lightgray, very thick] (-60:1) -- ++ (30:4);
            \draw[lightgray, very thick] (-60:1) -- ++ (30:-4);
            \draw[lightgray, very thick] (-60:1) -- ++ ($(-60:1)!5!(0, 0)$);
            \draw[lightgray, very thick] (-60:1) -- ++ ($(-60:1)!3!(-60:2)$);
            \coordinate (tangential) at (3.2, 0.95);
            \node[gray, rotate around={30:(tangential)}] at (tangential) {tangential};
            \coordinate (radial) at (-1.1, 2.3);
            \node[gray, rotate around={-60:(radial)}] at (radial) {radial};
            \draw[lightgray, thick] (-60:2) -- ++ (-3, 0);
            \draw[lightgray, thick, <->] (-1.8, 0) -- ++ (0, -1.75) node[left, midway, gray] {\(b\)};
            \draw[lightgray, thick] (0, 0) -- (1, 0);
        \end{tikzpicture}
        \caption{Two hard spheres colliding.}
        \label{fig:hard sphere scattering}
    \end{figure}
    
    The differential cross section is independent of the scattering angle in this case and therefore we call it \defineindex{isotropic}, meaning the same in all directions.
    The total cross section is
    \begin{align}
        \sigma^* &= \int_{S^2} \diff{\sigma^*}{\Omega^*}\dd{\Omega^*} = R^2 \int_{S^2}\dl{\Omega^2} = 4\pi R^2.
    \end{align}
    Notice that this is the cross sectional area of the cylinder of radius \(2R\) in which the centres of both spheres must be in order for a collision to occur.
    
    \section{Rutherford Scattering}
    One of the first scattering experiments ever was Rutherford scattering, in which alpha particles where fired at a thin gold foil.
    The alpha particles scatter off of the nucleus through the electromagnetic interaction.
    Suppose the incident alpha particle starts with speed \(v\) far from the scattering centre.
    It's energy at this point is purely kinetic, and given by \(E = mv^2/2\).
    The magnitude of the angular momentum about the target nucleus is \(L = mvb\).
    
    Since electrostatic energy decreases with distance and the nucleus is much more massive than the alpha particle conservation of energy means that the initial and final speeds of the alpha particle must be the same.
    
    The direction of the particle changes, so its momentum changes.
    The change in momentum is
    \begin{equation}
        \Delta p = 2mv\sin\frac{\vartheta}{2}.
    \end{equation}
    See \cref{fig:rutherford scattering} for the direction of \(\vv{\Delta p}\).
    The change in momentum is also given by the net impulse, which is
    \begin{equation}
        \Delta p = \int_{-\infty}^{\infty} F\cos\Phi \dd{t} = \int_{-\infty}^{\infty} \frac{qQ}{4\pi\varepsilon_0r^2}\cos\Phi \dd{t}.
    \end{equation}
    Here \(q\) and \(Q\) are the charge of the alpha particle and nucleus, \(r\) is the distance between them and \(\Phi\) is the polar angle in plane polar coordinates with the \(x\)-axis in the direction of the beam and the origin at the nucleus.
    
    \begin{figure}
        \tikzsetnextfilename{rutherford-scattering}
        \begin{tikzpicture}
            \draw[lightgray] (0, 0) -- (6, 0);
            \draw[lightgray] (0, -1) -- (4, -1);
            \draw[lightgray] (4, 0) -- (6, 2);
            \draw[<->, lightgray] (2, 0) -- (2, -1) node[midway, left, gray] {\(b\)};
            \draw[lightgray, thick, ->] (4, -1) -- ++ (-2, 4) node[left, gray] {\(\vv{\Delta p}\)};
            \draw[lightgray, thick, ->] (4, -1) -- ++ (0, 1.3) node[pos=0.4, gray, right] {\(\vv{r}\)};
            \path (2, 3) coordinate (C) -- (4, -1) coordinate (B) -- (4, 0) coordinate (A) pic [draw, "\(\Phi\)", angle radius=0.8cm, angle eccentricity=0.8, lightgray, text=gray] {angle};
            \draw[colder, ultra thick, ->] (0, 0) .. controls (4, 0) .. (6, 2);
            \path (6, 2) coordinate (C) -- (4, 0) coordinate (B) -- (6, 0) coordinate (A) pic [draw, "\(\vartheta\)", angle radius=0.7cm, lightgray, text=gray] {angle};
            \fill[highlight] (0, 0) circle [radius = 0.2cm];
            \fill[highlight] (4, -1) circle [radius = 0.2cm];
        \end{tikzpicture}
        \caption{Rutherford scattering.}
        \label{fig:rutherford scattering}
    \end{figure}
    
    Since the alpha particle is in a central potential its angular momentum is conserved and hence
    \begin{equation}
        L = mvb = mr^2\diff{\Phi}{t} \implies \diff{\Phi}{t} = \frac{vb}{r^2} \implies \dl{t} = \dl{\Phi}\frac{r^2}{vb}.
    \end{equation}
    
    Combining this result with the change in momentum, and noticing that \(2\Phi_{\max} + \vartheta = \pi\), where \(\Phi\) is the largest polar angle achieved, we have
    \begin{equation}
        2mv\sin\frac{\vartheta}{2} = \int_{-\Phi_{\max}}^{\Phi_{\max}} \frac{qQ}{4\pi\varepsilon_0r^2}\cos\Phi \frac{r^2}{vb} \dd{\Phi} = \frac{qQ}{4\pi\varepsilon_0vb}[\sin\Phi]_{-\Phi_{\max}}^{\Phi_{\max}} = \frac{2qQ}{4\pi\varepsilon_0vb}\cos\frac{\vartheta}{2}.
    \end{equation}
    From this we have
    \begin{equation}
        b = \frac{qQ}{4\pi\varepsilon_0mv^2}\cot\frac{\vartheta}{2}.
    \end{equation}
    
    We can write this result as \(b = a\cot(\vartheta/2)\).
    It turns out that \(a\) is related to \(r_c\), the minimum separation of incident and target particles, by
    \begin{equation}
        r_c = a + \sqrt{a^2 + b^2}.
    \end{equation}
    It then follows that
    \begin{equation}
        \diff{\sigma}{\Omega} = \frac{b}{\sin\vartheta} \abs*{\diff{b}{\vartheta}} = \frac{a^2}{4} \frac{1}{\sin^4(\vartheta/2)}.
    \end{equation}
    This is strongly dependent on the energy of the incoming particle and the scattering angle.
    It also goes to infinity as \(\vartheta \to 0\).
    The total cross section is also infinite.
    This is a consequence of the infinite range of the Coulomb force.
    Even particles with very large impact parameters are scattered, albeit through very small angles.
    
    \subsection{History}
    In 1911 Ernest Rutherford, and his students, Hans Geiger and Ernest Marsden carried out a series of scattering experiments with alpha particles and gold foil.
    They confirmed the predictions above of the cross section.
    They found that even when the alpha particles where high enough energy that one would expect \(r_c \le \qty{e-14}{\metre}\) they still followed this prediction.
    From this he concluded that the nucleus must be smaller than \qty{e-14}{\metre}.
    
    Subsequent experiments showed deviation from these predictions at higher energies, suggesting that the alpha particles where penetrating the nucleus, and hence the interactions differed.
    In the simplest case, which is what Rutherford expected, the nucleus is a charged sphere and hence the charge, and by extension force, inside should be zero.
    It was later shown however that the electromagnetic force alone could not predict the scattering of these higher energy particles.
    This was some of the early evidence for the existence of the strong nuclear force.
    
    \section{Moving Between Inertial Frames}
    Consider \cref{fig:scattering vector diagram}.
    As \(m_1/m_2 \to 0\) we expect that the quantities in the \CM{} frame should approach those in the \LAB{} frame.
    
    In general however the two frames are different.
    Often it is useful to be able to find the cross section in one frame and convert to the other.
    To do so is fairly easy.
    Note that the relative motion of the frames changes the scattering angle, \(\vartheta\), and solid angle, \(\Omega\), but not the azimuthal angle, \(\varphi\), or \(\dl{\sigma}\).
    Therefore we have
    \begin{align}
        \diff{\sigma}{\Omega} &= \diff{\sigma}{\Omega^*}\diff{\Omega^*}{\Omega}\\
        &= \diff{\sigma^*}{\Omega^*}\diff{\Omega^*}{\Omega}\\
        &= \diff{\sigma^*}{\Omega^*} \frac{\sin\vartheta^* \dd{\varphi}\dd{\vartheta^*}}{\sin\vartheta \dd{\varphi} \dd{\vartheta}}\\
        &= \diff{\sigma^*}{\Omega^*} \frac{\sin\vartheta^*}{\sin\vartheta} \diff{\vartheta^*}{\vartheta^*}\\
        &= \diff{\sigma^*}{\Omega^*} \diff{(\cos\vartheta^*)}{(\cos\vartheta^*)}.
    \end{align}
    
