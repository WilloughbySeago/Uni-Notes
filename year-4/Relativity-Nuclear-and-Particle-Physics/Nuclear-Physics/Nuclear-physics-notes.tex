\documentclass[fleqn]{NotesClass}

%% Packages
\usepackage[version=4]{mhchem}
\usepackage{csquotes}

% Tikz stuff
\usepackage{tikz}
\tikzset{>=latex}
% External
\usetikzlibrary{external}
\tikzexternalize[prefix=tikz-external/]
%\tikzexternaldisable
% Tikz Feynman
\usepackage[compat=1.1.0]{tikz-feynman}
\tikzfeynmanset{warn luatex=false}
\usepackage{pgfplots}
\usepackage{pgfplotstable}


% Siunitx
\usepackage{siunitx}
\DeclareSIUnit{\fermi}{\femto\meter}
\DeclareSIUnit{\MeV}{\mega\electronvolt}
\DeclareSIUnit{\clight}{\text{\ensuremath{c}}}
\DeclareSIUnit[per-mode=symbol]{\MeVpercsquared}{\MeV\per\clight\squared}
\DeclareSIUnit[per-mode=symbol]{\MeVperc}{\MeV\per\clight}
\DeclareSIUnit{\bohrmagneton}{\ensuremath{\upmu_{\mathrm{B}}}}
\DeclareSIUnit{\nuclearmagneton}{\ensuremath{\upmu_{\mathrm{N}}}}
\DeclareSIUnit{\amu}{u}

% References, should be last things loaded
\usepackage{hyperref}  % Should be loaded second last (cleveref last)
\colorlet{hyperrefcolor}{blue!60!black}
\hypersetup{colorlinks=true, linkcolor=hyperrefcolor, urlcolor=hyperrefcolor}
\usepackage[
capitalize,
nameinlink,
noabbrev
]{cleveref} % Should be loaded last

% My packages
\usepackage{NotesBoxes}
\usepackage{NotesMaths}
\usepackage{ParticlesPackage}


% Title page info
\title{Relativity, Nuclear, and Particle Physics\\{\Huge---Nuclear Physics}}
\author{Willoughby Seago}
\date{September 20, 2021}
% \subtitle{}
% \subsubtitle{}

% Highlight colour
%\definecolor{highlight}{HTML}{F36619}
\definecolor{darker}{HTML}{933808}
\definecolor{colder}{HTML}{863F86}
\definecolor{tetrad green}{HTML}{A8F31B}
\definecolor{tetrad blue}{HTML}{1BA8F3}
\definecolor{tetrad purple}{HTML}{671BF3}

% Commands
% Particles
\makeatletter
\newcommand{\Pdeuteron}{\ensuremath{\particletypeface{D}}}
\newcommand{\PBASE@pion}{\uppi}
\newcommand{\Ppion}{\ensuremath{\PBASE@pion}}
\newcommand{\Ppionp}{\ensuremath{\Ppion^+}}
\newcommand{\Ppionm}{\ensuremath{\Ppion^-}}
\newcommand{\Ppionzero}{\ensuremath{\Ppion^0}}
\newcommand{\Ppip}{\Ppionp}
\newcommand{\Ppim}{\Ppionm}
\newcommand{\Ppizero}{\Ppionzero}
\newcommand{\PBASE@Delta}{\Updelta}
\newcommand{\PDeltap}{\ensuremath{\PBASE@Delta^+}}
\newcommand{\PDeltazero}{\ensuremath{\PBASE@Delta^0}}
\newcommand{\Pee}{\PBASE@electron}
\makeatother

% Text

% Maths
\newcommand{\e}{\mathrm{e}}

% Include
\includeonly{}

\begin{document}
    \frontmatter
    \titlepage
    \title{Relativity, Nuclear and Particle Physics (Nuclear Physics)}
    \innertitlepage{tikz-external/nucleon-nucleon-interaction.pdf}
    \tableofcontents
    \mainmatter
    
    \chapter{The Big Bang and Virtual Particles}
    \section{The Big Bang}
    13.7 billion years ago the Big Bang occurred.
    About \qty{e-3}{\second} after this the temperature of the universe was \(T \sim \qty{e12}{\kelvin}\).
    At this point the universe was cool enough that protons, \Pp, and neutrons, \Pn, could bind together to form the first nuclei, specifically they formed deuterons, \Pdeuteron, that is proton-neutron pairs which form the nucleus of the deuterium atom, \ce{^2H}.
    This happens in the reaction
    \begin{equation}
        \Pproton + \Pneutron \leftrightharpoons \Pdeuteron + \Pphoton.
    \end{equation}

    There are two things to notice about this.
    First, the photon is needed since the mass of the deuteron is less than the combined mass of the proton and neutron, and this excess energy has to go somewhere.
    Note that it can't go to the kinetic energy of the deuteron since we can always transform to the frame of the deuteron where it has no kinetic energy.
    This reaction is, at this point in time, reversible since there are so many high energy photons flying around that can collide with deuterons and cause them to photodisintegrate into protons and neutrons.
    
    A few seconds after the Big Bang the universe had cooled to \(T \sim \qty{e9}{\kelvin}\).
    At this point there weren't enough high energy photons present to cause the photodisintegration of deuterons and so the reaction is essentially one way:
    \begin{equation}
        \Pproton + \Pneutron \rightarrow \Pdeuteron + \Pphoton.
    \end{equation}
    The energy of most photons at this point is below the \qty{2.22}{\mega\electronvolt} deuteron binding energy.
    
    After deuteron's were produced they underwent further reactions to create the other light nuclei \ce{^3He}, \ce{^4He}, \ce{^6Li}, and \ce{^7Li}.
    These were the only nuclei produced in the Big Bang.
    Originally many scientists thought that all elements were created in the Big Bang.
    Most notably a Russian physicist George Gamow\footnote{Russian, so w is pronounced like v, \textit{gam-ov}}.
    He wrote a paper expressing these ideas with his PhD student Ralph Alpher.
    They decided to also add the name of Gamow's friend, the physicist Hans Bethe\footnote{pronounced \textit{beta}}, simply for the comedic effect of being able to write a paper by Alpher, Bethe and Gamow\footnote{pronounced \textit{alpha}, \textit{beta}, and \textit{gam-ov}}.
    Unfortunately they turned out to be wrong, heavier elements were formed after the big bang in stars and super novae.
    
    The abundance of these lighter elements is some of the earliest evidence that we have for the Big Bang.
    In total Big Bang nucleosynthesis lasted only about 3 minutes.
    It wasn't until approximately 300,000 years after the Big Bang that the universe had cooled enough that electrons could bind to nuclei to form neutral atoms (recall that the binding energy of an electron to a proton, i.e. \ce{^1H}, is only \qty{13.6}{\electronvolt}, much less than the \qty{2.22}{\mega\electronvolt} binding energy of the deuteron).
    At this point the neutral particles in the universe become pretty much transparent to electromagnetic waves and we get the cosmic microwave background radiation.
    
    At this point we ask a question that will motivate much of the following work:
    \begin{important}
        What force binds nuclei together?
    \end{important}
    It can't be the electromagnetic force, protons are positive and neutrons are negative, so at best there is no electromagnetic interaction and at worst they actively repel each other.
    It also can't be gravity, it simply isn't strong enough.
    Both gravity (at least, Newtonian gravity) and the electrostatic force follow inverse square laws, therefore at a given distance the force due to gravity between two nucleons satisfies
    \begin{equation}
        F_Gr^2 = GMm = - (\qty{6.67e-11}{\newton\meter\squared\per\kilogram\squared})(\qty{1.67e-27}{\kilogram})^2 = \qty{1.86e-64}{\newton\meter\squared}
    \end{equation}
    whereas the electrostatic force between two protons satisfies
    \begin{equation}
        F_Er^2 = \frac{e^2}{4\pi\varepsilon_0} = \frac{(\qty{1.6e-19}{\coulomb})^2}{4\pi (\qty{8.85e-12}{\farad\per\meter})} = \qty{2.30e-28}{\newton\meter\squared}.
    \end{equation}
    We can see from this that gravity is a factor of \num{e36} times weaker than the electrostatic force.
    So gravity alone can't be holding together a nucleus of more than one proton.
    
    The answer is that we need a new force, called the \defineindex{strong force} which is strong enough to hold together the nucleus, but doesn't have a noticeable effect on macroscopic scales.
    To understand this force we will need to learn the basic principles of how microscopic particles interact according to quantum mechanics and quantum field theory (QFT)\glossary[acronym]{QFT}{quantum field theory}.
    
    \section{Energy-Time Uncertainty Relation}
    A familiar result from quantum mechanics is that the uncertainty in the energy, \(\Delta E\), of a quantum system is related to the time interval, \(\Delta t\), over which the system changes appreciably:
    \begin{equation}
        \Delta E \Delta t \sim \hbar
    \end{equation}
    Where we use \(\sim\) informally to mean \enquote{is about the same size as}.
    
    A consequence of this is that all unstable states, that is states with a finite lifetime (since decay is certainly an appreciable change) have finite (here meaning nonzero) energy uncertainty.
    We can define the \defineindex{energy width}, \(\Gamma\), which relates this uncertainty to the lifetime, \(\tau\), of the quantum system, by
    \begin{equation}
        \Gamma\tau = \hbar.
    \end{equation}
    When we say \enquote{energy width} what we mean is a measure of the width of the distribution of energy values.
    Since this is usually a Lorentzian distribution by width we mostly mean the full width at half maximum.
    
    A result of this is that only completely stable systems, where \(\tau = \infty\), can have \(\Gamma = \Delta E = 0\).
    
    Another feature of the uncertainty relation is it allows for temporary violation of the conservation of energy over short time periods, \(\Delta t\), energy conservation can be violated by up to \(\Delta E\).
    This allows for effects that would otherwise be impossible.
    
    For example, a laser involves lots of atoms emitting and absorbing photons of the same frequency.
    Since these atoms aren't perfectly still every photon will have some Doppler shift associated with its frequency.
    If atoms could only absorb photons with exactly the same energy that they emit then they would never be able to absorb, instead they can actually absorb within \(\Gamma/2\) of this value, and so can absorb photons from other atoms.
    Since lasers clearly work this is evidence for the uncertainty principle.
    
    Critically while instantaneous energy conservation no longer applies over a sufficiently long period of time energy must still be conserved.
    We can thing of this as \enquote{borrowing} energy from the uncertainty principle and having to pay it back.
    
    \section{Yukawa Exchange Model of Nucleon-Nucleon Interactions}
    In this section we consider nucleon-nucleon interactions.
    Nucleon simply means proton or neutron.
    For simplicity we will focus on \Pn-\Pn{} interactions as this allows us to ignore electromagnetic effects but the same analysis applies to \Pp-\Pn{} and \Pp-\Pp{} interactions.
    
    This model of interaction was developed by Japanese physicist Hideki Yukawa in 1935.
    It posits an \defineindex{exchange particle}, also known as a \define{virtual particle}\index{virtual particle|see{exchange particle}} or \define{field quantum}\index{field quantum|see{exchange particle}}.
    The simplest way to view the interaction is with a Feynman diagram, showing the world lines of the particles involved.
    This is shown for \Pn-\Pn{} in \cref{fig:n-n feynman diagram}.
    
    \begin{figure}
        \tikzsetnextfilename{n-n-interaction}
        \begin{tikzpicture}
            \begin{feynman}
                \vertex (a);
                \vertex[below=of a] (b);
                \vertex (n in 1) at ($(a) + (-2, 0.5)$) {\Pn};
                \vertex (n in 2) at ($(b) + (-2, -0.5)$) {\Pn};
                \vertex (n out 1) at ($(a) + (2, 0.5)$) {\Pn};
                \vertex (n out 2) at ($(b) + (2, -0.5)$) {\Pn};
                \diagram {
                    (n in 1) -- [fermion] (a) -- [fermion] (n out 1);
                    (n in 2) -- [fermion] (b) -- [fermion] (n out 2);
                    (a) -- [scalar, edge label={\scriptsize Virtual Particle}] (b);
                };
            \end{feynman}
        \end{tikzpicture}
        \caption{The Feynman diagram for simple \Pn-\Pn{} interaction exchanging a virtual particle.}
        \label{fig:n-n feynman diagram}
    \end{figure}
    
    This represents two nucleons coming in on the left, labelled as neutrons here but one or more of them could also be a proton.
    They come closer together, interact by exchanging a virtual particle, and then start moving apart.
    
    Considering the energy at the vertices shows that energy conservation must be violated by at least \(\Delta E = mc^2\) where \(m\) is the mass of the exchange particle.
    We can always work in a frame where a given nucleon is at rest and hence has no kinetic energy, and therefore either it emits a particle with at least the energy due to its mass or it absorbs one.
    
    The range, \(R\), of this interaction is related to the time, \(\Delta t\), over which the virtual particle is exchanged, and the maximum speed of information transfer between the particles, the speed of light, \(c\):
    \begin{equation}
        R \coloneqq c\Delta t.
    \end{equation}
    Using the energy uncertainty principle, \(\Delta E \Delta t = \hbar\) this gives us
    \begin{equation}\label{eqn:R = h/mc}
        R = \frac{\hbar c}{\Delta E} = \frac{\hbar c}{mc^2} = \frac{\hbar}{mc}.
    \end{equation}
    From this we see that the higher the mass of the exchange particle the shorter range the interaction.
    
    Experimental data on nucleon-nucleon interactions finds that the strong force is very strongly repulsive at short ranges (\(r < \qty{1}{\femto\meter} = \qty{e-15}{\meter}\)) and attractive for slightly longer distances (\qtyrange{1}{2}{\femto\meter}).
    We can model this by assuming a central potential, like the one shown in \cref{fig:strong force central potential}.
    The force is given by the negative gradient so we see the steep downwards part as repulsive, it then reaches a minimum, then there is an upwards slope which is attractive, and then the potential is pretty much flat, and hence the force is negligible.
    
    \begin{figure}
        \tikzsetnextfilename{strong-force-central-potential}
        \begin{tikzpicture}[scale=1.5]
            \draw[<->, thick] (0, 2.5) node[above] {\(V/\unit{\MeV}\)} -- (0, 0) -- (3, 0) node[right] {\(r/\unit{\fermi}\)};
            \draw[thick] (0, -1.5) -- (0, 0);
            \draw[domain=0.45:3, samples=400, highlight, very thick] plot (\x, {5*((\x+0.5)^-12 - (\x+0.5)^-6)});
            \node at (0.63, 0.12) {\(1\)};
            \node at (1.26, 0.12) {\(2\)};
            \node at (1.89, 0.12) {\(3\)};
            \node at (2.52, 0.12) {\(4\)};
        \end{tikzpicture}
        \caption{The central potential for the strong force. Strongly repulsive for \(r < \qty{1}{\fermi}\) and attractive in the range \qtyrange{1}{2}{\fermi}. Negligible over large distances.}
        \label{fig:strong force central potential}
    \end{figure}
    
    Choosing \(R = \qty{1.5}{\fermi}\) as the midpoint of the attractive region rearranging \cref{eqn:R = h/mc} gives us
    \begin{equation}
        m = \frac{\hbar}{Rc} \approx \qty{132}{\MeVpercsquared}
    \end{equation}
    Considering the whole attractive range we find that \(m\) is between \qtyrange{99}{197}{\MeVpercsquared}.
    
    It turns out that the exchange particle in this interaction is actually a \defineindex{pion}, also known as \index{pi mesons}.
    There are actually three different pions, \Ppip, \Ppim, and \Ppizero, these have charge \(\qty{1}{\elementarycharge}\), \(\qty{-1}{\elementarycharge}\), and 0 respectively.
    Both \Ppip{} and \Ppim{} have mass \qty{139.6}{\MeVpercsquared}, and \Ppizero{} is slightly lighter at \qty{135.0}{\MeVpercsquared}.
    So we conclude that the strong force is associated with the exchange of a single virtual pion.
    
    We cannot detect exchange particles as to do so would necessarily involve absorbing them in some way.
    This would prevent them from returning their energy and result in a permanent violation of the conservation of energy.
    This is not allowed.
    That we cannot possibly detect it is why we call the exchange particle a virtual particle.
    
    The exchange of virtual particles turns out to be the mechanism behind both the strong force and electromagnetic force, as well as the weak force.
    Many hypotheses have also been posited for virtual particle exchange mediating gravity, but none have yet been proven experimentally to the satisfaction of the wider scientific community.
    
    \begin{figure}
        \tikzsetnextfilename{nucleon-nucleon-interaction}
        \begin{tikzpicture}
            \begin{feynman}
                \vertex (a);
                \vertex[below=of a] (b);
                \vertex (n in 1) at ($(a) + (-2, 0.5)$) {\Pn};
                \vertex (n in 2) at ($(b) + (-2, -0.5)$) {\Pn};
                \vertex (n out 1) at ($(a) + (2, 0.5)$) {\Pn};
                \vertex (n out 2) at ($(b) + (2, -0.5)$) {\Pn};
                \diagram {
                    (n in 1) -- [fermion] (a) -- [fermion] (n out 1);
                    (n in 2) -- [fermion] (b) -- [fermion] (n out 2);
                    (a) -- [scalar, edge label=\Ppizero] (b);
                };
            \end{feynman}
            \begin{scope}[xshift=5cm]
                \begin{feynman}
                    \vertex (a);
                    \vertex[below=of a] (b);
                    \vertex (p in 1) at ($(a) + (-2, 0.5)$) {\Pp};
                    \vertex (p in 2) at ($(b) + (-2, -0.5)$) {\Pp};
                    \vertex (p out 1) at ($(a) + (2, 0.5)$) {\Pp};
                    \vertex (p out 2) at ($(b) + (2, -0.5)$) {\Pp};
                    \diagram {
                        (p in 1) -- [fermion] (a) -- [fermion] (p out 1);
                        (p in 2) -- [fermion] (b) -- [fermion] (p out 2);
                        (a) -- [scalar, edge label=\Ppizero] (b);
                    };
                \end{feynman}
            \end{scope}
            \begin{scope}[xshift=2.5cm, yshift=-3cm]
                \begin{feynman}
                    \vertex (a);
                    \vertex[below=of a] (b);
                    \vertex (n in) at ($(a) + (-2, 0.5)$) {\Pn};
                    \vertex (p in) at ($(b) + (-2, -0.5)$) {\Pp};
                    \vertex (n out) at ($(a) + (2, 0.5)$) {\Pn};
                    \vertex (p out) at ($(b) + (2, -0.5)$) {\Pp};
                    \diagram {
                        (n in) -- [fermion] (a) -- [fermion] (n out);
                        (p in) -- [fermion] (b) -- [fermion] (p out);
                        (a) -- [scalar, edge label=\Ppizero] (b);
                    };
                \end{feynman}
            \end{scope}
        \end{tikzpicture}
        \caption{Nucleon-nucleon interactions by exchange of a virtual \Ppizero.}
    \end{figure}
    
    \subsection{Yukawa Potential}
    Yukawa showed that his single pion exchange mechanism is equivalent to an attractive potential given by
    \begin{equation}
        V(r) = -V_0 \frac{\e^{-r/R}}{r}
    \end{equation}
    The form of this potential can be derived by considering the relativistic formula
    \begin{equation}
        E^2 = m^2c^4 + p^2c^2.
    \end{equation}
    We can make this into a wave equation by substituting for the operators \(E \to -i\hbar\diffp*{}/{t}\) and \(\vv{p}\to -i\hbar\grad\), and including a wave function, \(\varphi\):
    \begin{equation}
        -\hbar\diffp[2]{\varphi}{t} = (m^2c^4 - \hbar^2 c^2\laplacian)\varphi.
    \end{equation}
    These interactions can occur at any time and so the solution to this must be time independent.
    Therefore it reduces to
    \begin{equation}
        \left( \laplacian - \frac{m^2c^2}{\hbar^2} \right)\varphi = 0.
    \end{equation}
    We also assume spherical symmetry, so \(\varphi = \varphi(r)\), in which case
    \begin{equation}
        \laplacian \varphi = \frac{1}{r}\diffp*[2]{(r\varphi)}{r}
    \end{equation}
    and so the equation reduces further to
    \begin{equation}
        \diffp*[2]{(r\varphi)}{r} = \frac{m^2c^2}{\hbar^2}r\varphi.
    \end{equation}
    The full solution to this is
    \begin{equation}
        \varphi(r) = \frac{A \e^{-\frac{cmr}{\hbar}}}{r}+\frac{B\hbar  \e^{\frac{cmr}{\hbar}}}{2cmr}
    \end{equation}
    for some constants \(A\) and \(B\).
    We discard the second term since it doesn't vanish at infinity and we end up with
    \begin{equation}
        \varphi(r) = A\frac{\e^{-\frac{cmr}{\hbar}}}{r} = -V_0\frac{e^{-r/R}}{r},
    \end{equation}
    where, for an attractive potential, \(V_0 = -A > 0\).
    
    Electromagnetic interactions have infinite range.
    From this we can conclude that the exchange particle for them has zero mass.
    Indeed, this is the case as the exchange particle for electromagnetic interactions is the photon.
    In this case the Yukawa potential reduces to
    \begin{equation}
        V(r) = -\frac{V_0}{r},
    \end{equation}
    which leads us to Coulomb's law when \(V_0 = q/(4\pi\varepsilon_0)\).
    
    It turns out that exchange particles are bosons, meaning they have integer spin, whereas the interacting particles are fermions, so have half integer spin (in units where \(\hbar = 1\)).
    In particular we are considering nucleons which have spin \(\pm 1/2\).
    In a system of two nucleons the net spin can either be \(1\) or \(0\).
    Hence the spin can only change by an integer amount (\(\pm 1\)) and so the exchange particles can only carry integer spin.
    
    \chapter{The Structure of Nucleons}
    \section{Evidence for a Substructure}
    \subsection{\texorpdfstring{\PDeltap{} Particle}{Delta Plus Particle}}
    The delta particle, \(\PDeltap\), is a highly unstable fermion with spin \(S = 3\hbar/2\).
    It can be produced by bombarding a proton with high energy gamma-rays, of energy \(\sim \qty{300}{\MeV}\), in the reaction
    \begin{equation}
        \Pproton + \Pphoton \longrightarrow \PDeltap
    \end{equation}

    When bombarding protons with photons of a range of energies it was found that the energy distribution of photons absorbed had a width o approximately \(\Gamma = \qty{200}{\MeV}\).
    Hence the lifetime of \PDeltap{} is approximately \(\tau = \hbar/\Gamma \approx \qty{e-24}{\second}\).
    This is far too short to allow us to directly detect \PDeltap{} particles.
    Instead we detect the result of their decays.
    There are two possible decay processes:
    \begin{align}
        \PDeltap &\longrightarrow \Pproton + \Ppionzero,\\
        \PDeltap &\longrightarrow \Pneutron + \Ppionp.
    \end{align}
    
    Noticing that the \PDeltap{} is simply the result of bombarding a proton with a photon it is reasonable to declare it to be an excited state of the proton.
    It only achieves the status of being a different particle as it was discovered and named before someone made this connection.
    
    However, this raises a question.
    Electrons don't, alone, have excited states.
    It is only possible to excite an electron in a bound state with some other particle.
    This is because electrons are fundamental.
    The fact that we can excite a proton in isolation then suggests that it is not fundamental and is instead a bound system.
    
    A similar particle, \PDeltazero, exists as an excited state of the neutron and has the same consequences implying the neutron is not fundamental.
    
    \subsection{Magnetic Moment}
    Relativistic quantum theory, as developed by Dirac, predicts that for fundamental fermions the magnetic dipole moment is
    \begin{equation}
        \mu = \frac{q\hbar}{2m}
    \end{equation}
    where \(q\) is the charge of the fermion and \(m\) its mass.
    
    Recall that classically the magnetic dipole moment is most simply associated with a current loop of area \(A\) carrying current \(I\) and is given by \(\mu = AI\).
    The magnetic dipole moment of a particle is associated with its spin, which we can imagine as the charge of the particle spinning in a sort of microscopic current loop.
    Of course, this isn't really what's happening but it will be a useful picture to keep in mind for this section.
    
    For an electron we predict that
    \begin{equation}
        \mu_{\Pee} = \frac{e\hbar}{2m_{\Pee}} \coloneqq \qty{1}{\bohrmagneton}
    \end{equation}
    where \(m_{\Pee}\) is the mass of an electron.
    We define this quantity to be equal to \(\qty{1}{\bohrmagneton}\), where \(\unit{\bohrmagneton}\) is called the \defineindex{Bohr magneton}, and is the standard unit of atomic magnetism.
    
    A full calculation from quantum electrodynamics (QED)\glossary[acronym]{QED}{quantum electrodynamics} predicts that
    \begin{equation}
        \mu_{\Pee} = \qty{1.001596524(4)}{\bohrmagneton}
    \end{equation}
    This differs from the simple relativistic quantum mechanics prediction by including higher order terms which account self-interaction, such as virtual emission and absorption of photons and electron-positron pairs.
    These terms can each be summarised in a Feynman diagram.
    The QED Feynman diagrams for the self-interaction of an electron start
    \begin{align*}
        &
        \tikzsetnextfilename{qed-electron-selfinteraction-term-1}
        \begin{tikzpicture}
            \begin{feynman}
                \vertex (a);
                \vertex[right=of a] (b);
                \diagram {
                    (a) -- [fermion] (b);
                };
            \end{feynman}
        \end{tikzpicture}
        +
        \tikzsetnextfilename{qed-electron-selfinteraction-term-2}
        \begin{tikzpicture}
            \begin{feynman}
                \vertex (a);
                \vertex[right=of a] (b);
                \vertex[right=of b] (c);
                \vertex (d) at ($(b) + (0, 1)$);
                \vertex (e) at ($(d) + (0, 1)$);
                \diagram {
                    (a) -- [fermion] (b) -- [fermion] (c);
                    (b) -- [photon] (d) -- [half left, looseness=1.6, fermion] (e);
                    (d) -- [half right, looseness=1.6, anti fermion] (e);
                };
            \end{feynman}
        \end{tikzpicture}
        +
        \tikzsetnextfilename{qed-electron-selfinteraction-term-3}
        \begin{tikzpicture}
            \begin{feynman}
                \vertex (a);
                \vertex[right=of a] (b);
                \vertex[right=of b] (c);
                \vertex[right=of c] (d);
                \diagram {
                    (a) -- [fermion] (b) -- [fermion] (c) -- [fermion] (d);
                    (b) -- [photon, half left, looseness=2] (c);
                };
            \end{feynman}
        \end{tikzpicture}
        \\
        +\, &
        \tikzsetnextfilename{qed-electron-selfinteraction-term-4}
        \begin{tikzpicture}
            \begin{feynman}
                \vertex (a);
                \vertex[right=of a] (b);
                \vertex[right=of b] (c);
                \vertex (d) at ($(b) + (0, 1)$);
                \vertex (e) at ($(d) + (0, 1)$);
                \vertex[right=of c] (f);
                \vertex (g) at ($(c) + (0, 1)$);
                \vertex (i) at ($(g) + (0, 1)$);
                \diagram {
                    (a) -- [fermion] (b) -- [fermion] (c) -- [fermion] (f);
                    (b) -- [photon] (d) -- [half left, looseness=1.6, fermion] (e);
                    (d) -- [half right, looseness=1.6, anti fermion] (e);
                    (c) -- [photon] (g) -- [half left, looseness=1.6, fermion] (i);
                    (g) -- [half right, looseness=1.6, anti fermion] (i);
                };
            \end{feynman}
        \end{tikzpicture}
        \\
        +\, &
        \tikzsetnextfilename{qed-electron-selfinteraction-term-5}
        \begin{tikzpicture}
            \begin{feynman}
                \vertex (a);
                \vertex[right=of a] (b);
                \vertex[right=of b] (c);
                \vertex (d) at ($(b) + (0, 1)$);
                \vertex (e) at ($(d) + (0, 1)$);
                \vertex[right=of c] (f);
                \vertex[right=of f] (g);
                \diagram {
                    (a) -- [fermion] (b) -- [fermion] (c) -- [fermion] (f) -- [fermion] (g);
                    (b) -- [photon] (d) -- [half left, looseness=1.6, fermion] (e);
                    (d) -- [half right, looseness=1.6, anti fermion] (e);
                    (c) -- [photon, half left, looseness=2] (f);
                };
            \end{feynman}
        \end{tikzpicture}
        \\
        +\, &
        \tikzsetnextfilename{qed-electron-selfinteraction-term-6}
        \begin{tikzpicture}
            \begin{feynman}
                \vertex (a);
                \vertex[right=of a] (b);
                \vertex[right=of b] (c);
                \vertex (d) at ($(b) + (0, 1)$);
                \vertex (e) at ($(d) + (0, 1)$);
                \vertex[left=of a] (g);
                \vertex[left=of g] (f);
                \diagram {
                    (f) -- [fermion] (g) -- [fermion] (a) -- [fermion] (b) -- [fermion] (c);
                    (b) -- [photon] (d) -- [half left, looseness=1.6, fermion] (e);
                    (d) -- [half right, looseness=1.6, anti fermion] (e);
                    (g) -- [photon, half left, looseness=2] (a);
                };
            \end{feynman}
        \end{tikzpicture}
        \\
        +\, &
        \tikzsetnextfilename{qed-electron-selfinteraction-term-7}
        \begin{tikzpicture}
            \begin{feynman}
                \vertex (a);
                \vertex[right=of a] (b);
                \vertex[right=of b] (c);
                \vertex[right=of c] (d);
                \vertex[right=of d] (e);
                \vertex[right=of e] (f);
                \diagram {
                    (a) -- [fermion] (b) -- [fermion] (c) -- [fermion] (d) -- [fermion] (e) -- [fermion] (f);
                    (b) -- [photon, half left, looseness=2] (c);
                    (d) -- [photon, half left, looseness=2] (e);
                };
            \end{feynman}
        \end{tikzpicture}
    \end{align*}
    The first term accounts for no self-interaction.
    The second term shows an electron positron pair being created, annihilating, and the photon being reabsorbed.
    The third term shows a virtual photon being emitted and absorbed.
    The terms continue on in this way to account for the infinite number of ways an electron can interact with itself.
    Fortunately, each interaction, meaning each vertex in the diagram, contributes a factor of \(\alpha\), the \defineindex{fine structure constant}, to the sum, and since \(\alpha \approx 1/137\) the terms quickly become negligible.
    
    Experiments to measure \(\mu_\Pee\) measured
    \begin{equation}
        \mu_{\Pee} = \qty{1.001596524(2)}{\bohrmagneton}.
    \end{equation}
    This is in remarkable agreement with the QED value.
    
    For the proton we predict a magnetic dipole moment of
    \begin{equation}
        \mu_{\Pp} = \frac{e\hbar}{2m_{\Pp}} \coloneqq \qty{1}{\nuclearmagneton}.
    \end{equation}
    We define this quantity to be \qty{1}{\nuclearmagneton}, where \unit{\nuclearmagneton} is the \defineindex{nuclear magneton}, the standard unit of nuclear magnetisation.
    
    The problem is that we actually measure \(\mu_{\Pp}\) to be
    \begin{equation}
        \mu_{\Pp} = \qty{2.79}{\nuclearmagneton}.
    \end{equation}
    So our prediction is off by a factor of \(2.79\).
    
    For a neutron we predict a magnetic dipole moment of \(\mu_{\Pn} = 0\) since the neutron is neutral.
    However, we measure the magnetic dipole moment to be
    \begin{equation}
        \mu_{\Pn} = \qty{-1.91}{\nuclearmagneton}.
    \end{equation}
    
    One way of interpreting this is to say the neutron has a net positive charge at its centre and a net negative charge further from the centre.
    In our picture where the magnetic dipole moment comes from physical spin this would result in a nuclear magnetic moment that is anti-aligned with the spin, and hence negative.
    
    The disparity between our prediction and the measured results is more evidence that protons and neutrons are not fundamental particles, since this was an assumption upon which our predictions were based.
    
    \section{Simplified Quark Model}
    It is now accepted that protons and neutrons aren't fundamental.
    Instead they are made of \define{quarks}\index{quark}.
    In particular they are made of \define{up quarks}\index{quark!up}, \Pu, and \define{down quarks}\index{quark!down}, \Pd.
    These quarks have fractional charges \(q_{\Pu} = 2e/3\) and \(q_{\Pd} = -e/3\) respectively.
    It then follows that the proton is made of two up quarks and a down quark, \(\Pp = \Pu\Pu\Pd\), and the neutron is made of two down quarks and an up quark, \(\Pn = \Pu\Pd\Pd\).
    
    Up and down quarks are the lightest quarks with masses of a few \unit{\MeVpercsquared}.
    The exact masses are unknown as free quarks have never been isolated.
    For comparison the mass of the proton is \(m_{\Pp} = \qty{938.28}{\MeVpercsquared}\) and the mass of the neutron is \(m_{\Pn} = \qty{939.57}{\MeVpercsquared}\).
    We conclude that most of the proton and neutron's mass is associated with the strong interaction between the confined quarks, in an analogous way to the mass of the hydrogen atom being unequal to the mass of a proton plus the mass of an electron.
    It's just that in this case the interactions involved are due to the strong force and so the effect is much greater than the proton-electron interactions due to the magnetic force which result in an energy/mass difference of only \qty{13.6}{\electronvolt}.
    
    Quarks have \defineindex{colour charge} of either red, green, or blue.
    Overall nucleons are colourless, meaning they contain a red quark, a green quark, and a blue quark.
    These colour charges are \emph{not} real colours, given the size of quarks it doesn't make sense for them to have a real colour.
    Instead they are more like spin, or electric charge, an abstract quantity that determines the strength of interactions, in this case the colour determines the strength of interactions with the strong force.
    
    The problem with using the quark model to derive properties of nuclei is that it is hard.
    Quantum chromodynamics (QCD)\glossary[acronym]{QCD}{quantum chromodynamics} is our best theory for strong force interactions, including between quarks.
    A nucleus is made of many quarks.
    
    The comparison with the electromagnetic force and standard quantum mechanics would be a multi-electron atom.
    We would usually treat such a system with perturbation theory, for example we can model \ce{^4He} as two independent electrons in a central potential from the nucleus and then add a perturbation from the electron-electron interaction.
    We then expand this perturbation in a series of terms decreasing in size.
    
    This same approach doesn't work with QCD for a few reasons.
    First, the force between quarks \emph{increases} with distance up to about \qty{1}{\fermi}.
    In comparison the electromagnetic force follows an inverse square law.
    Second, the \define{gluons}\index{gluon}, which are the force carriers for the strong force, are themselves coloured and so interact with the strong force, meaning more gluons, which interact through the strong force and so on.
    In comparison photons are neutral and so we don't have this problem considering electromagnetic forces.
    Yet another issue is colour confinement.
    Essentially it takes so much energy to separate quarks or gluons bound by the strong force that before enough energy has been input into the system to separate them some of that energy will have created a particle-antiparticle pair which then pair off with the original particles we were trying to separate and we have to start again.
    This means we can't treat just a single particle, we always have to consider at least pairs of particles.
    
    \chapter{The Deuteron}
    Ignoring a single proton as the hydrogen nucleus, which is really the object of particle physics, the deuteron is the simplest nuclear system.
    It consists of neutron-proton pair bound by the strong force.
    Recall the potential between two nucleons as shown in \cref{fig:strong force central potential}.
    This potential can be applied to \Pn-\Pn, \Pp-\Pp, or \Pn-\Pp{} pairs, so long as the net spin of the system is \(S = 0\).
    This is called the charge independence of the nuclear force and is due to the common underlying pion-exchange mechanism for the longer distance attractive component.
    The problem is that the potential is not deep enough to produce bound two nucleon systems.
    This is evidenced by the fact that we don't see stable \ce{^2He} (\Pp-\Pp) or di-neutron (\Pn-\Pn) systems.
    So how come we see stable deuterons?
    
    The answer is that stable deuterons have net spin \(S = 1\).
    It isn't possible to create \ce{^2He} or di-neutron systems with net spin \(S = 1\) as this requires the spins be aligned and this is forbidden by the Pauli exclusion principle.
    However, since protons and neutrons are distinct particles they can have the same spin with no issues.
    We conclude that the strong force is spin dependent.
    
    Experimentally the magnetic dipole moment of the deuteron has been measured as
    \begin{equation}
        \mu_{\Pdeuteron} = \qty{0.857}{\nuclearmagneton} \ne \mu_{\Pp} + \mu_{\Pn} = \qty{0.879}{\nuclearmagneton}.
    \end{equation}
    This small, \qty{2.5}{\percent}, difference compared to the sum of the free nucleons may not seem significant but it is.
    We can attribute the difference to a small, nonzero, orbital angular momentum, \(l\), component in the deuteron wave function\footnote{later in the course we will see that odd values of \(l\) correspond to parity changes and the strong force conserves parity, meaning that wave functions cannot have dependence on odd values of \(l\)}:
    \begin{equation}
        \psi_{\Pdeuteron} = a\psi_{l=0} + b\psi_{l=2}
    \end{equation}
    where \(a\) and \(b\) are constants and \(\psi_{l}\) are wave functions with fixed values of \(l\).
    This mixture of values of \(l\) means that orbital angular momentum is not conserved and \(l\) is not a good quantum number for the deuteron system.
    The result of this is that the nucleon-nucleon interaction is not fully symmetric.
    This means that the potential for the strong force has a non-central component, which we call the \defineindex{tensor force}.
    
    A macroscopic analogy is the force between two bar magnets, which depends not only on their orientation but also their relative positions.
    Consider the two pairs of magnets in \cref{fig:bar magnets}.
    We see that one pair attract and the other repel, even though the directions are the same, the relative positions are also important.
    
    \begin{figure}
        \tikzsetnextfilename{magnets}
        \definecolor{magnet red}{HTML}{cc0000}
        \definecolor{magnet blue}{HTML}{0000cc}
        \begin{tikzpicture}
            \fill[magnet blue] (0, 0) rectangle (1, 2);
            \fill[magnet red] (0, 2) rectangle (1, 4);
            \node[white] at (0.5, 0.5) {\Large\textsf{S}};
            \node[white] at (0.5, 3.5) {\Large\textsf{N}};
            \begin{scope}[xshift=2cm]
                \fill[magnet blue] (0, 0) rectangle (1, 2);
                \fill[magnet red] (0, 2) rectangle (1, 4);
                \node[white] at (0.5, 0.5) {\Large\textsf{S}};
                \node[white] at (0.5, 3.5) {\Large\textsf{N}};
            \end{scope}
            \begin{scope}[xshift=6cm, yshift=2.5cm]
                \fill[magnet blue] (0, 0) rectangle (1, 2);
                \fill[magnet red] (0, 2) rectangle (1, 4);
                \node[white] at (0.5, 0.5) {\Large\textsf{S}};
                \node[white] at (0.5, 3.5) {\Large\textsf{N}};
            \end{scope}
            \begin{scope}[xshift=6cm, yshift=-2.5cm]
                \fill[magnet blue] (0, 0) rectangle (1, 2);
                \fill[magnet red] (0, 2) rectangle (1, 4);
                \node[white] at (0.5, 0.5) {\Large\textsf{S}};
                \node[white] at (0.5, 3.5) {\Large\textsf{N}};
            \end{scope}
        \end{tikzpicture}
        \caption{Two sets of aligned bar magnets. In the case on the left the net force is repulsive and on the right it is attractive}
        \label{fig:bar magnets}
    \end{figure}
    
    Experimentally this non-spherical nature is evidenced by a small, but nonzero, positive electric quadrupole moment.
    A spherically symmetric arrangement would have zero quadrupole moment.
    
    \chapter{Nuclear Properties}
    \section{Nuclear Size}
    Electrons are the tool of choice for probing the nucleus to measure its size.
    This is because they are fundamental particles and leptons, and hence don't interact through the strong force.
    This means that the interactions are much simpler and therefore easier to reason with.
    
    An incoming electron is scattered off of the nucleus in an electromagnetic interaction by exchanging a virtual photon, as shown in \cref{fig:electron nucleus scattering}.
    
    \begin{figure}
        \tikzsetnextfilename{electron-nucleus-scattering}
        \begin{tikzpicture}
            \begin{feynman}
                \vertex (interaction);
                \vertex (e in) at ($(interaction) + (-1.5, -0.5)$);
                \vertex (e out) at ($(interaction) + (1.5, -0.5)$);
                \vertex[above=of interaction] (nucleus);
                \fill[colder] (nucleus) ++(0, 0.5) circle [radius = 0.5];
                \diagram {
                    (e in) -- [fermion] (interaction) -- [fermion] (e out);
                    (interaction) -- [photon] (nucleus);
                };
            \end{feynman}
        \end{tikzpicture}
    \caption{An electron scattering off of a nucleus.}
    \label{fig:electron nucleus scattering}
    \end{figure}

    Let \(\Delta p\) be the momentum transferred to the recoiling nucleus.
    Then we can study the size of the nucleus to a spatial resolution of \(\Delta r\) which must follow Heisenberg's uncertainty relation
    \begin{equation}
        \Delta p \Delta r \approx \hbar.
    \end{equation}
    A useful approximation when computing values in nuclear physics is \(\hbar c \approx \qty{197}{\MeV\fermi}\).
    The true value is actually closer to \qty{197.33}{\MeV\fermi}, but this approximation is good enough for our purposes.
    Suppose we wish to determine the spatial resolution to within \(\qty{1}{\fermi}\).
    Then we need the momentum exchanged to be
    \begin{equation}
        \Delta p \approx \frac{\hbar}{\Delta r} = \frac{\hbar c}{c\Delta r} \approx \frac{\qty{197}{\MeV\fermi}}{(\qty{1}{\fermi}) c} \approx \qty{200}{\MeVperc}.
    \end{equation}
    
    When talking about particle collisions we usually talk about the energy of the particles, rather than their momentum.
    Since this is an electron and \(m_{\Pe} = \qty[per-mode=symbol]{511}{\kilo\electronvolt\per\clight\squared}\) we can ignore the mass contribution to the energy and take the relativistic energy to be \(E \approx pc\).
    Hence we need electrons with energies in excess of \qty{200}{\MeV}.
    In reality we need even higher as the electron won't lose all of its momentum to the nucleus.
    
    \subsection{Diffraction Effects in Nuclear Scattering}
    When the electron scatters off of the nucleus we cannot ignore diffraction effects.
    We assume the nucleus behaves like a circular disk of radius \(R\).
    The de Broglie wavelength of the electron is \(\lambda = h/p\).
    It can be shown that in such a case the first minimum occurs when the scattering angle, \(\vartheta\), satisfies
    \begin{equation}
        \sin\vartheta = 0.61\frac{\lambda}{R}.
    \end{equation}
    This factor of \(0.61\) comes from the Bessel function, \(J_1\), which appears as Fourier transform of a circular transfer function.
    The first root of \(J_1\) is at \(1.22\pi\), the \(\pi\) cancels with another \(\pi\) and their is a factor of \(1/2\) which results in the factor of \(1.22/2 = 0.61\).
    
    One problem that this method of electron scattering faces is that electrons are sensitive to the charge distribution within the nucleus, that is the arrangement of protons.
    
    One way round this is to instead use protons.
    Ideally we would use neutrons, since they are neutral, and therefore interact almost identically with protons and neutrons, with only a slight difference due to different magnetic dipole moments.
    Practically however since neutrons are neutral it is hard to accelerate them and so we use protons.
    Another issue is that protons are composite systems, which makes everything more difficult.
    
    In reality the slight electromagnetic interaction is not too much of a big deal. 
    We can get around it since the scattering cross section decreases with the energy of the particles and so, by using high energy protons, we can make it so that they interact essentially exclusively through the strong force.
    
    In scattering experiments with protons we see that the minima do not go to zero.
    This is because nuclei aren't, as we've been treating them so far, hard spheres.
    Instead they are better modelled by spheres with fuzzy edges.
    
    We can parametrise the nuclear matter distribution with the function
    \begin{equation}
        \rho(r) = \frac{\rho_0}{1 + \e^{(r - R)/a}}
    \end{equation}
    where \(\rho\) is the number of nucleons per fermi cubed, that is the \emph{number} density, not the mass density, \(R\) is the nuclear radius, defined as the distance at which \(\rho = \rho_0/2\), and \(a\) is a constant factor depending on the element and isotope which represents the diffuseness of the nuclear surface.
    For most nuclei \(a \approx \qty{0.5}{\fermi}\).
    Finally, \(r\) is the distance from the centre of the nucleus.
    
    For small values of \(r\) the nucleon density is approximately constant in the central region of the nucleus and tends to a limiting value of approximately \num{0.17} nucleons per fermi cubed.
    
    It can then be shown that in this central region the average nearest-neighbour separation is approximately \qty{1.8}{\fermi}.
    Importantly this means that only nearest neighbours interact non-negligibly through the strong force.
    
    Nuclear radius has been experimentally shown to be closely proportional to \(A^{1/3}\), where \(A\) is the \defineindex{mass number}, that is the total number of nucleons.
    In particular we find
    \begin{equation}
        R = r_0 A^{1/3}
    \end{equation}
    where \(r_0 = \qty{1.2}{\fermi}\).
    The dependence on \(A^{1/3}\) means that nuclear radius increases much faster for lighter elements.
    This relation is consistent with all nuclei having a constant central nucleonic density, \(\rho_0\), since we expect the product of the nuclear volume and nuclear density to be the number of nucleons:
    \begin{align}
        \frac{4}{3}\pi R^{3} \rho_0 = A \implies R = (\frac{3}{4}\frac{1}{\pi \rho_0})^{1/3}A^{1/3} \propto A^{1/3}.
    \end{align}
    
    These observations lead to one of the earliest models of the nucleus, the \defineindex{liquid drop model}.
    In this model a nucleus is treated like a drop of liquid.
    The justification for this is that the nucleus has an interior region of approximately uniform density and there are short range attractive forces between nearest neighbours.
    
    \section{Nuclear Mass}
    The standard unit for atomic mass is the \defineindex{dalton}, which has the symbol \unit{\dalton}.
    This unit is also sometimes\footnote{the unit for nuclear mass accepted for use with SI units is the dalton, but since the atomic mass unit is identical it is still commonly used.} referred to as the \define{atomic mass unit}\index{atomic mass unit|see{dalton}}, with the symbol \unit{\amu}.
    This unit is defined such that \qty{1}{\dalton} is \(1/12\) the mass of an unbound, neutral atom of \ce{^12C} in its nuclear and electronic ground state.
    For conversion purposes
    \begin{equation}
        \qty{1}{\dalton} = \qty{931.494}{\MeVpercsquared} = \qty{1.66054e-27}{\kilogram}.
    \end{equation}
    
    The \defineindex{nuclear binding energy} is defined as the amount of energy required to completely separate\footnote{this is the opposite definition to that used in atomic physics where the binding energy is the energy released when forming the nucleus. The result is a sign change in the definition.} the nucleons in a nucleus of \(Z\) protons and \(N\) neutrons.
    It is calculated by
    \begin{equation}
        B(N, Z) \coloneqq (Zm_{\ce{H}} + Nm_{\Pn} - M(N, Z))c^2.
    \end{equation}
    Here \(m_{\ce{H}}\) is the mass of a hydrogen atom, \(m_{\Pn}\) is the mass of a neutron, and \(M(N, Z)\) is the neutral atomic mass, that is the mass of the atom with all its electrons.
    Our interest is in the nucleus but despite this we are forced to consider atomic quantities that include electrons since it is too difficult to separate the nucleus from its electrons for the purpose of measuring its mass.
    
    More often we are actually interested in the binding energy per nucleon since this can be compared between isotopes and elements.
    For a nucleus of \(A\) nucleons this is given by
    \begin{equation}
        \frac{B(N, Z)}{A}.
    \end{equation}
    This is plotted in \cref{fig:binding energy per nucleon}.
    
    \begin{figure}
        \tikzsetnextfilename{binding-energy-per-nucleon}
        \begin{tikzpicture}
            \begin{axis}[
                xlabel=Mass Number,
                ylabel=Binding Energy Per Nucleon/\unit{\MeV},
                width=11cm,
                xmin=0,
                ymin=0,
                xmax=265
                ]
                \addplot[highlight, very thin] table [
                x=mass, y=bindingEnergy,
                mark=none
                ]{C:/Users/willo/Documents/Uni/Year 4/Relativity Nuclear and Particle Physics/Nuclear Physics/Code/processed-binding-energy.csv};
                \coordinate (4He) at (axis cs:4, 7.074);
                \draw[->] (4He) -- ++ (0.2cm, 1.2cm) node[above] {\ce{^4He}};
                \coordinate (58Fe) at (axis cs:58, 8.792);
                \draw[->] (58Fe) -- ++ (0.5cm, 0.3cm) node[right] {\ce{^58Fe}};
            \end{axis}
        \end{tikzpicture}
        \caption{The binding energy per nucleon.}
        \label{fig:binding energy per nucleon}
    \end{figure}
    
    There are a few features worth discussing.
    First, for lighter nuclei, with \(A \lessapprox 50\), the binding energy per nucleon increases rapidly with the mass.
    It then reaches a peak, at \ce{58Fe}.
    It then slowly decreases with mass.
    
    The \ce{^4He} nuclei also has a much higher binding energy per nucleon than its neighbours.
    The result of this is that \ce{^4He} nuclei are very common, after all this is an alpha particle.
    
    We can interpret the mass dependence using the liquid drop model.
    Lighter nuclei have a relatively high proportion of their nucleons in the surface region.
    Here the strong nuclear force is not saturated as they are not surrounded by nucleons.
    The total energy is therefore approximately proportional to the surface area, which is proportional to \(R^2\), which is proportional to \(A^{2/3}\).
    The effect on the binding energy \emph{per nucleon} is then proportional to \(A^{-1/3}\).
    As the mass increases this effect becomes less important.
    
    Once we get to medium nuclei the binding energy pre nucleon is approximately constant at around \qty{8}{\MeV}.
    In analogy to the liquid drop this is because the total volume is proportional to \(R^3\), which is proportional to \(A\), and so the binding energy \emph{per nucleon} is approximately constant as the factors of \(A\) cancel.
    
    
    The gradual decrease beyond \(A \approx 60\) is accounted for by the fact that the Coulomb potential energy repelling between two photons scales as \(Q^2/R\) where \(Q\) is the charge of a the nucleus.
    This is proportional to \(Z^2/A^{1/3}\) and so decreases the binding energy.
    We can see this as a result of long range effects, i.e. electromagnetism, starting to be more important than short range effects, i.e. the strong force, as the size increases.
    
    %Appendicies
    %\appendixpage
    %\begin{appendices}
    %    \include{}
    %\end{appendices}
    
    \backmatter
    \renewcommand{\glossaryname}{Acronyms}
    \printglossary[acronym]
    \printindex
\end{document}
