% !TeX program = lualatex
\documentclass[fleqn]{NotesClass}

\strictpagecheck

%% Packages
\usepackage{csquotes}
\usepackage{siunitx}
\usepackage{subcaption}

\usepackage{slashed}
\declareslashed{}{\not}{.1}{.5}{A}
\declareslashed{}{\not}{.05}{.5}{p}
\declareslashed{}{\not}{.1}{.5}{\partial}

% Tikz stuff
\usepackage{tikz}
%\tikzset{>=latex}
% £xternal
\usetikzlibrary{external}
\tikzexternalize[prefix=tikz-external/]
% Other libraries

\usepackage[compat=1.1.0]{tikz-feynman}

% References, should be last things loaded
\usepackage[pdfauthor={Willoughby Seago},pdftitle={Particle Physics},pdfkeywords={particle physics, Feynman diagrams, scattering, Dirac equation, QED, QCD, weak interactions},pdfsubject={particle physics}]{hyperref}  % Should be loaded second last (cleveref last)
\colorlet{hyperrefcolor}{blue!60!black}
\hypersetup{colorlinks=true, linkcolor=hyperrefcolor, urlcolor=hyperrefcolor}
\usepackage[
capitalize,
nameinlink,
noabbrev
]{cleveref} % Should be loaded last

% My packages
\usepackage{NotesBoxes}
\usepackage{NotesMaths}

\setmathfont[range={\int, \oint}]{Latin Modern Math}

% Highlight colour
\definecolor{Burgandy}{HTML}{8F2D56}
\definecolor{Teal}{HTML}{218380}
\definecolor{Blue}{HTML}{73D2DE}
\definecolor{Pink}{HTML}{D81159}
\definecolor{Yello}{HTML}{FFBC42}

\colorlet{highlight}{Burgandy}

% Title page info
\title{Particle Physics}
\author{Willoughby Seago}
\date{}
% \subtitle{}
% \subsubtitle{}

% Open Question environment ---------------------------------------------------
\BeforeBeginEnvironment{openquestion}{%
    \vspace*{0pt plus 3pt} % To help with underfull vbox warnings
    \refstepcounter{equation}%
}

\AfterEndEnvironment{openquestion}{%
    \vspace*{0pt plus 3pt} % To help with underfull vbox warnings
}

\newtcbtheorem[number freestyle={\noexpand\theequation},%
crefname={Open Question}{Open Questions}]%
{openquestion}{Open Question}%
{enhanced jigsaw,%
    enforce breakable,%
    shrink break goal=0.5\baselineskip,%
    segmentation style=solid,%
    theorem style=plain,%
    %    before title app={{\tiny\ensuremath{\blacksquare}}\nobreakspace},%
    fonttitle=\sffamily\upshape\bfseries\small,%
    coltitle=black,%
    description font=\sffamily\upshape\bfseries\small,%
    description color=black,%
    colframe=highlight,%
    colback=azure(web)(azuremist)!45,%
    boxrule=0pt,%
    leftrule=4pt,%
    sharp corners,%
    description delimiters={}{},%
    separator sign={\nobreakspace {\color{black}---}},%
    terminator sign={\ },%
    label type=app,%
    label separator=,
}{}
% End application environment -----------------------------------------------

% Commands

% Particles
\newcommand{\Pparticle}[1]{\symup{#1}}
\newcommand{\Pu}{\ensuremath{\Pparticle{u}}}
\newcommand{\Pd}{\ensuremath{\Pparticle{d}}}
\newcommand{\Ps}{\ensuremath{\Pparticle{s}}}
\newcommand{\Pc}{\ensuremath{\Pparticle{c}}}
\newcommand{\Pt}{\ensuremath{\Pparticle{t}}}
\newcommand{\Pb}{\ensuremath{\Pparticle{b}}}
\newcommand{\Pe}{\ensuremath{\Pparticle{e}^{-}}}
\newcommand{\Pmu}{\ensuremath{\text{μ}^{-}}}
\newcommand{\Ptau}{\ensuremath{\text{τ}^{-}}}
\newcommand{\Pnue}{\ensuremath{\text{ν}_{\symup{e}}}}
\newcommand{\Pnumu}{\ensuremath{\text{ν}_{\text{μ}}}}
\newcommand{\Pnutau}{\ensuremath{\text{ν}_{\text{τ}}}}
\newcommand{\PH}{\ensuremath{\Pparticle{H}}}
\newcommand{\PZ}{\ensuremath{\Pparticle{Z}}}
\newcommand{\PW}{\ensuremath{\Pparticle{W}}}
\newcommand{\PWpm}{\ensuremath{\Pparticle{W}^{\pm}}}
\newcommand{\PWp}{\ensuremath{\Pparticle{W}^{+}}}
\newcommand{\PWm}{\ensuremath{\Pparticle{W}^{-}}}
\newcommand{\Pphoton}{\ensuremath{\text{γ}}}
\newcommand{\Pg}{\ensuremath{\Pparticle{g}}}
\newcommand{\Pq}{\ensuremath{\Pparticle{q}}}
\newcommand{\Ppip}{\ensuremath{\text{π}^{+}}}
\newcommand{\Ppim}{\ensuremath{\text{π}^{-}}}
\newcommand{\Ppizero}{\ensuremath{\text{π}^{0}}}
\newcommand{\Prhozero}{\ensuremath{\text{ρ}^{0}}}
\newcommand{\Pf}{\ensuremath{\Pparticle{f}}}




\newcommand{\APantiparticle}[1]{\overbar{#1}}
\newcommand{\APu}{\ensuremath{\APantiparticle{\Pparticle{u}}}}
\newcommand{\APd}{\ensuremath{\APantiparticle{\Pparticle{d}}}}
\newcommand{\APs}{\ensuremath{\APantiparticle{\Pparticle{s}}}}
\newcommand{\APc}{\ensuremath{\APantiparticle{\Pparticle{c}}}}
\newcommand{\APt}{\ensuremath{\APantiparticle{\Pparticle{t}}}}
\newcommand{\APb}{\ensuremath{\APantiparticle{\Pparticle{b}}}}
\newcommand{\APe}{\ensuremath{\Pparticle{e}^{+}}}
\newcommand{\APmu}{\ensuremath{\text{μ}^{+}}}
\newcommand{\APtau}{\ensuremath{\text{τ}^{+}}}
\newcommand{\APnue}{\ensuremath{\APantiparticle{\text{ν}}_{\symup{e}}}}
\newcommand{\APnumu}{\ensuremath{\APantiparticle{\text{ν}}_{\text{μ}}}}
\newcommand{\APnutau}{\ensuremath{\APantiparticle{\text{ν}}_{\text{τ}}}}
\newcommand{\APH}{\ensuremath{\Pparticle{H}}}
\newcommand{\APZ}{\ensuremath{\Pparticle{Z}}}
\newcommand{\APWpm}{\ensuremath{\Pparticle{W}^{\mp}}}
\newcommand{\APWp}{\ensuremath{\Pparticle{W}^{-}}}
\newcommand{\APWm}{\ensuremath{\Pparticle{W}^{+}}}
\newcommand{\APphoton}{\ensuremath{\text{γ}}}
\newcommand{\APg}{\ensuremath{\Pparticle{g}}}
\newcommand{\APq}{\ensuremath{\APantiparticle{\Pparticle{q}}}}

% Text
\newcommand{\course}[1]{\textit{#1}}

% Maths
\newcommand{\strongCoupling}{g_{\symrm{s}}}
\newcommand{\wCoupling}{g_{\symrm{W}}}
\newcommand{\zCoupling}{g_{\symrm{Z}}}
\newcommand{\EM}{\text{EM}}
\newcommand{\strongForce}{\text{S}}
\newcommand{\e}{\symrm{e}}
\newcommand{\probability}{\symbb{P}}
\DeclareMathOperator{\sinc}{sinc}
\newcommand{\amplitude}{\symcal{M}}
\newcommand{\dalembertian}{\partial^2}
\DeclarePairedDelimiterX{\anticommutator}[2]{\{}{\}}{#1 , #2}
\newcommand{\hermit}{\dagger}
\newcommand{\minkowskiSpace}{\reals^{1,3}}
\newcommand{\diracadjoint}[1]{\overbar{#1}}
\newcommand{\dirac}{\symup{D}}
\newcommand{\minkowskiMetric}{\eta}
\newcommand{\lagrangianDensity}{\symcal{L}}
\newcommand{\trans}{\top}

\includeonly{parts/appParticleData, parts/appFermiGoldenRule, parts/appPolarisation}

\begin{document}
    \frontmatter
    \titlepage
    \innertitlepage{tikz-external/standard-model}
    \tableofcontents
    \listoffigures
    \mainmatter
    
    \chapter{Preliminaries}
    This course is not interested in the finer mathematical details, but the broader principles.
    For details see the \course{Quantum Field Theory} course.
    Knowledge is assumed of quantum mechanics (see the \course{Principles of Quantum Mechanics} and \course{Quantum Theory} courses), and special relativity (see the \course{Relativity, Nuclear, and Particle Physics} course, or \course{Classical Electrodynamics}).
    
    We use natural units where \(c = \hbar = \varepsilon_0 = \mu_0 = 1\).
    This means that masses, momentums, and energies are all measured in electron volts, force in electron volts squared, lengths and times in inverse electron volts, and speeds and angular momentums are dimensionless.
    Electric charge is also dimensionless, but the charge of an electron is  \emph{not} set to unity, although particle charges are given in terms of the (absolute value of the) electron charges.
    Instead we have the fine structure constant,
    \begin{equation}
        \alpha = \frac{e^2}{4\pi \varepsilon_0 \hbar c} \quad (\text{SI}) \qquad \alpha = \frac{e^2}{4\pi} \quad (c = \hbar = \varepsilon_0 = 1).
    \end{equation}
    The charge of an electron then has magnitude
    \begin{equation}
        e = \sqrt{4\pi \alpha}.
    \end{equation}
    Since \(\alpha\) is dimensionless its value is the same in any system of units, and is approximately \(1/137\), so
    \begin{equation}
        e \approx \num{0.303}.
    \end{equation}
    
    We use the Minkowski metric with the \(({+}{-}{-}{-})\) convention:
    \begin{equation}
        \eta_{\mu\nu} = \eta^{\mu\nu} = 
        \begin{pmatrix}
            1 & 0 & 0 & 0\\
            0 & -1 & 0 & 0\\
            0 & 0 & -1 & 0\\
            0 & 0 & 0 & -1
        \end{pmatrix}
        .
    \end{equation}
    
    \begin{openquestion}{}{}
        Particle physics is full of many unanswered questions.
        Relevant questions will be flagged in a box like this one.
        These are typically beyond the scope of the course, and are mentioned simply for interest.
    \end{openquestion}
        
    \chapter{Introduction}
    \section{The Standard Model}
    The \defineindex{standard model of particle physics}, or the standard model for short, is our best model of the fundamental constituents of matter and fundamental forces.
    The standard model actually refers to a collection of quantum field theories (QFT)\glossary[acronym]{QFT}{Quantum Field Theory} describing these forces as due to particle exchanges.
    The standard model is theoretically self consistent, all of the maths checks out.
    It can be used to make predictions, which can then be checked against measurements.
    So far all such tests have validated the standard model.
    As a model the standard model does not predict everything, instead there are about 20 parameters that have to be measured separately, including the masses of various particles, the coupling strengths of various fields, and mixing parameters.
    
    The standard model cannot explain everything, a nonexhaustive list of unexplained phenomena by the standard model is as follows:
    \begin{itemize}
        \item general relativity/accelerating expansion of the universe;
        \item dark energy/dark matter;
        \item neutrino masses;
        \item matter/antimatter imbalance;
        \item why the parameters have the values they do.
    \end{itemize}
    
    \section{Fundamental Particles}
    \epigraph{A particle with no charge. What's the point in that?}{Victoria Marting}
    \define{Spin}\index{spin} is a property that every particle has, it's a quantum number, just a label, like charge or mass, but without such a simple interpretation.
    There are two spin operators, the total spin operator, \(\operator{S}^2\), and the component of spin in the \(z\)-direction, \(\operator{S}_z\).
    If \(\ket{\psi}\) is a spin eigenstate then the action of these operators on \(\ket{\psi}\) is
    \begin{equation}
        \operator{S}^2 \ket{\psi} = s(s + 1)\ket{\psi}, \qand \operator{S}_z \ket{\psi} = m_s \hbar \ket{\psi}.
    \end{equation}
    Here \(s\) takes on a nonnegative half integer value, \(s = 0, \pm 1/2, \pm 1, \pm 3/2, \dotsc\).
    The value of \(m_s\) is then constrained to lie between \(-s\) and \(s\) increasing in integer steps, so \(m_s = -s, 1 - s, \dotsc, 0, \dotsc, s - 1, s\).
    
    The fundamental particles in the standard model have spin \(s = 0, 1/2, 1\).
    Specifically, the Higgs boson has spin 0, the quarks and leptons have spin \(1/2\), and the force carriers have spin 1.
    When a particle has spin \(1/2\) there are two possible values of \(m_s\), \(\pm 1/2\), which we call spin up (\(+1/2\)) and spin down (\(-1/2\)).
    When a particle has spin \(1\) there are three possible values of \(m_s\), \(0\) and \(\pm 1\), which we refer to as polarisations.
    A photon can only have \(m_s = \pm 1\), which is where this terminology comes from.
    
    We broadly split all particles into two types, \define{fermions}\index{fermion}, with half integer spin, and \define{bosons}\index{boson}, with integer spin.
    The fermions in the standard model further split into \define{quarks}\index{quark} and \define{leptons}\index{lepton}.
    
    There are 6 quarks, which split into two types up-type, \define{up}\index{up quark}, \define{charm}\index{charm quark}, and \define{top quarks}\index{top quark}, or \Pu\index{u@\Pu|see{up quark}}, \Pc\index{c@\Pc|see{charm quark}}, and \Pt\index{t@\Pt|see{top quark}}, and down-type quarks, the \define{down}\index{down quark}, \define{strange}\index{strange quark}, and \define{bottom quarks}\index{bottom quark}, or \Pd\index{d@\Pd|see{down quark}}, \Ps\index{s@\Ps|see{strange quark}}, and \Pb\index{b@\Pb|see{bottom quark}}.
    Up-type quarks have charge \(+2/3\) in units of electron charge, and bottom-type quarks have charge \(-1/3\).
    All types of quarks also have \enquote{colour charge}, relating to the strong force, and \enquote{weak isospin}, relating to the weak force, we'll see this in more detail later in the course.
    Each type (up/down) consists of three generations of quarks, and as we go down the generations, from \Pu{}, to \Pc, to \Pt, they get more massive.
    
    Similarly the leptons are split into two, first we have \define{electrons}\index{electron}, \Pe\index{e-@\Pu|see{electron}}, \define{muons}\index{muon}, \Pmu\index{\Pmu|see{muon}}, and \define{tau}\index{tau lepton} particles, \Ptau\index{\Ptau|see{tau lepton}}, these all have a charge of \(-1\).
    Then, there are the three \define{neutrinos}\index{neutrino}, the \define{electron}\index{electron neutrino}, \define{muon}\index{muon neutrino}, and \define{tau neutrinos}\index{tau neutrino}, \Pnue\index{\Pnue|see{electron neutrino}}, \Pnumu\index{\Pnumu|see{muon neutrino}}, and \Pnutau\index{\Pnutau|see{tau neutrino}}, these are all electrically neutral.
    All leptons have zero colour charge, but they do have weak isospin.
    
    We split the bosons into two parts, the spin zero bosons, or \define{scalar bosons}\index{scalar boson}, which is just the \defineindex{Higgs boson}, \PH\index{H@\PH|see{Higgs boson}}.
    Then there are the force carrying bosons, also known as \define{gauge bosons}\index{gauge bosons} or \define{vector bosons}\index{vector boson}, which have spin 1.
    These consist of the \defineindex{photon}, \Pphoton\index{\Pphoton|see{photon}}, gluons, \Pg\index{g@\Pg|see{gluon}}, \defineindex{\PZ{} boson}, and \defineindex{\PWpm{}-bosons}.
    The photon is the force carrier in electromagnetism and quantum electrodynamics (QED)\glossary[acronym]{QED}{Quantum Electrodynamics}.
    The gluons are the force carriers in quantum chromodynamics (QCD)\glossary[acronym]{QCD}{Quantum Chromodynamics}.
    The \PZ-boson gives neutral currents in weak interactions and \PWpm-bosons give charged currents.
    Finally, there is the hypothetical graviton, which if it exists will be a spin 2 boson, or a \defineindex{tensor boson}.
    Note that these names, scalar, vector, and gauge, come from what type of object the fields describing the particle are, so the photon is described by the electromagnetic field, \(A^\mu\), whereas the graviton is described by the energy-momentum tensor, \(T^{\mu\nu}\).
    
    \begin{figure}
        \tikzsetnextfilename{standard-model}
        \begin{tikzpicture}[
            level 1/.style = {sibling distance = 5cm},
            level 2/.style = {sibling distance = 2cm, anchor=north}
            ]
            \node {Elementary Particles}
            child {node {Fermions}
                child {node [xshift=-0.2cm] {\parbox{2cm}{\centering Quarks\\ \Pu, \Pd, \Pc, \Ps, \Pt, \Pb}}}
                child {node [xshift=0.2cm] {\parbox{2cm}{\centering Leptons\\ \Pe, \Pmu, \Ptau, \Pnue, \Pnumu, \Pnutau}}}
            }
            child {node {Bosons}
                child {node {\parbox{2cm}{\centering Spin 0\\ \PH}}}
                child {node {\parbox{2cm}{\centering Spin 1\\ \Pphoton, \Pg, \PZ, \PWpm}}}
                child [dashed] {node {\parbox{2cm}{\centering Spin 2\\ Graviton?}}}
            };
        \end{tikzpicture}
        \caption{All of the particles in the standard model. The graviton is hypothesised but has not been observed.}
    \end{figure}
    
    \subsection{Antiparticles}
    Every particle has a corresponding \defineindex{antiparticle}, although some particles are there own antiparticles, such as the photon, Higgs boson, and \PZ{} boson.
    The \PWp{} and \PWm{} are mutually each others antiparticles, and the antiparticle of a gluon is another gluon.
    The antiparticles are defined by being identical, but with opposite charges.
    By charges here we mean electric charge, colour charge, and weak isospin, which are the charges telling us how strongly the particle couples to the electromagnetic field, strong force, and weak force respectively.
    
    The naming convention is to just stick the prefix \enquote{anti} in front of the particle's name.
    The one exception to the naming convention is the electron, whose antiparticle is the \defineindex{positron}.
    Most antiparticles are denoted as the same symbol with a bar, so \APu{} is an antiup quark.
    If the symbol usually has a sign, such as \Pe, or \PWp, then the antiparticle is denoted with the opposite sign, so \APe\index{e+@\APe|see{positron}}, or \PWm.
    
    \begin{openquestion}{}{}
        It is not known if the neutrinos are their own antiparticles or not.
        Fermions which are their own antiparticles are called Majorana fermions, whereas fermions which which aren't are called Dirac fermions.
    \end{openquestion}
    
    \section{Composite Particles}
    As well as the fundamental particles, of which there are 18 particles, there are \emph{a lot} of \define{composite particles}\index{composite particle}, particles formed by combining quarks and/or antiquarks into a bound state.
    Note that there aren't composite leptons because the strong force is required to form bound states.
    
    We call a composite particle formed from quarks a \defineindex{hadron}.
    The hadrons split two types, first \define{baryons}\index{baryon}, which are formed of three quarks\footnote{here \Pq{} denotes an arbitrary quark, in particular there is no requirement that all three quarks in a baryon are the same, even if we call all three \Pq.}, \Pq\Pq\Pq, and \define{antibaryons}\index{antibaryon}, which are formed of three antiquarks, \APq\APq\APq.
    The other class is \define{mesons}\index{meson}, which are quark-antiquark pairs, \Pq\APq.
    
    The antiparticle of a composite particle has its quarks replaced with the equivalent antiquarks, and antiquarks replaced with the equivalent quarks.
    For example, the \defineindex{negative pion}, \Ppim\index{\Ppim|see{negative pion}}, is a meson with quark content \Pd\APu, and its antiparticle is the \defineindex{positive pion}, \Ppip\index{\Ppip|see{positive pion}}, another meson with quark content \Pu\APd.
    
    As well as the hadrons and mesons there have been other composite particles observed recently, although these aren't on the course as they aren't well understood yet.
    We've seen \defineindex{tetraquarks}, formed from two quarks and two antiquarks, \Pq\Pq\APq\APq, and pentaquarks, formed from four quarks and an antiquark, \Pq\Pq\Pq\Pq\APq.
    It is possible that these states aren't really new particles, but instead are \enquote{molecules} of either pairs of mesons in the tetraquark case or a hadron and a meson in the pentaquark case.
    The difference being that in the \enquote{molecules} not all quarks would be equally bound to each other, whereas if they truly are composite particles there will be no difference in how bound any pair is, apart from, for example, differences due to differing electric charges.
    
    \chapter{Feynman Diagrams}
    \define{Feynman diagrams}\index{Feynman diagram} are a way of depicting interactions between particles.
    They also correspond to the amplitude for that interaction occurring in that way.
    Each piece of the diagram can be converted into a term in some expression for the amplitude.
    More abstractly we can view Feynman diagrams as terms in a series expansion of some amplitude.
    
    When reading Feynman diagrams in this course we follow the convention that time increases to the right.
    
    \section{Currents}
    The most basic part of a Feynman diagram is a \enquote{current}.
    This is a line representing the movement of a particle through spacetime.
    The phrase current comes from electrons, where an electron moving through space is interpreted as a current.
    
    The way we depict a current depends on the type of particle.
    Fermions are depicted as a straight line with an arrow, so a fermion current looks like
    \begin{equation}
        \tikzsetnextfilename{fd-fermion-current}
        \feynmandiagram [horizontal=i to o] {
            i -- [fermion] o
        };
    \end{equation}
    Antifermions are then depicted as fermions travelling back in time, so an antifermion current looks like
    \begin{equation}
        \tikzsetnextfilename{fd-antifermion-current}
        \feynmandiagram [horizontal=i to o] {
            i -- [anti fermion] o
        };
    \end{equation}
    Spin 1 bosons, apart from gluons, so photons, \PZ{} bosons, and \PWpm{} bosons, are depicted with a wavy line:
    \begin{equation}
        \tikzsetnextfilename{fd-boson-current}
        \feynmandiagram [horizontal=i to o] {
            i -- [boson] o
        };
    \end{equation}
    Gluons are depicted with a curly line,
    \begin{equation}
        \tikzsetnextfilename{fd-gluon-current}
        \feynmandiagram [horizontal=i to o] {
            i -- [gluon] o
        };
    \end{equation}
    The Higgs boson is depicted with a dashed line,
    \begin{equation}
        \tikzsetnextfilename{fd-higgs-current}
        \feynmandiagram [horizontal=i to o] {
            i -- [scalar] o
        };
    \end{equation}
    
    It's fine not to draw an arrow for bosons as either the particle is its own antiparticle, in which case the direction does not matter, or in the case of the \PWpm{} bosons or gluons the antiparticle is of the same type, so changing direction corresponds to changing the charge (electric charge for \PWpm, and colour charge for gluons).
    
    \section{Vertex}
    The building block of any Feynman diagram is the \defineindex{vertex}, this is where currents come together.
    The exact rules for which currents can form a vertex depends on the theory in question.
    A vertex is not, by itself, a valid process.
    Instead, a diagram is a combination of vertices, connected in such a way that all conservation laws are obeyed.
    
    For now we focus on vertices involving a fermion.
    These will always have exactly two fermion currents and a boson current, so for example,
    \begin{equation}
        \tikzsetnextfilename{fd-vertex}
        \feynmandiagram [horizontal=i to v, inline=(i)] {
            i -- [fermion] v,
            v -- [boson] o1,
            v -- [fermion] o2
        };
    \end{equation}
    
    If we want to specify particular particles we can do so by labelling the lines:
    \begin{equation}
        \tikzsetnextfilename{fd-vertex-muon-photon}
        \feynmandiagram [horizontal=i to v, inline=(i)] {
            i [particle=\Pmu] -- [fermion] v,
            v -- [boson] o1 [particle=\Pphoton],
            v -- [fermion] o2 [particle=\Pmu]
        };
        \qqand
        \tikzsetnextfilename{fd-vertex-strange-gluon}
        \feynmandiagram [horizontal=i to v, inline=(i)] {
            i [particle=\Ps] -- [fermion] v,
            v -- [gluon] o1 [particle=\Pg],
            v -- [fermion] o2 [particle=\Ps]
        };
    \end{equation}
    These represent the processes \(\Pmu \to \Pmu\Pphoton\) and \(\Ps \to \Ps\Pg\).
    
    At every vertex there are conservation laws we have to apply.
    Again, the details depend on exactly what the fermions and bosons are.
    We must always conserve energy, momentum, and charge.
    There are also other quantities that must be conserved, such as muon lepton number, which is the number of muons and muon neutrinos minus the number of antimuons and antimuon neutrinos.
    We must also conserve the quark number, which is the number of quarks minus the number of antiquarks.
    
    Energy and momentum conservation can be combined into conservation of four-momentum, where
    \begin{equation}
        p^\mu = (E/c, \vv{p})
    \end{equation}
    is the four-momentum of a particle with energy \(E\) and momentum \(\vv{p}\).
    
    Given a valid vertex if we rotate it we will get another valid vertex.
    For example, rotating the \(\Ps \to \Ps\Pg\) vertex above we get the vertices
    \begin{equation}
        \tikzsetnextfilename{fd-vertex-strange-gluon-rotated-1}
        \feynmandiagram [horizontal=i to v, inline=(v)] {
            i [particle=\Pg] -- [gluon] v,
            v -- [fermion] o1 [particle=\Ps],
            v -- [anti fermion] o2 [particle=\APs]
        };
        \qqand
        \tikzsetnextfilename{fd-vertex-strange-gluon-rotated-2}
        \feynmandiagram [horizontal=v to o, inline=(v)] {
            i1 [particle=\Ps] -- [fermion] v,
            i2 [particle=\APs] -- [anti fermion] v,
            v -- [gluon] o [particle=\Pg] 
        };
    \end{equation}
    The only thing that changes here is the interpretation, these processes now represent \(\Pg \to \Ps\APs\) and \(\Ps\APs \to \Pg\).
    
    \subsection{Forces}
    \subsubsection{Electromagnetic Vertex}
    An electromagnetic vertex has a photon and any charged fermion.
    If the fermion has charge \(Q\) then the coupling constant, which we'll see later relates to how strong the interaction is, is \(Q\).
    There will be no fermion flavour change.
    \begin{equation}
        \tikzsetnextfilename{fd-vertex-electromagnetic-vertex}
        \begin{tikzpicture}[baseline=(i)]
            \begin{feynman}[horizontal=i to v, inline=(i)]
                \diagram[small]{
                    i -- [fermion] v,
                    v -- [boson] o1 [particle=\Pphoton],
                    v -- [fermion] o2,
                };
                \node [right] at (v) {\(Q\)};
            \end{feynman}
        \end{tikzpicture}
    \end{equation}
    
    \subsubsection{Strong Vertex}
    A strong vertex has a gluon and a quark.
    The coupling constant is \(\strongCoupling\).
    There will be no quark flavour change.
    \begin{equation}
        \tikzsetnextfilename{fd-vertex-strong-vertex}
        \begin{tikzpicture}[baseline=(i)]
            \begin{feynman}[horizontal=i to v, inline=(i)]
                \diagram[small]{
                    i [particle=\Pq] -- [fermion] v,
                    v -- [gluon] o1 [particle=\Pg],
                    v -- [fermion] o2 [particle=\Pq],
                };
                \node [right, xshift=0.1cm] at (v) {\(\strongCoupling\)};
            \end{feynman}
        \end{tikzpicture}
    \end{equation}
    
    \subsubsection{Neutral Current Weak Vertex}
    A neutral current weak vertex has a \PZ{} boson and any fermion.
    The coupling constant is \(\zCoupling\).
    There will be no fermion flavour change.
    \begin{equation}
        \tikzsetnextfilename{fd-vertex-weak-neutral-vertex}
        \begin{tikzpicture}[baseline=(i)]
            \begin{feynman}[horizontal=i to v, inline=(i)]
                \diagram[small]{
                    i -- [fermion] v,
                    v -- [boson] o1 [particle=\PZ],
                    v -- [fermion] o2,
                };
                \node [right] at (v) {\(\zCoupling\)};
            \end{feynman}
        \end{tikzpicture}
    \end{equation}
    
    \subsubsection{Charged Current Weak Vertex}
    A charged current weak vertex has a \PWpm{} boson and any fermion.
    The coupling constant is \(\wCoupling\).
    There will always be a fermion flavour change since it is required for charge conservation.
    \begin{equation}
        \tikzsetnextfilename{fd-vertex-weak-charged-vertex}
        \begin{tikzpicture}[baseline=(i)]
            \begin{feynman}[horizontal=i to v]
                \diagram[small]{
                    i [particle=\(\Pf\)] -- [fermion] v,
                    v -- [boson] o1 [particle=\PW],
                    v -- [fermion] o2 [particle=\(\Pf'\)],
                };
                \node [right] at (v) {\(g_{\PW}\)};
            \end{feynman}
        \end{tikzpicture}
    \end{equation}
    
    \subsubsection{Higgs}
    A Higgs vertex has a Higgs boson and any fermion, except, at least for the purposes of this course, neutrinos.
    If the fermion has mass \(m_{\Pf}\) then the coupling constant is \(m_{\Pf}/v\), where \(v\) is a constant of value \(v \approx \qty{246}{\giga\electronvolt}\).
    There will be no fermion flavour change.
    \begin{equation}
        \tikzsetnextfilename{fd-vertex-higgs}
        \begin{tikzpicture}[baseline=(i)]
            \begin{feynman}[horizontal=i to v]
                \diagram[small]{
                    i -- [fermion] v,
                    v -- [scalar] o1,
                    v -- [fermion] o2,
                };
                \node [right] at (v) {\(m_{\Pf}/v\)};
            \end{feynman}
        \end{tikzpicture}
    \end{equation}
    
    \begin{openquestion}{}{}
        It is not known whether neutrinos couple to the Higgs field, or get their mass through another mechanism.
        Since neutrino masses are incredibly small no coupling has ever been observed, but cannot be ruled out.
    \end{openquestion}
    
    \subsection{Coupling Constants}
    The coupling constants come with related quantities, which are also referred to as the coupling constants, given by
    \begin{equation}
        \alpha = \frac{g^2}{4\pi},
    \end{equation}
    where \(g\) is the relevant coupling constant:
    \begin{itemize}
        \item \(g_{\EM} = e\) for an electromagnetic vertex, where \(e\) is the charge of the fermion involved in the interaction, not the charge of an electron;
        \item \(\strongCoupling\) for a strong vertex;
        \item \(\zCoupling\) for a neutral current weak vertex; and
        \item \(\wCoupling\) for a charged current weak vertex.
    \end{itemize}
    
    The value of \(\alpha\) is a measure of how likely a particular vertex is to occur.
    The strong force is, unsurprisingly, the strongest by which we mean that, assuming it's an allowed interaction, a strong vertex is more likely to occur than any other sort.
    This is expressed through the (relatively) high value of \(\alpha\) for the strong force:
    \begin{equation}
        \alpha_{\strongForce} \sim 1,
    \end{equation}
    as an order of magnitude estimate.
    This causes problems when we try to do perturbation theory with the strong force, as it means very often things don't converge.
    
    Next up in the list is the weak force, which has
    \begin{equation}
        \alpha_{\PW} \approx \alpha_{\PZ} \sim \frac{1}{40}.
    \end{equation}
    Last, is electromagnetism, with
    \begin{equation}
        \alpha_{\EM} = \alpha \approx \frac{1}{137}.
    \end{equation}
    
    The actual likelihood of any particular interaction depends on much more than the value of the coupling constant.
    In reality we have to consider all possible ways the interaction can occur, compute the sum of the amplitudes for these processes, and square it to get a probability.
    This means that things like the small range of the strong and weak interactions means that they are actually far less relevant on a human scale than electromagnetism and gravity.
    However, on the scale of fundamental particles, and small composite particles, the strong and weak interactions cannot be ignored.
    
    Notice that we aren't giving precise values for the coupling constants.
    This is because, despite the name, these aren't really constants.
    Instead their values depend on the energy scale at which the interaction occurs.
    This is called running of the coupling constant, and we'll see more detail later.
    
    \subsection{Boson Vertices}
    As well as vertices involving a fermion we can vertices with bosons.
    These are more varied, and we'll meet examples throughout the course.
    For now, a few examples of valid boson vertices are shown in \cref{fig:boson verticies}.
    
    \begin{figure}[p]
        \tikzsetnextfilename{fd-boson-vertex-1}
        \subcaptionbox{Two \PW{} bosons and a \PZ{} boson or a photon.}[0.31\textwidth]{
            \feynmandiagram [inline=(i)] {
                i -- [boson] v,
                v -- [boson] o1,
                v -- [boson] o2
            };
        }
        \tikzsetnextfilename{fd-boson-vertex-2}
        \subcaptionbox{Either two \PW{} bosons or two \PZ{} bosons and a Higgs boson.}[0.31\textwidth]{
            \feynmandiagram [inline=(i)] {
                i -- [boson] v,
                v -- [boson] o1,
                v -- [scalar] o2
            };
        }
        \tikzsetnextfilename{fd-boson-vertex-3}
        \subcaptionbox{Three Higgs bosons.}[0.31\textwidth]{
            \feynmandiagram [inline=(i)] {
                i -- [scalar] v,
                v -- [scalar] o1,
                v -- [scalar] o2
            };
        }
        \tikzsetnextfilename{fd-boson-vertex-4}
        \subcaptionbox{Two \PW{} bosons and either two \PW{} bosons, two \PZ{} bosons, a \PZ{} boson and a photon, or two photons.}[0.31\textwidth]{
            \feynmandiagram [inline=(v)] {
                i1 -- [boson] v,
                i2 -- [boson] v,
                v -- [boson] o1,
                v -- [boson] o2,
            };
        }
        \tikzsetnextfilename{fd-boson-vertex-5}
        \subcaptionbox{Four Higgs bosons.}[0.31\textwidth]{
            \feynmandiagram [inline=(v)] {
                i1 -- [scalar] v,
                i2 -- [scalar] v,
                v -- [scalar] o1,
                v -- [scalar] o2,
            };
        }
        \tikzsetnextfilename{fd-boson-vertex-6}
        \subcaptionbox{Two \PW{} bosons or two \PZ{} bosons and two Higgs bosons.}[0.31\textwidth]{
            \feynmandiagram [inline=(v)] {
                i1 -- [boson] v,
                i2 -- [boson] v,
                v -- [scalar] o1,
                v -- [scalar] o2,
            };
        }
        \tikzsetnextfilename{fd-boson-vertex-7}
        \subcaptionbox{Three gluons.}[0.31\textwidth]{
            \feynmandiagram [inline=(i)] {
                i -- [gluon] v,
                v -- [gluon] o1,
                v -- [gluon] o2
            };
        }
        \tikzsetnextfilename{fd-boson-vertex-8}
        \subcaptionbox{Four gluons.}[0.31\textwidth]{
            \feynmandiagram [inline=(v)] {
                i1 -- [gluon] v,
                i2 -- [gluon] v,
                v -- [gluon] o1,
                v -- [gluon] o2,
            };
        }
        \caption{Some allowed boson vertices.}
        \label{fig:boson verticies}
    \end{figure}
    
    \section{Feynman Diagrams}
    A complete Feynman diagram is then a combination of two or more vertices such that all conservation laws are obeyed.
    Exactly what the conservation laws are we'll get to later, but the basic conserved properties are charge, lepton number, quark number, and of course momentum and energy.
    
    We can consider three broad classes of interactions, first, we have \defineindex{scattering}, where two particles come in, interact, then go their separate ways.
    A simple scattering diagram is
    \begin{equation}
        \tikzsetnextfilename{fd-scattering}
        \feynmandiagram [inline=(v1), vertical=v1 to v2] {
            i1 -- [fermion] v1,
            i2 -- [fermion] v2,
            v1 -- [boson] v2,
            v1 -- [fermion] o1,
            v2 -- [fermion] o2
        };
    \end{equation}
    
    Another possibility is \defineindex{annihilation}, where two particles come in, annihilate into a boson, and then the boson produced decays into another two particles.
    A simple annihilation diagram is
    \begin{equation}
        \tikzsetnextfilename{fd-annihilation}
        \feynmandiagram [inline=(v1), horizontal=v1 to v2] {
            i1 -- [fermion] v1,
            i2 -- [fermion] v2,
            v1 -- [boson] v2,
            v1 -- [fermion] o1,
            v2 -- [fermion] o2
        };
    \end{equation}
    
    The third possibility is \defineindex{decay}, where a single particle comes in, decays into a fermion and a boson, and the boson then decays into another two fermions.
    A simple decay diagram is
    \begin{equation}
        \tikzsetnextfilename{fd-decay}
        \feynmandiagram [inline=(v1), horizontal=i to v1, layered layout] {
            i -- [fermion] v1,
            v1 -- [fermion] o1,
            v1 -- [boson] v2,
            v2 -- [fermion] o2,
            v2 -- [anti fermion] o3
        };
    \end{equation}
    
    It is important to note that Feynman diagrams are \emph{not} physical.
    We set up an experiment with some ingoing particles, and measure the outgoing particles.
    What happens between is a black box process.
    The actual process should be viewed more like the diagram
    \begin{equation}
        \rotatebox{45}{
            \feynmandiagram [inline=(v)] {
                i1 -- [fermion] v [blob] -- [fermion] o1,
                i2 -- [fermion] v -- [fermion] o2
            };
        }
    \end{equation}
    where the blob in the middle is all possible intermediate processes, all occurring at once.
    After all, from the perspective of quantum field theory a Feynman diagram is just a single term in an infinite expansion summing over all possible processes.
    
    \chapter{What do we Measure}
    \section{Types of Interactions}
    There are three different things that can occur and we can measure to get information about the particles and forces involved.
    The first is particles can decay.
    That is, we start with some particle, \(A\), and end up with particles \(1, 2, \dotsc, n\),
    \begin{equation}
        A \to 1 + 2 + \dotsb + n.
    \end{equation}
    We measure the decay rate, that is the rate at which a given decay from one particle into some other set of other particles, occurs, and compare this to theoretical predictions.
    Often we compare ratios of decay rates to get rid of constant factors and make comparison easier.
    
    The second thing that can occur is scattering.
    Here two particles, say \(A\) and \(B\), interact and produce particles \(1, 2, \dotsc, n\),
    \begin{equation}
        A + B \to 1 + 2 + \dotsb + n.
    \end{equation}
    Note that its possible that particles \(1, 2, \dotsc, n\) contain particles that are the same as the incoming particles, \(A\) and \(B\).
    However, we shouldn't think of these as the same particles, since from a quantum field theory perspective all scattering occurs by annihilating all particles, and then creating new particles.
    That said, one of the most common scattering occurrences is \(A + B \to A + B\), which is essentially the two particles bouncing off each other.
    
    The third thing that can occur is bound states, such as \defineindex{positronium} (\(\Pe\APe\)), \defineindex{charmonium} (\(\Pc\APc\)), or \defineindex{bottomonium} (\(\Pb\APb\)).
    Measuring the properties of these, such as their lifetimes, can also tell us a lot about the particles and the forces binding them.
    
    More generally we measure the transition rate for the system to go from an initial quantum state to some other final quantum state.
    
    \section{Feynman Diagrams vs.\texorpdfstring{\@}{} the Lab Picture}
    Suppose we are doing a scattering experiment scattering electrons off quarks, the quarks in term being wrapped up in a proton, since we can't get unbound quarks.
    Consider a particular scattering event in which no new particles are produced, so \(\Pe\Pq \to \Pe\Pq\).
    A Feynman diagram for this is
    \begin{equation}
        \tikzsetnextfilename{fd-feynman-diagram-picture}
        \feynmandiagram [inline=(v1), vertical=v1 to v2] {
            i1 [particle=1 \Pe] -- [fermion] v1,
            i2 [particle=2 \Pq]-- [fermion] v2,
            v1 -- [boson] v2,
            v1 -- [fermion] o1 [particle=\Pe{} 3],
            v2 -- [fermion] o2 [particle=\Pq{} 4],
        };
    \end{equation}
    Here we label the particles, with 1 and 2 being the incoming particles, and 3 and 4 being the outgoing particles.
    
    The scattering experiment could be of two types, either two beams, or a fixed target.
    However these are really the same but related by a change of frame.
    So, for simplicity, we assume two beams.
    We also assume that, like most real scattering experiments, the two beams are head on.
    Then the interaction looks something like this:
    \begin{equation}
        \tikzsetnextfilename{scattering-beams}
        \begin{tikzpicture}[baseline=(current bounding box)]
                \draw[highlight, very thick, text=black] (-3, 0) node [left] {1} .. controls (0, 0) .. (3, 2) node [above right] {3};
                \draw[highlight, very thick, text=black] (3, 0) node [right] {2} .. controls (0, 0) .. (-3, -2) node [below left] {4};
        \end{tikzpicture}
    \end{equation}
    Here the labels 1 and 2 refer to the incoming electron and quark, and the labels 3 and 4 are the outgoing electron and quark.
    
    Momentum is conserved in this interaction.
    This is a relativistic statement, so we mean that the four-momentum is conserved.
    Let \(p_1\) be the four-momentum of the incoming electron, \(p_2\) the four-momentum of the incoming quark, \(p_3\) the four-momentum of the outgoing electron, and \(p_4\) the four momentum of the outgoing quark.
    Then we have
    \begin{equation}
        p_1 + p_2 = p_3 + p_4.
    \end{equation}
    
    We will generally be interested in measuring/predicting the rate of the process, say \(1 + 2 \to 3 + 4\), that is, of all scatterings with particles \(1\) and \(2\) incoming with four-momenta \(p_1\) and \(p_2\) how many result in particles 3 and 4 with four-momenta \(p_3\) and \(p_4\).
    
    In this experiment the initial state is something that we control, we choose the particles to scatter and set their momenta based on how we create the beams.
    The final state is then what we measure.
    
    \section{Mandelstam Variables}
    Consider the following three processes:
    \begin{equation}
        \tikzsetnextfilename{fd-mandelstam-s-channel}
        \feynmandiagram[small, horizontal=v1 to v2, baseline=(current bounding box)]{
            i1 [particle=2] -- [fermion] v1,
            i2 [particle=1] -- [anti fermion] v1,
            v1 -- [boson] v2,
            v2 -- [anti fermion] o1 [particle=3],
            v2 -- [fermion] o2 [particle=4]
        };
        \qquad
        \tikzsetnextfilename{fd-mandelstam-t-channel}
        \feynmandiagram[small, vertical=v1 to v2, baseline=(current bounding box)]{
            i1 [particle=1] -- [fermion] v1,
            i2 [particle=3] -- [anti fermion] v1,
            v1 -- [boson] v2,
            v2 -- [anti fermion] o1 [particle=2],
            v2 -- [fermion] o2 [particle=4]
        };
        \qquad
        \tikzsetnextfilename{fd-mandelstam-u-channel}
        \begin{tikzpicture}[baseline=(current bounding box)]
            \begin{feynman}
                \diagram [vertical=a to b, small] {
                    i1 [particle=1] -- [fermion] a -- [draw=none] f1 [particle=3],
                    a -- [boson] b,
                    i2 [particle=2] -- [fermion] b -- [draw=none] f2 [particle=4],
                };
                \diagram* [small] {
                    (a) -- [fermion] (f2),
                    (b) -- [fermion] (f1),
                };
            \end{feynman}
        \end{tikzpicture}
    \end{equation}
    These are called the \define{\(\symbf{s}\)-channel}\index{s-channel@\(s\)-channel}, \define{\(\symbf{t}\)-channel}\index{t-channel@\(t\)-channel}, and \define{\(\symbf{u}\)-channel}\index{u-channel@\(u\)-channel} processes.
    For each one we can define a Lorentz invariant from the four momenta of the particles:
    \begin{equation}
        s \coloneqq (p_1 + p_2)^2, \qquad t \coloneqq (p_1 - p_3)^2, \qqand u \coloneqq (p_1 - p_4)^2.
    \end{equation}
    These are called the \defineindex{Mandelstam variables}
    
    The most useful of these quantities is \(s\).
    Expanding it out we get
    \begin{equation}
        s = p_1^2 + p_2^2 + 2 p_1 \cdot p_2.
    \end{equation}
    Now, if we work in the rest frame of a particle with mass \(m\) its four momentum is simply \(p = (E, \vv{0})\), and \(E^2 = m^2 + \vv{0}^2\), so \(E = m\).
    Hence \(p = (m, \vv{0})\) and so \(p^2 = m^2\).
    This is true in the rest frame, but \(p^2\) is Lorentz invariant so it must be true in any frame.
    Thus, we have
    \begin{equation}
        s = m_1^2 + m_2^2 + 2p_1 \cdot p_2,
    \end{equation}
    where \(m_1\) and \(m_2\) are the masses of particles 1 and 2.
    
    In the lab frame the four-momentum of each particle is \(p = (E, \vv{p})\), where \(E\) is the energy and \(\vv{p}\) the three-momentum.
    In our set up the two beams are head on, so the angle between the incoming particles is \ang{180}.
    This is one of the reasons \(s\) is more useful than \(t\) or \(v\), these involve \(p_3\) or \(p_4\), and we can't control the direction of the outgoing particles.
    Using this we have
    \begin{equation}
        p_1 \cdot p_2 = E_1E_2 - \vv{p_1}\cdot \vv{p_2} = E_1E_2 - \abs{\vv{p_1}}\abs{\vv{p_2}}\sin(\ang{180}) = E_1E_2 + \abs{\vv{p_1}}\abs{\vv{p_2}}.
    \end{equation}
    
    Now, we know that \(E^2 = m^2 + \abs{\vv{p}}^2\).
    Suppose we have very energetic particles, so that \(m \ll \abs{\vv{p}}\), then \(E \approx \abs{\vv{p}}\).
    Hence, we have
    \begin{equation}
        p_1 \cdot p_2 = E_1E_2 + \abs{\vv{p_1}}\abs{\vv{p_2}} \approx E_1 E_2 + E_1E_2 = 2E_1E_2.
    \end{equation}
    Further, assuming that \(m_1 \ll E_1\) and \(m_2 \ll E_2\) we have
    \begin{equation}
        s = m_1^2 + m_2^2 + 2p_1 \cdot p_2 \approx m_1^2 + m_2^2 + 4E_1E_2 \approx 4E_1E_2.
    \end{equation}
    
    If both beams have the same energy, \(E\), then \(s = 4E^2\), and so \(\sqrt{s} = 2E\), which is the total energy of the two particles scattering.
    This means that \(\sqrt{s}\) has a nice interpretation as the energy of the collision.
    
    \section{Fermi's Golden Rule}
    We are interested in the transition rate for going from some state \(\ket{i}\) to some state \(\ket{f}\).
    We suppose that the system evolves under the Hamiltonian
    \begin{equation}
        \operator{H} = \operator{H}_0 + \operator{H}'(x),
    \end{equation}
    where \(\operator{H}_0\) is a time independent Hamiltonian for which we can solve the Schrödinger equation and \(\operator{H}'(x)\) is the interaction Hamiltonian, which induces the transitions between states.
    The interaction is generally position dependent.
    The state of the system then evolves according to the Schrödinger equation:
    \begin{equation}
        i \diffp{}{t}\ket{\psi, t} = \operator{H}\ket{\psi, t} = [\operator{H}_0 + \operator{H}'(x)]\ket{\psi, t}.
    \end{equation}
    
    Fermi's golden rule says that the transition rate from state \(\ket{i}\) to state \(\ket{f}\) is given by
    \begin{equation}
        \Gamma_{fi} = 2\pi \abs{T_{fi}}^2 \rho(E_i)
    \end{equation}
    where \(\Gamma_{fi}\) is the transition rate, \(T_{fi}\) is the transition matrix, and \(\rho(E_i)\) is the density of accessible states.
    The full derivation is in \cref{app:fermi's golden rule}.
    
    To first order the transition matrix is simply the matrix elements of the perturbation Hamiltonian,
    \begin{equation}
        T_{fi} = \bra{f} \operator{H}' \ket{i}.
    \end{equation}
    This represents a process where the state evolves directly from \(\ket{i}\) to \(\ket{f}\).
    To second order the transition matrix is
    \begin{equation}
        T_{fi} = \bra{f} \operator{H}' \ket{i} + \sum_{k \ne i} \frac{\bra{f} \operator{H}' \ket{k} \bra{k} \operator{H}' \ket{i}}{E_i - E_k}.
    \end{equation}
    The second term accounts for all transitions via an intermediate state, \(\ket{k}\).
    Continuing on the next term would have transitions via two intermediate states and so on.
    
    \subsection{Density of States}
    We want our processes to be physical.
    This means they have to obey several conservation laws, such as energy and momentum conservation.
    We enforce this by having a Dirac delta in our terms, such as
    \begin{equation}
        \delta(E_i - E_f), \qquad \delta^3(\vv{p_i} - \vv{p_f}), \qqand \delta^4(p_i - p_f),
    \end{equation}
    where \(E_i\) and \(E_f\) are the total energy of the initial and final states, \(\vv{p_i}\) and \(\vv{p_f}\) are the total three-momenta of the initial and final states, and \(p_i\) and \(p_f\) are the total four-momenta of the initial and final states.
    
    The density of final accessible states depends on the kinematics of the process being considered.
    In particular, the initial energy, \(E_i\), is the total amount of energy available to the final states.
    The density of states is given by integrating over all states with energy conservation enforced by a Dirac delta:
    \begin{equation}
        \rho(E_i) = \int \delta(E_i - E) \dd{n} = \int \delta(E_i - E) \diff{n}{E} \dd{E} = \diff{n}{E}[E = E_i].
    \end{equation}
    Here \(\dl{n}\) gives the number of states with energies between \(E\) and \(E + \dl{E}\).
    
    Fermi's golden rule is then that
    \begin{equation}
        \Gamma_{fi} = 2\pi \int \abs{T_{fi}}^2 \delta(E_i - E) \dd{n}.
    \end{equation}
    
    The complication with this is that states are quantised, so this integral is nontrivial.
    A spinless, massless particle can be described with a plane wave
    \begin{equation}
        \psi = A\e^{-ip\cdot x}.
    \end{equation}
    We consider the particle to be in a box of side length \(a\) and impose vanishing boundary conditions.
    Normalising the wave function gives us \(A = 1/V\) where \(V = a^3\) is the volume of the box.
    The momentum of the particle is then quantised as
    \begin{equation}
        p_i = \frac{2\pi n_i}{a}.
    \end{equation}
    
    The volume occupied by a state in momentum space is then
    \begin{equation}
        \dl{^3\vv{p}} = \dl{p_x} \dd{p_y}\dd{p_z} = \left( \frac{2\pi}{a} \right)^3 = \frac{(2\pi)^3}{V}.
    \end{equation}
    The number of states, \(\dl{n}\), with a momentum in the range \(p\) to \(\dl{p}\) is the volume in momentum space of the spherical shell of inner radius \(p\) and thickness \(\dl{p}\) divided by the volume of a single state:
    \begin{equation}
        \dl{n} = 4\pi p^2 \dd{p} \frac{V}{(2\pi)^3}.
    \end{equation}
    We can choose our box such that we have one particle per unit volume, and call this volume 1, we then have
    \begin{equation}
        \dl{n} = \frac{4p^2 \dd{p}}{(2\pi)^3} = \frac{\dl{^3\vv{p}}}{(2\pi)^3}.
    \end{equation}
    
    For example, consider a decay \(A \to 1 + 2 + \dotsb + N\).
    Then
    \begin{equation}
        \dl{n} = (2\pi)^3 \prod_{i=1}^{N} \frac{\dl{^3\vv{p_i}}}{(2\pi)^3} \delta^3\left( \vv{p_A} - \sum_{i = 1}^{N} \vv{p_i} \right).
    \end{equation}
    
    So far we haven't considered the effects of relativity.
    Our box of unit volume will be Lorentz contracted in the direction of motion.
    The contraction is proportional to \(\gamma\), which for a single particle of mass \(m\) and energy \(E\) is given by \(\gamma = E/m\).
    The fix is to redefine the wave function to \(\psi'\), so that
    \begin{equation}
        \int \psi'^*\psi' \dd{V} \propto E.
    \end{equation}
    The convention is to choose the constant of proportionality to be 2, and then set \(\psi' = \sqrt{2E}\psi\).
    We can then define a Lorentz invariant matrix element:
    \begin{equation}
        \amplitude_{fi} = \sqrt{2E_1 \cdot 2E_2 \dotsm 2E_N} T_{fi}
    \end{equation}
    where the product is over the energies of all particles, in both the initial and final states.
    
    The transition rate is then
    \begin{equation}
        \Gamma_{fi} = 2\pi \int \abs{T_{fi}}^2 \delta(E_i - E) \dd{n}.
    \end{equation}
    For simplicity consider the decay \(A \to 1 + 2\), the transition rate is
    \begin{equation}
        \Gamma_{fi} = 2\pi \int \abs{T_{fi}}^2 \delta(E_A - E_1 - E_2) \dd{n}.
    \end{equation}
    Moving to momentum space we have
    \begin{equation}
        \dl{n} = (2\pi)^3 \frac{\dl{^3\vv{p_1}}}{(2\pi)^3} \frac{\dl{^3\vv{p_2}}}{(2\pi)^3} \delta^3(\vv{p_A} - \vv{p_1} - \vv{p_2}).
    \end{equation}
    Making this Lorentz invariant we have
    \begin{equation}
        \amplitude_{fi} = \sqrt{2E_12E_22E_A} T_{fi} \implies T_{fi} = \frac{\amplitude_{fi}}{\sqrt{2E_12E_22E_A}}
    \end{equation}
    and so
    \begin{equation}
        \Gamma_{fi} = \frac{(2\pi)^4}{2E_A} \int \abs{\amplitude_{fi}}^2 \delta^4(p_A - p_1 - p_2) \frac{\dl{^3\vv{p_1}}}{(2\pi)^32E_1} \frac{\dl{^3\vv{p_2}}}{(2\pi)^32E_2}.
    \end{equation}
    
    This is just a rewriting of Fermi's golden rule, but now in an explicitly Lorentz invariant form.
    There are several key features:
    \begin{itemize}
        \item The Lorentz invariant phase space measure, \(\dl{^3\vv{p}}/[(2\pi)^32E]\).
        \item Energy and Momentum conservation from the Dirac delta.
        \item The decay rate is inversely proportional to the energy fo the decaying particle.
        \item The Lorentz invariant \(\amplitude_{fi}\) is the part that encodes the physics of the process, and working out what value this should have for a given interaction is the goal of a large part of quantum field theory.
    \end{itemize}
    
    Consider the \(A \to 1 + 2\) decay again, we can partially evaluate the integral and we find that
    \begin{equation}
        \Gamma_{fi} = \frac{\abs{\vv{p}^*}}{32\pi^2 m_A^2} \int \abs{\amplitude_{fi}}^2 \dd{\Omega}
    \end{equation}
    where \(\abs{\vv{p}^*}\) is the magnitude of the three-momenta of particles 1 and 2 in the rest frame of \(A\), note that they must have the same magnitude of three-momenta as the initial momentum is zero in this frame.
    The integral is over all angles, with \(\dl{\Omega} = \sin\theta \dd{\varphi} \dd{\vartheta}\).
    This is valid for all two body decays.
    
    If instead we consider scattering, \(A + B \to 1 + 2\), again in the centre of mass frame then we find that
    \begin{equation}
        \sigma = \frac{1}{64\pi^2s} \frac{\abs{\vv{p_f}^*}}{\abs{\vv{p_i}^*}} \int \abs{\amplitude_{fi}}^2 \dd{\Omega}
    \end{equation}
    where \(\abs{\vv{p_i}^*}\) and \(\abs{\vv{p_f}^*}\) are the magnitude of the three-momenta of the initial and final states in the centre of mass frame, and \(s = (p_A + p_B)^2\) is a Mandelstam variable.
    Note that this is Lorentz invariant, despite the appearance of three-vectors.
    
    \chapter{Deriving the Dirac Equation}
    \section{Schrödinger}
    Start with the nonrelativistic energy-momentum relation
    \begin{equation}
        E = \frac{p^2}{2m} + V,
    \end{equation}
    where \(p\) is the magnitude of the particles momentum, \(m\) is its mass, and \(V\) is some potential.
    To get the Schrödinger equation we substitute in operators:
    \begin{equation}
        \operator{p}_j = -i\diffp{}{x^j} = -i\partial_j, \qquad \vecoperator{p} = -i\grad, \qqand E = i\diffp{}{t} = i\partial_t.
    \end{equation}
    Acting with the resulting operator on a wave function we get the \defineindex{Schrödinger equation}:
    \begin{equation}
        i\diffp{\psi}{t} = \left( -\frac{1}{2m} \laplacian + V \right)\psi.
    \end{equation}
    Note that this is first order in time, and second order in space, which makes it not relativistic.
    
    \section{Klein--Gordon Equation}
    If we want a relativistic equation we should start with the relativistic energy-momentum relation,
    \begin{equation}
        E^2 = \vv{p}^2 + m^2.
    \end{equation}
    Substituting in the same operators we get
    \begin{equation}
        \diffp[2]{\varphi}{t} = (\laplacian - m^2)\varphi \implies (\dalembertian + m^2)\varphi = 0.
    \end{equation}
    Here \(\dalembertian = \partial_\mu \partial^\mu\) is the d'Alembert operator.
    This is the \defineindex{Klein--Gordon equation}, it describes free spin 0 bosons.
    The problem is that we have negative energy solutions, and worse still, if we try to interpret \(\varphi\) as a wave function then we can get negative probabilities.
    
    \section{Dirac Equation}
    The negative energies appear due to the presence of \(E^2\), giving a positive and negative branch when we take the square root.
    To avoid this Dirac looked for a first order relationship between energy and momentum.
    He started with
    \begin{equation}
        \operator{E}\psi = (\vv{\alpha} \cdot \vecoperator{p} + \beta m)\psi
    \end{equation}
    where \(\vv{\alpha}\) and \(\beta\) are some constants.
    Inserting the operators we have
    \begin{equation}
        i\diffp{\psi}{t} = \left( -i\alpha_x \diffp{}{x} - i\alpha_y \diffp{}{y} - i\alpha_z \diffp{}{z} + \beta m \right)\psi.
    \end{equation}
    Defining \(\gamma^0 = \beta\), \(\gamma^i = \beta \alpha_i\) some rearranging gives this equation in the form
    \begin{equation}
        \left( i\gamma^0 \diffp{}{t} + i\gamma^1 \diffp{}{x} + i\gamma^2 \diffp{}{y} + i\gamma^3 \diffp{}{z} - m \right) \psi = 0.
    \end{equation}
    This can be written more compactly as
    \begin{equation}
        (i\gamma^\mu \partial_\mu - m)\psi = 0.
    \end{equation}
    Defining \(\slashed{a} \coloneqq \gamma^\mu a_\mu\)\index{\(\slashed{a} \coloneqq \gamma^\mu a_\mu\)} we can write this even more compactly as
    \begin{equation}
        (i\slashed{\partial} - m)\psi = 0.
    \end{equation}
    This is the \defineindex{Dirac equation}, it describes spin 1/2 fermions.
    
    \section{The \texorpdfstring{\(\gamma\)}{gamma} Coefficients}
    Suppose that \(\psi\) satisfies both the Dirac equation, we can then write
    \begin{equation}
        (-i\slashed{\partial} - m)(i\slashed{\partial} - m)\psi = 0.
    \end{equation}
    Expanding this we get
    \begin{equation}
        (\slashed{\partial}^2 +m^2)\psi = 0.
    \end{equation}
    This looks slightly like the Klein--Gordon equation, so suppose that \(\psi\) also satisfies this, that is
    \begin{equation}
        (\dalembertian + m^2)\psi = 0.
    \end{equation}
    Expanding the the \(\slashed{\partial} = \gamma^\mu \partial_\mu\) we get
    \begin{align}
        \slashed{\partial}^2 &= \gamma^\mu\partial_\mu \gamma^\nu\partial_\nu\\
        &= (\gamma^0)^2\partial_t^2 + (\gamma^1)^2\partial_x^2 + (\gamma^2)^2\partial_y^2 + (\gamma^3)^2\partial_z^2\\
        &+ \gamma^0\gamma^1 \partial_t\partial_x + \gamma^0\gamma^2 \partial_t\partial_y + \gamma^0\gamma^3 \partial_t\partial_z\\
        &+ \gamma^1\gamma^2 \partial_x\partial_y + \gamma^1\gamma^3 \partial_x\partial_z + \gamma^2\gamma^3 \partial_y\partial_z.
    \end{align}
    Compare this to the expanded d'Alembertian:
    \begin{equation}
        \dalembertian = \partial_t^2 - \partial_x^2 - \partial_y^2 - \partial_z^2.
    \end{equation}
    In order for \(\psi\) to satisfy both the Klein--Gordon equation and the Dirac equation we must have
    \begin{equation}
        (\gamma^0)^2 = 1, \qquad (\gamma^i)^2 = -1, \qqand \gamma^\mu \gamma^\nu = -\gamma^\nu \gamma^\mu.
    \end{equation}
    More compactly we can write these requirements as\index{\(\gamma^\mu\)}
    \begin{equation}
        \anticommutator{\gamma^\mu}{\gamma^\nu} = 2\eta^{\mu\nu}
    \end{equation}
    where \(\anticommutator{A}{B} \coloneqq AB + BA\) is the \defineindex{anticommutator} and \(\eta\) is the metric tensor.
    
    \itshape
    While this isn't a proof that \(\gamma^\mu\) must satisfy this we take it to be so.
    The relation
    \begin{equation}
        \anticommutator{\gamma^\mu}{\gamma^\nu} = 2\varepsilon^{\mu\nu}
    \end{equation}
    defines a Clifford algebra, \(\operatorname{Cl}_{1,3}(\reals)\), which can be thought of as the space of all linear operators from spacetime, \(V\), to itself, that is, \(\operatorname{End}(V)\).
    We can then think of \(\gamma^\mu\) as the generators of this algebra, or a representation of this algebra.
    \normalshape
    
    The lowest dimensional set of matrices satisfying the required properties\footnote{That is, the lowest dimensional nontrivial representation of \(\operatorname{Cl}_{1,3}(\reals)\).} is
    \begin{alignat}{2}
        \gamma^0 &= 
        \begin{pmatrix}
            1 & 0 & 0 & 0 \\
            0 & 1 & 0 & 0 \\
            0 & 0 & -1 & 0 \\
            0 & 0 & 0 & -1
        \end{pmatrix}
        , \qquad & \gamma^1 &= 
        \begin{pmatrix}
            0 & 0 & 0 & 1 \\
            0 & 0 & 1 & 0 \\
            0 & -1 & 0 & 0\\
            -1 & 0 & 0 & 0
        \end{pmatrix}
        ,\\
        \gamma^0 &= 
        \begin{pmatrix}
            1 & 0 & 0 & 0 \\
            0 & 1 & 0 & 0 \\
            0 & 0 & -1 & 0 \\
            0 & 0 & 0 & -1
        \end{pmatrix}
        , \qquad & \gamma^1 &= 
        \begin{pmatrix}
            0 & 0 & 1 & 0 \\
            0 & 0 & 0 & -1 \\
            1 & 0 & 0 & 0\\
            0 & -1 & 0 & 0
        \end{pmatrix}
        .
    \end{alignat}
    This can be written more compactly as
    \begin{equation}
        \gamma^0 = 
        \begin{pmatrix}
            I & 0\\
            0 & -I
        \end{pmatrix}
        = \sigma^3 \otimes I, \qqand \gamma^i =
        \begin{pmatrix}
            0 & \sigma^i\\
            -\sigma^i
        \end{pmatrix}
        = i\sigma^2 \otimes \sigma^i
    \end{equation}
    where \(I\) is the 2-dimensional identity matrix and \(\sigma^i\) are the \defineindex{Pauli spin matrices}:
    \begin{equation}
        I = 
        \begin{pmatrix}
            1 & 0 \\
            0 & 1
        \end{pmatrix}
        , \quad \sigma^1 = 
        \begin{pmatrix}
            0 & 1\\
            1 & 0
        \end{pmatrix}
        , \quad \sigma^2 = 
        \begin{pmatrix}
            0 & -i\\
            i & 0
        \end{pmatrix}
        , \qand \sigma^3 = 
        \begin{pmatrix}
            1 & 0\\
            0 & -1
        \end{pmatrix}
        .
    \end{equation}
    
    \chapter{Solving the Dirac Equation}
    \section{Spinors}
    The Dirac equation,
    \begin{equation}
        (i\slashed{\partial} - m)\psi = 0,
    \end{equation}
    can be written in terms of \(4\times 4\) matrices, \(\slashed{\partial} = \gamma^\mu\partial_\mu\) and \(m\) (with an implicit factor of the \(4\times 4\) identity, \(I_4\)), acting on \(\psi\).
    This means that \(\psi\) must have four components.
    We can think of it as a \(4 \times 1\) matrix, but it is \emph{not} a four-vector\footnote{That is, it doesn't transform as a four-vector under \(\specialOrthogonal^+(1,3)\), they transform under a different representation which cannot be constructed from tensor products of the fundamental representation in the usual way.}.
    We call objects like this \define{spinors}\index{spinor}.
    
    We make the ansatz that \(\psi\) has the form
    \begin{equation}
        \psi(x) = u(p) \e^{-ip\cdot x},
    \end{equation}
    where \(u(p)\) is some spinor and \(\exp[-ip\cdot x]\) is a phase factor.
    This means that
    \begin{equation}
        \psi(x) = 
        \begin{pmatrix}
            \psi^1(x)\\ \psi^2(x)\\ \psi^3(x)\\ \psi^4(x)
        \end{pmatrix}
        =
        \begin{pmatrix}
            \varphi^1(p)\\ \varphi^2(p)\\ \varphi^3(p)\\ \varphi^4(p)
        \end{pmatrix}
        \e^{-ip\cdot x} = u(p)\e^{-ip\cdot x}.
    \end{equation}
    
    Now notice that \(i\partial_\mu\) acting on \(\e^{-ip\cdot x}\) gives \(p_\mu \e^{-ip\cdot x}\), and in particular \(\partial_\mu\) has no effect on \(u(p)\), since it has no position dependence.
    We can then write the Dirac equation as
    \begin{equation}
        (\slashed{p} - m)u(p) = 0,
    \end{equation}
    having divided through by the phase factor.
    
    We can write this in the significantly less compact form
    \begin{equation}
        \begin{pmatrix}
            E - m & 0 & -p_z & -p_x + ip_y\\
            0 & E - m & -p_x - ip_y & p_z\\
            p_z & p_x - ip_y & -E - m & 0\\
            p_x + ip_y & -p_z & 0 & -E - m
        \end{pmatrix}	
        \begin{pmatrix}
            \varphi^1(p)\\ \varphi^2(p)\\ \varphi^3(p)\\ \varphi^4(p)
        \end{pmatrix}
        = 0.
    \end{equation}
    
    To solve this move to the rest frame of the particle, so \(p_i = 0\), this then becomes
    \begin{equation}
        \begin{pmatrix}
            E - m & 0 & 0 & 0\\
            0 & E - m & 0 & 0\\
            0 & 0 & -E - m & 0\\
            0 & 0 & 0 & -E - m
        \end{pmatrix}	
        \begin{pmatrix}
            \varphi^1(p)\\ \varphi^2(p)\\ \varphi^3(p)\\ \varphi^4(p)
        \end{pmatrix}
        = 0.
    \end{equation}
    This is just an eigenvalue equation, for the matrix \(\diag(E, E, -E, -E)\), and there are four eigenspinors
    \begin{equation}
        u^1 = N
        \begin{pmatrix}
            1\\ 0\\ 0\\ 0
        \end{pmatrix}
        , \quad u^2 = N
        \begin{pmatrix}
            0\\ 1\\ 0\\ 0
        \end{pmatrix}
        , \quad u^3 = N
        \begin{pmatrix}
            0\\ 0\\ 1\\ 0
        \end{pmatrix}
        , \qand u^4 = N
        \begin{pmatrix}
            0\\ 0\\ 0\\ 1
        \end{pmatrix}
        .
    \end{equation}
    Here \(N\) is a normalisation factor, which is usually \(\sqrt{E + m}\) so \(u^\hermit u = 2E\) for all four spinors.
    The first two solutions here correspond to eigenvalues of \(E = +m\), whereas the second two correspond to eigenvalues of \(E = -m\).
    This means that despite Dirac's best attempts we \emph{still} have negative energy solutions.
    The full solution to the Dirac equation for a particle at rest is then
    \begin{equation}
        \psi^1 = N
        \begin{pmatrix}
            1\\ 0\\ 0\\ 0
        \end{pmatrix}
        \e^{-imt}, \quad u^2 = N
        \begin{pmatrix}
            0\\ 1\\ 0\\ 0
        \end{pmatrix}
        \e^{-imt}, \quad u^3 = N
        \begin{pmatrix}
            0\\ 0\\ 1\\ 0
        \end{pmatrix}
        \e^{imt}, \qand u^4 = N
        \begin{pmatrix}
            0\\ 0\\ 0\\ 1
        \end{pmatrix}
        \e^{imt}.
    \end{equation}
    We get four solutions, which is quite different to the Klein--Gordon equation.
    The reason for this is the Klein--Gordon equation represents spin 0 particles, whereas the Dirac equation represents spin \(1/2\) particles, and the four solutions correspond to four types of spin \(1/2\) particles:
    \begin{itemize}
        \item \(u^1\) corresponds to a positive energy spin up solution.
        \item \(u^2\) corresponds to a positive energy spin down solution.
        \item \(u^3\) corresponds to a negative energy spin down solution.
        \item \(u^4\) corresponds to a negative energy spin up solution.
    \end{itemize}
    
    \section{Interpreting Negative Energies}
    \subsection{Dirac Sea}
    The Dirac sea interpretation says that the vacuum state corresponds to all negative energy states being filled.
    The exclusion principle then prevents the entire system from decaying to arbitrarily large negative energies.
    What we view as a particle is then actually a particle being excited to a positive energy state from a negative energy state.
    At the same time this leaves a hole in the negative energy states, which is what we view as an antiparticle.
    
    This interpretation means that the process \(\Pphoton \to \Pe \APe\) is just an electron in a negative energy state absorbing a photon, and being excited to a positive energy state, leaving behind a hole.
    Similarly the process \(\Pe\APe\to\Pphoton\) is the electron falling back to the negative energy state, emitting a photon and filling the hole.
    
    This interpretation lead to Dirac predicting the existence of the positron before it was experimentally discovered.
    
    \subsection{Feynman--St\"uckelberg}
    The more modern interpretation is that the negative energy solutions correspond to negative energy particles propagating backwards in time, which correspond to positive energy antiparticles propagating forward in time.
    This is (part of) the reason we use arrows going backwards in time to represent antiparticles in a Feynman diagram.
    
    In this interpretation the process \(\Pe\APe \to \Pphoton\) corresponds either to an electron and positron annihilating, or to an electron with positive energy coming in, emitting a photon to have negative energy, and then leaving going backwards in time.
    
    \section{General Solution}
    It is possible to solve the Dirac equation in a general frame\footnote{see \course{Quantum Theory} or \course{Quantum Field Theory} for details}, not just the rest frame of the particle.
    The result is again four different spinors, first
    \begin{equation}
        u^1(p) = N
        \begin{pmatrix}
            1\\ 0\\
            \frac{p_z}{E + m}\\
            \frac{p_x + ip_y}{E + m}
        \end{pmatrix}
        , \qqand
        u^2(p) = N
        \begin{pmatrix}
            0\\ 1\\
            \frac{p_x - ip_y}{E + m}\\
            \frac{-p_z}{E + m}
        \end{pmatrix}
    \end{equation}
    where \(E = + \sqrt{m^2 + \vv{p}^2}\).
    Then we also have the negative energy solutions
    \begin{equation}
        u^3(p) = N
        \begin{pmatrix}
            \frac{p_z}{E - m}\\
            \frac{p_x + ip_y}{E - m}\\
            1\\ 0
        \end{pmatrix}
        , \qqand
        u^4(p) = N
        \begin{pmatrix}
            \frac{p_x - ip_y}{E - m}\\
            \frac{-p_z}{E - m}\\
            0\\ 1
        \end{pmatrix}
    \end{equation}
    where \(E = -\sqrt{m^2 + \vv{p}^2}\).
    
    To get rid of negative energy solutions we instead define explicit antiparticle spinors,
    \begin{equation}
        v^1(p) = u^4(-p), \qqand v^2(p) = u^3(-p).
    \end{equation}
    Note that \(4\) and \(1\) go together and \(2\) and \(3\) go together, which is probably not what one would expect, and corresponds to the weirdness with \(u^1\), \(u^2\), \(u^3\), and \(u^4\) representing spin up, down, down, and up solutions respectively.
    The antiparticle spinors are solutions to
    \begin{equation}
        (\slashed{p} \mathbin{\textcolor{highlight}{+}} m)v = 0.
    \end{equation}
    
    \section{Helicity}
    Recall that we have two operators measuring spin:
    \begin{equation}
        \vecoperator{S}^2\ket{\psi} = s(s + 1)\ket{\psi}, \qqand \operator{S}_z\ket{\psi} = m_s\ket{\psi}.
    \end{equation}
    Here \(\operator{S}_z\) measures the spin along the \(z\)-axis.
    Rather than measure along an arbitrarily chosen axis it is usually preferable to measure spin relative to the direction of travel of the particle.
    This can be done by defining the \defineindex{helicity}
    \begin{equation}
        \operator{h} \coloneqq \frac{\vecoperator{S} \cdot \vecoperator{p}}{\abs{\vv{p}}}.
    \end{equation}
    
    For a spin \(1/2\) particle \(m_s\) can be \(\pm 1/2\).
    Similarly \(h\), defined by \(\operator{h}\ket{\psi} = h\ket{\psi}\), can take the values\footnote{some texts define helicity with a factor of 2, so that \(h\) takes values of \(\pm 1\)} \(\pm 1/2\).
    If it takes the value \(1/2\) then the spin is aligned in the direction of travel and we say the particle is \defineindex{right handed}.
    If \(h\) takes the value \(-1/2\) then the spin is antialigned in the direction of travel and we say the particle is \defineindex{left handed}.
    
    Note that the helicity is not a Lorentz invariant, except for in the case where the particle is massless and helicity coincides with chirality.
    
    If we define the direction of the momentum by a unit vector,
    \begin{equation}
        \vh{n} = (\sin\vartheta \cos\varphi, \sin\vartheta\cos\varphi, \cos\vartheta),
    \end{equation}
    then we can work with helicity eigenstates given by
    \begin{alignat}{2}
        u_{\uparrow} &= N \left(
        \begin{array}{r}
            \cos(\vartheta/2)\\
            \e^{i\varphi}\sin(\vartheta/2)\\
            \xi\cos(\vartheta/2)\\
            \xi\e^{i\varphi}\sin(\vartheta/2)
        \end{array}
        \right), \qquad & u_{\downarrow} &= N \left(
        \begin{array}{r}
            -\sin(\vartheta/2)\\
            \e^{i\varphi}\cos(\vartheta/2)\\
            \xi\sin(\vartheta/2)\\
            -\xi\e^{i\varphi}\cos(\vartheta/2)
        \end{array}
        \right),\\
        v_{\uparrow} &= N \left(
        \begin{array}{r}
            \xi\sin(\vartheta/2)\\
            -\xi\e^{i\varphi}\cos(\vartheta/2)\\
            -\sin(\vartheta/2)\\
            \e^{i\varphi}\cos(\vartheta/2)
        \end{array}
        \right), \qquad & v_{\downarrow} &= N \left(
        \begin{array}{r}
            \xi\cos(\vartheta/2)\\
            \xi\e^{i\varphi}\sin(\vartheta/2)\\
            \cos(\vartheta/2)\\
            \e^{i\varphi}\sin(\vartheta/2)
        \end{array}
        \right),
    \end{alignat}
    where \(\xi = \abs{\vv{p}}/(E + m)\).
    
    Notice that the transformation
    \begin{equation}
        \vh{n} \mapsto (\sin(\vartheta + 2\pi)\cos\varphi, \sin(\vartheta + 2\pi), \cos(\vartheta + 2\pi))
    \end{equation}
    doesn't result in the same spinors, instead it gives us back their negatives.
    It requires a \(4\pi\) \enquote{rotation} to take the spinors back to their original value.
    
    One nice thing about these eigenstates is that for low momentum particles they become significantly simpler if we neglect the \(\abs{\vv{p}}/(E + m)\) terms.
    Many of our calculations will use these helicity eigenstates as a basis.
    
    \chapter{Getting Probabilities}
    \section{Interpreting the Spinor Wave Function}
    In nonrelativistic quantum mechanics we have a wave function, \(\psi\), which is a complex number valued function encoding all of the information we have about a quantum system.
    This wave function is the solution to the Schrödinger equation.
    We have the further restriction that \(\psi\) must be square integrable, to allow us to normalise probabilities.
    To extract this information we consider \(\rho(\vv{x}, t) \coloneqq \psi^*(t, \vv{x}) \psi(t, \vv{x}) = \abs{\psi(t, \vv{x})}^2\), which we interpret as a probability density.
    We can also define a three-vector probability current, \(\vv{j}(t, \vv{x})\)
    To do this we consider the rate of change of the total probability in some region, \(V\), of spacetime:
    \begin{equation}
        \diffp{}{t}\int_V \rho \dd{V}.
    \end{equation}
    The rate of change of the total probability is equal to the flux out through the boundary of \(V\), denoted \(\partial V\).
    This is given by 
    \begin{equation}
        \diffp{}{t} \int_V \rho \dd{V} = -\int_{\partial V} \vv{j} \cdot \dl{\vv{S}} = -\int_V \div\vv{j} \dd{V}
    \end{equation}
    where the last step is the divergence theorem.
    We then have the continuity equation
    \begin{equation}
        \diffp{\rho}{t} = -\div\vv{j} \implies \diffp{\rho}{t} + \div\vv{j} = 0.
    \end{equation}
    It is also possible to express \(\vv{j}\) in terms of the wave function:
    \begin{equation}
        \vv{j} = -\frac{i}{2m} \left( \psi^* \grad \psi - \psi \grad \psi^* \right).
    \end{equation}
    We can write this relativistically by defining \(j^\mu = (\rho, \vv{j})\) and then the continuity equation is
    \begin{equation}
        \partial_\mu j^\mu = 0.
    \end{equation}
    
    Things don't change that much when we consider spinors.
    We still have a wave function, \(\psi\), but now it's spinor valued\footnote{Formally if \(\rho \colon \operatorname{Spin}(p, q) \to \generalLinear(V)\) is a representation of the spin group, which is the double cover of \(\specialOrthogonal^+(p, q, \reals)\), which is the (proper orthochronous) Lorentz group when \(p = 1\) and \(q = 3\), then spinors are elements of \(V\).}.
    
    So, to complete our probability finding step we just need to find a way to \enquote{square} a spinor to get a probability density and a probability current.
    Appealing to the normal wave function case the obvious generalisation from the one (complex) dimensional complex-number-valued wave function to the four (in the Dirac representation) dimensional spinor-valued wave function is to replace \(\psi^*\psi\) with \(\psi^\hermit \psi\).
    If we have a four-component spinor
    \begin{equation}
        \psi = 
        \begin{pmatrix}
            \varphi^1\\ \varphi^2\\ \varphi^3\\ \varphi^4
        \end{pmatrix}
        \e^{-ip\cdot x},
    \end{equation}
    then we can define \(\psi^\hermit\) to be
    \begin{equation}
        \psi^\hermit \coloneqq 
        \begin{pmatrix}
            \varphi^{1*} & \varphi^{2*} & \varphi^{3*} & \varphi^{4*}
        \end{pmatrix}
        \e^{ip\cdot x}.
    \end{equation}
    
    Since we want to work relativistically we look directly for \(j^\mu\), rather than finding \(\rho\) and \(\vv{j}\) separately.
    To do so we need to combine our spinor with something with a Lorentz index, in order to get a four vector.
    The obvious choice when it comes to spinors is a gamma matrix, \(\gamma^\mu\).
    After some consideration we see that if we consider \(\psi^\hermit \gamma^\mu \psi\) then, for a fixed value of \(\mu\), this just gives a number, so allowing \(\mu\) to take values \(\mu = 0, 1, 2, 3\) we get a four-vector.
    
    However, there is a problem.
    This quantity is not Lorentz invariant\footnote{see \course{Quantum Field Theory} for details}.
    It turns out that a small modification fixes this, we should instead consider \(j^\mu \coloneqq \psi^\hermit \gamma^0\gamma^\mu \psi\).
    This is a four-vector with components
    \begin{equation}
        j^\mu = \psi^\hermit \gamma^0 \gamma^\mu \psi = (\psi^\hermit \psi, \psi^\hermit \gamma^0\gamma^1\psi, \psi^\hermit \gamma^0\gamma^2\psi, \psi^\hermit \gamma^0\gamma^3\psi),
    \end{equation}
    where we've used \((\gamma^0)^2 = 1\).
    Quantities of the form \(a^\hermit \gamma^0\) are common enough that we define \(\overbar{a} \coloneqq a^\hermit \gamma^0\)\index{\(\overbar{\psi} \coloneqq a^\hermit\gamma^0\)}.
    Hence,
    \begin{equation}
        j^\mu = \overbar{\psi} \gamma^\mu \psi = (\overbar{\psi} \gamma^0 \psi, \overbar{\psi} \gamma^1 \psi, \overbar{\psi} \gamma^2 \psi, \overbar{\psi} \gamma^3 \psi),
    \end{equation}
    which looks more symmetric in time and space.
    We call \(\overbar{\psi}\) the \defineindex{Dirac adjoint} of \(\psi\) (as opposed to the Hermitian adjoint, \(\psi^\hermit\)).
    It is then possible to show that \(\partial_\mu j^\mu = 0\) as expected.
    
    \section{Calculating Probabilities}
    Our goal now is to use Fermi's golden rule,
    \begin{equation}
        \Gamma_{fi} = 2\pi \abs{T_{fi}}^2 \rho(E_i),
    \end{equation}
    to compute transition rates.
    We'll do so using the Lorentz invariant form
    \begin{equation}
        \Gamma_{fi} = \frac{\abs{\vv{p}^*}}{32\pi^2 m_i^2} \int \abs{\amplitude_{fi}}^2 \dd{\Omega}
    \end{equation}
    for a decay rate for a decay \(a \to 1 + 2\), and
    \begin{equation}
        \sigma = \frac{1}{64\pi^2s} \frac{\abs{\vv{p_f}^*}}{\abs{\vv{p_i}^*}} \int \abs{\amplitude_{fi}}^2 \dd{\Omega}
    \end{equation}
    for the cross section for the scattering \(a + b \to 1 + 2\).
    
    Our goal for the next couple of chapters will be to compute the amplitude for the process \(\Pe\Ptau \to \Pe\Ptau\) with a photon as the intermediate state.
    We'll consider the \(t\)-channel process
    \begin{equation}
        \tikzsetnextfilename{fd-electron-positron-scattering}
        \feynmandiagram[vertical=v1 to v2, small, baseline=(current bounding box)]{
            i1 [particle=\Pe] -- [fermion] v1 -- [fermion] o1 [particle=\Pe],
            i2 [particle=\APe] -- [anti fermion] v2 -- [anti fermion] o2 [particle=\APe],
            v1 -- [photon, edge label=\Pphoton] v2
        };
    \end{equation}
    For now we'll focus on kinematic details of an arbitrary \(t\)-channel process, and in the next chapter we'll look at the specifics of this interaction.
    
    To second order in perturbation theory the transition matrix element, \(T_{fi}\), is given by
    \begin{equation}
        T_{fi} = \bra{f} V \ket{i} + \sum_{j \ne i} \frac{\bra{f} V \ket{j} \bra{j} V \ket{i}}{E_i - E_j}.
    \end{equation}
    The first term represents no interaction occurring, but we're considering the \(t\)-channel process, where there is an interaction, so its the second term here that we're interested in.
    
    Now consider the explicitly time-ordered interaction \(a + b \to c + d\) where \(a\) emits some intermediate particle, \(x\), and turns into \(c\), then \(b\) absorbs it and turns into \(d\).
    We can draw this as
    \begin{equation}
        \tikzsetnextfilename{fd-electron-positron-scattering-time-ordered}
        \begin{tikzpicture}[baseline=(current bounding box)]
            \begin{feynman}
                \vertex (i1) at (0, 1) {\(a\)};
                \vertex (i2) at (0, -1) {\(b\)};
                \vertex (v1) at (1, 0.5);
                \vertex (v2) at (1.5, -0.5);
                \vertex (o1) at (3, 1) {\(c\)};
                \vertex (o2) at (3, -1) {\(d\)};
                \diagram* [small] {
                    (i1) -- [fermion] (v1) -- [fermion] (o1),
                    (i2) -- [anti fermion] (v2) -- [anti fermion] (o2),
                    (v1) -- [photon, edge label=\(x\)] (v2)
                };
            \end{feynman}
        \end{tikzpicture}
    \end{equation}
    
    We then have three states, the initial state, \(\ket{a, b}\), the intermediate state, \(\ket{c, x, b}\), and the final state \(\ket{c, d}\).
    The energy of the initial state is \(E_a + E_b\), and the energy of the intermediate state is \(E_c + E_x + E_b\).
    Putting this together we get the contribution to the second term from this time ordering:
    \begin{equation}
        T_{fi}^{ab} = \frac{\bra{c, d} V \ket{c, x, b} \bra{c, x, b} V \ket{a, b}}{(E_a + E_b) - (E_c + E_x + E_b)} = \frac{\bra{c, d} V \ket{c, x, b} \bra{c, x, b} V \ket{a, b}}{E_a - E_c - E_x}.
    \end{equation}
    Here we write \(T_{fi}^{ab}\) for the process \(a + b \to c + d\) specifically time ordered so that \(a\) emits a particle which is then absorbed by \(x\).
    
    Recall that \(T_{fi}\) is given by the amplitude, \(\amplitude_{fi}\), divided by the square root of the product of twice the energies of all particles in the initial and final state.
    We can use this to work out the amplitudes \(\bra{c, d} V \ket{c, x, b}\) and \(\bra{c, x, b} V \ket{a, b}\), which are just transition matrix elements for the processes \(x + b \to d\) and \(a \to c + x\) respectively.
    We then have
    \begin{equation}
        \bra{c, d} V \ket{c, x, b} = \frac{\amplitude_{fi}^{a \to c + x}}{\sqrt{2E_a2E_c2E_x}} = \frac{g_a}{\sqrt{2E_a2E_c2E_x}},
    \end{equation}
    where \(g_a\) measures the strength of the interaction \(a \to c + x\).
    Similarly,
    \begin{equation}
        \bra{c, x, b} V \ket{a, b} = \frac{\amplitude_{fi}^{x + b \to d}}{\sqrt{2E_d2E_x2E_b}} = \frac{g_b}{\sqrt{2E_d2E_x2E_b}},
    \end{equation}
    where \(g_b\) measures the strength of the interaction \(x + b \to d\).
    
    We can then compute the total amplitude for this particular time ordering of this process.
    First we compute the total transition matrix element:
    \begin{equation}
        T_{fi}^{ab} = \frac{g_ag_b}{2E_x\sqrt{2E_a2E_b2E_c2E_d}} \frac{1}{\sqrt{E_a - E_x - E_c}},
    \end{equation}
    and then the amplitude is
    \begin{equation}
        \amplitude_{fi}^{ab} = g_ag_b \sqrt{2E_a2E_b2E_c2E_d} T_{fi}^{ab} = \frac{g_ag_b}{2E_x(E_a - E_c - E_x)}.
    \end{equation}
    Note that \(\amplitude_{fi}^{ab}\) is \emph{not} Lorentz invariant, since we are explicitly defining a time ordering for the processes occurring, which a Lorentz transformation may change.
    
    We can similarly consider the process where \(b\) emits the antiparticle \(\overbar{x}\), which is then absorbed by \(a\):
    \begin{equation}
        \tikzsetnextfilename{fd-electron-positron-scattering-alternate-order}
        \begin{tikzpicture}[baseline=(current bounding box)]
            \begin{feynman}
                \vertex (i1) at (0, 1) {\(a\)};
                \vertex (i2) at (0, -1) {\(b\)};
                \vertex (v1) at (1.5, 0.5);
                \vertex (v2) at (1, -0.5);
                \vertex (o1) at (3, 1) {\(c\)};
                \vertex (o2) at (3, -1) {\(d\)};
                \diagram* [small] {
                    (i1) -- [fermion] (v1) -- [fermion] (o1),
                    (i2) -- [anti fermion] (v2) -- [anti fermion] (o2),
                    (v1) -- [photon, edge label=\(\overbar{x}\)] (v2)
                };
            \end{feynman}
        \end{tikzpicture}
    \end{equation}
    An identical calculation gives
    \begin{equation}
        \amplitude_{fi}^{ba} = \frac{g_ag_b}{2E_x(E_b - E_d - E_x)}.
    \end{equation}
    
    The total amplitude for this \(t\)-channel process is
    \begin{align}
        \amplitude_{fi}^t &= \amplitude_{fi}^{ab} + \amplitude_{fi}^{ba}\\
        &= \frac{g_ag_b}{2E_x}\left( \frac{1}{E_a - E_c - E_x} + \frac{1}{E_b - E_d - E_x} \right).
    \end{align}
    Conservation of energy tells us that \(E_a + E_b = E_c + E_d\), and so we can write this as
    \begin{align}
        \amplitude_{fi}^t &= \frac{g_ag_b}{2E_x}\left( \frac{1}{E_a - E_c - E_x} - \frac{1}{E_a - E_c + E_x} \right)\\
        &= \frac{g_ag_b}{2E_x} \frac{2E_x}{(E_a - E_c)^2 - E_x^2}\\
        &= \frac{g_ag_b}{(E_a - E_c)^2 - E_x^2}.
    \end{align}
    The relativistic energy-momentum relation tells us that \(E_x^2 = \vv{p_x}^2 + m_x^2\) and if we consider the first time ordering then we have \(\vv{p_x} = \vv{p_a} - \vv{p_c}\), so
    \begin{align}
        \amplitude_{fi}^t &= \frac{g_ag_b}{(E_a - E_c)^2 - (\vv{p_a} - \vv{p_c})^2 - m_x^2}\\
        &= \frac{g_ag_b}{(p_a - p_b)^2 - m_x}\\
        &= \frac{g_ag_b}{q^2 - m_x^2}\\
        &= \frac{g_ag_b}{t - m_x^2}.
    \end{align}
    Here \(q = p_a - p_c\) is the four-momentum of the exchange particle and \(t\) is the Mandelstam invariant.
    This is a fairly common form for the amplitude to take.
    
    Notice that we do \emph{not} require \(q^2 = m_x^2\), in fact, if this were the case then the amplitude would be singular.
    Suppose we have elastic scattering, then \(p_a = (E, \vv{p_a})\) and \(p_c = (E, \vv{p_c})\), and so we have
    \begin{equation}
        q^2 = (p_a - p_c)^2 = (E - E)^2 - (\vv{p_a} - \vv{p_c})^2 = -(\vv{p_a} - \vv{p_c})^2 \le 0.
    \end{equation}
    So we don't even need to have \(q^2\) be positive.
    This is all because \(x\) is a \defineindex{virtual particle}, which is a particle which is \defineindex{off-shell}, meaning \(q^2 \ne m_x^2\).
    If \(q^2 = m_x^2\) then we say that the particle is \defineindex{on-shell}.
    We cannot have virtual particles in the initial and final states, but there perfectly valid intermediate states.
    This is mostly just an artefact of us reading too literally into what's happening in a process defined by a Feynman diagram.
    
    \chapter{Quantum Electrodynamics}
    \section{Modifying the Dirac Equation}
    We can modify the Dirac equation to allow for electromagnetic interactions by the \defineindex{minimal coupling prescription}\footnote{See \course{Quantum Theory}}, where we modify the momentum and energy according to
    \begin{equation}
        \vv{p} \mapsto \vv{p} - q\vv{A}, \qqand E \mapsto E - q\varphi
    \end{equation}
    where \(q\) is the charge of the particle, \(\vv{A}\) is the vector potential and \(\varphi\) the electrostatic potential.
    We can write this modification as
    \begin{equation}
        p^\mu \mapsto p^\mu - qA^\mu
    \end{equation}
    where \(A^\mu = (\varphi, \vv{A})\) is the four-potential\footnote{See \course{Classical Electrodynamics}}.
    We then make the corresponding modification to the quantum mechanical operators, \(\vv{p} = -i\grad\) and \(E = i\partial_t\), or \(p^\mu = i\partial^\mu\) to get
    \begin{equation}
        p^\mu = i\partial^\mu \mapsto p^\mu - qA^\mu = i\partial^\mu - qA^\mu.
    \end{equation}
    Hence, the Dirac equation in the presence of an electric field is modified according to
    \begin{equation}
        (i\slashed{\partial} - m)\psi = 0 \mapsto (i\slashed{\partial} - q\slashed{A} - m)\psi = 0.
    \end{equation}
    
    As motivation consider what we get if we rearrange this slightly.
    First multiply through by \(\gamma^0\) to get
    \begin{equation}
        i\gamma^0 \slashed{\partial}\psi - q\gamma^0\slashed{A}\psi - m\gamma^0\psi = 0.
    \end{equation}
    Then using \((\gamma^0)^2 = 1\) we have
    \begin{align}
        0 &= i\gamma^0\slashed{\partial}\psi - q\gamma^0\slashed{A}\psi - m\gamma^0\psi\\
        &= i\gamma^0\gamma^\mu\partial_\mu\psi - q\gamma^0\slashed{A}\psi - m\gamma^0\psi\\
        &= i(\gamma^0)^2\partial_t\psi - i\gamma^0\gamma^i\partial_i\psi - q\gamma^0\slashed{A} - m\gamma^0\psi\\
        &= i\diffp{\psi}{t} - i\gamma^0\vv{\gamma} \cdot \grad\psi - q\gamma^0 \slashed{A} - m\gamma^0\psi.
    \end{align}
    Moving the first term to the other side we have
    \begin{equation}
        i\diffp{\psi}{t} = i\gamma^0\vv{\gamma} \cdot \grad \psi + m\gamma^0\psi + q\gamma^0\slashed{A}.
    \end{equation}
    We then interpret the left hand side, \(i\partial_t \psi\), as the Hamiltonian acting on the spinor, \(\operator{H}\psi\).
    The terms on the right then correspond to the kinetic energy, mass-energy, and potential energy respectively.
    
    We can then write the potential energy operator for a spinor satisfying the Dirac equation in an electromagnetic field as
    \begin{equation}
        V_{\dirac} = q \gamma^0\slashed{A}.
    \end{equation}
    Note that the time-like contribution, \(\mu = 0\), to \(q\gamma^0\gamma^\mu A_\mu\) is \(q(\gamma^0)^2A_0 = q\varphi\), which is just the usual electrostatic potential.
    
    \section{Photons}
    Before we get to the calculation we should consider what's happening in the interaction.
    We start with two fermions, these are described by the Dirac equation.
    We end with two fermions, also described by the Dirac equation.
    In between we'll have a photon, which is a spin 1 boson.
    Noninteracting photons are described by Maxwell's equation\footnote{for relativistic use of Maxwell's equations see \course{Classical Electrodynamics}, for use in relativistic quantum theory see \course{Quantum Field Theory}}, \(\dalembertian A^\nu - \partial^\nu\partial_\mu A^\mu = 0\).
    This has plane wave solutions,
    \begin{equation}
        A_\mu = \varepsilon_\mu^{(\lambda)} \e^{-ip\cdot x}.
    \end{equation}
    Here \(\varepsilon_\mu^{(\lambda)}\) is the polarisation vector, with \(\lambda\) labelling the polarisation state.
    This is just a normal four vector, since \(A_\mu\) is just a four-vector.
    Since polarisation is a purely spatial phenomenon the time component of \(\varepsilon^{(\lambda)}\) is zero (at least we can always choose for it to be such).
    For massless particles, like the photon, it's not possible for them to be polarised in the direction of travel, since this would, intuitively, require the field to oscillate in the direction of travel, so when oscillating forward it would propagate a signal faster than the speed of light.
    There are therefore two polarisation states,
    \begin{equation}
        \varepsilon^{(1)} = 
        \begin{pmatrix}
            0\\ 1\\ 0\\ 0
        \end{pmatrix}
        , \qqand \varepsilon^{(2)} = 
        \begin{pmatrix}
            0\\ 0\\ 1\\ 0
        \end{pmatrix}
        .
    \end{equation}
    A virtual photon has two more polarisation states, it can be polarised in the time direction or in the direction of travel, corresponding to the states
    \begin{equation}
        \varepsilon^{(0)} = 
        \begin{pmatrix}
            1\\ 0\\ 0\\ 0
        \end{pmatrix}
        , \qqand \varepsilon^{(3)} = 
        \begin{pmatrix}
            0\\ 0\\ 0\\ 1
        \end{pmatrix}
    \end{equation}
    respectively.
    A general photon state will be a superposition of polarisations, meaning the solution will be of the form
    \begin{equation}
        A^\mu = \sum_\lambda \varepsilon_\mu^{(\lambda)} \e^{-ip\cdot x}.
    \end{equation}
    
    \section{Calculating the Amplitude}
    Recall that the amplitude for \(t\)-channel scattering \(a + b \to c + d\) is
    \begin{equation}
        \amplitude_{fi}^t = \frac{g_ag_b}{q^2 - m_x^2}.
    \end{equation}
    The values \(g_a\) and \(g_b\) are supposed to stand for the strength of the interactions \(a \to x + c\) and \(x + b \to d\) respectively.
    They are simply the matrix elements for these processes:
    \begin{equation}
        g_a = \bra{\psi_c} V \ket{\psi_a}, \qqand g_b = \bra{\psi_d} V \ket{\psi_b}.
    \end{equation}
    Here \(\ket{\psi}\) is the spinor state corresponding to the relevant particles.
    
    We now consider the process \(\Pe\Ptau \to \Pe\Ptau\) via a single photon as an intermediate state in a \(t\)-channel diagram.
    Since the exchange particle is a photon it is massless, and so we can replace the \(1/(q^2 - m_x^2)\) term with \(1/q^2\).
    The potential is replaced with the potential \(V_{\dirac} = q\gamma^0\slashed{A}\).
    The initial state corresponds to an electron and a tau particle, and so does the final state, although the momenta may have changed (although the total momentum is of course conserved).
    Hence,
    \begin{equation}
        g_{\Pe} = \bra{\psi_{\Pe,c}} V \ket{\psi_{\Pe,a}}.
    \end{equation}
    The state for the electron before and after then interaction is
    \begin{equation}
        \ket{\psi_{\Pe,a}} = u(p_a), \qqand \ket{\psi_{\Pe,c}} = u(p_c).
    \end{equation}
    Hence, we have
    \begin{equation}\label{eqn:electron-photon interaction strength}
        g_{\Pe} = u^\hermit(p_c) e\gamma^0\slashed{A} u(p_a) = e\diracadjoint{u}(p_c) \slashed{A} u(p_a).
    \end{equation}
    Similarly for the tau particle we have
    \begin{equation}
        g_{\Ptau} = e\diracadjoint{u}(p_d) \slashed{A}^* u(p_b).
    \end{equation}
    The only difference is the complex conjugate here.
    This arises because one particle emits the photon and the other absorbs it, however these processes are related by time reversal, and when we reverse time phase factors change as
    \begin{equation}
        \e^{-iEt} \mapsto \e^{-iE(-t)} = \e^{iEt} = (\e^{-iEt})^*,
    \end{equation}
    so we have to take complex conjugates\footnote{See \course{Quantum Field Theory} for details}.
    We don't know which of the particles absorbs the photon and which emits it, but it doesn't matter, the important thing is the relative complex conjugate.
    Putting this together we have
    \begin{equation}
        \amplitude_{fi}^{t} = e^2\diracadjoint{u}(p_c) \gamma^0\slashed{A} u(p_b) \frac{1}{q^2} \diracadjoint{u}(p_d) \gamma^0\slashed{A}^* u(p_d).
    \end{equation}
    Now we can deal with the \(\slashed{A}\) terms.
    First notice that \(A_\mu\) is just a number, its the component of some four-vector, and so commutes with everything.
    Since it is the same photon that leaves one particle and is absorbed by the other, with no other interactions, its polarisation cannot change.
    This means that our sum over polarisations should be just a single sum, rather than summing each factor of \(A_\mu\) over polarisations, so we have
    \begin{equation}
        A_\mu A_\nu^* = \sum_{\lambda} \varepsilon_\mu^{(\lambda)}\e^{-ip\cdot x}(\varepsilon_\nu^{(\lambda)} \e^{-ip\cdot x})^* = \sum_{\lambda} \varepsilon_\mu^{(\lambda)} (\varepsilon_\nu^{(\lambda)})^*.
    \end{equation}
    We then have
    \begin{align}
        \amplitude_{fi}^t &= e^2 \diracadjoint{u}(p_c) \gamma^0\gamma^\mu A_\mu u(p_b) \frac{1}{q^2} \diracadjoint{u}(p_d) \gamma^0\gamma^\nu A_\nu^* u(p_d)\\
        &= \sum_\lambda e^2 \diracadjoint{u}(p_c) \gamma^0 \gamma^\mu u(p_b) \frac{\varepsilon_\mu^{(\lambda)} \varepsilon_\nu^{(\lambda)}}{q^2} \diracadjoint{u}(p_d) \gamma^0\gamma^\nu u(p_d)\\
        &= -e^2 \diracadjoint{u}(p_c) \gamma^0 \gamma^\mu u(p_b) \frac{\minkowskiMetric_{\mu\nu}}{q^2} \diracadjoint{u}(p_d) \gamma^0\gamma^\nu u(p_d).\label{eqn:electron tau scattering amplitude}
    \end{align}
    In the last step we used the result
    \begin{equation}
        \sum_\lambda \varepsilon_\mu^{(\lambda)}(\varepsilon_\nu)^{(\lambda)} = -g_{\mu\nu}
    \end{equation}
    derived in \cref{eqn:spin sum}.
    
    
    % Appdendix
    \appendixpage
    \begin{appendices}
        \chapter{List of Particles}
\section{Fundamental Particles}
\subsection{Fermions}
\subsubsection{Leptons}
\begin{center}
    \begin{tabular}{lrrSl}\toprule
        Particle & Symbol & Charge & Mass & \\ \hline
        Electron & \Pe & \(-1\) & 511.0 & \si{\kilo\electronvolt} \\
        Muon & \Pmu & \(-1\) & 105.6 & \si{\mega\electronvolt} \\
        Muon & \Ptau & \(-1\) & 1776.9 & \si{\mega\electronvolt} \\
        Electron Neutrino & \Pnue & \(0\) & < 0.120 & \si{\electronvolt} \\
        Muon Neutrino & \Pnumu & \(0\) & < 0.120 & \si{\electronvolt} \\
        Tau Neutrino & \Pnutau & \(0\) & < 0.120 & \si{\electronvolt} \\
        \bottomrule
    \end{tabular}
\end{center}

\subsubsection{Quarks}
\begin{center}
    \begin{tabular}{lrrSl}\toprule
        Particle & Symbol & Charge & Mass & \\ \hline
        Up Quark & \Pu & \(+2/3\) & 2.3 & \si{\mega\electronvolt} \\
        Down Quark & \Pd & \(-1/3\) & 4.8 & \si{\mega\electronvolt} \\
        Charm Quark & \Pc & \(+2/3\) & 1.275 & \si{\giga\electronvolt} \\
        Strange Quark & \Ps & \(-1/3\) & 95 & \si{\mega\electronvolt} \\
        Top Quark & \Pt & \(+2/3\) & 173.2 & \si{\giga\electronvolt} \\
        Bottom Quark & \Pb & \(-1/3\) & 4.18 & \si{\giga\electronvolt} \\
        \bottomrule
    \end{tabular}
\end{center}

\subsubsection{Vector Bosons}
\begin{center}
    \begin{tabular}{lrrSl}\toprule
        Particle & Symbol & Charge & Mass & \\ \hline
        Photon & \Pphoton & \(0\) & 0 & \\
        Gluon & \Pg & \(0\) & 4.8 & \\
        \PZ{} boson & \PZ & \(0\) & 91.19 & \si{\giga\electronvolt} \\
        \PWpm{} boson & \PWpm & \(\pm 1\) & 80.38 & \si{\giga\electronvolt} \\
        \bottomrule
    \end{tabular}
\end{center}
Note that recent measurements place the mass of the \PWpm{} boson at the slightly higher value of \qty{80.43}{\giga\electronvolt}.

\subsubsection{Scalar Bosons}
\begin{center}
    \begin{tabular}{lrrSl}\toprule
        Particle & Symbol & Charge & Mass & \\ \hline
        Higgs Boson & \PH & \(0\) & 125.25 & \si{\giga\electronvolt} \\
        \bottomrule
    \end{tabular}
\end{center}
        \chapter{Fermi's Golden Rule}
\label{app:fermi's golden rule}
\begin{rmk}
    This derivation is repeated, in varying levels of detail with differing formalisms, in \course{Principles of Quantum Mechanics} and \course{Quantum Theory}.
\end{rmk}
\section{Derivation to First Order}
Consider the Hamiltonian
\begin{equation}
    \operator{H} = \operator{H}_0 + \operator{H}'(t, \vv{x})
\end{equation}
where \(\operator{H}_0\) is a time independent Hamiltonian for which we can solve the Schrödinger equation and \(\operator{H}'(t, \vv{x})\) is the interaction Hamiltonian.
Let \(\varphi_k(t, \vv{x})\) be a normalised solution to the Schrödinger equation for the unperturbed Hamiltonian, that is
\begin{equation}
    \operator{H}_0\varphi_k = E_k\varphi_k, \qqand \braket{j}{k} = \delta_{jk}
\end{equation}
where \(\varphi_k(\vv{x}) = \braket{\vv{x}}{k}\).

The Schrödinger equation in the presence of the interaction Hamiltonian is
\begin{equation}
    i\diffp{\psi}{t} = [\operator{H}_0 + \operator{H}'(t, \vv{x})]\psi.
\end{equation}
The wave function, \(\psi\), can be expressed in terms of the complete set of states of the eigenstates of the unperturbed Hamiltonian:
\begin{equation}
    \psi(t, \vv{x}) = \sum_k c_k(t) \varphi_k(\vv{x}) \e^{-iE_kt}.
\end{equation}
Substituting this into the Schrödinger equation we get
\begin{equation}
    i \sum_k \left[ \diff{c_k}{t} \varphi_k \e^{-iE_kt} - iE_kc_k\varphi_k\e^{-iE_kt} \right] = \sum_k \left[ c_k\operator{H}_0\varphi_k \e^{-iE_kt} + \operator{H}'c_k\varphi_k\e^{-iE_kt} \right]
\end{equation}
Using \(\operator{H}_0\varphi_k = E_k\varphi_k\) the second term on the left cancels with the first on the right and we're left with
\begin{equation}\label{eqn:fermi's golden rule DE for cf}
    i \sum_k \diff{c_k}{t} \varphi_k\e^{-iE_kt} = \sum_k \operator{H}'c_k(t) \varphi_k \e^{-iE_kt}.
\end{equation}

Suppose that at time \(t = 0\) the initial state is \(\varphi_i\), and the coefficients are \(c_k(0) = \delta_{ik}\), that is \(c_k(0) = 0\) for all \(k \ne i\).
Suppose also that the perturbing Hamiltonian is constant for \(t > 0\), we can imagine switching it on at time \(t = 0\) and then leaving it on, and that its small enough that at all times \(c_i(t) \approx 1\) and \(c_k(t) \approx 0\) for \(k \ne i\).
Then, we have
\begin{equation}
    i \sum_k \diff{c_k}{t} \varphi_k \e^{-iE_kt} \approx \operator{H}'\varphi_i\e^{-iE_it}.
\end{equation}
Rewriting this in terms of kets we have
\begin{equation}
    i \sum_k \diff{c_k}{t} \e^{-iE_kt} \ket{k} \approx \e^{-iE_it}\operator{H}'\ket{i}.
\end{equation}
Taking the product with some particular final state, \(\bra{f}\), we have
\begin{equation}
    i \sum_k \diff{c_k}{t} \e^{-iE_kt} \braket{\varphi_f}{\varphi_i} = i\diff{c_f}{t} \e^{-iE_ft} \approx \e^{-iE_it} \bra{f} \operator{H}' \ket{i}.
\end{equation}
Rearranging this we get
\begin{equation}
    \diff{c_f}{t} = -i \bra{f} \operator{H}' \ket{i} \e^{i(E_f - E_i)t}.
\end{equation}
Here we use the usual inner product
\begin{equation}
    \bra{f} \operator{H}' \ket{i} = \int \varphi_f^*(\vv{x}) \operator{H}' \varphi_i(\vv{x}) \dd{^3\vv{x}}.
\end{equation}

Define the transition matrix element, \(T_{fi} = \bra{f} \operator{H}' {i}\).
At time \(t\) the amplitude for transitions to the state \(\ket{f}\) is given by
\begin{equation}
    c_f(t) = -i \int_0^t T_{fi} \e^{i(E_f - E_i)t'} \dd{t'}.
\end{equation}
If the perturbing Hamiltonian is time independent then so is \(\bra{f} \operator{H}' \ket{i}\), and so
\begin{equation}\label{eqn:fermi's golden rule cf}
    c_f(t) = -iT_{fi} \int_0^t \e^{i(E_f - E_i)t'} \dd{t'}.
\end{equation}
The probability to transition to the state \(\ket{f}\) if we start in the state \(\ket{i}\) is then
\begin{equation}
    \probability(f \to i) = c_f^*(t)c_f(t) = \abs{T_{fi}}^2 \int_{0}^{t} \int_{0}^{t} \e^{i(E_f - E_i)t'} \e^{-i(E_f - E_i)t''} \dd{t'} \dd{t''}.
\end{equation}

The transition rate, \(\dl{\Gamma_{fi}}\), to transition from the given initial state to the particular final state, \(\ket{f}\), is then
\begin{equation}
    \dl{\Gamma_{fi}} = \frac{1}{t} \probability(f \to i) = \frac{1}{t} \abs{T_{fi}}^2 \int_0^t \int_0^t \e^{i(E_f - E_i)t'} \e^{-i(E_f - E_i)t''} \dd{t'} \dd{t''}.
\end{equation}
We can make the substitution \(t' \to t' + t/2\) and \(t'' \to t'' + t/2\).
This shifts the integration limits, without changing the integrand, since
\begin{equation}
    \e^{kt'}\e^{-kt''} \to \e^{kt' + kt/2}\e^{-kt'' - kt/2} = \e^{kt'}\e^{kt''},
\end{equation}
with \(k = i(E_f - E_i)\).

Performing this integral we get
\begin{equation}
    4\sinc^2\left( \frac{t}{2}[E_f - E_i] \right).
\end{equation}
\begin{cde}{}{}
    \begin{lstlisting}[gobble=8, language=mathematica, mathescape]
        Integrate[
        Integrate[
        Exp[I k t1] Exp[-I k t2],
        {t1, -t/2, t/2}],
        {t2, -t/2, t/2},
        Assumptions -> {k $\in$ Reals, t $\in$ Reals}]
        4 Sin[kt/2]^2/k^2
    \end{lstlisting}
\end{cde}
It is well known\footnote{see \course{Methods of Mathematical Physics}} that \(\sinc^2 x\) approximates \(\delta(x)\).
This means that the transition rate is only significant to states with \(E_f \approx E_i\).
This is energy conservation within the limits of the uncertainty relation, \(\Delta E \, \Delta t \approx 1/2\).
We use this to symmetrically extend the limits to \(\pm \infty\):
\begin{equation}
    \dl{\Gamma_{fi}} = \abs{T_{fi}}^2 \lim_{t \to \infty} \left[ \frac{1}{t} \int_{-t/2}^{t/2} \int_{-t/2}^{t/2} \e^{i(E_f - E_i)t'} \e^{-i(E_f - E_i)t''} \dd{t'} \dd{t''} \right]
\end{equation}

We can then recognise the integral representation of the Dirac delta:
\begin{equation}
    \delta(x) = \frac{1}{2\pi} \int_{-\infty}^{\infty} \int_{-\infty}^{\infty} \e^{ipx} \dd{p}.
\end{equation}
This comes from realising that the Fourier transform of \(\delta(x)\) is
\begin{equation}
    \fourierTransform \{\delta(x)\} = \int_{-\infty}^{\infty} \delta(x) \e^{-ipx} \dd{x} = \e^{0} = 1,
\end{equation}
and then the above expression for \(\delta(x)\) is simply \(\inverseFourierTransform\{\fourierTransform\{\delta(x)\}\}\).

We can use this to write
\begin{equation}
    \dl{\Gamma_{fi}} = 2\pi\abs{T_{fi}}^2 \lim_{t \to \infty} \left[ \frac{1}{t} \int_{-t/2}^{t/2} \e^{i(E_f - E_i)t'} \delta(E_f - E_i) \dd{t'} \right].
\end{equation}
Suppose there are \(\dl{n}\) accessible final states in the range \([E_f, E_f + \dl{E_f}]\).
Then the total transition rate is given by integrating over the final states with accessible energy, which we convert to an integral over their energies using the density of states\footnote{see the \course{Statistical Mechanics} part of the \course{Thermal Physics} course.}:
\begin{align}
    \Gamma_{fi} &= 2\pi \int \abs{T_{fi}}^2 \diff{n}{E_f} \delta(E_f - E_i) \lim_{t \to \infty} \left[ \frac{1}{t} \int_{-t/2}^{t/2} \dl{t} \right] \dd{E_f}\\
    &= 2\pi \int \abs{T_{fi}} \diff{n}{E_f} \delta(E_f - E_i) \dd{E_f}\\
    &= 2\pi \abs{T_{fi}} \diff{n}{E_f}[E_i].
\end{align}
The last term here is the density of states:
\begin{equation}
    \rho(E_i) = \diff{n}{E_f}[E_i].
\end{equation}
Thus, we have derived Fermi's golden rule
\begin{equation}
    \Gamma_{fi} = 2\pi \abs{T_{fi}}^2 \rho(E_i).
\end{equation}

\section{Improvement to Second Order}
To first order we have
\begin{equation}
    T_{fi} = \bra{f} \operator{H}' \ket{i}.
\end{equation}
We assumed that \(c_k(t) \approx 0\) for \(k \ne i\).
An improved derivation would again take \(c_i(t) \approx 1\) and substitute the expression for \(c_k(t)\) from \cref{eqn:fermi's golden rule cf} and substitute it into \cref{eqn:fermi's golden rule DE for cf}.
Then again taking the inner product with \(\ket{f}\) we get
\begin{equation}
    \diff{c_f}{t} \approx -i \bra{f} \operator{H} \ket{i} \e^{i(E_f - E_i)t} + (-i)^2 \sum_{k \ne i} \bra{f} \operator{H}'\ket{k} \e^{i(E_f - E_k)t} \int_{0}^{t} \bra{k} \operator{H}' \ket{i} \e^{i(E_k - E_i)t'} \dd{t'}.
\end{equation}

The perturbation is not present at time \(t = 0\) and for \(t > 0\) the perturbation is constant, which lets us write
\begin{equation}
    \int_0^t \bra{k} \operator{H}' \ket{i} \e^{i(E_k - E_i)t'} \dd{t'} = \bra{k} \operator{H}'\ket{i} \frac{\e^{i(E_k-E_i)t}}{i(E_k - E_i)}.
\end{equation}
This gives us an improved approximate differential equation for the coefficients \(c_f(t)\):
\begin{equation}
    \diff{c_f}{t} \approx -i \left[ \bra{f} \operator{H}' \ket{i} + \sum_{k \ne i} \frac{\bra{f} \operator{H}' \ket{k} \bra{k} \operator{H}' \ket{i}}{E_i - E_k} \right] \e^{i(E_f - E_i)t}.
\end{equation}
Then, to second order, the transition matrix elements are
\begin{equation}
    T_{fi} = \bra{f} \operator{H}' \ket{i} + \sum_{k \ne i} \frac{\bra{f} \operator{H}' \ket{k} \bra{k} \operator{H}' \ket{i}}{E_i - E_k}.
\end{equation}
We interpret the first term as a direct transition, \(\ket{i} \to \ket{f}\).
The second term then corresponds to some indirect transition, \(\ket{i} \to \ket{k} \to \ket{f}\), where we sum over all possible intermediate states, \(\ket{k}\).
The next higher order term would then allow two intermediate states, and so on.
        \chapter{Gauge Boson Polarisations}
\section{Classical Electrodynamic Gauge Invariance}
Maxwell's equations are given by
\begin{equation}
    \partial_\mu F^{\mu\nu} = j^\nu \qqand \partial_\mu F^{*\mu\nu} = 0
\end{equation}
where
\begin{equation}
    F^{\mu\nu} \coloneqq \partial^\mu A^\nu - \partial^\nu A^\mu
\end{equation}
is the electric field strength tensor, \(A^\mu = (\varphi, \vv{A})\) is the four-potential, \(j^\nu = (\rho, \vv{j})\) is the four-current, and
\begin{equation}
    F^{*\mu\nu} = \frac{1}{2}\varepsilon^{\mu\nu\rho\sigma}F_{\rho\sigma}
\end{equation}
is the dual field strength tensor.

The first equation can be expanded in terms of \(A\) to get
\begin{equation}
    \partial_\mu\partial^\mu A^\nu - \partial_\mu\partial^\nu A^\mu = \dalembertian A^\nu - \partial^\nu\partial_\mu A^\mu = j^\nu.
\end{equation}

We know that in electromagnetism we have a gauge freedom where the transformation
\begin{equation}
    \varphi \mapsto \varphi' - \diffp{\chi}{t}, \qqand \vv{A} \mapsto \vv{A} + \grad \chi
\end{equation}
for some function \(\chi\) with continuous second derivatives doesn't change Maxwell's equations.
This can easily be shown by substituting these transformed potentials into Maxwell's equations, through \(\vv{B} = \curl (\vv{A} + \grad\chi)\) and \(\vv{E} = -\grad\varphi -\partial_t(\vv{A} + \grad\chi)\) and then using the identities \(\div(\curl\vv{X}) = 0\) and \(\curl(\grad f) = 0\).

Relativistically this gives the gauge freedom
\begin{equation}
    A_\mu \mapsto A_\mu - \partial_\mu \chi.
\end{equation}
Under this we have
\begin{equation}
    \partial_\mu A^\mu \mapsto \partial_\mu A'^\mu = \partial_\mu (A^\mu - \partial^\mu \chi) = \partial_\mu A^\mu - \dalembertian \chi.
\end{equation}
We can choose \(\dalembertian\chi = \partial_\mu A^\mu\), then our transformed potential has \(\partial_\mu A'^\mu = 0\).
This is called the \defineindex{Lorenz gauge} condition.
In this gauge Maxwell's equation becomes
\begin{equation}
    \dalembertian A^\mu = j^\mu.
\end{equation}


\section{Photon Polarisations}
In a vacuum, \(j^\nu = 0\), the photon field satisfies \(\dalembertian A^\mu = 0\).
This has plane wave solutions:
\begin{equation}
    A^\mu = \varepsilon^\mu(q) \e^{-iq\cdot x}
\end{equation}
where \(q\) is the four-momentum of the photon and \(\varepsilon^\mu(p)\) describes the polarisation of the electromagnetic field.
Substituting this into the equation of motion we get
\begin{equation}
    \dalembertian A^\mu = -q^2\varepsilon^\mu(q)\e^{-iq\cdot x} = 0.
\end{equation}
Hence the plane wave solutions must have \(q^2 = 0\) for a free photon field.

A spin 1 boson, such as the photon, has three degrees of freedom, spins \(-1\), \(0\), and \(+1\).
It's not immediately obvious how these correspond to the four degrees of freedom in defining \(\varepsilon^\mu\).
The solution is that gauge freedom fixes one of the degrees of freedom.
In the Lorenz gauge we must have \(\partial_\mu A^\mu = 0\), and so
\begin{equation}
    0 = \partial_\mu (\varepsilon^\mu \e^{-iq\cdot x}) = -iq_\mu\varepsilon^\mu\e^{-iq \cdot x}.
\end{equation}
We must therefore have
\begin{equation}
    q_\mu \varepsilon^\mu = 0,
\end{equation}
fixing one degree of freedom in \(\varepsilon^\mu\).

There is still further freedom to make the gauge transformation
\begin{equation}
    A_\mu \mapsto A_\mu - \partial_\mu \Lambda(x)
\end{equation}
where \(\Lambda\) is some function satisfying \(\dalembertian\Lambda = 0\).
Consider the gauge transformation defined by \(\Lambda = -ia\e^{-iq \cdot x}\), which satisfies \(\dalembertian\Lambda = -q^2\Lambda = 0\), since \(q^2 = 0\).
Under this transformation \(A_\mu\) becomes
\begin{align}
    A_\mu \mapsto A'_\mu &= A_\mu - \partial_\mu\Lambda\\
    &= \varepsilon_\mu \e^{-iq\cdot x} + ia\partial_\mu\e^{-iq\cdot x}\\
    &= \varepsilon_\mu \e^{-iq\cdot x} + ia(-iq_\mu)\e^{-iq\cdot x}\\
    &= (\varepsilon_\mu + aq_\mu)\e^{-iq\cdot x}.
\end{align}
We can then interpret this gauge freedom as the ability to make the transformation
\begin{equation}
    \varepsilon_\mu \mapsto \varepsilon_\mu + aq_\mu
\end{equation}
without changing the physics.
The \defineindex{Coulomb gauge} corresponds to the choice of \(a\) making the time component of the polarisation, \(\varepsilon_0\), vanish.
The Lorenz gauge condition, \(\varepsilon_\mu q^\mu\), then becomes \(\vv{\varepsilon} \cdot \vv{q} = 0\).

These requirements are satisfied by
\begin{equation}
    \varepsilon^{(1)} = 
    \begin{pmatrix}
        0\\ 1\\ 0\\ 0
    \end{pmatrix}
    , \qqand \varepsilon^{(2)} = 
    \begin{pmatrix}
        0\\ 0\\ 1\\ 0
    \end{pmatrix}
    .
\end{equation}

\section{Polarisation of Massive Spin-1 Particles}
A massless noninteracting spin 1 field has the Lagrangian\footnote{see \course{Classical Electrodynamics} for a derivation}
\begin{equation}
    \lagrangianDensity_0 = -\frac{1}{4}F^{\mu\nu}F_{\mu\nu}.
\end{equation}
Here \(F^{\mu\nu}\) is the field-strength tensor, given by \(F^{\mu\nu} = \partial^\mu B^\nu - \partial^\nu B^\mu\), where we now use \(B\) instead of \(A\) because we no longer want to focus only on photons.
For massive particles we add in a mass term\footnote{see \course{Quantum Field Theory} for more details}:
\begin{equation}
    \lagrangianDensity_m = -\frac{1}{4}F^{\mu\nu}F_{\mu\nu} + \frac{1}{2}m^2B^\mu B_\mu.
\end{equation}
The Euler--Lagrange equations then give\footnote{see \course{Classical Electrodynamics} for a detailed calculation of \(\diffp{F^{\mu\nu}F_{\mu\nu}}/{(\partial_\mu B^\nu)}\).}
\begin{equation}
    (\dalembertian + m^2)B^\mu - \partial^\mu\partial_\nu B^\nu = 0.
\end{equation}

An alternative way to obtain this equation is to note that for a massless scalar particle the Klein--Gordon equation is \(\dalembertian\varphi = 0\), whereas for a massive scalar field it's \((\dalembertian + m^2)\varphi = 0\), suggesting that applying the prescription \(\dalembertian \mapsto \dalembertian + m^2\) to a massless field equation gives the equivalent equation for a massive particle.
Which we can then apply to the massless photon free equation
\begin{equation}
    \dalembertian A^\mu - \partial^\mu\partial_\nu A^\nu = 0
\end{equation}
to get the result.

Taking the derivative of the resulting equation of motion we get
\begin{equation}
    0 = (\dalembertian + m^2) \partial_\mu B^\mu - \partial_\mu\partial^\mu\partial_\nu B^\nu = (\dalembertian + m^2) \partial_\mu B^\mu - \dalembertian \partial_\mu B^\mu = m^2\partial_\mu B^\mu = 0.
\end{equation}
This implies that massive spin 1 particles automatically satisfy the Lorenz condition,
\begin{equation}
    \partial_\mu B^\mu = 0.
\end{equation}
Using this the equation of motion becomes
\begin{equation}
    (\dalembertian + m^2) B^\mu = 0.
\end{equation}
So we just get a Klein--Gordon equation for each component.

A massive particle with four-momentum \(q\) has \(q^2 = m^3\) and so
\begin{equation}
    \dalembertian \e^{-iq\cdot x} = -q^2\e^{-q\cdot x} = -m^2\e^{-iq\cdot x}
\end{equation}
implying plane wave solutions of the form
\begin{equation}
    B^\mu = \varepsilon^\mu \e^{-iq\cdot x}.
\end{equation}
The Lorenz condition implies that
\begin{equation}
    q_\mu \varepsilon^\mu = 0,
\end{equation}
exactly the same as in the massless case.

There is no further gauge freedom, there isn't a massive equivalent of the Coulomb gauge, and so we have three different polarisations, we can take the two used for the photon and as the third
\begin{equation}
    \varepsilon^{(3)} = 
    \begin{pmatrix}
        0\\ 0\\ 0\\ 1
    \end{pmatrix}
    .
\end{equation}

\section{Polarisation Sums}
A polarisation sum is a sum of the form
\begin{equation}
    \sum_\lambda \varepsilon^{(\lambda)}_\mu (\varepsilon^{(\lambda)}_\nu)^*.
\end{equation}

\subsection{Massive Gauge Bosons}
We can interpret \(\varepsilon^{(\lambda)}_\mu(\varepsilon^{(\lambda)}_\nu)^*\) as the components of some two-index tensor, or a matrix.
Rather than work with the polarisation vectors we've discussed, instead we use the left and right circular polarisations
\begin{equation}
    \varepsilon^{(-)} = \frac{1}{\sqrt{2}}
    \begin{pmatrix}
        0\\ 1\\ -i\\ 0
    \end{pmatrix}
    , \qand 
    \varepsilon^{(+)} = -\frac{1}{\sqrt{2}}
    \begin{pmatrix}
        0\\ 1\\ i\\ 0
    \end{pmatrix}
    .
\end{equation}
As the third polarisation state we then choose something orthogonal to both of these, so it must be of the form \((\alpha, 0, 0, \beta)^\trans\).
The relationship between \(\alpha\) and \(\beta\) is fixed by the requirement that \(q_\mu \varepsilon^\mu = 0\), so we must have \(\alpha E - \beta p_z = 0\).
Hence we choose the longitudinal polarisation to be
\begin{equation}
    \varepsilon^{(\symup{L})} = \frac{1}{m}
    \begin{pmatrix}
        p_z\\ 0\\ 0\\ E
    \end{pmatrix}
    ,
\end{equation}
where the normalisation is chosen such that in the rest frame we just have \((0, 0, 0, 1)^\trans\).

Using this our polarisation sum becomes
\begin{align}
    \sum_\lambda \varepsilon_\mu^{(\lambda)} (\varepsilon_\nu^{(\lambda)})^* &= \varepsilon_\mu^{(+)}(\varepsilon_\nu^{(+)})^* + \varepsilon_\mu^{(-)}(\varepsilon_\nu^{(-)})^* + \varepsilon_\mu^{(\symup{L})}(\varepsilon_\nu^{(\symup{L})})^*\\
    &= \frac{1}{2}
    \begin{pmatrix}
        0 & 0 & 0 & 0\\
        0 & 1 & -i & 0\\
        0 & -i & 1 & 0\\
        0 & 0 & 0 & 0
    \end{pmatrix}
    + \frac{1}{2}
    \begin{pmatrix}
        0 & 0 & 0 & 0\\
        0 & 1 & -i & 0\\
        0 & i & 1 & 0\\
        0 & 0 & 0 & 0
    \end{pmatrix}
    \\
    &\qquad\qquad+ \frac{1}{m^2}
    \begin{pmatrix}
        p_z^2 & 0 & 0 & Ep_z\\
        0 & 0 & 0 & 0\\
        0 & 0 & 0 & 0\\
        Ep_z & 0 & 0 & E^2
    \end{pmatrix}
    \\
    &= 
    \begin{pmatrix}
        -1 & 0 & 0 & 0\\
        0 & 1 & 0 & 0\\
        0 & 0 & 1 & 0\\
        0 & 0 & 0 & 0
    \end{pmatrix}
    + \frac{1}{m^2}
    \begin{pmatrix}
        (q^0)^2 & 0 & 0 & q^0q^3\\
        0 & 0 & 0 & 0\\
        0 & 0 & 0 & 0\\
        q^0q^3 & 0 & 0 & (q^3)^2
    \end{pmatrix}
    .
\end{align}
In the last step we consider that the particle is moving in the \(z\)-direction, by definition, and wrote \(E^2 = m^2 + p_z^2\), so \(p_z^2/m^2 = (E^2 - m^2)/m^2 = E^2 - 1\), and similarly \(E^2/m^2 = (m^2 + p_z^2)/m^2 = 1 + p_z^2\), and then we wrote \(p_z\) and \(E\) in terms of components of the four-momentum, \(q\).
This generalises to a particle moving along an arbitrary direction:
\begin{equation}
    \sum_\lambda \varepsilon_\mu^{(\lambda)}(\varepsilon_\nu^{(\lambda)})^* = -\minkowskiMetric_{\mu\nu} + \frac{q_\mu q_\nu}{m^2}.
\end{equation}

\subsection{Real Photons}
For photons the gauge freedom makes matters slightly more complicated.
For a real photon travelling in the \(z\)-direction we can use the polarisations \(\varepsilon^{(1)} = (0, 1, 0, 0)^\trans\) and \(\varepsilon^{(2)} = (0, 0, 1, 0)^\trans\), and these are the only polarisation states.
Hence,
\begin{equation}
    \sum_\lambda \varepsilon_\mu^{(\lambda)}(\varepsilon_\nu^{(\lambda)})^* = 
    \begin{pmatrix}
        0 & 0 & 0 & 0\\
        0 & 1 & 0 & 0\\
        0 & 0 & 1 & 0\\
        0 & 0 & 0 & 1
    \end{pmatrix}
    .
\end{equation}
As before this generalises to a particle travelling in an arbitrary direction:
\begin{equation}
    \sum_T \varepsilon_i^{(T)}(\varepsilon_j^{(T)})^* = \delta_{ij} - \frac{q_iq_j}{\abs{\vv{q}}^2}.
\end{equation}
Here the sum is over the transverse polarisation states, indexed by \(T\).

If we want to sum over all possible components of the polarisation vector then we need to account for extra gauge freedom.
To do so consider the process \(\Pq \to \Pq \Pphoton\), such as occurs as part of the decay \(\Prhozero \to \Ppizero \Pphoton\).
This is shown in the diagram
\begin{equation}
    \tikzsetnextfilename{fd-quark-to-quark-photon}
    \feynmandiagram[horizontal'=i to v, inline=(v)]{
        i [particle=\Pq] -- [fermion,  edge label'=\(p\)] v -- [fermion, edge label=\(p'\)] o1 [particle=\Pq],
        v -- [photon, edge label=\(q\)] o2 [particle=\Pphoton]
    };
\end{equation}
The amplitude for this vertex is
\begin{equation}
    \amplitude = Q_{\Pq} \diracadjoint{u}(p')\gamma^\mu u(p) \varepsilon_\mu^{(\lambda)}(q),
\end{equation}
which follows in a similar way to \cref{eqn:electron-photon interaction strength}.
We can write this as
\begin{equation}
    \amplitude = j^\mu \varepsilon^{(\lambda)}_\mu(q), \qqwhere j^\mu = Q_{\Pq} \diracadjoint{u}(p')\gamma^\mu u(p).
\end{equation}
Then, summing over all polarisations, we have
\begin{equation}
    \sum_\lambda \abs{\amplitude}^2 = \sum_\lambda j^\mu (j^\nu)* \varepsilon^{(\lambda)}_\mu (\varepsilon_\nu^{(\lambda)})^*.
\end{equation}
In the Coulomb gauge this takes the form
\begin{equation}
    \sum_{T = 1}^2 \abs{\amplitude}^2 = j^\mu(j^\nu)^* \sum_{T = 1}^2 \varepsilon_\mu^{(T)}(\varepsilon_\nu^{(T)})^*.
\end{equation}
In the frame where the photon travels along the \(z\)-direction this becomes
\begin{equation}
    \sum_{T = 1}^{2} \abs{\amplitude}^2 = j_1j_1^* + j_2j_2^*.
\end{equation}
We can write this as a sum over all four components by subtracting off the unwanted components:
\begin{equation}
    \sum_{T = 1}^{2} \abs{\amplitude}^2 = j_1j_1^* + j_2j_2^* = -\minkowskiMetric^{\mu\nu}j_\mu j_\nu^* + j_0j_0^* - j_3j_3^*.
\end{equation}

We saw that the polarisation vectors \(\varepsilon_\mu\) and \(\varepsilon_{\mu} + aq^\mu\) describe the same physics, so the amplitude must be invariant under this, meaning we must have
\begin{equation}
    \amplitude = j^\mu \varepsilon_\mu^{(\lambda)}(q) = j^\mu \varepsilon^{(\lambda)}_\mu(q) + j^\mu q_\mu.
\end{equation}
Hence, we must have
\begin{equation}
    j_\mu q^\mu = 0.
\end{equation}
For a photon with \(q^\mu = (q, 0, 0, q)\) (so \(q^2 = 0\)) we have \(qj_0 - qj_3 = 0\), and so the invariance condition gives \(j_0 = j_3\).
Hence, 
\begin{equation}
    \sum_{T = 1}^{2} \abs{\amplitude}^2 = -\minkowskiMetric^{\mu\nu}j_\mu j_\nu.
\end{equation}
Thus, the sum over polarisation states for a real photon gives
\begin{equation}
    \sum_{T = 1}^{2} \varepsilon^{(\lambda)}_{\mu} (\varepsilon^{(\lambda)}_\nu)^* = -\minkowskiMetric_{\mu\nu}.
\end{equation}

\subsection{Virtual Photons}
For off-shell photons \(q^2 \ne 0\) and we cannot ignore the other two polarisation states.
The amplitude for \(\Pe\Ptau \to \Pe\Ptau\) scattering via a photon in a \(t\)-channel process is, as given in \cref{eqn:electron tau scattering amplitude},
\begin{equation}
    \amplitude \propto \sum_{\lambda = 1}^{4} j_\mu^{(\Pe)} j_\nu^{(\Ptau)} \frac{\varepsilon^{(\lambda)}_\mu(\varepsilon^{(\lambda)}_\nu)^*}{q^2}.
\end{equation}
We can treat the virtual photon as having effective mass \(m^2 = q^2\), and so we can use the massive spin 1 result:
\begin{equation}
    \sum_{\lambda = 1}^{4} \varepsilon^{(\lambda)}_\mu(\varepsilon^{(\lambda)}_\nu)^* = -\minkowskiMetric_{\mu\nu} + \frac{q_\mu q_\nu}{q^2}.
\end{equation}
Now, the \(t\)-channel process is the only one at this order in in QED.
There is no \(s\) channel, as we'd need a boson with charge \(-2\) to propagate between the two vertices.
There is no \(u\) channel, since the two particles are distinct.
This means that this term must be gauge invariant on its own.
So, consider the gauge transformation \(\varepsilon^\mu \mapsto \varepsilon^\mu + aq^\mu\) and \(\varepsilon^\nu \mapsto \varepsilon^\nu + bq^\nu\), we can take different gauge transformations at different points in space time as long as the variation between \(a\) and \(b\) occurs linearly in spacetime, so vanishes when we take the second derivative.
Hence,
\begin{equation}
    \sum_{\lambda = 1}^{4} j_\mu^{(\Pe)} j_\nu^{(\Ptau)} \varepsilon_\mu^{(\lambda)}(\varepsilon_\nu^{(\lambda)})^* = \sum_{\lambda = 1}^4 j_\mu^{(\Pe)} j_\nu^{(\Ptau)} (\varepsilon_\mu^{(\lambda)} + aq^\mu)((\varepsilon_\nu^{(\lambda)})^* + bq^\nu)
\end{equation}
for all \(a\) and \(b\).
This means we have
\begin{equation}
    j_\mu^{(\Pe)} j_\nu^{(\Ptau)} q^\mu q^\nu = 0.
\end{equation}
We can then conclude that the \(q_\mu q_\nu/q^2\) term does not contribute to the amplitude.
Hence, we have
\begin{equation}\label{eqn:spin sum}
    \sum_{\lambda = 1}^{4} \varepsilon^{(\lambda)}_\mu (\varepsilon_\nu^{(\lambda)})^* = -\minkowskiMetric^{\mu\nu}.
\end{equation}

The Feynman rule for a photon propagator in a first order diagram is then
\begin{equation}
    -i\frac{\minkowskiMetric_{\mu\nu}}{q^2}.
\end{equation}
For higher order diagrams we cannot guarantee the gauge invariance of a single diagram's contribution to the amplitude and instead the photon propagator is
\begin{equation}
    -\frac{i}{q^2}\left[ \minkowskiMetric_{\mu\nu} + (1 - \xi) \frac{q^\mu q^\nu}{q^2} \right]
\end{equation}
where \(\xi\) is a gauge dependent parameter.
Most calculations are performed in the Feynman gauge in which \(\xi = 1\).
    \end{appendices}
    
    \backmatter
    \renewcommand{\glossaryname}{Acronyms}
    \printglossary[acronym]
    \printindex
\end{document}