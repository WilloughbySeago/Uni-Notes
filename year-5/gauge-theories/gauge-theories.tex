% !TeX program = lualatex
\documentclass[fleqn]{NotesClass}

\strictpagecheck

%% Packages
\usepackage{tensor}
\usepackage{csquotes}
\usepackage{scalerel,stackengine}  % needed for d with slash
\usepackage{simpler-wick}
\usepackage{siunitx}
\usepackage{slashed}
\declareslashed{}{\not}{.1}{.5}{A}
\declareslashed{}{\not}{.05}{.5}{p}
\declareslashed{}{\not}{.1}{.5}{\partial}
\declareslashed{}{\not}{-.1}{.5}{a}
\declareslashed{}{\not}{.05}{.5}{D}

\let\widetildeOLD\widetilde
\AtBeginDocument{\let\widetilde\widetildeOLD}
\let\widehatOLD\widehat
\AtBeginDocument{\let\widehat\widehatOLD}

% Tikz stuff
\usepackage{tikz}
%\tikzset{>=latex}
% External
\usetikzlibrary{external}
\tikzexternalize[prefix=tikz-external/]
% Other libraries
\usetikzlibrary{decorations.markings, decorations.pathmorphing, patterns.meta, decorations.pathreplacing, calligraphy}

% Feynman diagram styles
\tikzset{scalar/.style={dashed}}
\tikzset{charged scalar/.style={dashed, postaction={decorate, decoration={markings, mark=at position #1 with {\arrow{Latex[width=1.5mm]}}}}}}
\tikzset{charged scalar/.default=0.5}
\tikzset{photon/.style={decoration={snake, amplitude=1.25, segment length=6}, decorate}}
\tikzset{electron/.style={postaction={decorate, decoration={markings, mark=at position #1 with {\arrow{Latex[width=1.5mm]}}}}}}
\tikzset{electron/.default=0.5}
\tikzset{positron/.style={postaction={decorate, decoration={markings, mark=at position #1 with {\arrowreversed{Latex[width=1.5mm]}}}}}}
\tikzset{positron/.default=0.5}
\tikzset{momentum injection/.style={->}}
\pgfdeclarepatternformonly{south west lines}{\pgfqpoint{-0pt}{-0pt}}{\pgfqpoint{3pt}{3pt}}{\pgfqpoint{3pt}{3pt}}{
    \pgfsetlinewidth{0.4pt}
    \pgfpathmoveto{\pgfqpoint{0pt}{0pt}}
    \pgfpathlineto{\pgfqpoint{3pt}{3pt}}
    \pgfpathmoveto{\pgfqpoint{2.8pt}{-.2pt}}
    \pgfpathlineto{\pgfqpoint{3.2pt}{.2pt}}
    \pgfpathmoveto{\pgfqpoint{-.2pt}{2.8pt}}
    \pgfpathlineto{\pgfqpoint{.2pt}{3.2pt}}
    \pgfusepath{stroke}}
\tikzset{blob/.style={thick, pattern={south west lines}}}
\tikzset{gluon/.style={decoration={coil, aspect=0.75, segment length=1.5mm}, decorate}}
\tikzset{mirror gluon/.style={decoration={coil, aspect=0.75, segment length=1.5mm, mirror}, decorate}}
\tikzset{quark/.style={postaction={decorate, decoration={markings, mark=at position #1 with {\arrow{Latex[width=1.5mm]}}}}}}
\tikzset{quark/.default=0.5}
\tikzset{antiquark/.style={postaction={decorate, decoration={markings, mark=at position #1 with {\arrowreversed{Latex[width=1.5mm]}}}}}}
\tikzset{antiquark/.default=0.5}
\tikzset{momentum injection/.style={->}}
\tikzset{WZ boson/.style={photon}}
\tikzset{higgs/.style={scalar}}
\tikzset{ghost/.style={thick, dotted, postaction={decorate, decoration={markings, mark=at position #1 with {\arrow{Latex[width=1.5mm]}}}}}}
\tikzset{ghost/.default=0.5}
\tikzset{antighost/.style={thick, dotted, postaction={decorate, decoration={markings, mark=at position #1 with {\arrowreversed{Latex[width=1.5mm]}}}}}}
\tikzset{antighost/.default=0.5}

\usepackage[compat=1.1.0]{tikz-feynman}

% References, should be last things loaded
\usepackage[pdfauthor={Willoughby Seago},pdftitle={Gauge Theoreis in Particle Physics},pdfkeywords={quantum field theory, QFT, gauge theory, particle physics, Feynman diagrams, QED, QCD, electroweak, EW, lattice, renormalisation},pdfsubject={Gauge Theories}]{hyperref}  % Should be loaded second last (cleveref last)
\colorlet{hyperrefcolor}{blue!60!black}
\hypersetup{colorlinks=true, linkcolor=hyperrefcolor, urlcolor=hyperrefcolor}
\usepackage[
capitalize,
nameinlink,
noabbrev
]{cleveref} % Should be loaded last

% My packages
\usepackage{NotesBoxes}
\usepackage{NotesMaths}

\setmathfont[range={\int, \oint, \otimes, \oplus, \bigotimes, \bigoplus}]{Latin Modern Math}

% Highlight colour
\definecolor{Yellow}{HTML}{F9C80E}
\definecolor{Orange}{HTML}{F86624}
\definecolor{Red}{HTML}{EA3546}
\definecolor{Purple}{HTML}{662E9B}
\definecolor{Blue}{HTML}{43BCCD}

\colorlet{highlight}{Red}

% Title page info
\title{Gauge Theories in Particle Physics}
\author{Willoughby Seago}
\date{January 16, 2023}
% \subtitle{}
% \subsubtitle{}

% Commands
% Text
\newcommand*{\course}[1]{\textit{#1}}
\newcommand{\MSbar}{\ensuremath{\overline{\text{MS}}}}
    
% Particles
\newcommand{\Pparticle}[1]{\mathrm{#1}}
\newcommand{\Pphoton}{\ensuremath{\upgamma}}
\newcommand{\PZboson}{\ensuremath{\Pparticle{Z}}}
\newcommand{\Pu}{\ensuremath{\Pparticle{u}}}
\newcommand{\Pd}{\ensuremath{\Pparticle{d}}}
\newcommand{\Ps}{\ensuremath{\Pparticle{s}}}
\newcommand{\Pc}{\ensuremath{\Pparticle{c}}}
\newcommand{\Pt}{\ensuremath{\Pparticle{t}}}
\newcommand{\Pb}{\ensuremath{\Pparticle{b}}}
\newcommand{\Pe}{\ensuremath{\Pparticle{e}^{-}}}
\newcommand{\Pex}{\ensuremath{\Pparticle{e}}}
\newcommand{\Pmu}{\ensuremath{\upmu^{-}}}
\newcommand{\Pmux}{\ensuremath{\upmu}}
\newcommand{\Ptau}{\ensuremath{\uptau^{-}}}
\newcommand{\Ptaux}{\ensuremath{\uptau}}
\newcommand{\PeLeft}{\ensuremath{\Pparticle{e}^{-}_{\raisebox{0ex}[0pt][0.5ex]{\(\scriptstyle\symrm{L}\)}}}}
\newcommand{\PmuLeft}{\ensuremath{\upmu^{-}_{\raisebox{0ex}[0pt][0.5ex]{\(\scriptstyle\symrm{L}\)}}}}
\newcommand{\PtauLeft}{\ensuremath{\uptau^{-}_{\raisebox{0ex}[0pt][0.5ex]{\(\scriptstyle\symrm{L}\)}}}}
\newcommand{\Pnu}{\ensuremath{\upnu}}
\newcommand{\Pnue}{\ensuremath{\upnu_{\mathrm{e}}}}
\newcommand{\Pnumu}{\ensuremath{\upnu_{\upmu}}}
\newcommand{\Pnutau}{\ensuremath{\upnu_{\uptau}}}
\newcommand{\PnueLeft}{\ensuremath{\upnu_{\symrm{eL}}}}
\newcommand{\PnumuLeft}{\ensuremath{\upnu_{\upmu\symrm{L}}}}
\newcommand{\PnutauLeft}{\ensuremath{\upnu_{\uptau\symrm{L}}}}
\newcommand{\PH}{\ensuremath{\Pparticle{H}}}
\newcommand{\PZ}{\ensuremath{\Pparticle{Z}}}
\newcommand{\PW}{\ensuremath{\Pparticle{W}}}
%\newcommand{\PWpm}{\ensuremath{\Pparticle{W}^{\pm}}}
%\newcommand{\PWp}{\ensuremath{\Pparticle{W}^{+}}}
%\newcommand{\PWm}{\ensuremath{\Pparticle{W}^{-}}}
\newcommand{\Pg}{\ensuremath{\Pparticle{g}}}
\newcommand{\Pq}{\ensuremath{\Pparticle{q}}}
\newcommand{\Ppi}{\ensuremath{\uppi}}
\newcommand{\Ppip}{\ensuremath{\uppi^{+}}}
\newcommand{\Ppim}{\ensuremath{\uppi^{-}}}
\newcommand{\Ppizero}{\ensuremath{\uppi^{0}}}
%\newcommand{\Prhozero}{\ensuremath{\uprho^{0}}}
\newcommand{\Pf}{\ensuremath{\Pparticle{f}}}
%\newcommand{\Pred}{\ensuremath{\textcolor{redgluoncolor}{\Pparticle{r}}}}
%\newcommand{\Pgreen}{\ensuremath{\textcolor{greengluoncolor}{\Pparticle{g}}}}
%\newcommand{\Pblue}{\ensuremath{\textcolor{bluegluoncolor}{\Pparticle{b}}}}
%\newcommand{\PJpsi}{\ensuremath{\Pparticle{J}/\text{\normalfont ψ}}}
\newcommand{\Pproton}{\ensuremath{\Pparticle{p}}}
\newcommand{\Pneutron}{\ensuremath{\Pparticle{n}}}
\newcommand{\PKm}{\ensuremath{\Pparticle{K}^-}}
\newcommand{\PKzero}{\ensuremath{\Pparticle{K}^0}}
\newcommand{\PKp}{\ensuremath{\Pparticle{K}^+}}
\newcommand{\PK}{\ensuremath{\Pparticle{K}}}
\newcommand{\Phiggs}{\ensuremath{\Pparticle{H}}}
%\newcommand{\Peta}{\ensuremath{\upeta}}
\newcommand{\Pp}{\ensuremath{\Pparticle{p}}}
\newcommand{\Pn}{\ensuremath{\Pparticle{n}}}
\newcommand{\PSigmap}{\ensuremath{\upSigma^{p}}}
\newcommand{\PLambdazero}{\ensuremath{\upLambda^{0}}}
\newcommand{\PXizero}{\ensuremath{\upXi^{0}}}
\newcommand{\POmegam}{\ensuremath{\upOmega^{-}}}

\newcommand{\APantiparticle}[1]{\bar{#1}}
\newcommand{\APu}{\ensuremath{\APantiparticle{\Pparticle{u}}}}
\newcommand{\APd}{\ensuremath{\APantiparticle{\Pparticle{d}}}}
\newcommand{\APs}{\ensuremath{\APantiparticle{\Pparticle{s}}}}
\newcommand{\APc}{\ensuremath{\APantiparticle{\Pparticle{c}}}}
\newcommand{\APt}{\ensuremath{\APantiparticle{\Pparticle{t}}}}
\newcommand{\APb}{\ensuremath{\APantiparticle{\Pparticle{b}}}}
\newcommand{\APe}{\ensuremath{\Pparticle{e}^{+}}}
\newcommand{\APmu}{\ensuremath{\text{\normalfont μ}^{+}}}
%\newcommand{\APtau}{\ensuremath{\text{\normalfont τ}^{+}}}
\newcommand{\APnu}{\ensuremath{\upnu}}
\newcommand{\APnue}{\ensuremath{\APantiparticle{\upnu}_{\mathrm{e}}}}
\newcommand{\APnumu}{\ensuremath{\APantiparticle{\upnu}_{\text{μ}}}}
%\newcommand{\APnutau}{\ensuremath{\APantiparticle{\upnu}_{\text{τ}}}}
%\newcommand{\APH}{\ensuremath{\Pparticle{H}}}
%\newcommand{\APZ}{\ensuremath{\Pparticle{Z}}}
%\newcommand{\APWpm}{\ensuremath{\Pparticle{W}^{\mp}}}
%\newcommand{\APWp}{\ensuremath{\Pparticle{W}^{-}}}
%\newcommand{\APWm}{\ensuremath{\Pparticle{W}^{+}}}
%\newcommand{\APphoton}{\ensuremath{\upgamma}}
%\newcommand{\APg}{\ensuremath{\Pparticle{g}}}
%\newcommand{\APq}{\ensuremath{\APantiparticle{\Pparticle{q}}}}
%\newcommand{\APf}{\ensuremath{\APantiparticle{\Pparticle{f}}}}
%\newcommand{\APred}{\ensuremath{\textcolor{antiredgluoncolor}{\APantiparticle{\Pparticle{r}}}}}
%\newcommand{\APgreen}{\ensuremath{\textcolor{antigreengluoncolor}{\APantiparticle{\Pparticle{g}}}}}
%\newcommand{\APblue}{\ensuremath{\textcolor{antibluegluoncolor}{\APantiparticle{\Pparticle{b}}}}}
%\newcommand{\APKzero}{\ensuremath{\APantiparticle{\Pparticle{K}}^0}}
    
% Maths
% DOC
\newcommand{\e}{\symrm{e}}
\newcommand{\diracadjoint}[1]{\overbar{#1}}
\newcommand{\covariantDerivative}{D}
\newcommand{\hermit}{\dagger}
\newcommand{\lagrangianDensity}{\symcal{L}}
\newcommand{\amplitude}{\symcal{A}}
\newcommand{\interaction}{{\symrm{int}}}
\DeclareMathOperator{\timeOrdering}{T}
\newcommand{\DL}[1]{\symcal{D}{#1}}
\newcommand{\DD}[1]{\,\symcal{D}{#1}}
\DeclarePairedDelimiter{\correlator}{\langle}{\rangle}
\newcommand{\dhat}[1]{\hat{\symrm{d}}{#1}}
\newcommand{\dalembertian}{\partial^2}
\newcommand{\bare}{\symrm{B}}
\newcommand{\order}{\symcal{O}}
\newcommand{\minkowskiMetric}{\eta}
\DeclareMathOperator{\tr}{tr}
\newcommand{\measured}{\symrm{meas}}
\newcommand{\strongCoupling}{\alpha_{\symrm{S}}}
\newcommand{\bohrMagneton}{\mu_{\symrm{B}}}
\DeclarePairedDelimiterX{\anticommutator}[2]{\{}{\}}{#1, #2}
\DeclarePairedDelimiterX{\commutator}[2]{[}{]}{#1, #2}
\newcommand{\action}{\mathbin{.}}
\newcommand{\adj}{\symrm{adj}}
\newcommand{\adjointRep}{\symrm{A}}
\newcommand{\numberFermions}{N_{\symrm{f}}}


% RDB
\newcommand{\chargeConjugation}{\symsf{C}}
\newcommand{\parity}{\symsf{P}}
\newcommand{\timeReversal}{\symsf{T}}
\newcommand{\Left}{\symrm{L}}
\newcommand{\Right}{\symrm{R}}
\AtBeginDocument{\renewcommand{\mapstochar}{\rule[0.5pt]{0.5pt}{4.3pt}\mkern-1mu}}
\newcommand{\trans}{\top}
\newcommand{\fermiConst}{G_{\symrm{F}}}
\newcommand{\phaseSpaceMeasure}[1][n]{(\dl{\symrm{PS}})_{#1}}
\newcommand{\cabibboangle}{\vartheta_{\symrm{c}}}

\includeonly{}

\begin{document}
    \frontmatter
    \titlepage
    \innertitlepage{tikz-external/consequences-A-psi-psibar-to-all-orders}
    \tableofcontents
    \listoffigures
    \mainmatter
    
    \chapter{Introduction}
    \section{Other Relevant Courses}
    This course follows on directly from the \course{Quantum Field Theory} course, so much of the relevant background is contained in those notes.
    A lot of the notation is also taken from this course.
    It is also expected that students on this course will have taken \course{Symmetries of Particles and Fields}, and the prerequisite for that course, \course{Symmetries of Quantum Mechanics}, so for any group theory related topics see the notes of one of these courses.
    Other relevant courses will be flagged throughout the notes.
    
    \section{Conventions}
    \begin{itemize}
        \item We will mostly work in natural units, where \(c = \hbar = 1\).
        \item We use the mostly-minuses metric, \(({+}{-}{-}{-})\), or \(({+}{-}\dotsb{-})\) in \(D\) spacetime dimensions.
        \item We will use the Einstein summation convention where repeated indices are summed over.
        In \(d + 1\) dimensions Greek letters, \(\mu, \nu, \rho, \dotsc\), run from 0 to \(d\) and Latin letters, \(i, j, k, \dotsc\), run from \(1\) to \(d\).
        \item We use the Fourier transform
        \begin{equation}
            \tilde{f}(p) = \int_{-\infty}^{\infty} \dl{x} \, \e^{ipx}f(x), \qqand f(x) = \int_{-\infty}^{\infty} \frac{\dl{p}}{2\pi} \e^{-ipx}\tilde{f}(p).
        \end{equation}
        \item Feynman diagrams are drawn with time increasing to the right.
        \item The electric charge, \(e\), is taken to be positive, so an electron has charge \(-e\).
    \end{itemize}
    
    \section{Dirac Algebra}
    Recall from \course{Quantum Field Theory} that the \define{gamma matrices}\index{gamma matrix}, \(\gamma^\mu\), are defined such that
    \begin{equation}
        \anticommutator{\gamma^\mu}{\gamma^\nu} \coloneqq 2\minkowskiMetric^{\mu\nu}
    \end{equation}
    where \(\anticommutator{A}{B} \coloneqq AB + BA\) is the \defineindex{anticommutator}.
    This means that
    \begin{equation}
        (\gamma^0)^2 = 1, \qqand (\gamma^i)^2 = -1.
    \end{equation}
    We can also impose that
    \begin{equation}
        (\gamma^0)^{\hermit} = \gamma^{0}, \qqand (\gamma^i)^{\hermit} = \gamma^0 \gamma^i \gamma^0 = -\gamma^i,
    \end{equation}
    but this comes with a choice of normalisation.
    
    The \defineindex{Dirac adjoint} of a spinor, \(\psi\), is defined to be \(\diracadjoint{\psi} \coloneqq \psi^\hermit \gamma^0\).
    
    We also define
    \begin{equation}
        \gamma^5 \coloneqq i\gamma^0\gamma^1\gamma^2\gamma^3,
    \end{equation}
    which is such that
    \begin{equation}
        \anticommutator{\gamma^5}{\gamma^\mu} = 0, \qquad (\gamma^5)^\hermit = \gamma^5, \qqand (\gamma^5)^2 = 1.
    \end{equation}
    
    When contracting a gamma matrix and a four vector we use \defineindex{slash notation}, writing \(\slashed{a} \coloneqq a_\mu \gamma^\mu\).
    
    The following identities hold in \(D = 4\) dimensions:
    \begin{alignat}{3}
        \gamma^\mu \gamma_\mu &= 4,\\
        \gamma^\mu \slashed{a} \gamma_\mu &= -2\slashed{a}, \qquad\qquad & \gamma^\mu \gamma^\nu \gamma_\mu &= -2\gamma^\nu,\\
        \gamma^\mu \slashed{a} \slashed{b} \gamma_\mu &= 4 a \cdot b, \qquad\qquad & \gamma^\mu \gamma^\nu \gamma^\rho \gamma_\mu &= 4\minkowskiMetric^{\nu\rho},\\
        \gamma^\mu \slashed{a} \slashed{b} \slashed{c} \gamma_\mu &= -2 \slashed{c}\slashed{b}\slashed{a}, \qquad\qquad & \gamma^\mu \gamma^\nu \gamma^\rho \gamma^\sigma \gamma_\mu &= -2 \gamma^\sigma \gamma^\rho \gamma^\nu.
    \end{alignat}
    The following identities hold in \(D\) dimensions:
    \begin{alignat}{3}
        \gamma^\mu \gamma_\mu &= D,\\
        \gamma^\mu \slashed{a} \gamma_\mu &= (2 - D)\slashed{a}, \qquad\qquad & \gamma^\mu \gamma^\nu \gamma_\mu &= (2 - D) \gamma^\nu.
    \end{alignat}
    The following identities hold for traces in \(D = 4\) dimensions:
    \begin{align}
        \tr(\text{odd number of gamma matrices}) &= 0,\\
        \tr(\text{odd number of gamma matrices times } \gamma^5) &= 0,\\
        \tr(\gamma^\mu \gamma^\nu) &= 4\minkowskiMetric^{\mu\nu},\\
        \tr(\gamma^\mu \gamma^\nu \gamma^\rho \gamma^\sigma) = 4(\minkowskiMetric^{\mu\nu}\minkowskiMetric^{\rho\sigma} + {}&\minkowskiMetric^{\mu\sigma}\minkowskiMetric^{\nu\rho} - \minkowskiMetric^{\mu\rho}\minkowskiMetric^{\nu\sigma}),\\
        \tr(\gamma^5) = \tr(\gamma^\mu \gamma^\nu \gamma^5) &= 0,\\
        \tr(\gamma^\mu\gamma^\nu\gamma^\rho\gamma^\sigma\gamma^5) &= -4i\varepsilon^{\mu\nu\rho\sigma}.
    \end{align}
    Remember that the trace is cyclic, so terms may appear in different orders.
    
    \part{Quantum Electrodynamics}
    \chapter{Classical Electrodynamics}
    Quantum electrodynamics (QED)\glossary[acronym]{QED}{Quantum Electrodynamics} is the quantum field theory (QFT)\glossary[acronym]{QFT}{Quantum Field Theory} of electromagnetism.
    It supersedes classical electrodynamics (CED)\glossary[acronym]{CED}{Classical Electrodynamics} as a theory for predicting what happens to electric charges, as well as electromagnetic radiation, light.
    We'll start this section by discussing CED and some of its short comings which necessitate the development of QED.
    For more details on CED see the \course{Classical Electrodynamics} course.
    
    \section{Classical Action}
    Classical electrodynamics follows from the action
    \begin{equation}
        S = \int \dl{^4x} \, \left[ -\frac{1}{4} F^{\mu\nu}F_{\mu\nu} - J_\mu A^\mu \right]
    \end{equation}
    where \(A_\mu\) is the electromagnetic field (also called the electromagnetic potential), \(J_\mu\) is the current, and \(F_{\mu\nu} = \partial_\mu A_\nu - \partial_\nu A_\mu\) is the electromagnetic field strength (confusingly also called the electromagnetic field in some contexts).
    
    The first term in the action tells us how the electromagnetic field evolves and the second term encodes interactions.
    One common case is a current made of particles of charge \(q_i\).
    The current due to the \(i\)th particle is \(J^\mu = q_i u_i^\mu = q_i \diff{x_i^\mu}/{\tau_i}\) (no sum on \(i\)).
    In this case we can write the interaction term as a sum over particles and the action from each particle is given by an integral over the particle's world line.
    Thus the action can be written as
    \begin{equation}
        S = \int \dl{^4x} \, \left[ -\frac{1}{4}F^{\mu\nu}F_{\mu\nu} \right] - \sum_{i} \int \dl{\tau_i} \, q_i A_\mu(x_i(\tau_i)) \diff{x_i^\mu(\tau_i)}{\tau_i} + \dotsb.
    \end{equation}
    The \enquote{\(\dotsb\)} here accounts for other effects not already covered, such as spin corrections.
    These terms are suppressed by factors of \(1/m_i\) or more, \(m_i\) being the mass of the \(i\)th particle.
    
    \section{Problems with CED}
    The theory of point-like electric charges developed as above is not entirely satisfactory.
    Recall that the Lorentz force on a particle with charge \(q_i\) and four-velocity \(u_i^\mu\) in an electromagnetic field with field strength tensor \(F^{\mu\nu}\) is
    \begin{equation}
        \diff{p_i^\mu}{\tau} = q_i F^{\mu\nu}(x_i(\tau)) u_{i\nu}(x_i(\tau))
    \end{equation}
    where \(p_i = m_i u_i = m_i \diff{x_i}/{\tau}\) is the four-momentum of the particle, with \(m_i\) being the mass of the \(i\)th particle.
    This law follows from conservation of energy, so it's pretty fundamental.
    
    Now consider the Coulomb field of particle \(1\) in particle 1's rest frame.
    At a point \(x\) a distance \(r\) from particle 1, which has position \(x_1\), the Coulomb field is
    \begin{equation}
        A^0(x) = \frac{q_i}{4\pi r} = \frac{q_i}{4\pi} \frac{1}{\sqrt{(x^1 - x_1^1)^2 + (x^2 - x_1^2)^2 + (x^3 - x_1^3)^2}}.
    \end{equation}
    If we can write this in a covariant way then it will apply in all frames.
    We can do this using \(u_1 = (1, \vv{0})\), which is the four-velocity of the particle in it's rest frame in units where \(c = 1\).
    We then have
    \begin{equation}
        [u_1 \cdot (x - x_1)]^2 - (x - x_1)^2 = (x^0 - x_1^0)^2 - (x^0 - x_1^0)^2 + (x^i - x_1^i)^2.
    \end{equation}
    Using \(u_1^\mu\) to get a \(\mu\) index we have
    \begin{equation}
        A^\mu(x) = \frac{q_i}{4\pi} \frac{u_1^\mu}{\sqrt{[u_1 \cdot (x - x_1)]^2 - (x - x_1)^2}}.
    \end{equation}
    This is covariant and so holds in any inertial frame.
    
    Now suppose that particles 1 and 2 interact.
    The force on particle one is
    \begin{equation}
        \diff{p_1^\mu}{\tau} = q_1 F^{\mu\nu}(x_1(\tau))u_{1\nu}(\tau).
    \end{equation}
    The \(F^{\mu\nu}\) appearing here is the total electromagnetic field strength.
    This proves to be a problem because part of \(F^{\mu\nu}\) is the Coulomb field of particle 1, \(A^\mu\).
    We can see from the expression above that this is singular when \(x = x_1\), which is exactly the case when we try to compute the force above.
    So we get a divergent result which we have to somehow make sense of.
    
    The standard solution in CED to avoid this problem is to just use the electromagnetic field strength due to particle 2, and ignore the field from particle 1.
    Then, so long as the two particles can't have the same position, \(F^{\mu\nu}\) is nonsingular and we avoid the problem with infinity.
    This workaround works well at low energies (small velocities).
    
    The problem is that this workaround is not consistent with conservation of energy.
    For example, consider a classical atom, formed from a nucleus of charge \(Ze\), which we take to have infinite mass, and an electron of charge \(-e\).
    The Lorentz force applied to the electron, calculated using the electric field only from the nucleus, is consistent with stable circular orbits.
    The problem is that as the electron is orbiting it is changing direction, and so accelerating, and therefore must radiate.
    There is only one possible source for this energy, the potential, and so the orbit must decay.
    
    Since the orbit decays there must be some force we have not accounted for causing this decay.
    It is possible to account for this force only if we include the electron's own field in the calculation of the force upon the electron.
    This means we have to account for interactions of particles with their own fields.
    The correction that we get when doing so is suppressed by inverse powers of \(c\), which is why removing the electron's electromagnetic field works well enough for many purposes at low energies.
    
    Once we accept that particles interact with their own fields there is another problem.
    The Coulomb field of a point-like particle contains an infinite electrostatic energy.
    The energy density of an electromagnetic field is \((E^2 + B^2)/2\), meaning that the energy contained in the Coulomb field outside of a spherical region of radius \(r_{\min}\) centred on the electron is given by
    \begin{equation}
        \int \dl{^3x} \, \frac{1}{2}E^2 = \frac{1}{2}\frac{e^2}{(4\pi)^2} \int_{S^2} \dl{\Omega} \int_{r_{\min}}^{\infty} \dl{r} \, r^2 \frac{1}{r^4} = \frac{1}{8\pi} \frac{1}{r_{\min}}.
    \end{equation}
    Taking \(r_{\min} \to \infty\) the energy diverges.
    
    The classical physics solution to this is to say that the electron isn't point-like.
    Then if we choose \(r_{\min}\) to be the size of the electron so long as
    \begin{equation}
        mc^2 \gtrsim \frac{1}{2} \frac{e^2}{4\pi r_{\min}},
    \end{equation}
    so the energy of the field is less than the total energy available from the electron, things are fine.
    Setting \(c = 1\) again gives
    \begin{equation}
        r_{\min} \gtrsim \frac{1}{2} \frac{e^2}{4\pi} \frac{1}{m} = \alpha \ell_{\mathrm{Compton}}
    \end{equation}
    where \(\alpha = e^2/(4\pi)\) is the fine structure constant and \(\ell_{\mathrm{Compton}}\) is the Compton wavelength, which is the wavelength of a photon with the same energy as the rest mass of the particle.
    What this tells us is that scales at which quantum mechanical effects become important become important before the size of the electron becomes a problem classically, so if we're only interested in classical computations we don't need to worry about treating the electron as point-like.
    
    However, we know from experiments that the electron is point-like at least up to the \qty{1}{\tera\electronvolt} scale, which is about one millionth of the limit above.
    These point-like particles lead to divergences in classical theories, and also in quantum theories, like the ones above.
    Interestingly the divergence is actually not as bad in the quantum theory, being log divergent instead of going as \(1/r_{\min}\).
    In the quantum theory we can deal with these infinities by absorbing them into a finite number of measured parameters, such as \(e\) and \(m\), and we get a very powerful and predictive theory, QED.
    
    \chapter{QFT Recap}
    \epigraph{It's part of my job to give you problems}{Donal O'Connell}
    \section{QED Lagrangian}
    In QFT, in particular in QED, we replace the classical action with the \define{QED action}\index{QED Lagrangian}:
    \begin{equation}
        S = \int \dl{^Dx} \, \left[ -\frac{1}{4}F^{\mu\nu}F_{\mu\nu} + \diracadjoint{\psi} (i\slashed{\covariantDerivative} - m)\psi \right].
    \end{equation}
    Here \(F^{\mu\nu} = \partial^\mu A^\nu - \partial^\nu A^\mu\) as before.
    The change comes in the interaction term, where we now have the spinor field \(\psi\), which is acted on by the covariant derivative, which in QED is given by \(\covariantDerivative_\mu \coloneqq \partial_\mu -ieA_\mu\).
    Recall that \(\slashed{a} \coloneqq \gamma^\mu a_\mu\) where \(\gamma^\mu\) are the Dirac gamma matrices.
    Note that we leave the dimension as a variable, in preparation for dimensional regularisation later.
    
    Notice that there are no world lines appearing in the action now.
    This reflects the fact that we've replaced the particles with definite position with fields, which aren't localised in the same way.
    One of the biggest changes upon moving to a \emph{quantum} field theory is that we have a the new phenomenon of pair production.
    This creates new world lines, which is part of the reason we have to move away from actions involving sums over world lines.
    
    We use the Dirac Lagrangian in QED since we are mostly interested in electrons, which are spin \(1/2\) particles.
    If instead we have a spin 0 particle with complex scalar field \(\Phi\) then we can use the \defineindex{scalar QED} Lagrangian
    \begin{equation}
        S = \int \dl{^Dx} \, \left[ -\frac{1}{4}F^{\mu\nu}F_{\mu\nu} + (\covariantDerivative_\mu \Phi)^\hermit (\covariantDerivative^\mu \Phi) - m^2 \Phi^\hermit \Phi - V(\Phi) \right]
    \end{equation}
    where \(V\) is some potential.
    
    If we neglect spin corrections in QED, often a valid thing to do since the spin is on the order of \(\hbar\), then we get similar results in both normal and scalar QED, and these results are similar to those we get in classical electrodynamics.
    In this way scalar QED is more similar to classical electrodynamics, and we will make use of scalar QED as an example while focusing on normal QED for applications.
    In some ways normal QED is actually simpler than scalar QED, there is no potential in normal QED and in normal QED the largest number of fields comes from the \(-ie\diracadjoint{\psi}\slashed{A}\psi\) term, with three fields, whereas scalar QED has a four field term, \(-\diracadjoint{\psi}A_\mu^\hermit A_\mu \psi\).
    
    The main objects of physical interest in QFT are scattering amplitudes.
    We will make use of both canonical quantisation and path integral methods to compute these and other quantities.
    
    \section{Canonical Quantisation}
    We'll compute one term of a tree-level amplitude using the canonical quantisation approach.
    For more details and similar calculations see the first half of the \course{Quantum Field Theory} course.
    We'll consider a real scalar field with a cubic interaction
    \begin{equation}
        \lagrangianDensity_{\interaction} = -\frac{g}{3!}\varphi^3.
    \end{equation}
    The two-to-two amplitude\footnote{two changes here from \course{Quantum Field Theory}, in that course we called the amplitude \(\symcal{M}\) and the factor of \(i\) was missing, this phase doesn't effect the final physics which always depend on \(\abs{\amplitude}^2\)}, \(\amplitude\), is given by
    \begin{equation}
        i\amplitude  = \bra{p_1', p_2'} S \ket{p_1, p_2}
    \end{equation}
    where \(p_i\) are the momenta of the incoming particles and \(p_i'\) the momenta of the outgoing particles.
    The \(S\)-matrix, \(S\), is given by the Dyson expansion:
    \begin{align}
        S &= \sum_{n = 0}^{\infty} \frac{(-i)^n}{n!} \timeOrdering \int \dl{t_1} \dotsm \int \dl{t_n} \, H_{\interaction}(t_1) \dotsm H_{\interaction}(t_n)\\
        &= \sum_{n = 0}^{\infty} \frac{i^n}{n!} \timeOrdering \int \dl{^Dx_1} \dotsm \int \dl{x_n} \, \lagrangianDensity_{\interaction}(x_1) \dotsm \lagrangianDensity_{\interaction}(x_n)\\
        &= \timeOrdering \exp\left\{ i \int \dl{^Dx} \, \lagrangianDensity_{\interaction} \right\}\\
        &= \timeOrdering \exp\{ i S_{\interaction} \}
    \end{align}
    where \(\timeOrdering\) is the time ordering operator, acting on everything to its right, \(H_{\interaction}\) is the interaction Hamiltonian, related to the interaction Lagrangian by
    \begin{equation}
        \int \! \dl{^{D-1}x} \, \lagrangianDensity_{\interaction} = -H_{\interaction}
    \end{equation}
    and \(S_{\interaction}\) is the interaction action, given by
    \begin{equation}
        S_{\interaction} \coloneqq \int \dl{^Dx} \, \lagrangianDensity_{\interaction}.
    \end{equation}
    Note that the exponential is just a short hand for the expansion above based on the similarity with the Taylor series of the exponential.
    
    Suppose we are interested in \(i\amplitude\) at order \(g^2\).
    Then we consider the following term:
    \begin{multline*}
        i\amplitude^{(2)} =\\
        \bra{p_1', p_2'} \frac{i^2}{2} \left( -\frac{g}{3!} \right)^2 \int \dl{^Dx_1} \int \dl{^Dx_2} \, \timeOrdering \varphi(x_1) \varphi(x_1) \varphi(x_1) \varphi(x_2) \varphi(x_2) \varphi(x_2) \ket{p_1, p_2}.
    \end{multline*}
    We compute this using contractions.
    One particular set of contractions is
    \begin{equation*}
        \wick{\bra{\c4{p}_1', \c3{p}_2'} \frac{i^2}{2} \left( -\frac{g}{3!} \right)^2 \int \dl{^Dx_1} \int \dl{^Dx_2} \, \timeOrdering \c3{\varphi}(x_1) \c4{\varphi}(x_1) \c1{\varphi}(x_1) \c1{\varphi}(x_2) \c2{\varphi}(x_2) \c1{\varphi}(x_2) \ket{\c1{p}_1, \c2{p}_2}}.
    \end{equation*}
    We can interpret this in terms of creation and annihilation of particles.
    The fields at \(x_2\) contracted with the incoming particles annihilate them and then the fields at \(x_1\) contracted with the outgoing particles create the outgoing particles.
    The contraction between the fields at \(x_1\) and \(x_2\) gives a propagator between these points.
    This is best seen in a Feynman diagram,
    \begin{equation}
        \tikzsetnextfilename{qft-recap-canonical-quantisation-example}
        \begin{tikzpicture}[baseline=(A)]
            \draw[scalar] (0, 0) node [below right, xshift=-0.2cm] {\(x_1\)} coordinate (A) -- (1, 0) node [below left, xshift=0.2cm] {\(x_2\)};
            \draw[scalar] (0, 0) -- (120:1) node [left] {\(p_1\)};
            \draw[scalar] (0, 0) -- (240:1) node [left] {\(p_2\)};
            \draw[scalar] (1, 0) -- ++ (60:1) node [right] {\(p_1'\)};
            \draw[scalar] (1, 0) -- ++ (300:1) node [right] {\(p_2'\)};
        \end{tikzpicture}
        .
    \end{equation}
    The initial particles enter on the left, annihilate at \(x_1\), where there is a propagator to \(x_2\), where two new particles are created.
    
    \section{Path Integral}
    In the path integral formulation we don't compute amplitudes directly.
    Instead we compute correlators, which we can extract the amplitude from later.
    For more details and similar calculations see the second half of the \course{Quantum Field Theory} course.
    The starting point for using the path integral formalism is to define the generating functional, which for QED is
    \begin{equation}
        Z[J] = \int \DL{\varphi} \exp\left\{ i\int \dl{^Dx} \, [\lagrangianDensity(\varphi) + J(x)\varphi(x)] \right\}.
    \end{equation}
    We then define the \(n\) point \defineindex{correlator}
    \begin{align}
        G^{(n)}(x_1, \dotsc, x_n) &\coloneqq \bra{0} \timeOrdering \varphi(x_1) \dotsm \varphi(x_n) \ket{0}\\
        &\eqqcolon \correlator{\varphi(x_1) \dotsm \varphi(x_n)}\\
        &= \frac{1}{Z[0]} \left( \frac{1}{i} \diffd{}{J(x_1)} \right) \dotsm \left( \frac{1}{i} \diffd{}{J(x_n)} \right) Z[J] \bigg|_{J = 0}\\
        &= \int \DL{\varphi} \, \varphi(x_1) \dotsm \varphi(x_n) \exp\left\{ i \int \dl{^Dx} \, (\lagrangianDensity_{\symrm{free}} + \lagrangianDensity_{\interaction}) \right\}.\notag
    \end{align}
    The factor of \(1/Z[0]\) is just normalisation and we typically don't worry about it.
    The source is just there to allow us to pull down factors of \(\varphi\) by differentiating, which is why we set it to zero at the end.
    
    As the first example of the path integral we'll compute the three-point correlator to first order in \(g\).
    We'll choose our normalisation such that \(Z[0] = 1\).
    Then
    \begin{equation}
        G^{(3)}(x_1, x_2, x_3) = \int \DL{\varphi} \, \varphi(x_1) \varphi(x_2) \varphi(x_3) \e^{iS_{\symrm{free}} + iS_{\interaction}}.
    \end{equation}
    Expanding the interaction exponential to first order we get
    \begin{equation}
        G^{(3)}(x_1, x_2, x_3) \approx \int \DL{\varphi} \, \varphi(x_1) \varphi(x_2) \varphi(x_3) \left[ 1 - \frac{ig}{3!} \int \dl{^Dx} \varphi(x)^3 \right] \e^{iS_{\symrm{free}}}.
    \end{equation}
    We have now reduced this to a Gaussian path integral which can be computed with contractions.
    One set of contractions gives the result
    \begin{equation}
        C(x_1, x_2, x_3) = -ig \int \DL{\varphi} \, \wick{\c4{\varphi}(x_1) \c3{\varphi}(x_2) \c2{\varphi}(x_3) \int \dl{^Dx} \, \c2{\varphi}(x) \c3{\varphi}(x) \c4{\varphi}(x)} \e^{iS_{\symrm{free}}}.
    \end{equation}
    The contraction of two fields is given by the \defineindex{Feynman propagator}:
    \begin{equation}
        \wick{\c{\varphi}(x)\c{\varphi}(y)} = i\Delta(x - y) = \int \frac{\dl{^Dp}}{(2\pi)} \frac{i}{p^2 - m^2 + i\varepsilon} \e^{ip \cdot (x_1 - x_2)}.
    \end{equation}
    Here \(\varepsilon\) is a small positive real number included to make expressions converge.
    Note that \(\Delta(x - y) = \Delta(y - x)\).
    At this point we introduce the short hand notation \(\dhat{p} = \dl{p}/(2\pi)\).
    
    Using this result the contraction above can be calculated as
    \begin{equation}
        C(x_1, x_2, x_3) = -ig \int \dl{^Dx} \, i\Delta(x_1 - x) \, i\Delta(x_2 - x) \, i\Delta(x_3 - x).
    \end{equation}
    We can again summarise this in a diagram
    \begin{equation}
        \tikzsetnextfilename{qft-recap-path-integral-example-1}
        \begin{tikzpicture}[baseline=(A)]
            \draw[scalar] (0, 0) node [below] {\(x\)} coordinate (A) -- (330:1) node [above] {\(x_1\)};
            \draw[scalar] (0, 0) -- (90:1) node [right] {\(x_2\)};
            \draw[scalar] (0, 0) -- (210:1) node [left] {\(x_3\)};
        \end{tikzpicture}
    \end{equation}
    In fact, given a diagram we can read off the corresponding expression for a correlator through the following prescription:
    \begin{itemize}
        \item Each line, \tikzsetnextfilename{qft-recap-scalar-propagator}\tikz[baseline=(x.base)]{\draw[scalar] (0, 0) node [left] (x) {\(x\)} -- (1, 0) node [right] {\(y\)};}, is a factor of \(i\Delta(x - y)\).
        \item Each vertex, \tikzsetnextfilename{qft-recap-phi-cubed-vertex}\tikz[baseline=-0.15cm]{\foreach \angle in {90, 210, 330} \draw[scalar] (0, 0) coordinate (A) -- (\angle:0.3); \node at (A) [below] {\(x\)};}, is a factor of \(-ig \int \symrm{d}^Dx\).
        \item Conserve momentum at each vertex.
    \end{itemize}
    
    There are other possible contractions, one such contraction contributing to \(\correlator{\varphi(x_1)\varphi(x_2)\varphi(x_3)}\) is
    \begin{align}
        &\hphantom{=} -ig \int \DL{\varphi} \, \wick{\c3{\varphi}(x_1) \c1{\varphi}(x_2) \c1{\varphi}(x_3) \int \dl{^Dx} \, \c3{\varphi}(x) \c1{\varphi}(x) \c1{\varphi}(x)} \e^{iS_{\symrm{free}}}\\
        &= -ig \int \dl{^D x} \, i\Delta(x_1 - x) \, i\Delta(x - x) \, i\Delta(x_2 - x_3)\\
        &= \tikzsetnextfilename{qft-recap-path-integral-example-2}
        \begin{tikzpicture}[baseline=-0.3cm]
            \draw[scalar] (0, 0) node [left] {\(x_1\)} -- (0.75, 0) node [above left, xshift=0.1cm] {\(x\)};
            \draw[scalar] (1, 0) circle [radius = 0.25];
            \draw[scalar] (0, -0.5) node [left] {\(x_2\)} -- (1, -0.5) node [right] {\(x_3\)};
        \end{tikzpicture}
        .
    \end{align}
    
    This diagram is disconnected.
    Often we are only interested in correlators involving connected diagrams, since these are the only diagrams that contribute to quantities such as \(\log(Z[J])\), as we saw in \course{Quantum Field Theory}.
    
    For another example consider the four-point correlator
    \begin{align}
        G^{(4)}(y_1, y_2, z_1, z_2) &= \bra{0} \timeOrdering \varphi(y_1) \varphi(y_2) \varphi(z_1) \varphi(z_2) \ket{0}\\
        &= \int \DL{\varphi} \, \varphi(y_1) \varphi(y_2) \varphi(z_1) \varphi(z_2) \exp\left\{ i \int \dl{^Dx} (\lagrangianDensity_{\symrm{free}} + \lagrangianDensity_{\interaction}) \right\}.
    \end{align}
    If we want to evaluate the order \(g^2\) contribution to this correlator then we can do so by considering the quadratic term after expanding \(\exp\{iS_{\interaction}\}\):
    \begin{multline*}
        \int \DL{\varphi} \varphi(y_1) \varphi(y_2) \varphi(z_1) \varphi(z_2)\\
        \times\left[ \frac{1}{2}\left( -\frac{ig}{3!} \right)^2 \int \dl{^Dx_1} \int \dl{^Dx_2} \, \varphi(x_1) \varphi(x_1) \varphi(x_1) \varphi(x_2) \varphi(x_2) \varphi(x_2) \right] \e^{iS_{\symrm{free}}}
    \end{multline*}
    Again this is a Gaussian integral and can be computed using contractions.
    One particular contraction we may want to consider is
    \begin{equation}
        \wick{\c4\varphi(y_1) \c3\varphi(y_2) \c2\varphi(z_1) \c1\varphi(z_2) \ \c1\varphi(x_1) \c2\varphi(x_1) \c1\varphi(x_1) \c1\varphi(x_2) \c3\varphi(x_2) \c4\varphi(x_2)}
    \end{equation}
    where we've only written the fields, not any of the integrals, constants, or exponentials, to fit it all on one line.
    This corresponds to the diagram
    \begin{equation}
        \tikzsetnextfilename{qft-recap-path-integral-example-3}
        D = 
        \begin{tikzpicture}[baseline=(A)]
            \draw[scalar] (0, 0) node [below right, xshift=-0.2cm] {\(x_1\)} coordinate (A) -- (1, 0) node [below left, xshift=0.2cm] {\(x_2\)};
            \draw[scalar] (0, 0) -- (120:1) node [left] {\(y_1\)};
            \draw[scalar] (0, 0) -- (240:1) node [left] {\(y_2\)};
            \draw[scalar] (1, 0) -- ++ (60:1) node [right] {\(z_1\)};
            \draw[scalar] (1, 0) -- ++ (300:1) node [right] {\(z_2\)};
        \end{tikzpicture}
        .
    \end{equation}
    There are multiple different contractions which all give the same diagram, and so the same contribution to the correlator.
    We combine these into one, including a \defineindex{symmetry factor} counting the number of such diagrams.
    The factor of \(1/3!\) has been chosen to cancel out this symmetry factor, in this case because we can permute the three \(\varphi(x_1)\) fields and the three \(\varphi(x_2)\) fields without changing anything, giving \((3!)^2\) as a symmetry factor, which cancels the \((3!)^2\) from expanding the exponential at second order.
    The result of evaluating all contractions giving diagram \(D\) is
    \begin{equation*}
        D = \int \dl{^Dx_1} \int \dl{^Dx_2} \, (-ig)^2 \, i\Delta(y_1 - x_2) \, i\Delta(y_2 - x_3) \, i\Delta(x_2 - x_1) \, i\Delta(x_1 - z_1) \, iD(x_1 - z_2).
    \end{equation*}
    
    Usually we prefer to work in momentum space.
    The simplest way to move to momentum space here is to replace each propagator with the inverse Fourier transform of the Fourier transform:
    \begin{equation}
        \Delta(x - y) = \int \dhat{p} \e^{ip \cdot (x - y)} \underbrace{\frac{1}{p^2 - m^2 + i\varepsilon}}_{= \widetilde{\Delta}(p)}.
    \end{equation}
    We can then manipulate the result until it is of the form \(D = \inverseFourierTransform\{\widetilde{D}\}\) and then identify \(\widetilde{D} = \fourierTransform\{D\}\).
    Making this replacement of propagators we get the somewhat unwieldy
    \begin{align}
        D &= (-ig)^2 \int \dl{^Dx_1} \, \dl{^Dx_2} \int \dhat{p_1} \, \dhat{p_2} \, \dhat{p_2} \, \dhat{p_3} \, \dhat{p_4} \, \dhat{q} \notag\\
        &\quad\times i\e^{ip_1 \cdot (y_1 - x_2)} i\e^{ip_2 \cdot (y_2 - x_2)} i\e^{ip_3 \cdot (x_1 - z_1)} i\e^{ip_4 \cdot (x_1 - z_2)} i\e^{iq \cdot (x_2 - x_1)} \notag\\
        &\quad\times \frac{1}{q^2 - m^2 + i\varepsilon} \prod_{j=1}^{4} \frac{1}{p_j^2 - m^2 + i\varepsilon}.
    \end{align}
    We use \(q\) for the momentum of the internal propagator to distinguish it from the external propagators.
    We can perform the integrals over \(x_i\) using the identity
    \begin{equation}
        \int \dl{^Dx} \, \e^{ip \cdot x} = (2\pi)^D\delta(p).
    \end{equation}
    Rewriting the exponentials slightly we get
    \begin{equation}
        \e^{-i(p_1 + p_2 - q) \cdot x_2} i\e^{ip_1 \cdot y_1} i\e^{ip_2 \cdot y_2} \e^{i(p_3 + p_4 - q) \cdot x_1} i\e^{-ip_3 \cdot z_1} i\e^{-ip_4 \cdot z_2} i
    \end{equation}
    so we'll get two Dirac deltas:
    \begin{align}
        D &= (-ig)^2 \int \dhat{p_1} \, \dhat{p_2} \, \dhat{p_2} \, \dhat{p_3} \, \dhat{p_4} \, \dhat{q} \, (2\pi)^D \delta(p_1 + p_2 - q) \notag\\
        &\quad\times  (2\pi)^D \delta(p_3 + p_4 - q) \e^{ip_1 \cdot y_1} \e^{ip_2 \cdot y_2}  \e^{-ip_3 \cdot z_1} \e^{-ip_4 \cdot z_2} \notag\\
        &\quad\times \frac{i}{q^2 - m^2 + i\varepsilon} \prod_{j=1}^{4} \frac{i}{p_j^2 - m^2 + i\varepsilon}.
    \end{align}
    We can then perform the \(q\) integral using the second Dirac delta to set \(q = p_3 + p_4\), giving
    \begin{align}
        D &= (-ig)^2 \int \dhat{p_1} \, \dhat{p_2} \, \dhat{p_2} \, \dhat{p_3} \, \dhat{p_4} \, \dhat{q} \, (2\pi)^D \delta(p_1 + p_2 - p_3 - p_4) \notag\\
        &\quad\times \e^{ip_1 \cdot y_1} \e^{ip_2 \cdot y_2}  \e^{-ip_3 \cdot z_1} \e^{-ip_4 \cdot z_2} \notag\\
        &\quad\times \frac{i}{(p_3 + p_4)^2 - m^2 + i\varepsilon} \prod_{j=1}^{4} \frac{i}{p_j^2 - m^2 + i\varepsilon}.
    \end{align}
    Note that the factor of \((2\pi)^D\) in front of the Dirac delta cancels with the hidden factor of \(1/(2\pi)^D\) in \(\dhat{p}\).
    This is now of the form \(D = \inverseFourierTransform\{\widetilde{D}\}\), so we can identify \(\widetilde{D} = \fourierTransform\{D\}\) as
    \begin{equation}
        \widetilde{D}(p_1, p_2, p_3, p_4) = (2\pi)^D\delta(p_1 + p_2 - p_3 - p_4) \frac{i}{(p_3 + p_4)^2 - m^2 + i\varepsilon} \prod_{j=1}^{4} \frac{i}{p_j^2 - m^2 + i\varepsilon}.
    \end{equation}
    Notice that the signs of \(p_i\) in the Dirac delta reflect a sign choice where \(p_1\) and \(p_2\) are incoming momenta and \(p_3\) and \(p_4\) are outgoing momenta.
    If all momenta are chosen to be incoming, as is sometimes the case, then all of the signs would be \(+\).
    This Dirac delta is simply telling us that the total momentum is conserved, so it's not that interesting and is not considered to be part of the amplitude.
    
    Similarly, the external line factors, the product above, don't carry any information, beyond the number of external lines, and aren't present in the amplitude.
    For this reason we consider \define{amputated correlators}\index{amputated correlator}, which are given by omitting this term in momentum space.
    In position space it's slightly more work to amputate a correlator, but it can be done by using
    \begin{equation}
        (\dalembertian_x + m^2) \, i\Delta(x - y) = i \int \dhat{^Dp} (-p^2 + m^2) \frac{\e^{-ip \cdot (x - y)}}{p^2 - m^2 + i\varepsilon} = -i\delta(x - y),
    \end{equation}
    where \(\dalembertian_x\) is the d'Alembert operator with respect to \(x\), rather than \(\diffp{}/{x}\) squared.
    This statement is simply that the propagator is a Green's function of the Klein--Gordon operator, which should be familiar from \course{Quantum Field Theory}.
    We also see from this that the \enquote{inverse} of \(\dalembertian + m^2\) is \(1/(p^2 - m^2 + i\varepsilon)\), another fact we saw in the previous course.
    Using this we can amputate a correlator in position space by acting on it with
    \begin{equation}
        \prod_{j=1}^{n} (+i)(\dalembertian_j + m_j^2)
    \end{equation}
    where \(\dalembertian_j\) is the d'Alembert with respect to \(x_j\) and \(m_j\) is the mass of the \(j\)th particle.
    
    \chapter{Quantum Electrodynamics}
    \epigraph{If you don't use dim reg you'll be shot.}{Donal O'Connell}
    \section{The QED Lagrangian}
    The QED Lagrangian is
    \begin{equation}
        \lagrangianDensity = -\frac{1}{4}F^{\mu\nu}F_{\mu\nu} + \diracadjoint{\psi} (i\slashed{\covariantDerivative} - m)\psi
    \end{equation}
    where \(F_{\mu\nu} = \partial_\mu A_\nu - \partial_\nu A_\mu\) and \(\covariantDerivative_\mu = \partial_\mu - ieA_\mu\).
    One important property of this Lagrangian is the presence of a \(\unitary(1)\) gauge symmetry, given by
    \begin{equation*}
        \psi(x) \mapsto \psi(x)\e^{i\alpha(x)}, \quad \diracadjoint{\psi}(x) \mapsto \diracadjoint{\psi}(x) \e^{-i\alpha(x)}, \qand A_\mu(x) \mapsto A_\mu(x) + \frac{1}{e}\partial_\mu \alpha(x)
    \end{equation*}
    where \(\alpha\) is some function of spacetime taking values in \([0, 2\pi)\) (with continuous second derivatives).
    We can see that this leaves the Lagrangian invariant by considering how each term transforms.
    First,
    \begin{align}
        F_{\mu\nu} &= \partial_\mu A_\nu - \partial_\nu A_\mu\\
        &\mapsto \partial_\mu \left( A_\nu + \frac{1}{e}\partial_\nu \alpha \right) - \partial_\nu \left( A_\mu + \frac{1}{e}\partial_\mu \alpha \right)\\
        &= \partial_\mu A_\nu + \frac{1}{e} \partial_\mu \partial_\nu \alpha - \partial_\nu A_\mu - \frac{1}{e} \partial_\mu \alpha\\
        &= F_{\mu\nu}.
    \end{align}
    Second,
    \begin{align}
        \diracadjoint{\psi}(i\slashed{\covariantDerivative} - m)\psi &= \diracadjoint{\psi}(i\slashed{\partial} + e\slashed{A})\psi\\
        &\mapsto \diracadjoint{\psi}\e^{-i\alpha}\left( i\slashed{\partial} + e\left( \slashed{A} + \frac{1}{e}\slashed{\partial}\alpha \right) - m \right)\psi\e^{i\alpha}\\
        &= \diracadjoint{\psi}\e^{-i\alpha} (i\slashed{\partial} + e\slashed{A} + \slashed{\partial}\alpha - m) \psi\e^{i\alpha}\\
        &= \diracadjoint{\psi}\e^{-i\alpha}\e^{i\alpha}(i\slashed{\partial}(i\alpha) + i\slashed{\partial} + e\slashed{A} + \slashed{\partial}\alpha - m)\psi\\
        &= \diracadjoint{\psi}(\slashed{\partial} + e\slashed{A} - m)\psi\\
        &= \diracadjoint{\psi}(i\slashed{\covariantDerivative} - m)\psi.
    \end{align}
    
    \section{Divergences}
    \epigraph{Personally I detest dim reg. It's weird, but it's easy. It's detestable.}{Donal O'Connell}
    Our focus in the first part of this course will be on renormalisation of divergences in QED.
    We will mainly look at correlators and how we can extract physics from renormalised correlators.
    We will do this through the process of renormalised perturbation theory.
    
    Loop diagrams, such as
    \begin{equation}
        \tikzsetnextfilename{qed-divergent-diagram-example}
        \begin{tikzpicture}[baseline=(current bounding box)]
            \draw[photon] (0, 0) -- (1, 0);
            \draw[electron=0.53] (1, 0) arc (180:0:0.5);
            \draw[positron=0.47] (1, 0) arc (-180:0:0.5);
            \draw[photon] (2, 0) -- (3, 0);
        \end{tikzpicture}
    \end{equation}
    are often divergent.
    We can absorb these divergences into a finite (in QED) set of measured parameters.
    
    To do this we distinguish between the fields and parameters appearing in the original Lagrangian, which we call \defineindex{bare} fields and parameters, and the renormalised fields and parameters.
    Add a label \(\bare\) to each bare quantity so the Lagrangian is
    \begin{equation}
        \lagrangianDensity = -\frac{1}{4}F_{\bare \mu\nu} F_{\bare}^{\mu\nu} + \diracadjoint{\psi}_{\bare} (i\slashed{\covariantDerivative}_{\bare} - m_{\bare})\psi_{\bare}
    \end{equation}
    where \(F_{\bare \mu\nu} = \partial_\mu A_{\bare \nu} - \partial_\nu A_{\bare \mu}\) and \(\covariantDerivative_{\bare \mu} = \partial_\mu - ie_{\bare} A_{\bare \mu}\).
    We can think of the original Lagrangian and the bare quantities as being \enquote{true}, corresponding to some high energy theory.
    Then the renormalised quantities are from the low energy limit of this theory.
    We define the following renormalised quantities in terms of the bare quantities:
    \begin{align}
        A_{\bare\mu} &= \sqrt{Z_3}A_\mu,\\
        \psi_{\bare} &= \sqrt{Z_2}\psi,\\
        m_{\bare} &= m + \delta m,\\
        e_{\bare} &= Z_e e.
    \end{align}
    
    We fix the values of the unknown parameters introduced in these definitions by requiring that correlators of renormalised fields are finite and choosing these unknown parameters in such a way that this is enforced.

    In \course{Quantum Field Theory} we didn't give a special notation for the bare quantities, and instead labelled the renormalised quantities, for example, \(\varphi_{\symrm{R}}\) was the renormalised scalar field and \(\varphi\) was the bare scalar field.
    This was because we didn't cover renormalisation until the end of the course, so the extra \(\bare\) labels would have been a nuisance.
    In this course we start with renormalisation, so we will work with the renormalised quantities more, so we don't give them a special label and instead label the bare quantities.
    
    The parameters \(Z_2\) and \(Z_3\) are called the \define{wave function renormalisation constants}\index{wave function renormalisation constant}.
    The labels 2 and 3 are convention, and we'll introduce \(Z_1\) shortly.
    The square roots are chosen as these fields appear squared in the Lagrangian.
    
    The definition of the renormalised mass above doesn't fit the pattern.
    We've chosen to think of the renormalised mass, \(m\), as simply being shifted by \(\delta m\) from the bare mass, \(m_{\bare}\).
    We could have followed the pattern and written \(m_{\bare} = Z_mm\) with \(Z_m = 1 + \delta m/m\).
    This is nice in QED where if \(m_{\bare} = 0\) then \(m = 0\).
    However in other theories, such as scalar QED this isn't the case, it is possible for the bare field to be massless but the renormalised field has a mass.
    This can work with \(m_{\bare} = Z_mm\), we just have to choose \(Z_m\) so that it diverges when \(m_{\bare} = 0\).
    The reason that we don't define \(A_\mu\) and \(\psi\) in the same way, i.e.\@ \(A_{\bare\mu} = A_{\mu} + \delta A_\mu\) and \(\psi_{\bare} = \psi + \delta \psi\), is because we don't have any other vectors or spinors in our theory to give us \(\delta A_\mu\) or \(\delta \psi\).
    The reason we don't define \(e\) in this way is that when \(e_{\bare} = 0\) the theory is non-interacting, and thus we should also have \(e = 0\).
    Later we will expand \(Z_i\) as \(1 + \delta_i\), which corresponds to \(\delta_m = \delta m/m\).
    
    To make the \(Z_i\) and \(\delta m\) well defined we need to pick a regulator.
    We'll use \defineindex{dimensional regularisation}, or \defineindex{dim reg}\index{dim reg|see{dimensional regularisation}}.
    We also need to pick a renormalisation scheme.
    We'll use \defineindex{modified minimal subtraction}, or \define{\MSbar}\index{MS@\MSbar}.
    
    In dim reg it is actually better to set
    \begin{equation}
        e_{\bare} = Z_e e \mu^\varepsilon
    \end{equation}
    since \(e_{\bare}\) is dimensionless in \(D = 4\), but in \(D = 4 - 2\varepsilon\) dimensions \(e_{\bare}\) has mass dimension \(\varepsilon\).
    We choose \(\mu\) to be a mass scale so that \(e\) is dimensionless in \(D = 4 - 2\varepsilon\) dimensions.
    Importantly \(\mu\) is not a parameter of the bare theory.
    This means that no physics can depend on \(\mu\) so \(\mu\) must cancel out in any computation giving a measurable result.
    The choice of \(\mu\) can effect how quickly perturbation theory converges, for terms of the form \(\log(m/\mu)\) are common, and if we choose \(\mu \approx m\) then this value will be small, whereas if \(\mu \gg m\) we'll get large logs, which we usually want to avoid.
    The parameter \(\mu\) is called the \defineindex{renormalisation point} or occasionally the \defineindex{'t Hooft scale}.
    
    The Lagrangian can the be rewritten in terms of the renormalised quantities.
    First,
    \begin{equation}
        F_{\bare \mu\nu} = \partial_\mu A_{\bare \nu} - \partial_\nu A_{\bare \mu} = \partial_\mu (\sqrt{Z_3} A_\mu) - \partial_\nu (\sqrt{Z_3} A_\nu) = \sqrt{Z_3}F_{\mu\nu},
    \end{equation}
    and so
    \begin{equation}
        F_{\bare\mu\nu}F_{\bare}^{\mu\nu} = Z_3F_{\mu\nu}F^{\mu\nu}.
    \end{equation}
    We also have
    \begin{equation}
        i\slashed{\covariantDerivative}_{\bare} = i\slashed{\partial} + e_{\bare}\slashed{A}_{\bare} = i\slashed{\partial} + Z_e\sqrt{Z_3} e\mu^{\varepsilon} \slashed{A}
    \end{equation}
    so \(\diracadjoint{\psi}_{\bare} i\slashed{\covariantDerivative}_{\bare} \psi_{\bare} = Z_eZ_2\sqrt{Z_3}\).
    We define \(Z_1 = Z_e Z_2 \sqrt{Z_3}\) for notational compactness.
    One imagines that this process was followed in the reverse when \(Z_i\) were named.
    The mass term gives
    \begin{equation}
        \diracadjoint{\psi}_{\bare}m_{\bare}\psi_{\bare} = Z_2m\diracadjoint{\psi}\psi + Z_2\delta m\diracadjoint{\psi}\psi.
    \end{equation}
    So the Lagrangian in terms of the renormalised quantities is
    \begin{equation}
        \lagrangianDensity = -\frac{1}{4}Z_3F^{\mu\nu}F_{\mu\nu} + Z_2\diracadjoint{\psi}(i\slashed{\partial} - m)\psi + Z_1 e\mu^{\varepsilon} \diracadjoint{\psi}\slashed{A}\psi - Z_2 \delta m \diracadjoint{\psi} \psi.
    \end{equation}
    
    It is not immediately clear that this Lagrangian has a gauge symmetry, but of course it does, since it inherits the gauge symmetry of the bare theory.
    This will be made more clear later when we show that \(Z_1 = Z_2\) (\cref{sec:charge renormalisation}), which allows us to write this with a covariant derivative again.
    
    To make sense of this we define \(Z_i = 1 + \delta_i\) and then the Lagrangian is
    \begin{multline}
        \lagrangianDensity = -\frac{1}{4}F^{\mu\nu}F_{\mu\nu} + \diracadjoint{\psi}(i\slashed{\partial} - m)\psi\\
        - \frac{1}{4}\delta_3 F^{\mu\nu} F_{\mu\nu} + \delta_2 \diracadjoint{\psi}(i\slashed{\partial} - m)\psi + \delta_1 e\mu^{\varepsilon}\diracadjoint{\psi}\slashed{A}\psi - \delta m\diracadjoint{\psi} \psi - \delta m \delta_2 \diracadjoint{\psi} \psi.
    \end{multline}
    This is of the form
    \begin{equation}
        \lagrangianDensity = \lagrangianDensity_{\symrm{classical}} + \lagrangianDensity_{\symrm{ct}}
    \end{equation}
    where
    \begin{equation}
        \lagrangianDensity_{\symrm{classical}} = -\frac{1}{4}F^{\mu\nu}F_{\mu\nu} + \diracadjoint{\psi}(i\slashed{\partial} - m)\psi
    \end{equation}
    is the \enquote{classical} Lagrangian, being of the same form as the Lagrangian in terms of the bare parameters and
    \begin{equation}
        \lagrangianDensity_{\symrm{ct}} = -\frac{1}{4}\delta_3 F^{\mu\nu} F_{\mu\nu} + \delta_2 \diracadjoint{\psi}(i\slashed{\covariantDerivative} - m)\psi + \delta_1 e\mu^{\varepsilon}\diracadjoint{\psi}\slashed{A}\psi - \delta m\diracadjoint{\psi} \psi - \delta m \delta_2 \diracadjoint{\psi} \psi
    \end{equation}
    are the \define{counterterms}\index{counterterm}, which we choose in such a way that the divergences cancel out.
    
    Note that in this expression \(\covariantDerivative_\mu = \partial_\mu - ie\mu^{\varepsilon}A_\mu\), with the factor of \(\mu^\varepsilon\).
    It is common to miss out writing in \(\mu^\varepsilon\) both here and in the Lagrangian since it can always be inserted by dimensional analysis and disappears in final results.
    For tree diagrams we can take \(\varepsilon \to 0\) anyway since there are no divergences, making it even more common to leave \(\mu^\varepsilon\) out.
    
    In QED divergences first appear at one loop, and since a single loop with external particles has at least two vertices these divergences are \(\order(e^2)\).
    Since we choose the renormalisation parameters to cancel these divergences they must be of the form \(Z_i = 1 + \order(e^2)\), with the \(1\) giving us the bare theory and the \(\order(e^2)\) cancelling the divergences.
    This means that \(\delta m \delta_2\) is \(\order(e^4)\), so it is only important if we are doing a next-to-next-to leading order (NNLO)\glossary[acronym]{NNLO}{Next-to-Next-to Leading Order} calculation or working with two or more loops.
    We'll only be doing leading order computations with one loop, so we'll neglect the final term of the Lagrangian.
    
    \section{Feynman Rules}
    The Feynman rules tell us how a diagram translates into an equation.
    For a given term involving a product of fields, such as \(\diracadjoint{\psi}\slashed{A}\psi\), the Feynman rules can be derived by considering the tree level correlator \(\correlator{\diracadjoint{\psi} \slashed{A} \psi}\).
    Since this is at tree level the terms just add linearly allowing us to consider them one at a time.
    We then amputate the correlator, working in momentum space, and drop the overall momentum conservation Dirac delta.
    What is left is the Feynman rule for this term.
    
    \subsection{\texorpdfstring{\(\diracadjoint{\psi}\psi\)}{psi-bar psi} Term}
    First we'll consider the \(\diracadjoint{\psi}\psi\) terms in the Lagrangian.
    These correspond to an interaction with the action
    \begin{equation}
        S_{\interaction} = \int \dl{^Dx} \left[ i\delta_2 \diracadjoint{\psi} \slashed{\partial} \psi - (\delta_2 m + \delta m) \diracadjoint{\psi} \psi \right].
    \end{equation}
    Diagrammatically this corresponds to the correlator
    \begin{equation}
        \tikzsetnextfilename{qed-psibar-psi-counterterm-correlator}
        \begin{tikzpicture}
            \draw[electron=0.6] (0, 0) -- (0.85, 0);
            \draw[electron=0.6] (1.15, 0) -- (2, 0);
            \draw (1, 0) circle [radius = 0.15];
            \foreach \angle in {45, 135, 225, 315} {
                \draw (1, 0) -- ++ (\angle:0.15);
            }
            \draw[->] (0.2, 0.2) -- (0.65, 0.2) node [midway, above] {\(p\)};
            \draw[->] (1.35, 0.2) -- (1.8, 0.2) node [midway, above] {\(p'\)};
        \end{tikzpicture}
        .
    \end{equation}
    The symbol \(\otimes\) is used to signify a counter term, since these are often 1-to-1 scattering processes which would otherwise look like just a propagator.
    
    Recall that in the canonical quantisation formalism we can expand \(\psi\) in terms of the electron annihilation operator, \(a_s(p)\), and the positron creation operator, \(b_s^\hermit(p)\):
    \begin{equation}
        \psi(x) = \sum_s \int \frac{\dhat{p}}{2E_p} \left[ a_s(p) u(p, s) \e^{-ip \cdot x} + b_s^\hermit(p) v(p, s) \e^{ip\cdot x} \right].
    \end{equation}
    Then we have
    \begin{equation}
        \slashed{\partial}\psi(x) = \sum_s \int \frac{\dhat{p}}{2E_p} \left[ -i\slashed{p} a_s(p) u(p, s) \e^{-ip \cdot x} + i \slashed{p} b_s^\hermit(p) v(p, s) \e^{ip\cdot x} \right].
    \end{equation}
    
    We want to calculate \(\correlator{\diracadjoint{\psi}\slashed{A}\psi}\) to tree level.
    To do this we expand \(\e^{iS} = 1 + iS + \dotsb\) to first order and consider the first order term, which corresponds to tree level processes (zeroth order corresponds to no interaction occurring).
    So for an incoming particle with momentum \(p\) and outgoing particle with momentum \(p'\) we need to calculate
    \begin{equation}
        \bra{p'} i \int \dl{^Dx} \left[ i \delta_2 \diracadjoint{\psi}\slashed{\partial}\psi - (\delta_2 m + \delta m) \diracadjoint{\psi}\psi \right] \ket{p}.
    \end{equation}
    To do this we need to contract the fields.
    We have an incoming and outgoing electron, which need to be created and destroyed and this can only be done one way.
    The field \(\psi\) can annihilate the incoming electron and the adjoint \(\diracadjoint{\psi}\) can create the outgoing electron.
    Thus we must contract as follows:
    \begin{equation}
        \wick{\bra{\c2{p}'} i \int \dl{^Dx} \, i\delta_2\c2{\diracadjoint{\psi}}\slashed{\partial}\c2{\psi} \ket{\c2{p}}} - \wick{\bra{\c2{p}'} i \int \dl{^Dx} \, (\delta_2 m + \delta m) \c2{\diracadjoint{\psi}}\c2{\psi} \ket{\c2{p}}}.
    \end{equation}
    Completing this contraction, and including the factor of \(i\slashed{p}\) we get from acting with the derivative we get the result
    \begin{equation}
        i \int \dl{^D x} \left[ \delta_2 (\slashed{p} - m) - \delta m \right] \diracadjoint{u}u \e^{-i(p - p') \cdot x}.
    \end{equation}
    Performing this integral we get
    \begin{equation}
        i (2\pi)^D \delta(p - p') \left[ \delta_2 (\slashed{p} - m) - \delta m \right] \diracadjoint{u}u.
    \end{equation}
    To get the Feynman rule we strip off the factors corresponding to external legs, since these are dealt with by other Feynman rules, so we remove \(\diracadjoint{u}u\), and we strip off the overall momentum conserving Dirac delta \((2\pi)^D\delta(p - p')\), since this is enforced by conserving momentum at each vertex, which in this case just corresponds to setting \(p = p'\).
    The resulting Feynman rule is
    \begin{equation}
        \tikzsetnextfilename{qed-psibar-psi-counterterm-correlator-feynman-rule}
        \begin{tikzpicture}[baseline=-0.05cm]
            \draw[electron=0.6] (0, 0) -- (0.85, 0);
            \draw[electron=0.6] (1.15, 0) -- (2, 0);
            \draw (1, 0) circle [radius = 0.15];
            \foreach \angle in {45, 135, 225, 315} {
                \draw (1, 0) -- ++ (\angle:0.15);
            }
            \draw[->] (0.2, 0.2) -- (0.65, 0.2) node [midway, above] {\(p\)};
            \draw[->] (1.35, 0.2) -- (1.8, 0.2) node [midway, above] {\(p\)};
        \end{tikzpicture}
        = i[\delta_2(\slashed{p} - m) - \delta m].
    \end{equation}
    
    \subsection{\texorpdfstring{\(F^{\mu\nu}F_{\mu\nu}\)}{Fmunu Fmunu} Term}
    \epigraph{Well you can't stop me. No one said we were going to use sane notation, just consistent notation.}{Donal O'Connell}
    Now consider the \(F^{\mu\nu}F_{\mu\nu}\) counterterm, which corresponds to the interaction action
    \begin{equation}
        S_{\interaction} = \int \dl{^Dx} \left[ -\frac{1}{4} \delta_3 F^{\mu\nu} F_{\mu\nu} \right].
    \end{equation}
    To calculate the Feynman rule associated with this interaction consider the diagram
    \begin{equation}
        \tikzsetnextfilename{qed-Fmunu-Fmunu-counterterm-correlator}
        \begin{tikzpicture}
            \draw[photon] (0, 0) -- (0.85, 0);
            \draw[photon] (1.15, 0) -- (2, 0);
            \draw (1, 0) circle [radius = 0.15];
            \foreach \angle in {45, 135, 225, 315} {
                \draw (1, 0) -- ++ (\angle:0.15);
            }
            \draw[->] (0.2, 0.2) -- (0.65, 0.2) node [midway, above] {\(k, \varepsilon\)};
            \draw[->] (1.35, 0.2) -- (1.8, 0.2) node [midway, above] {\(k', \varepsilon'\)};
        \end{tikzpicture}
        .
    \end{equation}
    Here \(k\) and \(k'\) are momenta and \(\varepsilon\) and \(\varepsilon'\) are polarisation vectors.
    To first order this diagram gives
    \begin{equation}
        \bra{k', \varepsilon'} i \int \dl{^Dx} \left[ -\frac{1}{4} \delta_3 F^{\mu\nu}F_{\mu\nu} \right] \ket{k, \varepsilon}.
    \end{equation}
    There are two possible ways to perform contractions on this:
    \begin{gather}
        \wick{\bra{\c2{k}', \varepsilon'} i \int \dl{^Dx} \left[ -\frac{1}{4} \delta_3 \c2{F}^{\mu\nu}\c2{F}_{\mu\nu} \right] \ket{\c2{k}, \varepsilon}},\\
        \wick{\bra{\c3{k}', \varepsilon'} i \int \dl{^Dx} \bigg[-\frac{1}{4} \delta_3 \c2{F}^{\mu\nu}\c3{F}_{\mu\nu} \bigg] \ket{\c2{k}, \varepsilon}}.
    \end{gather}
    Since we can freely commute \(F\) with itself and raise and lower the paired indices these two contractions are actually exactly the same.
    So we include a symmetry factor of 2 and only consider one of these contractions.
    We'll take the first contraction and compute
    \begin{equation}
        -i \frac{\delta_3}{2} \wick{\bra{\c2{k}', \varepsilon'} \int \dl{^Dx} \, \c2{F}^{\mu\nu} \c2{F}_{\mu\nu} \ket{\c2{k}, \varepsilon}}.
    \end{equation}
    The next simplification is that for any two index tensor \(X\) we have
    \begin{align}
        (X^{\mu\nu} - X^{\nu\mu})(X_{\mu\nu} - X_{\nu\mu}) &= X^{\mu\nu}(X_{\mu\nu} - X_{\nu\mu}) - X^{\nu\mu}(X_{\mu\nu} - X_{\nu\mu})\\
        \shortintertext{exchanging \(\mu\) and \(\nu\) in the second term}
        (X^{\mu\nu} - X^{\nu\mu})(X_{\mu\nu} - X_{\nu\mu}) &= X^{\mu\nu}(X_{\mu\nu} - X_{\nu\mu}) - X^{\mu\nu}(X_{\nu\mu} - X_{\mu\nu})\\
        &= 2X^{\mu\nu}(X_{\mu\nu} - X_{\nu\mu})
    \end{align}
    so, taking \(X^{\mu\nu} = \partial^\mu A^\nu\), we have
    \begin{equation}
        F^{\mu\nu}F_{\mu\nu} = 2(\partial^\mu A^\nu)(\partial_\mu A_\nu - \partial_\nu A_\mu).
    \end{equation}
    Hence the correlator is
    \begin{equation}
        -i \delta_3 \wick{\bra{\c2{k}', \varepsilon'} \int \dl{^Dx} \partial^\mu \c2{A}^\nu\c3{\overbrace{(\partial_\mu A_\nu - \partial_\nu A_\mu)}} \ket{\c3{k}, \varepsilon}}.
    \end{equation}
    As with the \(\psi\) case the derivatives act on the mode expansion to bring down a factor of \(\pm ik_\mu\) from terms like \(\varepsilon^\mu(k)a_s(k)\e^{-ik\cdot x}\).
    The result of performing the contractions is
    \begin{equation}
        -i\delta_3 \int \dl{^Dx} \, k'^\mu \varepsilon'^\nu (k_\mu \varepsilon_\nu - k_\nu \varepsilon_\mu) \e^{-ik\cdot x} \e^{ik' \cdot x}.
    \end{equation}
    Performing the integral gives a Dirac delta, and we'll also expand the bracket in terms of inner products while we're at it
    \begin{equation}
        -i\delta_3(2\pi)^D\delta(k - k')[(k \cdot k')(\varepsilon \cdot \varepsilon') - (k \cdot \varepsilon)(k' \cdot \varepsilon')].
    \end{equation}
    As before to get the Feynman rule we strip of terms corresponding to external legs and the momentum conservation Dirac delta, setting \(k = k'\), to get the Feynman rule
    \begin{equation}
        \tikzsetnextfilename{qed-Fmunu-Fmunu-counterterm-correlator-feynman-rule}
        \begin{tikzpicture}[baseline=(mu.base)]
            \draw[photon] (0, 0) node [left] (mu) {\(\mu\)} -- (0.85, 0);
            \draw[photon] (1.15, 0) -- (2, 0) node [right] {\(\nu\)};
            \draw (1, 0) circle [radius = 0.15];
            \foreach \angle in {45, 135, 225, 315} {
                \draw (1, 0) -- ++ (\angle:0.15);
            }
            \draw[->] (0.2, 0.2) -- (0.65, 0.2) node [midway, above] {\(k\)};
            \draw[->] (1.35, 0.2) -- (1.8, 0.2) node [midway, above] {\(k\)};
        \end{tikzpicture}
        = -i\delta_3(k^2 \minkowskiMetric_{\mu\nu} - k_\mu k_\nu).
    \end{equation}
    Notice that we have to include the metric from \(\varepsilon' \cdot \varepsilon = \minkowskiMetric_{\mu\nu}\varepsilon'^\mu \varepsilon^\nu\) in order for the indices to match up.
    
    \subsection{\texorpdfstring{\(\diracadjoint{\psi}\slashed{A}\psi\)}{psi-bar A-slashed psi} Term}
    The final counterterm to compute is the one from the \( \diracadjoint{\psi}\slashed{A}\psi\) term.
    The corresponding action is
    \begin{equation}
        S_{\interaction} = \int \dl{^Dx} [e\mu^\varepsilon\delta_1 \diracadjoint{\psi}\slashed{A}\psi].
    \end{equation}
    This has three fields, so contributes as a three point vertex.
    To compute the counterterm in this case we consider the correlator \(\correlator{\diracadjoint{\psi}(x_1)\psi(x_2) A^\mu(x_3)}\).
    Expanding the exponential to first order in the interaction we want to compute
    \begin{equation}
        \int \DL{A} \DD{\psi} \DD{\diracadjoint{\psi}} \, \diracadjoint{\psi}(x_1)\psi(x_2)A^\mu(x_3) \int \dl{^Dx} \, e\mu^\varepsilon \delta_1 \diracadjoint{\psi}(x)\slashed{A}^\nu(x) \psi(x) \e^{i S[\psi, \diracadjoint{\psi}, A]}
    \end{equation}
    where \(S[\psi, \diracadjoint{\psi}, A]\) is the free action.
    There is only one contraction not giving zero:
    \begin{equation}
        \int \DL{A} \DD{\psi} \DD{\diracadjoint{\psi}} \, \wick{\c4{\diracadjoint{\psi}}(x_1)\c3{\psi}(x_2)\c2{A}^\mu(x_3) \int \dl{^Dx} \, e\mu^\varepsilon \delta_1 \c3{\diracadjoint{\psi}}(x)\c2{\slashed{A}}^\nu(x) \c4{\psi}(x)}\e^{i S[\psi, \diracadjoint{\psi}, A]}.
    \end{equation}
    Computing the contractions we are left with external propagators, which we can drop, and then we get the amputated correlator \(ie\mu^{\varepsilon}\delta_1 \gamma^\mu\).
    This gives the Feynman rule
    \begin{equation}
        \tikzsetnextfilename{qed-psibar-A-psi-counterterm-correlator}
        \begin{tikzpicture}[baseline=(mu.base)]
            \draw[electron=0.6] (210:1) -- (210:0.15);
            \draw[positron=0.4] (150:1) -- (150:0.15);
            \draw[photon] (0.15, 0) -- (1, 0) node [right] (mu) {\(\mu\)};
            \draw (0, 0) circle [radius = 0.15];
            \foreach \angle in {45, 135, 225, 315} {
                \draw (0, 0) -- (\angle:0.15);
            }
        \end{tikzpicture}
        = ie\delta_1 \gamma^\mu \mu^\varepsilon
    \end{equation}
    
    \section{Vacuum Polarisation}
    In this section we will compute the vacuum polarisation, which is a correction to an internal photon propagator.
    Before we proceed we'll recap the Feynman rules for QED, including the two counterterms we've just computed.
    \index{QED Feynman rules}
    \begin{itemize}
        \item The \defineindex{QED vertex}:
        \begin{equation}
            \tikzsetnextfilename{qed-qed-vertex}
            \begin{tikzpicture}[baseline=(mu.base)]
                \draw[photon] (-1, 0) node [left] (mu) {\(\mu\)} -- (0, 0);
                \draw[electron=0.6] (0, 0) -- (45:1);
                \draw[positron=0.4] (0, 0) -- (-45:1);
            \end{tikzpicture}
            = ie\mu^{\varepsilon}\gamma^\mu.
        \end{equation}
        Note the factor of \(\mu^\varepsilon\) which we had not previously included as at tree level we were taking \(\varepsilon = 0\).
        \item The QED propagators:
        \begin{itemize}
            \item \define{Electron propagator}\index{electron propagator}:
            \begin{equation}
                \tikzsetnextfilename{qed-electron-propagator}
                \begin{tikzpicture}[baseline=-0.05cm]
                    \draw[electron=0.6] (0, 0) -- (1, 0);
                    \draw[->] (0.2, 0.3) -- (0.7, 0.3) node [midway, above] {\(p\)};
                \end{tikzpicture}
                = \frac{i(\slashed{p} + m)}{p^2 - m^2 + i\varepsilon}.
            \end{equation}
            \item \define{Photon propagator}\index{photon propagator}:
            \begin{equation}
                \tikzsetnextfilename{qed-photon-propagator}
                \begin{tikzpicture}[baseline=(mu.base)]
                    \draw[photon] (0, 0) node [left] (mu) {\(\mu\)} -- (1, 0) node [right] {\(\nu\)};
                    \draw[->] (0.2, 0.3) -- (0.7, 0.3) node [midway, above] {\(k\)};
                \end{tikzpicture}
                = \frac{-i}{k^2 + i\varepsilon} \left( \minkowskiMetric^{\mu\nu} - (1 - \xi) \frac{k^\mu k^\nu}{k^2} \right).
            \end{equation}
            Here \(\xi\) is a gauge parameter.
            The choice of \(\xi = 1\) is the \defineindex{Feynman gauge}, giving the propagator
            \begin{equation}
                \frac{-i\minkowskiMetric_{\mu\nu}}{k^2 + i\varepsilon},
            \end{equation}
            which is nice since the second term vanishes.
            The choice of \(\xi = 0\) is the \defineindex{Landau gauge} or \defineindex{Lorenz gauge}, giving the propagator
            \begin{equation}
                \frac{i}{k^2 + i\varepsilon}\left( \minkowskiMetric^{\mu\nu} - \frac{k^\mu k^\nu}{k^2} \right),
            \end{equation}
            which is nice because when we multiply by \(k_\mu\) this vanishes.
        \end{itemize}
        \item Conserve momentum at each vertex.
        \item Integrate over internal momenta which aren't fixed by momentum conservation, giving an integral
        \begin{equation}
            \int \dhat{^D\ell} = \int \frac{\dl{^D\ell}}{(2\pi)}.
        \end{equation}
        \item Each closed fermion loop gives a factor of \(-1\).
        \item Counterterms:
        \begin{itemize}
            \item electron counter term:
            \begin{equation}
                \vcenter{\hbox{\includegraphics{tikz-external/qed-psibar-psi-counterterm-correlator-feynman-rule}}} = i(\delta_2(\slashed{p} - m) - \delta m).
            \end{equation}
            \item photon counter term:
            \begin{equation}
                \vcenter{\hbox{\includegraphics{tikz-external/qed-Fmunu-Fmunu-counterterm-correlator-feynman-rule}}} = -i\delta_3(k^2\minkowskiMetric_{\mu\nu} - k_\mu k_\nu).
            \end{equation}
        \end{itemize}
        \item For an amputated correlator each external line simply gives a factor of 1.
    \end{itemize}
    
    Consider the correlator
    \begin{equation}
        \correlator{A_\mu(x) A_\nu(y)} = \int \DL{A} \DD{\psi} \DD{\diracadjoint{\psi}} \, A_\mu(x) A_\nu(y) \e^{iS}.
    \end{equation}
    At order \(e^0\) there is no interaction so the photon just propagates freely giving
    \begin{equation}
        \correlator{A_\mu(x) A_\nu(y)}_{\order(e^0)} = iD_{\mu\nu}(x - y)
    \end{equation}
    where \(D_{\mu\nu}\) is the free photon propagator.
    
    At order \(e^1\) if we expand \(\e^{iS}\) to get \(1 + \order(e)\) we get three \(A\) fields and two \(\psi\) fields with no way to contract them all.
    This means that there is no \(\order(e)\) contribution.
    
    At order \(e^2\) expanding \(\e^{iS}\) we get
    \begin{equation}
        \int \DL{A} \DD{\psi} \DD{\diracadjoint{\psi}} \, A_\mu(x) A_\nu(y) \frac{(ie)^2}{2} \int \dl{^Dx_1} \dd{^Dx_2} \, (\diracadjoint{\psi}\slashed{A}\psi)_{x_1} (\diracadjoint{\psi}\slashed{A}\psi)_{x_2} + \text{counterterm}
    \end{equation}
    where \((-)_{x}\) means that all the fields in the brackets are evaluated at \(x\).
    Diagrammatically this is
    \begin{equation}
        \tikzsetnextfilename{qed-AA-correlator-order-e-squared}
        \begin{tikzpicture}[baseline=(current bounding box)]
            \draw[photon] (-1.5, 0) node [above] {\(\mu\)} -- (-0.5, 0);
            \draw[photon] (0.5, 0) -- (1.5, 0) node [above] {\(\nu\)};
            \draw[electron=0.55] (-0.5, 0) arc (180:0:0.5);
            \draw[positron=0.45] (-0.5, 0) arc (-180:0:0.5);
            \node at (1.8, 0) {\(+\)};
            \draw[photon] (2.1, 0) node [above] {\(\mu\)} -- (2.95, 0);
            \draw[photon] (3.25, 0) -- (4.1, 0) node [above] {\(\nu\)};
            \draw (3.1, 0) circle [radius = 0.15];
            \foreach \angle in {45, 135, 225, 315} {
                \draw (3.1, 0) -- ++ (\angle:0.15);
            }
        \end{tikzpicture}
        .
    \end{equation}
    
    The interesting part of the correlator is the loop, so define
    \begin{equation}
        i\Pi_{\mu\nu}(x - y) \coloneqq \correlator{A_\mu(x)A_\nu(y)}|_{\text{1 loop, amputated}}
    \end{equation}
    to be just the loop without the counter term or external photon propagators.
    We want to work in momentum space so define
    \begin{equation}
        \widetilde{\Pi}_{\mu\nu}(k) = \int \dl{^Dx} \, \e^{ik \cdot x} \Pi_{\mu\nu}(x).
    \end{equation}
    Then, enforcing momentum conservation, we have
    \begin{equation}
        i\widetilde{\Pi}_{\mu\nu}(k) = 
        \tikzsetnextfilename{qed-AA-correlator-momentum-space}
        \begin{tikzpicture}[baseline=(current bounding box)]
            \draw[photon] (-1.5, 0) node [above] {\(\mu\)} -- (-0.5, 0);
            \draw[photon] (0.5, 0) -- (1.5, 0) node [above] {\(\nu\)};
            \draw[electron=0.55] (-0.5, 0) arc (180:0:0.5);
            \draw[positron=0.45] (-0.5, 0) arc (-180:0:0.5);
            \node at (1.8, 0) {\(+\)};
            \draw[photon] (2.1, 0) node [above] {\(\mu\)} -- (2.95, 0);
            \draw[photon] (3.25, 0) -- (4.1, 0) node [above] {\(\nu\)};
            \draw (3.1, 0) circle [radius = 0.15];
            \foreach \angle in {45, 135, 225, 315} {
                \draw (3.1, 0) -- ++ (\angle:0.15);
            }
            \draw[->] (-1.3, -0.3) -- (-0.8, -0.3) node [midway, below] {\(k\)};
            \draw[->] (0.8, -0.3) -- (1.3, -0.3) node [midway, below] {\(k\)};
            \draw[->] (110:0.7) arc (110:70:0.7) node [midway, above] {\(q\)};
            \draw[->] (-70:0.7) arc (-70:-110:0.7) node [midway, below] {\(q - k\)};
            \draw[->] (2.3, -0.3) -- (2.75, -0.3) node [midway, below] {\(k\)};
            \draw[->] (3.45, -0.3) -- (3.9, -0.3) node [midway, below] {\(k\)};
        \end{tikzpicture}
        .
    \end{equation}
    
    To evaluate this we use the Feynman rules, which give
    \begin{itemize}
        \item A factor of \(-1\) from the fermion loop.
        \item An integral \(\int \dhat{^Dq}\) for the undetermined momentum.
        \item Entering on the left we first come to a photon propagator, but this is external and we're considering an amputated correlator so it gives a factor of 1.
        \item Next we reach a QED vertex giving a factor of \(ie\mu^\varepsilon \gamma^\mu\).
        \item We then proceed \emph{backwards} along the fermion line, which is an electron propagator with momentum \(q - k\), giving a factor of
        \begin{equation}
            \frac{i(\slashed{q} - \slashed{k} + m)}{(q - k)^2 - m^2 + i\varepsilon}.
        \end{equation}
        \item Another QED vertex giving a factor of \(ie\mu^\varepsilon \gamma^\nu\).
        \item Continuing backwards along the fermion propagator with momentum \(q\) we get
        \begin{equation}
            \frac{i(\slashed{q} + m)}{q^2 - m^2 + i\varepsilon}.
        \end{equation}
    \end{itemize}
    Call these electron propagators \(S(q - k)\) and \(S(q)\) and write in the spinor indices.
    We then have a factor of
    \begin{align}
        \gamma^\mu_{ab} S_{bc}(q - k)\gamma^\nu_{cd}S_{da}(q) &= S_{da}(q)\gamma^\mu_{ab} S_{bc}(q - k)\gamma^\nu_{cd}\\
        &= \tr(S(q)\gamma^\mu S(q - k)\gamma^\nu)\\
        &= \tr(\gamma^\mu S(q - k)\gamma^\nu S(q)).
    \end{align}
    Putting this all together we get
    \begin{equation*}
        -\int \dhat{^Dq} (ie\mu^\varepsilon)^2 \tr\left[ \gamma^\mu \frac{i(\slashed{q} - \slashed{k} + m)}{(q - k)^2 - m^2 + i\varepsilon} \gamma^\nu \frac{i(\slashed{q} + m)}{q^2 - m^2 + i\varepsilon} \right] - i\delta_3(k^2\minkowskiMetric^{\mu\nu} - k^\mu k^\nu).
    \end{equation*}
    
    \chapter{Evaluating Loop Integrals}
    One loop integrals can be computed using the following algorithm\footnote{for examples using this method see the second half of \course{Quantum Field Theory}}
    \begin{enumerate}
        \item Use Feynman parametrisation to rewrite the denominators.
        The simple case is
        \begin{equation}
            \frac{1}{AB} = \int_0^1 \dl{x} \, \frac{1}{[xA + (1 - x)B]^2}
        \end{equation}
        and the more general case is
        \begin{multline*}
            \frac{1}{A_1^{\alpha_1} \dotsm A_n^{\alpha_n}} = \int_0^1 \dl{x_1} \, x^{\alpha_1 - 1} \dotsm \int_0^1 \dl{x_n} \, x^{\alpha_n - 1}\\
            \times \frac{\delta(1 - \sum_i \alpha_i)}{[x_1 A_1 + \dotsb + x_n A_n]^{\sum_i \alpha_i}} \frac{\Gamma(\sum_i \alpha_i)}{\Gamma(\alpha_1) \dotsm \Gamma(\alpha_n)}.
        \end{multline*}
        \item Shift the loop momentum to get the integral in the form
        \begin{equation}
            \int \dhat{^D\ell} \frac{N}{(\ell^2 - \Delta)^n}.
        \end{equation}
        \item Simplify the numerator. This step is new to QED since for scalar \(\varphi^3\) theory as seen in \course{Quantum Field Theory} the numerator is always one.
        We'll see several tricks for performing this simplification later.
        \item Wick rotate defining \(\ell^0 = i\ell^0_{\symrm{E}}\) and \(\ell^2 = -(\ell_{\symrm{E}})^2 - \vv{\ell}^2 = -\ell_{\symrm{E}}^2\) where \(\ell_{\symrm{E}}^2\) is the Euclidean inner product, \(\ell_{\symrm{E}}^2 = \ell_{\symrm{E}}^2 + \vv{\ell}^2\).
        \item Use the identity
        \begin{equation}
            \int \dhat{^D\ell_{\symrm{E}}} \, \frac{(\ell_{\symrm{E}}^2)^p}{(\ell_{\symrm{E}} + \Delta)^n} = \frac{\Gamma(n - p - D/2)\Gamma(p + D/2)}{(4\pi)^{D/2}\Gamma(n)\Gamma(D/2)} \Delta^{D/2 + p - n}.
        \end{equation}
    \end{enumerate}
    
    We can combine the fourth and fifth steps into one to get the identity
    \begin{equation*}
        \int \dhat{^D\ell} \, \frac{(\ell^2)^p}{(\ell^2 - \Delta)^n} = \frac{i}{(4\pi)^{D/2}} \frac{(-1)^{p + n} \Gamma(n - p - D/2)\Gamma(p + D/2)}{\Gamma(n)\Gamma(D/2)} \Delta^{D/2 + p - n}.
    \end{equation*}
    Note that the \(\Delta\) term can be found from dimensional analysis.
    We know from the \(\ell^2 - \Delta\) term that\footnote{\([X]\) is the mass dimension of \(X\), that is the power of mass (or energy or momentum) appearing in the dimensions} \([\Delta] = 2\).
    Then looking at the integral on the left we have \([\dhat{^D\ell}] = D\), \([(\ell^2)^p] = 2p\), and \([(\ell^2 - \Delta)^n] = 2n\), so the integral has mass dimension \(D + 2p - 2n\).
    Dimensions can only enter the right hand side as \(\Delta^x\), since we are integrating out \(\ell\), so we must have \([\Delta^x] = 2x = D + 2p - 2n\), so \(x = D/2 + p - n\).
    
    In QED we often encounter \define{tensor integrals}\index{tensor integral} such as
    \begin{equation}
        I^{\mu_1 \dotso \mu_r} \coloneqq \int \dhat{^D\ell} \frac{\ell^{\mu_1} \dotsm \ell^{\mu_r}}{(\ell^2 - \Delta)^n}.
    \end{equation}
    First consider the \(r = 1\) case:
    \begin{equation}
        I^\mu = \int \dhat{^D\ell} \frac{\ell^\mu}{(\ell^2 - \Delta)^n}.
    \end{equation}
    A change of variables from \(\ell\) to \(-\ell\) gives an overall negative from \(\dl{\ell} \to -\dl{\ell}\) as well as exchanging the limits from \((-\infty, \infty)\) to \((\infty, -\infty)\).
    We can change the limits back at the cost of another overall negative.
    We also get an overall negative from \(\ell^\mu \to \ell^\mu\) so
    \begin{equation}
        I^\mu = \int \dhat{^D\ell} \frac{-\ell^\mu}{(\ell^2 - \Delta)^n} = -I^\mu
    \end{equation}
    and so we must have \(I^\mu = 0\).
    This same logic can be applied whenever \(r\) is odd, so \(I^{\mu_1 \dotso \mu_r} = 0\) for all odd \(r\).
    
    Now consider the \(r = 2\) case:
    \begin{equation}
        I^{\mu\nu} = \int \dhat{^D\ell} \frac{\ell^\mu \ell^\nu}{(\ell^2 - \Delta)^n}.
    \end{equation}
    The result must also be a rank two tensor.
    Since we are integrating over \(\ell\) the result can only depend on \(\Delta\), which is just a scalar.
    The only other dependence in this integral is on the metric, \(\minkowskiMetric^{\mu\nu}\), so this must be where the indices come from.
    We make the ansatz that
    \begin{equation}
        I^{\mu\nu} = \minkowskiMetric^{\mu\nu} X(\Delta)
    \end{equation}
    where \(X\) is some function to be determined.
    Now notice that
    \begin{equation}
        \minkowskiMetric_{\mu\nu}I^{\mu\nu} = \tensor{I}{^\mu_\mu} = \minkowskiMetric_{\mu\nu}\minkowskiMetric^{\mu\nu}X(\Delta) = DX(\Delta) = \int \dhat{^D\ell} \frac{\ell^2}{(\ell^2 - \Delta)^n},
    \end{equation}
    and we know what this result integral is so we have
    \begin{equation}
        I^{\mu\nu} = \frac{1}{D} \int \dhat{^D\ell} \frac{\minkowskiMetric^{\mu\nu}\ell^2}{(\ell^2 - \Delta)^n}.
    \end{equation}
    To conclude, in a loop integral we can replace \(\ell^\mu \ell^\nu\) with \(\minkowskiMetric^{\mu\nu}\ell^2/D\) so long as the denominator is of the form \((\ell^2 - \Delta)^n\).
    Other even \(r\) cases can be worked out in a similar manor.
    
    \section{Vacuum Polarisation}\label{sec:vacuum polarisation}
    Let's return to the problem of computing the vacuum polarisation.
    Define a function corresponding to just the loop part of the vacuum polarisation:
    \begin{equation}
        i\widetilde{\Pi}^{\mu\nu}_{\symrm{L}}(k) = 
        \tikzsetnextfilename{loop-integrals-vacuum-polarisation-loop}
        \begin{tikzpicture}[baseline=(current bounding box)]
            \draw[photon] (-1.5, 0) node [above] {\(\mu\)} -- (-0.5, 0);
            \draw[photon] (0.5, 0) -- (1.5, 0) node [above] {\(\nu\)};
            \draw[electron=0.55] (-0.5, 0) arc (180:0:0.5);
            \draw[positron=0.45] (-0.5, 0) arc (-180:0:0.5);
            \draw[->] (-1.3, -0.3) -- (-0.8, -0.3) node [midway, below] {\(k\)};
            \draw[->] (0.8, -0.3) -- (1.3, -0.3) node [midway, below] {\(k\)};
            \draw[->] (110:0.7) arc (110:70:0.7) node [midway, above] {\(q\)};
            \draw[->] (-70:0.7) arc (-70:-110:0.7) node [midway, below] {\(q - k\)};
        \end{tikzpicture}
        .
    \end{equation}
    As we computed before using the Feynman rules gives the result
    \begin{equation}\label{eqn:photon-self-energy}
        i\widetilde{\Pi}^{\mu\nu}_{\symrm{L}}(k) = - (ie\mu^\varepsilon)^2 \int \dhat{^Dq} \frac{\tr[\gamma^\mu i(\slashed{q} - \slashed{k} + m)\gamma^\nu i(\slashed{q} + m)]}{[(q - k)^2 - m^2 + i\varepsilon][q^2 - m^2 + i\varepsilon]}.
    \end{equation}
    
    The first step of the algorithm is to use Feynman parametrisation.
    In this case we have two denominators, \((q - k)^2 - m^2 + i\varepsilon\) and \(q^2 - m^2 + i\varepsilon\).
    We have to decide which one goes with \(x\) and which goes with \(1 - x\) in
    \begin{equation}
        \frac{1}{AB} = \int_0^1 \dl{x} \, \frac{1}{[xA + (1 - x)B]^2}.
    \end{equation}
    A good idea is to put the more complicated factor with \(x\), so that we don't increase the complexity.
    This gives
    \begin{align}
        A &= x[(q - k)^2 - m^2 + i\varepsilon] = xq^2 - 2xq \cdot k + xk^2 - xm^2 + ix\varepsilon,\\
        B &= (1 - x)[q^2 - m^2 + i\varepsilon] = q^2 - xq^2 - m^2 + xm^2 + i\varepsilon - ix\varepsilon.
    \end{align}
    Hence, we have
    \begin{equation}
        xA + (1 - x)B = q^2 - 2xq \cdot k + xk^2 - m^2 + i\varepsilon.
    \end{equation}
    This allows us to rewrite the loop integral as
    \begin{equation}
        i\widetilde{\Pi}^{\mu\nu}_{\symrm{L}}(k) = e^2 \mu^{2\varepsilon} \int_0^1 \dl{x} \int \dhat{^Dq} \frac{\tr[(\gamma^\mu(\slashed{q} - \slashed{k} + m)\gamma^\nu(\slashed{q} + m))]}{q^2 - 2xq \cdot k + xk^2 - m^2 + i\varepsilon}.
    \end{equation}
    
    The next step of the algorithm is to shift the integration variable to get a denominator of the form \((\ell^2 - \Delta)^n\).
    To do this we choose \(\ell = q - xk\), then we have \(\ell^2 = q^2 - 2xq \cdot k + x^2 k^2\) and so
    \begin{equation}
        q^2 - 2xq \cdot k + xk^2 - m^2 + i\varepsilon = \ell^2 - m^2 + x(1 - x)k^2 + i\varepsilon = \ell^2 - \Delta
    \end{equation}
    with
    \begin{equation}
        \Delta \coloneqq m^2 - x(1 - x)k^2 - i\varepsilon.
    \end{equation}
    Then the loop integral is
    \begin{equation}
        i\widetilde{\Pi}^{\mu\nu}_{\symrm{L}}(k) = -e^2\mu^{2\varepsilon} \int_0^1 \dl{x} \int \dhat{^D\ell} \frac{N^{\mu\nu}}{(\ell^2 - \Delta)^2}
    \end{equation}
    with
    \begin{equation}
        N^{\mu\nu} = \tr[\gamma^\mu(\slashed{q} - \slashed{k} + m)\gamma^\nu (\slashed{q} + m)].
    \end{equation}
    
    The next step in the algorithm is to simplify the numerator.
    The first step is to write the numerator in terms of \(\ell\) which can be done by inverting \(\ell = q - xk\) to get \(q = \ell + xk\).
    We also have \(q - k = \ell + xk - k = \ell - (1 - x)k\).
    Thus the numerator is
    \begin{equation}
        N^{\mu\nu} = \tr[\gamma^\mu(\slashed{\ell} - (1 - x)\slashed{k} + m)\gamma^\nu(\slashed{\ell} + x\slashed{k} + m)].
    \end{equation}
    We can now expand the product in the trace.
    In doing so we keep only terms with an even power of gamma matrices, since the trace of an odd number of gamma matrices vanishes.
    We also keep only terms with an even power of \(\ell\), since under the integral any odd power of \(\ell\) vanishes.
    Note that this is only true under the integral, not in general, so rather than using equality we'll use the symbol \(\rightsquigarrow\):
    \begin{equation}
        N^{\mu\nu} \rightsquigarrow \tr[\gamma^\mu \slashed{\ell}\gamma^\nu\slashed{\ell} - x(1 - x)\gamma^\mu\slashed{k}\gamma^\nu\slashed{k} + m^2\gamma^\mu\gamma^\nu].
    \end{equation}
    To proceed we need the following two identities for traces of gamma matrices:
    \begin{align}
        \tr[\gamma^\mu \gamma^\nu] &= 4\minkowskiMetric^{\mu\nu},\\
        \tr[\gamma^\mu \gamma^\nu \gamma^\rho \gamma^\sigma] &= 4(\minkowskiMetric^{\mu\nu}\minkowskiMetric^{\rho\sigma} + \minkowskiMetric^{\sigma\mu}\minkowskiMetric^{\nu\rho} - \minkowskiMetric^{\mu\rho}\minkowskiMetric^{\nu\sigma}).
    \end{align}
    Note that the factor of 4 is the dimension of the spinors.
    It is possible to define the spinors in such a way that they are four component objects even in dimensional regularisation where the spacetime dimension is \(D = 4 - 2\varepsilon\).
    Consider the first term in the trace, we have
    \begin{align}
        \tr[\gamma^\mu \slashed{\ell} \gamma^\nu \slashed{\ell}] &= \ell_\rho \ell_\sigma \tr[\gamma^\mu \gamma^\rho \gamma^\nu \gamma^\sigma]\\
        &= 4 \ell_\rho \ell_\sigma [\minkowskiMetric^{\mu\rho} \minkowskiMetric^{\nu\sigma} + \minkowskiMetric^{\rho\nu} \minkowskiMetric^{\sigma\mu} - \minkowskiMetric^{\mu\nu}\minkowskiMetric^{\rho\sigma}]\\
        &= 4 [\ell^\mu \ell^\nu + \ell^\nu \ell^\mu - \minkowskiMetric^{\mu\nu}\ell^2]\\
        &= 4 [2 \ell^\mu \ell^\nu - \minkowskiMetric^{\mu\nu}\ell^2].
    \end{align}
    In general in traces like this if we have \(\gamma^\mu \slashed{\ell}\) then we the index \(\mu\) in the identity for traces of \(\gamma^\mu \gamma^\nu\) will be attached to the \(\ell\) in the final result.
    The second term in the numerator is the same but with \(k\)s in place of the \(\ell\)s, and an extra scalar factor.
    The last term in the trace is simply \(4m^2\minkowskiMetric^{\mu\nu}\).
    Combined we have
    \begin{equation}
        N^{\mu\nu} \rightsquigarrow 4[2\ell^\mu \ell^\nu - \minkowskiMetric^{\mu\nu}\ell^2 - x(1 - x)(2k^\mu k^\nu - \minkowskiMetric^{\mu\nu} k^2) + m^2\minkowskiMetric^{\mu\nu}].
    \end{equation}
    We can further simplify this by using \(\Delta = m^2 - x(1 - x)k^2 - i\varepsilon\) and so \(m^2 = \Delta + x(1 - x)k^2 + i\varepsilon\) giving
    \begin{equation}
        N^{\mu\nu} \rightsquigarrow 4[2\ell^\mu \ell^\nu - \minkowskiMetric^{\mu\nu}\ell^2 -2x(1 - x)(k^\mu k^\nu - \minkowskiMetric^{\mu\nu}k^2) + \Delta\minkowskiMetric^{\mu\nu}].
    \end{equation}
    The final simplification we can make is to use our earlier discovery that we can replace \(\ell^\mu\ell^\nu\) with \(\minkowskiMetric^{\mu\nu}\ell^2/D\) under a one loop integral with denominator \((\ell^2 - \Delta)^n\).
    This gives
    \begin{align}
        N^{\mu\nu} &\rightsquigarrow 4\left[ \frac{2}{D}\minkowskiMetric^{\mu\nu}\ell^2 - \minkowskiMetric^{\mu\nu}\ell^2 - 2x(1 - x)(k^\mu k^\nu - \minkowskiMetric^{\mu\nu}k^2) + \Delta\minkowskiMetric^{\mu\nu} \right]\\
        &= 4\left[ \left( \frac{2}{D} - 1 \right)\minkowskiMetric^{\mu\nu}\ell^2 - 2x(1 - x)(k^\mu k^\nu - \minkowskiMetric^{\mu\nu}) + \Delta \minkowskiMetric^{\mu\nu} \right].
    \end{align}
    
    We can now use the two following integrals, first
    \begin{equation}
        \int \dhat{^D\ell} \frac{1}{(\ell^2 - \Delta)^2} = \frac{i}{(4\pi)^{D/2}} \Delta^{D/2 - 2}\Gamma(2 - D/2)
    \end{equation}
    which follows by setting \(p = 0\) and \(n = 2\) in the general formula and using \(\Gamma(1) = 1\).
    Second, we need
    \begin{equation}
        \int \dhat{^D\ell} \frac{\ell^2}{(\ell^2 - \Delta)^2} = -\frac{i}{(4\pi)^{D/2}}\Delta^{D/2 - 1} \frac{\Gamma(1 - D/2)\Gamma(1 + D/2)}{\Gamma(D/2)}.
    \end{equation}
    This can be simplified using the identity \(\Gamma(1 + z) = z\Gamma(z)\), which is just the factorial recursive definition extended to the \(\Gamma\) function.
    With this we have
    \begin{equation}
        \Gamma(1 + D/2) = \frac{D}{2}\Gamma(D/2)
    \end{equation}
    and so
    \begin{equation}
        \int \dhat{^D\ell} \frac{\ell^2}{(\ell^2 - \Delta)^2} = -\frac{i}{(4\pi)^{D/2}}\Delta^{D/2 - 1} \frac{D}{2} \Gamma(1 - D/2).
    \end{equation}
    
    Using these integrals for each term of the full integral we get the result
    \begin{align}
        i\widetilde{\Pi}^{\mu\nu}_{\symrm{L}}(k) = -&4e^2\mu^{2\varepsilon} \int_0^1 \dl{x} \frac{i}{(4\pi)^{D/2}} \\
        &\times\bigg[ \frac{D}{2}\left( \frac{2}{D} - 1 \right)\Gamma(1 - D/2) \Delta^{D/2 - 1} \minkowskiMetric^{\mu\nu} \notag\\
        &\qquad - 2x(1 - x)\Gamma(2 - D/2)\Delta^{D/2 - 2}(k^\mu k^\nu - \minkowskiMetric^{\mu\nu}k^2) \notag\\
        &\qquad + \Gamma(2 - D/2)\Delta^{D/2 - 1}\minkowskiMetric^{\mu\nu} \bigg]. \notag
    \end{align}
    Now we can use
    \begin{equation}
        \frac{D}{2}\left( \frac{2}{D} - 1 \right)\Gamma(1 - D/2) = \left( 1 - \frac{D}{2} \right)\Gamma(1 - D/2) = \Gamma(2 - D/2),
    \end{equation}
    which is just another application of \(z\Gamma(z) = \Gamma(1 + z)\).
    Doing this the first and last term above are the same and so cancel out giving the final result
    \begin{equation}
        i\widetilde{\Pi}^{\mu\nu}_{\symrm{L}}(k) = \frac{8e^2\mu^{2\varepsilon}}{(4\pi)^{D/2}} (k^\mu k^\nu - \minkowskiMetric^{\mu\nu} k^2)\Gamma(2 - D/2) \int_0^1 \dl{x} \, x(1 - x) \Delta^{D/2 - 2}.
    \end{equation}
    Note that \(\Delta\) depends on \(x\) so we can't pull it outside of the integral.
    
    \subsection{Regularisation}
    We will use dimensional regularisation, setting the number of dimensions to \(D = 4 - 2\varepsilon\) for some \(\varepsilon \in \reals\) which we will later take to zero.
    This means that we can neglect terms on the order of \(\varepsilon\), although these terms may be important for two loop diagrams.
    We keep the poles, terms like \(1/\varepsilon\), and finite\footnote{finite here meaning finite and nonzero} terms, \(\order(\varepsilon^0)\).
    
    The gamma function appearing in the vacuum polarisation loop becomes
    \begin{equation}
        \Gamma(2 - D/2) = \Gamma(\varepsilon).
    \end{equation}
    We can then use the expansion of the gamma function about zero:
    \begin{equation}
        \Gamma(\varepsilon) \approx \frac{1}{\varepsilon} - \gamma + \order(\varepsilon)
    \end{equation}
    where
    \begin{equation}
        \gamma \approx 0.57721566
    \end{equation}
    is the \defineindex{Euler--Mascheroni constant}\index{\(\gamma\)|see{Euler--Mascheroni constant}}.
    It is actually more useful to write this expansion as
    \begin{equation}
        \frac{1}{\varepsilon}\e^{-\varepsilon \gamma} + \order(\varepsilon) \approx \frac{1}{\varepsilon}(1 - \varepsilon\gamma) + \order(\varepsilon) = \frac{1}{\varepsilon} - \gamma + \order(\varepsilon) \approx \Gamma(\varepsilon).
    \end{equation}
    
    Using this, and factoring out a minus sign from the tensor structure, we can write the vacuum polarisation loop as
    \begin{equation}
        i\widetilde{\Pi}_{\symrm{L}}^{\mu\nu}(k) = -\frac{8ie^2}{16\pi^2}(4\pi \mu^2 \e^{-\gamma})^{\varepsilon}(k^2 \minkowskiMetric^{\mu\nu} - k^\mu k^\nu) \frac{1}{\varepsilon} \int_0^1 \dl{x} \, x(1 - x) \Delta^{-\varepsilon}.
    \end{equation}
    The integral here is over a finite region, \([0, 1]\), and so diverges only if the integrand diverges.
    In dimensional regularisation we always get factors of \(4\pi \mu^2\e^{-\gamma}\), so it helps to define
    \begin{equation}
        \tilde{\mu}^2 = 4\pi\mu^2\e^{-\gamma} \approx 7.0555 \mu^2.
    \end{equation}
    
    Now write
    \begin{align}
        \tilde{\mu}^{2\varepsilon} \Delta^{-\varepsilon} &= \left( \frac{\Delta}{\tilde{\mu}^2} \right)^{-\varepsilon}\\
        &= \exp\left\{ \log \left( \frac{\Delta}{\tilde{\mu}^2} \right)^{-\varepsilon} \right\}\\
        &= \exp\left\{ -\varepsilon\log \frac{\Delta}{\tilde{\mu}^2} \right\}\\
        &\approx 1 - \varepsilon \log \frac{\Delta}{\tilde{\mu}^2} + \order(\varepsilon^2).
    \end{align}
    Then we have
    \begin{equation}
        i\widetilde{\Pi}_{\symrm{L}}^{\mu\nu}(k) = -\frac{ie^2}{2\pi^2} (k^2 \minkowskiMetric^{\mu\nu} - k^\mu k^\nu) \int_0^1 \dl{x} \, x(1 - x) \left[ \frac{1}{\varepsilon} - \log \frac{\Delta}{\tilde{\mu}^2} \right].
    \end{equation}
    This is now perfectly finite so long as \(\varepsilon \ne 0\) (and \(\Delta/\tilde{\mu}^2\) > 0).
    
    \subsection{Renormalisation}
    We require that correlators of renormalised fields are finite.
    Recall that the counter term appearing in the vacuum polarisation is
    \begin{equation}
        \vcenter{\hbox{\includegraphics{tikz-external/qed-Fmunu-Fmunu-counterterm-correlator-feynman-rule}}} = -i\delta_3(k^2\minkowskiMetric^{\mu\nu} - k^\mu k^\nu).
    \end{equation}
    Fortunately this has the same tensor structure, \(k^2 \minkowskiMetric^{\mu\nu} - k^\mu k^\nu\), as the loop part of the vacuum polarisation.
    This is important as it allows us to choose \(\delta_3\) to cancel the infinities which occur in the loop integral to get a finite result.
    This is not always the case.
    For example, if we use an ultraviolet cut-off, \(\Lambda\), then the cancellation earlier in the calculation doesn't occur and we don't have matching tensor structures.
    The reason for this is that dimensional regularisation respects gauge symmetry, but an ultraviolet cut-off doesn't, since, for example, a gauge transformation \(p_\mu \mapsto p_\mu + eA_\mu\) could result in a momentum with \(p^2 > \Lambda^2\).
    
    We can write the full one-loop vacuum polarisation as
    \begin{align}
        \widetilde{\Pi}^{\mu\nu}(k) &= 
        \vcenter{\hbox{\includegraphics{tikz-external/qed-AA-correlator-momentum-space}}}\\
        &= (k^2 \minkowskiMetric^{\mu\nu} - k^\mu k^\nu) \left[ -\delta_3 - \frac{e^2}{2\pi^2} \int_0^1 \dl{x} \, x(1 - x)\left[ \frac{1}{\varepsilon} - \log \frac{\Delta}{\tilde{\mu}^2} \right] \right].\notag
    \end{align}
    We can compute the first term in the integral,
    \begin{equation}
        \frac{1}{\varepsilon} \int_0^1 \dl{x} \, x(1 - x) = \frac{1}{6\varepsilon},
    \end{equation}
    giving
    \begin{align}
        \widetilde{\Pi}^{\mu\nu}(k) &= (k^2 \minkowskiMetric^{\mu\nu} - k^\mu k^\nu) \left[ -\delta_3 - \frac{e^2}{12\pi^2\varepsilon} - \frac{e^2}{2\pi^2} \int_0^1 \dl{x} \, x(1 - x) \log\frac{\Delta}{\tilde{\mu}^2} \right] \notag\\
        &= (k^2 \minkowskiMetric^{\mu\nu} - k^\mu k^\nu) \Pi(k^2)
    \end{align}
    where we define \(\Pi(k^2)\) to contain the scalar structure of the vacuum polarisation.
    
    We demand that this scalar structure is finite, however currently we have a pole at \(\varepsilon = 0\) from the \(1/\varepsilon\) term.
    The solution is to choose \(\delta_3\) to be of the form
    \begin{equation}\label{eqn:delta 3 QED}
        \delta_3 = -\frac{e^2}{12\pi^2 \varepsilon} + \text{finite terms}
    \end{equation}
    so that the \(1/\varepsilon\) terms cancel out and we are left only with the finite terms.
    
    \subsubsection{Choice of Renormalisation Scheme}
    Clearly in the above prescription for \(\delta_3\) we have some freedom in the choice of finite terms.
    In general we can add terms \(\order(\varepsilon^0)\) to \(\delta_3\) and the result will still be finite.
    This freedom is the freedom in choosing a \defineindex{renormalisation scheme}.
    We'll discuss two renormalisation schemes here.
    
    \paragraph{\MSbar{} Scheme}
    The \define{\MSbar}\index{MS@\MSbar} scheme, standing for \defineindex{modified minimal subtraction}, is defined by two requirements:
    \begin{itemize}
        \item We use
        \begin{equation}
            \tilde{\mu}^2 = 4\pi \mu^2 e^{-\gamma}
        \end{equation}
        as a parameter.
        This is the \enquote{modified} part of modified minimal subtraction, and simply reduces the number of \(4\pi\)s and \(\e^{-\gamma}\)s that we have to deal with.
        \item We require counterterms to contain only poles, so no finite terms, this is the \enquote{minimal subtraction} part of modified minimal subtraction.
        This means that all counterterms are of the form
        \begin{equation}
            \delta_i = \sum_{n = 1}^{\infty} \frac{a_n}{\varepsilon^n}
        \end{equation}
        for some finite quantities \(a_n\).
    \end{itemize}
    Thus in \MSbar{} the counterterm for the vacuum polarisation term is
    \begin{equation}
        \delta_3 = \frac{e^2}{12\pi^2\varepsilon} + \order\left( \frac{e^4}{\varepsilon^2} \right).
    \end{equation}
    The factor in front of any higher order term is just a number, it has no dependence on momentum, since it can be traced back to the Lagrangian where all momenta are explicitly accounted for.
    
    We will almost entirely use the \MSbar{} scheme, but it isn't the only choice.
    For the rest of this section we'll briefly discuss an alternative.
    
    \paragraph{On-shell Scheme}
    The \defineindex{on-shell scheme} is imposed by the normalisation condition \(\Pi(0) = 0\).
    If we write out \(\Pi(k^2)\) in full\footnote{we use \(\varepsilon\) here for the parameter of dimensional regularisation and \(\epsilon\) for the parameter of the \(i\epsilon\) prescription.},
    \begin{equation}
        \Pi(k^2) = -\left[ \delta_3 + \frac{e^2}{12\pi^2\varepsilon} - \frac{e^2}{2\pi^2} \int_0^1 \dl{x} \, x(1 - x) \log\left( \frac{m^2 - x(1 - x)k^2 - i\epsilon}{\tilde{\mu}^2} \right) \right]
    \end{equation}
    then we see that at \(k^2 = 0\) we can take \(\epsilon \to 0\) since \(m^2 > 0\).
    The integral is then easy to do, again giving \(1/6\), and so
    \begin{equation}
        \Pi(0) = -\left[ \delta_3 + \frac{e^2}{12\pi^2\varepsilon} - \frac{e^2}{12\pi^2} \log \frac{m^2}{\tilde{\mu}^2} \right].
    \end{equation}
    Demanding \(\Pi(0) = 0\) we get
    \begin{equation}
        \delta_3 = -\frac{e^2}{12\pi^2\varepsilon} + \frac{e^2}{12\pi^2} \log \frac{m^2}{\tilde{\mu}^2}.
    \end{equation}
    
    A sensible question is why we impose \(\Pi(0) = 0\) as a condition.
    There are at least two reasons for this.
    First, doing so ensures that the photon propagator takes on its tree form,
    \begin{equation}
        \frac{i\minkowskiMetric^{\mu\nu}}{k^2 + i\varepsilon}.
    \end{equation}
    Essentially, this means that the photon remains massless, since the propagator has a pole at the physical mass.
    
    The second reason can be demonstrated by considering \(2 \to 2\) scattering.
    For simplicity we'll consider scalars scattering, which just allows us to drop the spinors that would normally occur.
    Consider the process
    \begin{equation}
        \tikzsetnextfilename{loop-integrals-2-to-2-scattering-tree-level-on-shell}
        \begin{tikzpicture}[baseline=(current bounding box)]
            \draw[charged scalar] (0, 0) -- (2, 0.5);
            \draw[charged scalar] (2, 0.5) -- (4, 0);
            \draw[charged scalar] (0, 3) -- (2, 2.5);
            \draw[charged scalar] (2, 2.5) -- (4, 3);
            \draw[photon] (2, 0.5) -- (2, 2.5);
            \draw[->] (2.2, 1) -- (2.2, 2) node [midway, right] {\(k\)};
            \draw[->, yshift=-0.2cm] (0.4, 0.1) -- (1.6, 0.4) node [midway, below] {\(p_2\)};
            \draw[->, yshift=-0.2cm] (2.4, 0.4) -- (3.6, 0.1) node [midway, below, xshift=-0.1cm] {\(p_2 - k\)};
            \draw[->, yshift=0.2cm] (0.4, 2.9) -- (1.6, 2.6) node [midway, above] {\(p_1\)};
            \draw[->, yshift=0.2cm] (2.4, 2.6) -- (3.6, 2.9) node [midway, above, xshift=-0.1cm] {\(p_1 + k\)};
        \end{tikzpicture}
        .
    \end{equation}
    The amplitude for this at tree level is
    \begin{equation}
        i\amplitude_{\symrm{tree}} = \frac{ie^2}{k^2}(4p_1 \cdot p_2 - k^2).
    \end{equation}
    At large distances \(k^2\) is small, so
    \begin{equation}
        i\amplitude_{\symrm{tree}} \approx \frac{ie^2}{k^2}4p_1 \cdot p_2.
    \end{equation}
    
    At one loop we encounter diagrams like
    \begin{equation}
        \tikzsetnextfilename{loop-integrals-2-to-2-scattering-one-loop-level-on-shell}
        \begin{tikzpicture}[baseline=(current bounding box)]
            \draw[charged scalar] (0, 0) -- (2, 0.5);
            \draw[charged scalar] (2, 0.5) -- (4, 0);
            \draw[charged scalar] (0, 3) -- (2, 2.5);
            \draw[charged scalar] (2, 2.5) -- (4, 3);
            \draw[electron=0.55] (2, 1) arc (-90:90:0.5);
            \draw[positron=0.45] (2, 1) arc (270:90:0.5);
            \draw[photon] (2, 0.5) -- (2, 1);
            \draw[photon] (2, 2.5) -- (2, 2);
            \draw[->] (2.7, 1) -- (2.7, 2) node [midway, right] {\(k\)};
            \draw[->, yshift=-0.2cm] (0.4, 0.1) -- (1.6, 0.4) node [midway, below] {\(p_2\)};
            \draw[->, yshift=-0.2cm] (2.4, 0.4) -- (3.6, 0.1) node [midway, below, xshift=-0.1cm] {\(p_2 - k\)};
            \draw[->, yshift=0.2cm] (0.4, 2.9) -- (1.6, 2.6) node [midway, above] {\(p_1\)};
            \draw[->, yshift=0.2cm] (2.4, 2.6) -- (3.6, 2.9) node [midway, above, xshift=-0.1cm] {\(p_1 + k\)};
        \end{tikzpicture}
        .
    \end{equation}
    In fact, we can carefully choose our theory so that this is the dominant diagram at one-loop.
    Diagrams such as
    \begin{equation}
        \tikzsetnextfilename{loop-integrals-2-to-2-scattering-one-loop-level-on-shell-2}
        \begin{tikzpicture}[baseline=(current bounding box)]
            \draw[charged scalar] (0, 0) -- (1.5, 0.5);
            \draw[charged scalar] (1.5, 0.5) -- (3, 0);
            \draw[charged scalar] (0, 2) -- (1.5, 1.5);
            \draw[charged scalar] (1.5, 1.5) -- (3, 2);
            \draw[photon] (1.5, 0.5) -- (1.5, 1.5);
            \draw[photon] (0.3, 1.9) to[bend left] (2.7, 1.9);
        \end{tikzpicture}
    \end{equation}
    are suppressed by powers of \(1/m\), with \(m\) the mass of the external particles.
    By choosing a theory with lots of fermions we can also make the fermion loop the dominant loop.
    
    Notice that the diagram contains the loop diagram in the vacuum polarisation as a subdiagram.
    This means that we can just replace the normal photon propagator with this term, which we've already computed.
    This gives
    \begin{align}
        i\amplitude_{\symrm{loop}} &= (ie)^2(2p_1 + k)^\mu(2p_2 - k)^\nu \left( -\frac{i}{k^2} \right)(k^2\minkowskiMetric_{\mu\nu} - k_\mu k_\nu)\Pi(k^2)\\
        &= (ie)^24(k^2 (p_1 \cdot p_2) - (p_1 \cdot k)(p_2 \cdot k)) \frac{1}{k^4}\Pi(k^2)
    \end{align}
    where in the second line we've used \(k^\mu(k^2\minkowskiMetric_{\mu\nu} - k_\mu k_\nu) = 0\).
    At large distances we then have
    \begin{equation}
        \label{eqn:vacuum-polarisation-2-to-2-scattering}
        i(\amplitude_{\symrm{tree}} + \amplitude_{\symrm{loop}}) \approx \frac{i}{k^2}4 p_1 \cdot p_2 e^2 (1 + \Pi(k^2)).
    \end{equation}
    
    One way to measure the charge of a particle is to perform scattering experiments like this one.
    These are typically done at low momentum.
    What we actually measure in these experiments is the factor \(e^2(1 + \Pi(k^2))\), plus any higher order corrections.
    Choosing \(\Pi(0) = 0\) then means that the value we are measuring is the renormalised charge, which is the charge appearing in the Lagrangian after regularisation.
    
    \chapter{Interpreting The Vacuum Polarisation}
    Recall that in the last chapter we computed the vacuum polarisation
    \begin{align}
        \Pi^{\mu\nu}(k) &= \vcenter{\hbox{\includegraphics{tikz-external/qed-AA-correlator-momentum-space}}}\\
        &= (k^\mu k^\nu - \minkowskiMetric k^2)\Pi(k^2)
    \end{align}
    where
    \begin{equation}
        \Pi(k^2) =
        \begin{cases}
            \displaystyle \frac{e^2}{2\pi^2} \int_0^1 \dl{x} \, x(1 - x) \log \frac{\Delta}{\tilde{\mu}^2} & \MSbar,\\
            \displaystyle \frac{e^2}{2\pi^2} \int_0^1 \dl{x} \, x(1 - x) \log \frac{\Delta}{m^2} & \text{on-shell},
        \end{cases}
    \end{equation}
    where
    \begin{equation}
        \Delta = m^2 - x(1 - x)k^2 - i\epsilon.
    \end{equation}
    Now we're going to try to extract some physics from this result and discuss the effects of the two different renormalisation scales.
    
    \section{Scale Dependence of Charge}
    Previously (\cref{eqn:vacuum-polarisation-2-to-2-scattering}) we saw that the amplitude for \(2 \to 2\) scattering process between scalar particles exchanging a photon has
    \begin{equation}
        i\amplitude = \frac{4i}{k^2} p_1 \cdot p_2 e^2(1 + \Pi(k^2)).
    \end{equation}
    This includes the tree level process and a single fermion loop in the photon propagator, and it is assumed that \(k^2\) is small.
    
    We can use this process to measure the charge of the particle in scattering process with zero, or very little, momentum.
    We find that measured charge squared, at one loop accuracy, is then
    \begin{align}
        e_{\measured}^2 &= e^2(1 + \Pi(0))\\
        &= 
        \begin{cases}
            \displaystyle e^2\left( 1 + \frac{e^2}{12\pi^2} \log \frac{m^2}{\tilde{\mu}^2} \right) & \MSbar,\\
            e^2 & \text{on-shell}.
        \end{cases}
    \end{align}
    We can see that the relation between the Lagrangian parameters and the physical observables is scheme dependent.
    This isn't too much of a problem because in QED we can make a finite number of measurements to determine these parameters and from there we can use them to make scheme independent predictions.
    
    For example, at nonzero momentum there is a correction to the measured charge squared:
    \begin{align}
        e_{\measured}^2(k^2) - e_{\measured}^2(0) &= e^2(1 + \Pi(k^2)) - e^2(1 + \Pi(0))\\
        &= e^2(\Pi(k^2) - \Pi(0))\\
        &= \frac{e^4}{2\pi^2} \int_0^1 \dl{x} \, x(1 - x) \log \frac{\Delta}{m^2}
    \end{align}
    in the \MSbar{} scheme.
    This \(\order(e^4)\) correction to the measured charge is a physical prediction of QED.
    
    \section{Non-Relativistic Limit}
    We can take the non-relativistic limit in QFT to compare results to results from non-relativistic quantum mechanics.
    Recall that in quantum mechanics the Born approximation gives the scattering amplitude
    \begin{equation}
        \amplitude = -4m_1m_2 \int \dl{^3x} \, \e^{-i\vv{k} \cdot \vv{x}} V(\vv{x})
    \end{equation}
    where \(m_i\) are the masses of the two particles scattering and \(V(\vv{x})\) is the potential they are scattering in, with \(\vv{x}\) being the vector between the two particles.
    
    Note that this differs by a factor of \(2m\) from the amplitudes we have been calculating.
    This is due to a difference in normalisation, where in quantum mechanics
    \begin{equation}
        \braket{\vv{p}}{\vv{p}'} = \delta(\vv{p} - \vv{p}')
    \end{equation}
    and in QFT
    \begin{equation}
        \braket{p}{p'} = \frac{1}{2E_p}\delta(p - p')
    \end{equation}
    with \(2E_p \approx 2m\) in the non-relativistic limit.
    
    Consider Coulomb scattering, the amplitude calculated in quantum mechanics is then
    \begin{align}
        \amplitude &= -4m_1m_2 \int \dl{^3x} \, \e^{-i\vv{k} \cdot \vv{x}} \frac{e_1e_2}{4\pi\abs{\vv{x}}}\\
        &= -4m_1m_2\frac{e_1e_2}{\abs{\vv{k}}^2}.
    \end{align}
    
    For the non-relativistic QFT result we take \(k^2\) to be small.
    Note that \(p_1 \cdot p_2 \approx m_1 m_2\).
    There is an incoming (and hence on-shell) particle with momentum \(p_1\), so \(p_1^2 = m_1^2\).
    There is an outgoing (and hence on-shell) particle with momentum \(p_1 + k\), so
    \begin{equation}
        (p_1 + k)^2 = m_1^2 = m_1^2 + 2p_1 \cdot k + k^2.
    \end{equation}
    From this, neglecting \(k^2\) terms, we see that \(2p_1 \cdot k \approx 0\).
    Since \(p_1^0 \approx m_1\), which can be large, we must have \(k^0 \approx 0\), and so \(k^2 \approx -\vv{k}^2\).
    Then the amplitude calculated in QFT is
    \begin{equation}
        \amplitude = -\frac{4m_1m_2}{\abs{\vv{k}}^2} e_1e_2(1 + \Pi(-\vv{k}^2)).
    \end{equation}
    
    We can interpret this as having a one-loop correction to the potential, given in momentum space by
    \begin{equation}
        \Delta\tilde{V}(\vv{k}) = \frac{e_1e_2}{\vv{k}^2}\Pi(-\vv{k}^2)
    \end{equation}
    and in real space by
    \begin{equation}
        \Delta V(\vv{r}) = e_1 e_2 \int \dhat{^3k} \, \e^{-i\vv{k} \cdot \vv{r}} \frac{\Pi(-\vv{k}^2)}{\vv{k}^2}.
    \end{equation}
    This corrected potential is called the \defineindex{Uehling potential}.
    
    In the on-shell scheme for \(k^2 \ll m^2\) we can expand
    \begin{align}
        \Pi(k^2) &= \frac{e^2}{2\pi^2} \int_0^1 \dl{x} \, x(1 - x) \log\left( 1 - x(1 - x)\frac{k^2}{m^2} \right)\\
        &\approx \frac{e^2}{2\pi^2} \int_0^1 \dl{x} \, x(1 - x) \left[ -x(1 - x)\frac{k^2}{m^2} \right]\\
        &= \frac{e^2}{60\pi^2} \frac{\vv{k}^2}{m^2}.
    \end{align}
    Here we've used the expansion \(\log(1 + \alpha) \approx \alpha\) and we've set \(\epsilon = 0\) since the term in the logarithm is positive as \(k^2 < m^2\) and \(0 < x(1 - x) < 1/4\).
    
    Hence, the correction to the potential is
    \begin{equation}
        \Delta V(\vv{r}) = \frac{e_1 e_2 e^2}{60\pi^2m^2} \delta(\vv{r}).
    \end{equation}
    
    If we consider one of the particles to be a nucleus with \(Z\) protons then \(e_1 = -Ze\), taking the other particle to be an electron, \(e_2 = e\), we can interpret this as a shift in the energy levels of the atom.
    Note that since we're considering scalar particles scattering we are essentially ignoring spin-orbit effects, however these are subdominant terms since there is no direct spin-dependence in the potential.
    The energy shift is given by
    \begin{equation}
        \Delta E = \int \dl{^3x} \, \psi^*(\vv{x}) \Delta V(\vv{x}) \psi(\vv{x})
    \end{equation}
    where \(\psi\) is the wave function of the electron in the atom.
    Since we have a factor of \(\delta(\vv{r})\) this shift only applies to wave functions which are nonzero at the origin, meaning it only applies to \(\symrm{s}\) states.
    The energy shift to these states is
    \begin{equation}
        \Delta E = -\abs{\psi(0)}^2 \frac{4Z\alpha^2}{15m^2}
    \end{equation}
    with \(\alpha = e^2/(4\pi)\) being the \defineindex{fine structure constant}.
    
    This shift is one contribution to the Lamb shift, the difference between the energy levels \(\tensor[^2]{\symrm{S}}{_{1/2}}\) and \(\tensor[^2]{\symrm{P}}{_{1/2}}\) which is  not predicted by the Dirac equation alone.
    It is not the dominant contribution to the Lamb shift.
    This shift is more important for muonic atoms, where the electron is replaced by a muon, since heavier external particles make the one-fermion-loop diagram more dominant, as opposed to diagrams with, say, photon loops, since these are suppressed by factors of \(1/m\), with \(m\) the mass of the external particles.
    
    \section{Effective Charge Distribution}
    We can write the corrected potential as
    \begin{equation}
        V(\vv{r}) = e_1e_2 \int \dl{^3x} \dd{^3y} \, \frac{\eta(\vv{x})\eta(\vv{y})}{4\pi\abs{\vv{x} - \vv{y} + \vv{r}}}
    \end{equation}
    where
    \begin{equation}
        \eta(\vv{r}) = \delta(\vv{r}) + \frac{1}{2} \dhat{^3k} \, \Pi(-\vv{k}^2) \e^{i\vv{k} \cdot \vv{r}}.
    \end{equation}
    Note that we are implicitly dropping the \(\Pi^2\) term appearing in the product \(\eta(\vv{x})\eta(\vv{y})\) since this term is \(\order(e^6)\), which is of a higher order than the one loop calculations we have been doing, and so is not physically meaningful if we don't also include two loop calculations.
    To see how this works expand this form of the potential, dropping the \(\Pi^2\) term, giving
    \begin{align}
        V(\vv{r}) &= e_1e_2 \int \dl{^3x} \dd{^3y} \, \frac{1}{4\pi\abs{\vv{x} - \vv{y} + \vv{r}}} \left[ \delta(\vv{x}) + \frac{1}{2} \int \dhat{^3k} \, \Pi(-\vv{k}^2)\e^{i\vv{k} \cdot \vv{x}} \right] \notag\\
        &\qquad\qquad\times \left[ \delta(\vv{y}) + \frac{1}{2} \int \dhat{^3k} \, \Pi(-\vv{k}^2)\e^{i\vv{k} \cdot \vv{y}} \right]\\
        &= e_1e_2 \int \dl{^3x} \dd{^3y} \frac{1}{4\pi\abs{\vv{x} - \vv{y} + \vv{r}}} \bigg[ \delta(\vv{x})\delta(\vv{y})\\
        &\qquad + \frac{1}{2} \delta(\vv{x}) \int \dhat{^3k} \, \Pi(-\vv{k}^2)\e^{-i\vv{k}\cdot\vv{y}} + \frac{1}{2}\delta(\vv{y})\int\dhat{^3k} \, \Pi(-\vv{k}^2)\e^{-i\vv{k}\cdot\vv{x}} \bigg] \notag\\
        &= \frac{e_1e_2}{4\pi\abs{\vv{\vv{r}}}} + \frac{1}{2} \int \dl{^3y} \frac{e_1e_2}{4\pi\abs{-\vv{y} + \vv{r}}} \int \dhat{^3k} \, \Pi(-\vv{k}^2)\e^{-i\vv{k}\cdot\vv{y}}\notag\\
        &\qquad\qquad+ \frac{1}{2} \int \dl{^3y} \frac{e_1e_2}{4\pi\abs{\vv{x} + \vv{r}}} \int \dhat{^3k} \, \Pi(-\vv{k}^2)\e^{-i\vv{k}\cdot\vv{x}}\\
        &= \frac{e_1e_2}{4\pi\abs{\vv{r}}} + \int \dl{^3x} \, \frac{e_1e_2}{4\pi\abs{\vv{x} + \vv{r}}} \int \dhat{^3k} \, \Pi(-\vv{k}^2) \e^{-i\vv{k}\cdot\vv{x}}.
    \end{align}
    In the last step we make the transformation of variables \(\vv{y} \mapsto -\vv{x}\) and identify that both integrals are the same.
    Now recognise that we can write the first factor in the integral as the inverse Fourier transform of it's Fourier transform:
    \begin{equation}
        \frac{1}{4\pi\abs{\vv{x} + \vv{r}}} = \int \dhat{^3k} \, \frac{1}{\abs{\vv{k}}^2} \e^{-i\vv{k} \cdot (\vv{x} + \vv{r})}
    \end{equation}
    and so we get
    \begin{equation}
        V(\vv{r}) = \frac{e_1e_2}{4\pi\abs{\vv{r}}} + \int \dhat{^3k} \, \frac{1}{\abs{\vv{k}}^2} \Pi(-\vv{k}^2) \e^{-i\vv{k} \cdot \vv{r}}.
    \end{equation}
    This takes the form we had in the previous section of a Coulomb potential plus the correction from one loop diagrams.
    
    The interpretation of this rewriting of the potential is that the electron's charge distribution is a superposition of a point charge, given by the \(1/\abs{\vv{r}}\) Coulomb part of the potential, but also has some spatial distribution coming from this extra one loop term.
    The intuition is that this is similar to a point charge in a medium causing that medium to become polarised around it.
    The electron polarises the sea of virtual particles around it.
    This is why we call this term the vacuum polarisation, since it occurs outside of any medium.
    
    \section{Large Logs}
    In the on-shell scheme we have
    \begin{equation}
        \Pi(k^2) = \frac{e^2}{2\pi^2} \int_0^1 \dl{x} \, x(1 - x) \log\left[ 1 - \frac{x(1 - x)k^2 + i\epsilon}{m^2}. \right]
    \end{equation}
    In perturbation theory we expect that the one loop diagrams are small corrections to the tree level amplitude.
    Since \(e^2 = 4\pi \alpha \approx 4\pi/137 \approx 0.09\) is small many one loop corrections are small.
    However this isn't enough on its own.
    If \(k^2\) is large enough then the logarithm can become large.
    This is the problem of \defineindex{large logs}.
    
    This is a very real worry, for example, the LHC operates at \(\sqrt{s} = \qty{13}{\tera\electronvolt}\), meaning that \(k^2 \approx s = (\qty{13}{\tera\electronvolt})^2\), so an electron created in this process will have \(k^2/m^2 \sim 10^{14}\), so \(\log(k^2/m^2) \sim 34\), which is large enough to require higher order terms to achieve the desired accuracy.
    We also run into this issue if we try to take the \(m \to 0\) limit, which can be a useful thing to do.
    
    This problem of large logs is worse in QCD where \(\strongCoupling \approx 1/10\), so the coupling doesn't provide as much suppression of higher order terms.
    
    The way we get around this is to instead work in the \MSbar{} scheme, where this can still be an issue.
    We know that \(\tilde{\mu}\) is non-physical, and so can't appear in an amplitude calculated to all orders, however \(\tilde{\mu}\) does appear in truncated calculations.
    By carefully choosing the value of \(\tilde{\mu}\) we can avoid large logs.
    This leads to the somewhat unsettling observation that while \(\tilde{\mu}\) is non-physical it does affect the rate of \enquote{convergence} of our perturbative series, which certainly feels like it is a physical result.
    
    \chapter{All Orders Properties of QED}
    \section{Ward Identities}
    We've seen that \(i\Pi^{\mu\nu}(k) \propto (k^\mu k^\nu - \minkowskiMetric^{\mu\nu}k^2)\), and this means that \(k_\mu \Pi^{\mu\nu}(k) = 0\) at one loop.
    This is reminiscent of the Ward identities discussed in \course{Quantum Field Theory}.
    For processes with an external photon the amplitude contains a factor of \(\varepsilon^\mu(k)\), so we can define \(\amplitude_\mu\) to be such that \(\amplitude = \varepsilon^\mu(k)\amplitude_\mu\).
    It can then be shown that \(k^\mu\amplitude_\mu = 0\), this is called the \defineindex{Ward identity}, and is required for the gauge transformation of \(\varepsilon^\mu(k)\) to leave the physics invariant, since this transformation takes the form \(\varepsilon^\mu(k) \mapsto \varepsilon^\mu(k) + \alpha p^\mu\) under the gauge transformation \(\psi \mapsto \e^{i\alpha}\psi\).
    We will generalise this identity to correlators in QED and any loop level.
    
    
    \section{Rewriting the Action}
    The photon action, that is the terms in the action involving only the four-potential, is given in terms of the bare fields by
    \begin{equation}
        S_{\Pphoton} = \int \dl{^Dx} \left[ -\frac{1}{4} F_{\bare}^{\mu\nu} F_{\bare \mu\nu} - \frac{1}{2}\xi(\partial_\mu A_{\bare}^\mu)^2 \right].
    \end{equation}
    If we work in the Feynman gauge, which we will for the rest of this chapter, then \(\xi = 1\) and we have
    \begin{align}
        S_{\Pphoton} &= \int \dl{^Dx} \left[ -\frac{1}{4}(\partial^\mu A_{\bare}^\nu - \partial^\nu A_{\bare}^\mu)(\partial_\mu A_{\bare \nu} - \partial_\nu A_{\bare \mu}) - \frac{1}{2}(\partial_\mu A_{\bare}^\mu)^2 \right]\\
        &= \int \dl{^Dx} \bigg[ -\frac{1}{4}\{(\partial^\mu A_{\bare}^\nu)(\partial_\mu A_{\bare \nu}) - (\partial^\mu A_{\bare}^\nu)(\partial_\nu A_{\bare \mu}) \notag\\
        &\qquad\qquad - (\partial^\nu A_{\bare}^\mu)(\partial_\nu A_{\bare \mu}) + (\partial^\nu A_{\bare}^\mu)(\partial_\nu A_{\bare \mu})\} - \frac{1}{2}(\partial_\mu A_{\bare}^\mu)^2 \bigg]\\
        &= \int \dl{^Dx} \left[ -\frac{1}{2}\{(\partial^\mu A_{\bare}^\nu)(\partial_\mu A_{\bare\nu}) - (\partial^\mu A_{\bare}^\nu)(\partial_\nu A_{\bare \mu})\} - \frac{1}{2}(\partial_\mu A_{\bare}^\mu)^2 \right]\\
        &= \int \dl{^Dx} \left[ -\frac{1}{2}(\partial^\mu A_{\bare}^\nu)\{\partial_\mu A_{\bare\nu} - \partial_\nu A_{\bare \mu}\} - \frac{1}{2} (\partial_\mu A_{\bare}^\mu)^2 \right],
    \end{align}
    note that we could have gotten to this result using
    \begin{equation}
        F_{\bare}^{\mu\nu}F_{\bare \mu\nu} = (\partial^\mu A_{\bare}^\nu)(\partial_\mu A_{\bare \nu} - \partial_\nu A_{\bare \mu}),
    \end{equation}
    which follows using the antisymmetry of \(F_{\bare \mu\nu}\) to write \(F_{\bare}^{\mu\nu}\) as \(\partial^\mu A_{\bare}^\nu\), which has \(F_{\bare}^{\mu\nu}\) as its antisymmetric part.
    Now we have
    \begin{equation}
        \partial_\mu A_{\bare}^\mu = \partial^\mu A_{\bare}^\nu \minkowskiMetric_{\mu\nu},
    \end{equation}
    We can then write the action as
    \begin{equation}
        S_{\Pphoton} = \int \dl{^Dx} \left[ -\frac{1}{2}(\partial^\mu A_{\bare}^\nu)(\partial_\mu A_{\bare \nu} - \partial_\nu A_{\bare \mu} + \minkowskiMetric_{\mu\nu} \partial_\rho A_{\bare}^\rho) \right].
    \end{equation}
    We can now integrate by parts:
    \begin{align}
        S_{\Pphoton} = {} &-\frac{1}{2}\int \dl{^Dx} \, \partial^\mu \{A_{\bare}^\nu(\partial_\mu A_{\bare \nu} - \partial_\nu A_{\bare \mu} + \minkowskiMetric_{\mu\nu} \partial_\rho A_{\bare}^\rho)\} \notag\\
        &+ \frac{1}{2} \int \dl{^Dx} [A_{\bare}^\nu \partial^\mu (\partial_\mu A_{\bare \nu} - \partial_\nu A_{\bare \mu} + \minkowskiMetric_{\mu\nu} \partial_\rho A_{\bare}^\rho)].
    \end{align}
    As usual we use the divergence theorem to turn the first term into a surface integral, which we assume vanishes, leaving us with
    \begin{align}
        S_{\Pphoton} &= \frac{1}{2} \int \dl{^Dx} [A_{\bare}^\nu \partial^\mu (\partial_\mu A_{\bare \nu} - \partial_\nu A_{\bare \mu} + \minkowskiMetric_{\mu\nu} \partial_\rho A_{\bare}^\rho)]\\
        &= \frac{1}{2} \int \dl{^Dx} [A_{\bare}^\nu (\dalembertian A_{\bare \nu} - \partial^\mu \partial_\nu A_{\bare \mu} + \partial_\nu \partial_\rho A_{\bare}^\rho)].
    \end{align}
    After renaming \(\rho \to \mu\) and swapping raised and lowered indices in the last term the second and third term cancel, leaving us with
    \begin{equation}
        S_{\Pphoton} = \frac{1}{2}\int \dl{^Dx} \, A_{\bare}^\nu \dalembertian A_{\bare \nu}.
    \end{equation}
    One way we can tell that this is correct is that in the Feynman gauge the photon propagator goes as \(1/k^2\), and \(1/k^2\) is the inverse of \(\dalembertian\) in Fourier space, and the propagator should always be the inverse of the kernel in Fourier space.
    
    After regularisation this term becomes
    \begin{equation}
        S_{\Pphoton} = \frac{1}{2} \int \dl{^D x} \, Z_3 A_\mu \dalembertian A^\mu.
    \end{equation}
    The full action after regularising can then be written, in the Feynman gauge, as
    \begin{equation}
        S[A, \psi, \diracadjoint{\psi}] = \int \dl{^Dx} \left[ \frac{1}{2}Z_3A_\mu \dalembertian A^\mu + Z_2 \diracadjoint{\psi}(i \slashed{\partial} - m - \delta m)\psi + Z_1 e\diracadjoint{\psi}\slashed{A}\psi \right].
    \end{equation}
    
    We obtain the classical equations of motion by varying this action with respect to \(A_\mu\).
    The change in the action when \(A_\mu \mapsto A_\mu + \delta A_\mu\) is
    \begin{align}
        \delta S &= S[A + \delta A, \psi, \diracadjoint{\psi}] - S[A, \psi, \diracadjoint{\psi}]\\
         &= \int \dl{^D x} \bigg[ \frac{1}{2}Z_3 (A_\mu + \delta A_\mu)\dalembertian(A^\mu + \delta A^\mu) + Z_2(\diracadjoint{\psi} - m - \delta m)\psi \notag\\
         &\qquad\qquad + Z_1e\diracadjoint{\psi}\gamma^\mu(A_\mu + \delta A_\mu)\psi \bigg] - S[A, \psi, \diracadjoint{\psi}]\\
         &= \int \dl{^Dx} \left[ \frac{1}{2}Z_3 \delta A_\mu \dalembertian A^\mu + \frac{1}{2}Z_3 A_\mu \dalembertian \delta A^\mu + Z_1 e\diracadjoint{\psi}\gamma^\mu \delta A_\mu \psi \right].
    \end{align}
    Integrating the second term by parts twice, and again assuming that surface terms vanish, we move both derivatives from the \(\delta A^\mu\) term to the \(A_\mu\) term, and pick up two minus signs, leaving us with exactly the first term, giving
    \begin{equation}\label{eqn:change in action with A varying}
        \delta S = \int \dl{^Dx}\left[ Z_3 \dalembertian A^\mu + Z_1e\diracadjoint{\psi}\gamma^\mu \psi \right] \delta A_\mu.
    \end{equation}
    This must then vanish for all \(\delta A_\mu\) so we have
    \begin{equation}
        Z_3 \dalembertian A^\mu + Z_1 e\diracadjoint{\psi} \gamma^\mu \psi = 0.
    \end{equation}
    Defining the current to be
    \begin{equation}
        j_\mu(x) \coloneqq \diracadjoint{\psi}(x) \gamma_\mu \psi(x)
    \end{equation}
    the equations of motion become
    \begin{equation}
        Z_3 \dalembertian A^\mu = -eZ_1 j^\mu,
    \end{equation}
    which is just Maxwell's equations in terms of the renormalised fields, and factoring the charge out of the current.
    
    In the rest of this section we'll use the action in the form derived here to derive some identities which apply to all orders in QED.
    
    \section{Ward--Takahashi Identity}
    The QED Lagrangian has a symmetry given by
    \begin{equation}
        \psi \mapsto \e^{i\alpha}\psi, \qqand \diracadjoint{\psi} \mapsto \e^{-i\alpha}\diracadjoint{\psi}.
    \end{equation}
    Lets explore what happens if we promote \(\alpha\) to be a function of spacetime without changing the photon, \(A_\mu\).
    This ceases to be a symmetry of the Lagrangian, we need the extra term from transforming \(A_\mu\) to cancel the derivative of \(\alpha\) which appears.
    The change in the action comes entirely from the term with derivatives of \(\psi\), and is
    \begin{align}
        \delta S &= \int \dl{^Dx} \left[ Z_2 \e^{-i\alpha}\diracadjoint{\psi}i\slashed{\partial}(\e^{i\alpha}\psi) - Z_2 \diracadjoint{\psi}i\slashed{\partial}\psi \right]\\
        &= \int \dl{^Dx} \left[ Z_2 \e^{-i\alpha}\e^{i\alpha}\diracadjoint{\psi}(i\slashed{\partial}\psi - (\slashed{\partial}\alpha)\psi) - Z_2\diracadjoint{\psi}i\slashed{\partial}\psi \right]\\
        &= \int \dl{^Dx}\left[ Z_2 \diracadjoint{\psi}(-\slashed{\partial}\alpha) \psi \right].
    \end{align}
    Continuing on we integrate by parts, and assume the surface term vanishes
    \begin{align}
        \delta S &= -\int \dl{^Dx} \left[ Z_2 \diracadjoint{\psi}(\gamma^\mu \partial_\mu \alpha)\psi \right]\\
        &= -\int \dl{^Dx} \left[ Z_2 \partial_\mu (\diracadjoint{\psi}\gamma^\mu \alpha \psi) - Z_2 \alpha \partial_\mu (\diracadjoint{\psi} \gamma^\mu \psi) \right]\\
        &= \int \dl{^Dx} \left[ Z_2 \alpha \partial_\mu (\diracadjoint{\psi} \gamma^\mu \psi) \right]\\
        &= \int \dl{^Dx} \left[ Z_2 \alpha \partial^\mu j_\mu \right].
    \end{align}
    We are free to choose \(\alpha\), and choosing it to be a nonzero constant implies that \(\partial^\mu j_\mu = 0\), since despite not being a symmetry we clearly have \(\delta S = 0\) as \(\slashed{\partial} \alpha = 0\) for constant \(\alpha\) so before integrating by parts we had zero.
    So \(j_\mu\) is a conserved current.
    
    Now consider the correlator \(\correlator{\psi(x_1) \diracadjoint{\psi}(x_2)}\).
    This can be calculated using the path integral
    \begin{equation}
        \correlator{\psi(x_1) \diracadjoint{\psi}(x_2)} = \int \DL{A} \DD{\psi} \DD{\diracadjoint{\psi}} \, \psi(x_1) \diracadjoint{\psi}(x_2) \e^{iS[A, \psi, \diracadjoint{\psi}]}.
    \end{equation}
    The fields \(\psi\) and \(\diracadjoint{\psi}\) appear as integration variables, so we can consider a change of variables
    \begin{equation}
        \psi(x) \mapsto \psi'(x) = (1 + i\alpha(x))\psi(x), \qand \diracadjoint{\psi}(x) \mapsto \diracadjoint{\psi}'(x) = (1 - i\alpha(x))\diracadjoint{\psi}(x),
    \end{equation}
    which is just the linearised version of the transformation \(\psi \mapsto \e^{i\alpha}\psi\).
    It is reasonable to assume, but tricky to prove since path integrals aren't very well defined, that the integration measure is invariant under this, after all, we would expect that the measure transforms as
    \begin{equation}
        \DL{\psi}\DD{\diracadjoint{\psi}} \mapsto \e^{i\alpha}\DD{\psi} \, \e^{-i\alpha}\DD{\diracadjoint{\psi}} = \DL{\psi} \DD{\psi}.
    \end{equation}
    We will assume that this is the case.
    Then we have
    \begin{align}
        \correlator{\psi(x_1) \diracadjoint{\psi}(x_2)} &= \int \DL{A} \DD{\psi} \DD{\diracadjoint{\psi}} \, \psi(x_1) \diracadjoint{\psi}(x_2) \e^{iS[A, \psi, \diracadjoint{\psi}]}\\
        &= \int \DL{A} \DD{\psi'} \DD{\diracadjoint{\psi}'} \, \psi'(x_1) \diracadjoint{\psi'}(x_2) \e^{iS[A, \psi', \diracadjoint{\psi'}]}\\
        &= \int \DL{A} \DD{\psi} \DD{\diracadjoint{\psi}} \, \psi'(x_1) \diracadjoint{\psi'}(x_2) \e^{iS[A, \psi', \diracadjoint{\psi'}]}.
    \end{align}
    Now consider what we get if we subtract the last line here from the first.
    On the one hand we clearly get zero, since both lines are equal, on the other hand we can expand the third line as
    \begin{equation}
        \int \DL{A} \DD{\psi} \DD{\diracadjoint{\psi}} \, (1 + i\alpha(x_1))\psi(x_1) (1 - i\alpha(x_2))\diracadjoint{\psi}(x_2) \e^{iS[A, \psi', \diracadjoint{\psi}]}.
    \end{equation}
    Notice that since we're working to linear order we have
    \begin{equation}
        \e^{iS[A, \psi', \diracadjoint{\psi}']} = \e^{iS[A, \psi, \diracadjoint{\psi}] + i\delta S} \approx \left( 1 + i\int \dl{^Dx} \alpha(x) Z_2 \partial_\mu j^\mu(x) \right) \e^{iS[A, \psi, \diracadjoint{\psi}]}.
    \end{equation}
    Computing the difference then gives us
    \begin{multline}
        0 = \int \DL{A} \DD{\psi} \DD{\diracadjoint{\psi}} \bigg[ (i\alpha(x_1) - i\alpha(x_2))\psi(x_1)\diracadjoint{\psi}(x_2)\\
        + \psi(x_1)\diracadjoint{\psi}(x_2) i \int \dl{^Dx} \, \alpha(x) Z_2 \partial_\mu j^\mu(x)  \bigg] \e^{iS[A, \psi, \diracadjoint{\psi}]}.
    \end{multline}
    We can now drop the overall factor of \(i\), use some Dirac deltas to write \(\alpha(x_1)\) as \(\int \dl{^Dx} \, \alpha(x) \delta(x - x_1)\).
    This gives us
    \begin{align}
        0 &= \int \DL{A} \DD{\psi} \DD{\diracadjoint{\psi}} \int \dl{^Dx} [ (\alpha(x)\delta(x - x_1) - \alpha(x)\delta(x - x_2))\psi(x_1)\diracadjoint{\psi}(x_2) \notag\\
        &\qquad\qquad+ \psi(x_1)\diracadjoint{\psi}(x_2) \alpha(x) Z_2 \partial_\mu j^\mu(x) ]\e^{iS[A, \psi, \diracadjoint{\psi}]}\\
        &= \int \dl{^Dx} \, \alpha(x) [(\delta(x - x_1) - \delta(x - x_2)) \correlator{\psi(x_1) \diracadjoint{\psi}(x_2)} \notag\\
        &\qquad\qquad + Z_2 \partial_x^\mu \correlator{j_\mu(x)\psi(x_1)\diracadjoint{\psi}(x_2)}].
    \end{align}
    Note that the order of \(\psi\), \(\diracadjoint{\psi}\), and \(j^\mu\) in the path integral only matters up to the sign.
    Within a correlator the order doesn't matter at all, since we are time ordering.
    However, we have to be careful with the derivative, since the time ordering introduces factors of \(\theta(t - t')\), which the derivative acts on, so it's best to keep \(\partial_x^\mu\) out of the correlator.
    We could expand the correlator in terms of the time orderings and acting on the Heaviside step functions with the derivative would give Dirac deltas, and would just lead to us rederiving this result.
    
    Since this result must hold for all \(\alpha\) we can assume \(\alpha \ne 0\) and so we have
    \begin{equation}
        Z_2 \partial_x^\mu \correlator{j_\mu(x) \psi(x_1) \diracadjoint{\psi}(x_2)} = (\delta(x - x_2) - \delta(x - x_1))\correlator{\psi(x_1)\diracadjoint{\psi}(x_2)}.
    \end{equation}
    This is the \defineindex{Ward--Takahashi identity}.
    
    Classically one would expect that terms of the form \(\partial^\mu j_\mu\) vanish, and this identity tells us that this is almost the case, unless \(x\) and \(x_1\) or \(x_2\) happen to coincide, for this reason the term on the right is called a \defineindex{contact term}.
    
    We can generalise this result to include more fermions and gauge fields.
    The gauge fields are just bystanders in this, and don't change anything.
    Each pair of fermion fields picks up a Dirac delta term in exactly the same way, so the \defineindex{generalised Ward--Takahashi identity} is
    \begin{multline}
        Z_2 \partial_x^\mu \correlator*{j_\mu(x) \prod_{i = 1}^n \psi(y_i) \diracadjoint{\psi}(z_i) \prod_{j=1}^{n}A_{\mu_j}(w_j)}\\
        = \sum_{k = 1}^{n} (\delta(x - z_k) - \delta(x - y_k)) \correlator*{\prod_{i=1}^{n} \psi(y_i) \diracadjoint{\psi}(z_i) \prod_{j=1}^n A_{\mu_j}(w_j)}.
    \end{multline}
    
    \section{Schwinger--Dyson Equations}
    The Ward--Takahashi identity arose from classical symmetries.
    The Schwinger--Dyson equations arise similarly from the classical equations of motion.
    Consider the correlator
    \begin{equation}
        \correlator{A_\nu(x_1)} = \int \DL{A} \DD{\psi} \DD{\diracadjoint{\psi}} \, A_\nu(x_1) \e^{iS[A, \psi, \diracadjoint{\psi}]}.
    \end{equation}
    Consider the change of variables \(A_\mu(x) \mapsto A'_\mu(x) = A_\mu(x) + \zeta_\mu(x)\) where \(\zeta_\mu\) is some arbitrary vector field, not a gauge field.
    We think of \(\zeta_\mu\) as a variation on \(A_\mu\), hence the relation to the classical equations of motion.
    
    The change in the action from this transformation comes from the two terms in the action in which \(A_\mu\) appears, and using the result found for the classical equations of motion (\cref{eqn:change in action with A varying}) we have
    \begin{equation}
        \delta S = \int \dl{^Dx} \left[ Z_3 \dalembertian A^\mu(x) + Z_1 ej^\mu(x) \right]\zeta_\mu.
    \end{equation}
    
    We assume that the integration measure is invariant with respect to this change of variables, so \(\DL{A} = \DL{A'}\), or \(\DL{\zeta} = 0\), which seems reasonable.
    Then we have
    \begin{align}
        \correlator{A_\nu(x_1)} &= \int \DL{A} \DD{\psi} \DD{\diracadjoint{\psi}} A_\nu(x_1) \e^{iS[A, \psi, \diracadjoint{\psi}]}\\
        &= \int \DL{A'} \DD{\psi} \DD{\diracadjoint{\psi}} A'_\nu(x_1) \e^{iS[A', \psi, \diracadjoint{\psi}]}\\
        &= \int \DL{A} \DD{\psi} \DD{\diracadjoint{\psi}} A'_\nu(x_1) \e^{iS[A', \psi, \diracadjoint{\psi}]}.
    \end{align}
    We now consider the first line minus the third line.
    On the one hand clearly this gives zero, on the other hand we can expand the third line as
    \begin{equation}
        \int \DL{A} \DD{\psi} \DD{\diracadjoint{\psi}} (A_\nu(x_1) + \zeta_\nu(x_1))\e^{iS[A, \psi, \diracadjoint{\psi}] + i\delta S}
    \end{equation}
    which gives
    \begin{equation}
        \int \DL{A} \DD{\psi} \DD{\diracadjoint{\psi}} (A_\nu(x_1) + \zeta_\nu(x_1))(1 + i\delta S)\e^{iS[A, \psi, \diracadjoint{\psi}]}.
    \end{equation}
    Expanding the brackets we get \(A_\nu(x_1) + \zeta_\nu(x_1) + A_\nu(x_1) i\delta S + \zeta_\nu(x_1) i\delta S\).
    The \(A_\nu(x_1)\) term cancels with the first line and we're left with
    \begin{align}
        0 &= \int \DL{A} \DD{\psi} \DD{\diracadjoint{\psi}} (\zeta_\nu(x_1) + A_\nu(x_1) i\delta S + \zeta_\nu(x_1) i\delta S) \e^{iS[A, \psi, \diracadjoint{\psi}]}\\
        &= \int \DL{A} \DD{\psi} \DD{\diracadjoint{\psi}} \bigg[\zeta_\nu(x_1) \notag\\
        &\qquad + i \int\dl{^Dx} \{Z_3 \dalembertian A^\mu(x) + Z_1 ej^\mu(x)\}\zeta_\mu(x)(A_\nu(x_1) + \zeta_\nu(x_1))\bigg] \e^{iS[A, \psi, \diracadjoint{\psi}]}.
    \end{align}
    Choosing \(\zeta_\mu\) to be small and neglecting the quadratic term in \(\zeta_\mu\) we get
    \begin{multline}
        0 = \int \DL{A} \DD{\psi} \DD{\diracadjoint{\psi}} \bigg[\zeta_\nu(x_1) \notag\\
        + i \int\dl{^Dx} \{Z_3 \dalembertian A^\mu(x) + Z_1 ej^\mu(x)\}\zeta_\mu(x)A_\nu(x_1)\bigg] \e^{iS[A, \psi, \diracadjoint{\psi}]}.
    \end{multline}
    We can then write \(\zeta_\nu(x_1) = \int \dl{^Dx} \, \delta(x - x_1)\zeta_\nu(x)\) and \(\zeta_\nu(x) = \minkowskiMetric_{\mu\nu}\zeta^\mu(x)\) to give
    \begin{multline}
        0 = \int \dl{^Dx} \int \DL{A} \DD{\psi} \DD{\diracadjoint{\psi}} \, [ \delta(x - x_1) \minkowskiMetric_{\mu\nu}\\
        + i\{Z_3 \dalembertian A_\mu(x) + Z_1 ej_\mu(x)\}A_\nu(x_1) ]\zeta^\mu(x) \e^{iS[A, \psi, \diracadjoint{\psi}]}.
    \end{multline}
    Finally we can rewrite this in terms of correlators:
    \begin{equation}
        0 = \int \dl{^Dx} \, [\delta(x - x_1) \minkowskiMetric_{\mu\nu} + i\{Z_3\dalembertian \correlator{A_\mu(x)A_\nu(x_1)} + eZ_1\correlator{j_\mu(x)A_\nu(x_1)}\}]\zeta^\mu(x).
    \end{equation}
    Since we can choose \(\zeta_\mu\) arbitrarily we must have that
    \begin{equation}
        \delta(x - x_1)\minkowskiMetric_{\mu\nu} + i\{Z_3\dalembertian_x \correlator{A_\mu(x)A_\nu(x_1)} + eZ_1\correlator{j_\mu(x)A_\nu(x_1)}\} = 0,
    \end{equation}
    which tells us that
    \begin{equation}
        Z_3i\dalembertian_x\correlator{A_\mu(x)A_\nu(x_1)} = -eZ_1i\correlator{j_\mu(x)A_\nu(x_1)} - \minkowskiMetric_{\mu\nu}\delta(x - x_1).
    \end{equation}
    This is the \defineindex{Schwinger--Dyson equation}.
    Compare this to the classical equations of motion,
    \begin{equation}
        Z_3\dalembertian A^\mu = -eZ_1j^\mu,
    \end{equation}
    and we see that the Schwinger--Dyson equation is simply the classical equations of motion plus a contact term.
    Intuitively this contact term only comes into effect when \(x = x_1\), so \(A_\mu(x)\) and \(A_\nu(x_1)\) are evaluated at the same point.
    
    We can generalise this process to arbitrary correlators of fermion and gauge fields.
    The fermion fields have no effect, since they aren't modified by the change of variables, so we'll just consider some arbitrary product \(\psi \dotsm \diracadjoint{\psi} \dotsm\), in which every time we write this each fermion field is evaluated at some point which is the same each time we write this.
    Then we have the \defineindex{generalised Schwinger--Dyson equation}
    \begin{multline}
        Z_3i\dalembertian_x\correlator{A_\mu(x) A_{\nu_1}(x_1) \dotsb A_{\nu_n}(x_n) \psi \dotsm \diracadjoint{\psi} \dotsm}\\
        = -ieZ_1\correlator{j_\mu(x) A_{\nu_1}(x_1) \dotsm A_{\nu_n}(x_n) \psi \dotsm \diracadjoint{\psi} \dotsm}\\
        - \sum_{i = 1}^n \minkowskiMetric_{\mu\nu_i} \delta(x - x_i) \correlator{A_{\nu_1}(x_1) \dotsm \widehat{A_{\nu_i}(x_i)} \dotsm A_{\nu_n}(x_n)}
    \end{multline}
    where the notation \(\widehat{x}\) means that \(x\) is left out of the product, so, for example,
    \begin{equation}
        5 \cdot \widehat{4} \cdot 3 \cdot 2 \cdot 1 = 5 \cdot 3 \cdot 2 \cdot 1 = 30.
    \end{equation}
    
    \subsection{Schwinger--Dyson in Perturbation Theory}
    \subsubsection{Order \texorpdfstring{\(e^0\)}{e\textasciicircum{}0}}
    At \(\order(e^0)\) the Schwinger--Dyson equation gives
    \begin{equation}
        Z_3 i \dalembertian_x \correlator{A_\mu(x) A_\nu(y)} = -\minkowskiMetric_{\mu\nu}\delta(x - y).
    \end{equation}
    This is just the free propagator.
    Note that \(Z_3 = 1\) at order \(e^0\).
    
    For a photon, since \(m = 0\), we amputate a correlator by acting on it with \(i\dalembertian\), so we can interpret this as saying that if we amputate the photon propagator then we simply get a Dirac delta, which represents a point.
    So amputating a propagator, a line between points, gives a single point.
    \begin{equation}
        i\dalembertian_x (x, \mu \tikz[baseline=0.1cm]{\draw[photon] (0, 0) -- (1, 0);} y, \nu) = -\minkowskiMetric_{\mu\nu}\delta(x - y).
    \end{equation}
    
    \subsubsection{Order \texorpdfstring{\(e^1\)}{e\textasciicircum{}1}}
    At order \(e^1\) the Schwinger--Dyson equation gives
    \begin{equation}
        Z_3 i\dalembertian\correlator{A_\mu(x) \psi(x_1) \diracadjoint{\psi}(x_2)} = -e\correlator{j_\mu(x) \psi(x_1) \diracadjoint{\psi}(x_2)},
    \end{equation}
    we don't include the \(\minkowskiMetric_{\mu\nu}\delta(x)\) term since this is \(\order(e^0)\).
    Expanding \(j_\mu\) in terms of the fermion fields we get
    \begin{equation}
        \correlator{j_\mu(x)\psi(x_1)\diracadjoint{\psi}(x_2)} = \correlator{\diracadjoint{\psi}(x)\gamma_\mu\psi(x)\psi(x_1)\diracadjoint{\psi}(x_2)}.
    \end{equation}
    The only nonzero contraction, since we must contract each \(\psi\) with a \(\diracadjoint{\psi}\) and contracting fields at the same point gives zero, is
    \begin{equation}
        \wick{\c1{\diracadjoint{\psi}}(x)\gamma_\mu \c2{\psi}(x) \c1{\psi}(x_1) \c2{\diracadjoint{\psi}}(x_2)}.
    \end{equation}

    The correlator \(\correlator{A_\mu(x)\psi(x_1)\diracadjoint{\psi}(x_2)}\) corresponds to the QED vertex
    \begin{equation}
        \tikzsetnextfilename{all-orders-qed-qed-vertex}
        \begin{tikzpicture}[baseline=(current bounding box)]
            \draw[photon] (-1, 0) -- (0, 0);
            \draw[electron=0.6] (0, 0) -- (30:1);
            \draw[positron=0.4] (0, 0) -- (-30:1);
        \end{tikzpicture}
        .
    \end{equation}
    
    We are amputating the photon by acting with \(i\dalembertian\), which removes the photon, but still results in the injection of momentum from the photon.
    We draw this as
    \begin{equation}
        \tikzsetnextfilename{all-orders-qed-qed-vertex-amputated}
        \begin{tikzpicture}[baseline=(current bounding box)]
            \draw[momentum injection] (-1, 0) -- (0, 0);
            \draw[electron=0.6] (0, 0) -- (30:1);
            \draw[positron=0.4] (0, 0) -- (-30:1);
        \end{tikzpicture}
        .
    \end{equation}
    We can think of this as replacing the photon, \(A_\mu\), with a current, \(j_\mu\), of electrons and positrons, carrying momentum but no net charge.
    
    \subsubsection{General Order}
    From the zeroth and first order analysis we can now deduce what happens at arbitrary order.
    Consider some diagram
    \begin{equation}
        \tikzsetnextfilename{all-orders-qed-arbitrary-diagram}
        \begin{tikzpicture}[baseline=(current bounding box)]
            \draw[blob] (0, 0) circle [radius=0.5cm];
            \draw[photon] (-1.5, 0) -- (-0.5, 0);
            \draw[photon] (45:0.5) -- (45:1.5);
            \draw[electron] (20:0.5) -- (20:1.5);
            \draw[photon] (-20:0.5) -- (-20:1.5);
            \draw[electron] (-45:0.5) -- (-45:1.5);
        \end{tikzpicture}
        ,
    \end{equation}
    and focus on the incoming photon in particular, note that the other particles may be incoming or outgoing, and there may be any number of them.
    
    Acting on this with \(i\dalembertian\) amputates this photon, and there are two things that can happen, corresponding to the two terms on the right hand side of the Schwinger--Dyson equation.
    Either we get a propagator alone, which corresponds to the contact term, or we replace the photon with the amputated QED vertex.
    Diagrammatically this looks like
    \begin{equation*}
        \tikzsetnextfilename{all-orders-qed-arbitrary-order-schwinger-dyson}
        \begin{tikzpicture}[baseline=(current bounding box)]
            \draw[blob] (0, 0) circle [radius=0.5cm];
            \draw[photon] (-1.5, 0) node[left] {\(i\dalembertian\)} -- (-0.5, 0);
            \draw[photon] (45:0.5) -- (45:1.5);
            \draw[electron] (20:0.5) -- (20:1.5);
            \draw[photon] (-20:0.5) -- (-20:1.5);
            \draw[electron] (-45:0.5) -- (-45:1.5);
            \node at (2, 0) {\(=\)};
            \begin{scope}[xshift=3cm]
                \draw[blob] (0, 0) circle [radius=0.5cm];
                \draw[photon] (-0.5, -1.3) -- (1.3, -1.25);
                \draw[photon] (45:0.5) -- (45:1.5);
                \draw[electron] (20:0.5) -- (20:1.5);
                \draw[photon] (-20:0.5) -- (-20:1.5);
                \draw[electron] (-45:0.5) -- (-45:1.5);
            \end{scope}
            \node at (5, 0) {\(+\)};
            \begin{scope}[xshift=7.5cm]
                \draw[blob] (0, 0) circle [radius=0.5cm];
                \draw[momentum injection] (-2, 0)-- (-1.3, 0);
                \draw[electron=0.6] (-1.3, 0) -- ++ (22:1.05);
                \draw[positron=0.4] (-1.3, 0) -- ++ (-22:1.05);
                \draw[photon] (45:0.5) -- (45:1.5);
                \draw[electron] (20:0.5) -- (20:1.5);
                \draw[photon] (-20:0.5) -- (-20:1.5);
                \draw[electron] (-45:0.5) -- (-45:1.5);
            \end{scope}
        \end{tikzpicture}
        .
    \end{equation*}
    
    \chapter{Consequences}
    \section{Recap}
    In the last chapter we derived the Schwinger--Dyson equation:
    \begin{equation}
        Z_3 \dalembertian \correlator{A_\mu(x) \dotsm} = -eZ_1 \correlator{j_\mu(x) \dotsm} + \text{contact terms},
    \end{equation}
    and the Ward--Takahashi identity:
    \begin{equation}
        Z_2\partial_\mu \correlator{j^\mu(x)\psi(x_1)\diracadjoint{\psi}(x_2)} = (\delta(x - x_2) - \delta(x - x_1)) \correlator{\psi(x_1) \diracadjoint{\psi}(x_2)}.
    \end{equation}
    These results hold to all orders in QED.
    We will now derive and discuss several consequences of these results.
    
    \section{\texorpdfstring{\(k_\mu \Pi^{\mu\nu}(k) = 0\)}{k Pi = 0} to All Orders}
    We can generalise the definition of \(\Pi^{\mu\nu}(k)\) from just the vacuum polarisation to an all orders object
    \begin{equation}
        i\Pi^{\mu\nu}(x - y) = \correlator{A^\mu(x)A^\nu(y)}_{\text{amputated, connected}}.
    \end{equation}
    That is,
    \begin{equation}
        i\Pi^{\mu\nu}(x - y) = 
        \tikzsetnextfilename{consequences-pi-definition}
        \begin{tikzpicture}[baseline=(plus.base)]
            \draw[photon] (0, 0) -- (1, 0);
            \draw[photon] (2, 0) -- (3, 0);
            \draw[electron=0.55] (1, 0) arc (180:0:0.5);
            \draw[positron=0.45] (1, 0) arc (180:360:0.5);
            \node (plus) at (3.5, 0) {\(+\)};
            \draw[photon] (4, 0) -- (5, 0);
            \draw[photon] (6, 0) -- (7, 0);
            \draw[electron=0.55] (5, 0) arc (180:90:0.5);
            \draw[electron=0.55] (5.5, 0.5) arc (90:0:0.5);
            \draw[electron=0.55] (6, 0) arc (0:-90:0.5);
            \draw[electron=0.55] (5.5, -0.5) arc (270:180:0.5);
            \draw[photon] (5.5, 0.5) -- (5.5, -0.5);
            \node at (7.5, 0) {\(+ \dotsb\)};
        \end{tikzpicture}
        ,
    \end{equation}
    whereas before \(i\Pi^{\mu\nu}(x - y)\) was just the first term in this expansion.
    Here the diagrams stand for the amputated correlators, so really we are acting on each photon with \(i\dalembertian\).
    
    Since these diagrams are amputated it follows from the Schwinger--Dyson equation, applied twice, once to each external photon, that
    \begin{equation}
        \Pi^{\mu\nu}(x - y) \propto \correlator{A^\mu(x)A^\nu(y)}_{\text{amputated, connected}} \propto \correlator{j^\mu(x)j^\nu(y)}.
    \end{equation}
    Then we have
    \begin{align}
        ik_\mu \Pi^{\mu\nu}(k) &= ik_\nu \int \dl{^Dx} \, \e^{-ik\cdot x}\Pi^{\mu\nu}(x)\\
        &= \int \dl{^Dx} \, \e^{-ik\cdot x} \partial_\mu \Pi^{\mu\nu}(x)\\
        &\propto \int \dl{^Dx} \, \e^{-ik\cdot x} \partial_\mu\correlator{j^\mu(x)j^\nu(0)}\\
        &= 0.
    \end{align}
    Here we've used the fact that \(\fourierTransform\{\partial_\mu\} = ik_\mu\).
    The last step follows using the Ward--Takahashi identity recognising that \(j^\nu(0) = \diracadjoint{\psi}(0)\gamma^\nu \psi(0)\) and so we get \(\delta(x) - \delta(x) = 0\), at least when we consider the Dirac deltas as distributions under an integral.
    So, we have
    \begin{equation}
        k_\mu \Pi^{\mu\nu}(k) = 0
    \end{equation}
    to all orders in QED.
    
    We can go further in the case where the only four-vector is \(k^\mu\), then it must be that \(\Pi^{\mu\nu}(k) = \Pi_1(k^2)\minkowskiMetric^{\mu\nu} + \Pi_2(k^2)k^\mu k^\nu\) for some scalar functions \(\Pi_1\) and \(\Pi_2\).
    Using \(k_\mu \Pi^{\mu\nu}(k) = 0\) we must then have \(\Pi_1(k^2)k_\nu + \Pi_2(k^2)k^2k_\nu = 0\), and for \(k_\nu \ne 0\) we must then have \(\Pi_1(k^2) = -\Pi_2(k^2)k^2\), so
    \begin{equation}
        \Pi^{\mu\nu}(k) = (k^\mu k^\nu - \minkowskiMetric^{\mu\nu}k^2)\Pi(k^2)
    \end{equation}
    with \(\Pi = \Pi_2\).
    
    It is possible to introduce another four-vector quantity through a choice of non-Lorentz invariant gauge fixing, and this four-vector can then appear in the tensor structure above, but we won't do this.
    
    \section{Ward Identities}
    Any non-connected amplitude can be factored into connected amplitudes, so we'll only consider connected amplitudes.
    Given some amplitude, \(\amplitude(k)\), with \(k\) the momentum of an external photon we can write the amplitude as \(\amplitude(k) = \varepsilon^\mu(k)\amplitude_\mu(k)\) where \(\varepsilon^\mu\) is the polarisation vector of the photon and this equation defines \(\amplitude_\mu\).
    Then the \defineindex{Ward identity} is
    \begin{equation}
        k^\mu \amplitude_\mu(k) = 0.
    \end{equation}
    This is important as it means we have only two polarisation states and gauge invariance, both basic requirements of QED.
    This Ward identity follows from the Ward--Takahashi identity since \(\amplitude_\mu\) is proportional to the Fourier transform of \(\correlator{j_\mu(x) \dotsm}\), and the Ward--Takahashi identity  tells us that \(\partial^\mu\correlator{j_\mu(x) \dotsb} = 0\) for connected, amputated correlators.
    Taking the Fourier transform then turns the derivative into a \(k^\mu\) proving the identity.
    This proof shows that the Ward identity holds to all orders in perturbation theory, not just the tree level at which we proved in in \course{Quantum Field Theory}.
    
    \section{Charge Renormalisation}\label{sec:charge renormalisation}
    Consider the correlator \(\correlator{A_\mu(x) \psi(x_1) \diracadjoint{\psi}(x_2)}\).
    This is in position space, but we can identify it with it's Fourier transform.
    Then we can expand this as
    \begin{equation}
        \correlator{A_\mu(x)\psi(x_1) \diracadjoint{\psi}(x_2)} = \mkern-30mu
        \tikzsetnextfilename{consequences-A-psi-psibar-to-all-orders}
        \begin{tikzpicture}[baseline=(plus.base)]
            \draw[photon] (-1, 0) -- (0, 0);
            \draw[electron=0.6] (0, 0) -- (30:1);
            \draw[positron=0.4] (0, 0) -- (-30:1);
            \node (plus) at (2, 0) {\(+\)};
            \begin{scope}[xshift=4cm]
                \draw[photon] (-1, 0) -- (-0.15, 0);
                \draw[electron=0.6] (30:0.15) -- (30:1);
                \draw[positron=0.4] (-30:0.15) -- (-30:1);
                \draw (0, 0) circle [radius = 0.15cm];
                \foreach \angle in {45, 135, 225, 315} {
                    \draw (0, 0) -- (\angle:0.15);
                }
                \node at (0, -0.35) {\(\delta_1\)};
            \end{scope}
            \node at (-1.5, -1.5) {\(+\)};
            \begin{scope}[yshift=-1.5cm]
                \draw[photon] (-1, 0) -- (0, 0);
                \draw[electron=0.6] (0, 0) -- (30:1);
                \draw[positron=0.4] (0, 0) -- (-30:1);
                \draw[photon] (30:0.75) -- (-30:0.75);
            \end{scope}
            \node at (2, -1.5) {\(+\)};
            \begin{scope}[yshift=-1.5cm, xshift=4cm]
                \draw[photon] (-1, 0) -- (0, 0);
                \foreach \angle in {45, 135, 225, 315} {
                    \draw[shift={(30:0.5)}] (0, 0) -- (\angle:0.15);
                }
                \draw (30:0.5) circle [radius=0.15cm];
                \node at ($(30:0.5) + (0, 0.35)$) {\(\delta_2\)};
                \draw[electron=0.8] (0, 0) -- (30:0.35);
                \draw[electron=0.8] (30:0.65) -- (30:1);
                \draw[positron=0.4] (0, 0) -- (-30:1);
            \end{scope}
            \node at (-1.5, -3) {\(+\)};
            \begin{scope}[yshift=-3cm, xshift=0.5cm]
                \draw[photon] (-1.5, 0) -- (-1.1, 0);
                \draw[photon] (-0.4, 0) -- (0, 0);
                \draw[electron=0.55] (-1.1, 0) arc (180:0:0.35);
                \draw[positron=0.45] (-1.1, 0) arc (180:360:0.35);
                \draw[electron=0.6] (0, 0) -- (30:1);
                \draw[positron=0.4] (0, 0) -- (-30:1);
                \draw[photon] (30:0.75) -- (-30:0.75);
            \end{scope}
            \node at (2, -3) {\(+\)};
            \begin{scope}[yshift=-3cm, xshift=4cm]
                \draw[photon] (-1.5, 0) -- (-0.9, 0);
                \draw[photon] (-0.6, 0) -- (0, 0);
                \foreach \angle in {45, 135, 225, 315} {
                    \draw[xshift=-0.75cm] (0, 0) -- (\angle:0.15);
                }
                \draw (-0.75, 0) circle [radius = 0.15cm];
                \draw[electron=0.6] (0, 0) -- (30:1);
                \draw[positron=0.4] (0, 0) -- (-30:1);
                \draw[photon] (30:0.75) -- (-30:0.75);
                \node at (-0.75, -0.35) {\(\delta_3\)};
            \end{scope}
            \node at (5.5, -3) {\(+ \dotsb\).};
        \end{tikzpicture}
    \end{equation}
    We can then write this more compactly as
    \begin{equation}
        \correlator{A_\mu(x) \psi(x_1) \diracadjoint{\psi}(x_2)} =
        \tikzsetnextfilename{consequences-vertex-as-1PI}
        \begin{tikzpicture}[baseline=(current bounding box)]
            \draw[blob] (0, 0) circle [radius=0.25cm];
            \draw[blob] (-1.5, 0) circle [radius=0.25cm];
            \draw[photon] (-1.25, 0) -- (-0.25, 0);
            \draw[photon] (-2.75, 0) -- (-1.75, 0);
            \draw[electron=0.6] (30:0.25) -- (30:1);
            \draw[positron=0.4] (-30:0.25) -- (-30:1);
            \draw[blob] (30:1.25) circle [radius=0.25cm];
            \draw[blob] (-30:1.25) circle [radius=0.25cm];
            \draw[electron=0.6] (30:1.5) -- (30:2.25);
            \draw[positron=0.4] (-30:1.5) -- (-30:2.25);
        \end{tikzpicture}
    \end{equation}
    Here each circle represents a sum over all one particle irreducible (1PI) diagrams with the matching external states.
    
    Now we can apply the Schwinger--Dyson equation:
    \begin{equation}
        Z_3 \dalembertian \correlator{A_\mu(x) \psi(x_1) \diracadjoint{\psi}(x_2)} = -eZ_1\correlator{j_\mu(x) \psi(x_1) \diracadjoint{\psi}(x_2)}.
    \end{equation}
    Acting with \(\dalembertian\) amputates the incoming photon, reducing this series to
    \begin{equation}
        i\dalembertian\langle{A_\mu(x)\psi(x_1) \overline{\psi}(x_2)} = \mkern-30mu
        \tikzsetnextfilename{consequences-A-psi-psibar-to-all-orders-amputated}
        \begin{tikzpicture}[baseline=(plus.base)]
            \draw[momentum injection] (-1, 0) -- (0, 0);
            \draw[electron=0.6] (0, 0) -- (30:1);
            \draw[positron=0.4] (0, 0) -- (-30:1);
            \node (plus) at (2, 0) {\(+\)};
            \begin{scope}[xshift=4cm]
                \draw[momentum injection] (-1, 0) -- (-0.15, 0);
                \draw[electron=0.6] (30:0.15) -- (30:1);
                \draw[positron=0.4] (-30:0.15) -- (-30:1);
                \draw (0, 0) circle [radius = 0.15cm];
                \foreach \angle in {45, 135, 225, 315} {
                    \draw (0, 0) -- (\angle:0.15);
                }
                \node at (0, -0.35) {\(\delta_1\)};
            \end{scope}
            \node at (-1.5, -1.5) {\(+\)};
            \begin{scope}[yshift=-1.5cm]
                \draw[momentum injection] (-1, 0) -- (0, 0);
                \draw[electron=0.6] (0, 0) -- (30:1);
                \draw[positron=0.4] (0, 0) -- (-30:1);
                \draw[photon] (30:0.75) -- (-30:0.75);
            \end{scope}
            \node at (2, -1.5) {\(+\)};
            \begin{scope}[yshift=-1.5cm, xshift=4cm]
                \draw[momentum injection] (-1, 0) -- (0, 0);
                \foreach \angle in {45, 135, 225, 315} {
                    \draw[shift={(30:0.5)}] (0, 0) -- (\angle:0.15);
                }
                \draw (30:0.5) circle [radius=0.15cm];
                \node at ($(30:0.5) + (0, 0.35)$) {\(\delta_2\)};
                \draw[electron=0.8] (0, 0) -- (30:0.35);
                \draw[electron=0.8] (30:0.65) -- (30:1);
                \draw[positron=0.4] (0, 0) -- (-30:1);
            \end{scope}
            \node at (-1.5, -3) {\(+\)};
            \begin{scope}[yshift=-3cm, xshift=0.5cm]
                \draw[momentum injection] (-1.5, 0) -- (-1.1, 0);
                \draw[photon] (-0.4, 0) -- (0, 0);
                \draw[electron=0.55] (-1.1, 0) arc (180:0:0.35);
                \draw[positron=0.45] (-1.1, 0) arc (180:360:0.35);
                \draw[electron=0.6] (0, 0) -- (30:1);
                \draw[positron=0.4] (0, 0) -- (-30:1);
                \draw[photon] (30:0.75) -- (-30:0.75);
            \end{scope}
            \node at (2, -3) {\(+\)};
            \begin{scope}[yshift=-3cm, xshift=4cm]
                \draw[momentum injection] (-1.5, 0) -- (-0.9, 0);
                \draw[photon] (-0.6, 0) -- (0, 0);
                \foreach \angle in {45, 135, 225, 315} {
                    \draw[xshift=-0.75cm] (0, 0) -- (\angle:0.15);
                }
                \draw (-0.75, 0) circle [radius = 0.15cm];
                \draw[electron=0.6] (0, 0) -- (30:1);
                \draw[positron=0.4] (0, 0) -- (-30:1);
                \draw[photon] (30:0.75) -- (-30:0.75);
                \node at (-0.75, -0.35) {\(\delta_3\)};
            \end{scope}
            \node at (5.5, -3) {\(+ \dotsb\).};
        \end{tikzpicture}
    \end{equation}
    We can write the first two terms here as \(1 + \delta_1\) times the first term, since the only difference between the QED vertex and its counter term is a factor of \(\delta_1\).
    
    This correlator is infinite.
    We know that all photon-to-photon 1PI diagrams are exactly the diagrams contributing to
    \begin{equation}
        i\Pi^{\mu\nu}(x - y) = 
        \tikzsetnextfilename{consequences-Pi-as-1PI}
        \begin{tikzpicture}[baseline=-0.1cm]
            \draw[blob] (0, 0) circle [radius=0.25cm];
            \draw[photon] (-0.25, 0) -- (-1, 0);
            \draw[photon] (0.25, 0) -- (1, 0);
        \end{tikzpicture}
        .
    \end{equation}
    We also know that \(\partial_\mu \Pi^{\mu\nu}(x - y) = 0\) to all orders, so \(\partial^\mu\) kills all 1PI diagrams on a photon, that is
    \begin{equation}
        \partial^\mu \vcenter{\hbox{\includegraphics{tikz-external/consequences-Pi-as-1PI}}} = 0.
    \end{equation}
    Everything else appearing in the expansion has been renormalised, and so is finite.
    As a consequence we know that
    \begin{equation}
        \partial^\mu Z_3 \dalembertian \correlator{A_\mu(x) \psi(x_1) \diracadjoint{\psi}(x_3)} = \text{finite}.
    \end{equation}
    Using the Schwinger--Dyson equation we can rewrite this as
    \begin{equation}
        Z_1 \partial^\mu \correlator{j_\mu(x) \psi(x_1) \diracadjoint{\psi}(x_2)} = \text{finite}.
    \end{equation}
    This correlator corresponds to the diagram
    \begin{equation}
        \tikzsetnextfilename{consequences-alternative-form}
        \begin{tikzpicture}[baseline=-0.075cm]
            \draw[blob] (0, 0) circle [radius=0.25cm];
            \draw[photon] (-1.25, 0) -- (-0.25, 0);
            \draw[electron=0.6] (30:0.25) -- (30:1);
            \draw[positron=0.4] (-30:0.25) -- (-30:1);
            \draw[blob] (30:1.25) circle [radius=0.25cm];
            \draw[blob] (-30:1.25) circle [radius=0.25cm];
            \draw[electron=0.6] (30:1.5) -- (30:2.25);
            \draw[positron=0.4] (-30:1.5) -- (-30:2.25);
        \end{tikzpicture}
        ,
    \end{equation}
    which must then also be finite.
    Recalling that \(Z_1 = 1 + \delta_1\) is the counter term for the QED vertex it is not that surprising that this is finite, assuming that the two electron propagators are properly regulated.
    
    We can also apply the 
    \begin{equation}
        Z_2\partial^\mu\correlator{j_\mu(x) \psi(x_1) \diracadjoint{\psi}(x_2)} = \text{finite}
    \end{equation}
    by the Ward--Takahashi identity this becomes
    \begin{equation}
        \correlator{\psi(x_1) \diracadjoint{\psi}(x_2)} = \text{finite},
    \end{equation}
    which corresponds to a fermion propagator.
    This is surprising, by regulating the QED vertex with \(Z_2\) we've also regulated the fermion propagator.
    
    In the \MSbar{} scheme we have
    \begin{equation}
        Z_i = 1 + \sum_{n = 1}^{\infty} \frac{a_i}{\varepsilon^n},
    \end{equation}
    and so what we have shown here is that, in the \MSbar{} scheme \(Z_1 = Z_2\) to all orders in perturbation theory.
    More generally the divergent parts of \(Z_1\) and \(Z_2\) are equal to all orders in perturbation theory, but they may differ by finite terms.
    
    Recall that the Fermion Lagrangian involves the term
    \begin{equation}
        \diracadjoint{\psi}_{\bare}(i\slashed{\partial})\psi_{\bare} + e_{\bare}\diracadjoint{\psi}_{\bare}\slashed{A}_{\bare}\psi_{\bare} = Z_2\diracadjoint{\psi}(i\slashed{\partial})\psi + e\mu^\varepsilon \underbrace{Z_eZ_2\sqrt{Z_3}}_{Z_1} \diracadjoint{\psi}\slashed{A}\psi.
    \end{equation}
    This factorises nicely if \(Z_1 = Z_2\) allowing us to rewrite this term using the covariant derivative as
    \begin{equation}
        Z_1\diracadjoint{\psi}(i\slashed{\covariantDerivative})\psi
    \end{equation}
    for \(\covariantDerivative_\mu = \partial_\mu - ie\mu^\varepsilon A_\mu\).
    This is nice since it allows us to make gauge invariance more obvious.
    
    We previously found that the electric charge becomes scale dependent, with the result that
    \begin{equation}
        e_{\measured}^2(k^2) = e_{\measured}^2(0) + \frac{e^4}{2\pi^2} \int_0^1 \dl{x} \, x(1 - x) \log\left( \frac{x(1 - x)k^2}{\tilde{\mu}^2} \right)
    \end{equation}
    in the \MSbar{} scheme, neglecting the \(m^2\) term since we are interested in high energies.
    We found this by studying the vacuum polarisation of the electron, that is, one loop corrections to the photon propagator.
    This is a bit odd since it is the QED vertex which involves the charge, \(e\), and is used to renormalise the interaction, which is where the charge is important.
    The link between the vacuum polarisation and the QED vertex is that \(Z_1 = Z_2\), so the counter terms of these two diagrams are related and we can study either one to learn things about the other.
    
    
    \chapter{The QED \texorpdfstring{\(\beta\)}{beta} Function}
    \section{What is the QED \texorpdfstring{\(\beta\)}{beta} Function?}
    The bare charge, \(e_{\bare}\), is independent of \(\mu\), since we only introduce \(\mu\) in the regularisation process to hold the dimensions of \(e\) away from \(D = 4\).
    Thus we have
    \begin{equation}
        \mu \diff{e_{\bare}}{\mu} = 0.
    \end{equation}
    Note that it is common to work with the log-derivative,
    \begin{equation}
        \mu \diff{}{\mu} = \diff{}{(\log \mu)},
    \end{equation}
    in this context, rather than just \(\diff{}/{\mu}\).
    
    On the other hand after regularising we have \(e_{\bare} = e\mu^\varepsilon Z_3\), which clearly has \(\mu\) dependence.
    The solution is that \(eZ_3\) must also have \(\mu\) dependence, and this dependence must be such as to exactly cancel the dependence of the \(\mu^\varepsilon\) term.
    We also have
    \begin{equation}
        Z_1 = Z_e Z_2 \sqrt{Z_3},
    \end{equation}
    which combined with our recent derivation that \(Z_1 = Z_2\) to all orders in the \MSbar{} scheme gives
    \begin{equation}
        Z_e = \frac{1}{\sqrt{Z_3}}.
    \end{equation}
    Hence, we are looking to solve
    \begin{align}
        0 &= \mu \diff{}{\mu}(e\mu^\varepsilon Z_e)\\
        &= \mu \diff{}{\mu}\left( \frac{e\mu^\varepsilon}{\sqrt{Z_3}} \right)\\
        &= \mu \diff{e}{\mu} \mu^\varepsilon \frac{1}{\sqrt{Z_3}} + e\varepsilon \mu^\varepsilon \frac{1}{\sqrt{Z_3}} - \frac{1}{2} e\mu^\varepsilon \frac{1}{Z_3^{3/2}} \mu \diff{Z_3}{\mu}.
    \end{align}
    Cancelling a common factor of \(\mu^\varepsilon / \sqrt{Z_3}\) we get
    \begin{equation}
        0 = \mu \diff{e}{\mu} + e\varepsilon - \frac{1}{2} \frac{e}{Z_3} \mu \diff{Z_3}{\mu}.
    \end{equation}
    Recall that
    \begin{equation}
        Z_3 = 1 + \frac{e^2}{12\pi^2\varepsilon},
    \end{equation}
    which allows us to write this all in terms of \(\diff{e}/{\mu}\):
    \begin{equation}
        0 = \mu \diff{e}{\mu} + e\varepsilon - \frac{1}{2} \frac{e}{Z_3} \mu \frac{2e}{12\pi^2\varepsilon} \diff{e}{\mu}
    \end{equation}
    so we have
    \begin{align}
        0 &= e\varepsilon + \left( 1 - \frac{1}{Z_3}\frac{e^2}{12\pi^2\varepsilon} \right) \mu \diff{e}{\mu}\\
        &\approx e\varepsilon + \left( 1 - \frac{e^2}{12\pi^2\varepsilon} \right) \mu \diff{e}{\mu}
    \end{align}
    where we've used
    \begin{equation}
        \frac{1}{Z_3} \approx 1 - \frac{e^2}{12\pi^2\varepsilon}
    \end{equation}
    and dropped any terms of order \(e^3\) or higher.
    We then have
    \begin{align}
        \mu \diff{e}{\mu} &\approx -\frac{e\varepsilon}{1 - e^2/(12\pi^2\varepsilon)}\\
        &\approx -e\varepsilon\left( 1 + \frac{e^2}{12\pi^2\varepsilon} \right)\\
        &= -e\varepsilon + \frac{e^3}{12\pi^2}.
    \end{align}
    
    Finally, take the \(\varepsilon \to 0\) limit and define the \defineindex{beta function}\index{\(\beta\)}:
    \begin{equation}
        \beta(e) \coloneqq \mu \diff{e}{\mu} = \frac{e^3}{12\pi^2}.
    \end{equation}
    This is a \defineindex{renormalisation group equation}.
    
    \section{Solving the Renormalisation Group Equation}
    Set \(t = \log \mu\), then we have
    \begin{equation}
        \beta(e) = \diff{e}{t} = \frac{e^3}{12\pi^2}.
    \end{equation}
    Multiplying through by\footnote{sorry to any mathematicians reading, although I doubt any have made it this far since nothing converges and we've been computing divergent integrals for about 35 pages} \(\dl{t}\) to get
    \begin{equation}
        \frac{\dl{e}}{e^3} = \frac{\dl{t}}{12\pi^2}.
    \end{equation}
    Recognising that
    \begin{equation}
        \diff{(1/e^2)}{e} = -2\frac{1}{e^3}
    \end{equation}
    we have
    \begin{equation}
        \frac{\dl{e}}{e^3} = -\frac{1}{2} \dl{\left( \frac{1}{e^2} \right)}
    \end{equation}
    and so
    \begin{equation}
        \dl{\left( \frac{1}{e^2} \right)} = -\frac{\dl{t}}{6\pi^2}.
    \end{equation}
    Integrating from some \(t_0\) to \(t\) we have
    \begin{equation}
        \int_{t_0}^{t} \dl{(\frac{1}{e^2})} = -\frac{1}{6\pi^2} \int_{t_0}^{t} \dl{t}
    \end{equation}
    which gives
    \begin{equation}
        \frac{1}{e^2(t)} - \frac{1}{e^2(t_0)} = - \frac{1}{6\pi^2}(t - t_0).
    \end{equation}
    
    Usually we write this in terms of the fine structure constant, \(\alpha = e^2/(4\pi)\), in which case we have
    \begin{equation}
        \frac{1}{4\pi\alpha(t)} - \frac{1}{4\pi\alpha(t_0)} = -\frac{1}{6\pi^2}(t - t_0) \implies \frac{1}{\alpha(t)} - \frac{1}{\alpha(t_0)} = -\frac{2}{3\pi}(t - t_0).
    \end{equation}
    So, for fixed \(t_0\), we have found a linear relationship between \(1/\alpha(t)\) and \(t\), which is logarithmic in the energy.
    We can plot the result as in \cref{fig:fine structure constant plot}.
    From this we see that as \(t\), and hence the energy \(\mu\), grows \(1/\alpha(t)\) becomes smaller until it vanishes at some value \(t_*\).
    That is, \(\alpha(t)\) becomes larger and larger and eventually at \(t = t_*\) it takes on an infinite value.
    We call this a \defineindex{Landau pole}.
    
    \begin{figure}
        \tikzsetnextfilename{qed-beta-alpha-blows-up}
        \begin{tikzpicture}
            \draw[thick, ->] (0, -1) -- (0, 5) node[below left] {\(1/\alpha(t)\)};
            \draw[thick, ->] (-1, 0) -- (5, 0) node[below] {\(t\)};
            \draw[highlight, very thick] (-1, 4) -- (4, 0) node [below, black] {\(t_*\)};
        \end{tikzpicture}
        \caption{A plot of \(1/\alpha(t)\) as a function of \(t\). Notice that at some value \(t_*\) we have \(1/\alpha(t_*) = 0\), meaning \(\alpha(t_*)\) must be infinite.}
        \label{fig:fine structure constant plot}
    \end{figure}
    
    The modern understanding of this result is that QED is not well defined as a quantum theory to arbitrarily high energy.
    At some energy scale, \(\Lambda_{\symrm{QED}} = \e^{t_*}\) the theory breaks down and stops working.
    This isn't really too much of a worry for two reasons:
    \begin{itemize}
        \item the approximation that \(e(\mu)\) is small breaks far below this energy scale;
        \item \(\Lambda_{\symrm{QED}}\) is truly massive, using \(t_0 = \log m_{\PZboson}\), where measurements give \(\alpha(t_0) \approx 1/127\), we find that \(\Lambda_{\symrm{QED}} \sim 10^{270} \, \unit{\electronvolt}\).
    \end{itemize}
    
    We can rearrange this equation:
    \begin{align}
        \frac{1}{\alpha(t)} &= \frac{1}{\alpha(t_0)} - \frac{2}{3\pi}(t - t_0)\\
        \implies \alpha(t) &= \frac{1}{1/\alpha(t_0) - 2(t - t_0)/(3\pi)} = \frac{\alpha(t_0)}{2\alpha(t_0)(t - t_0)/(3\pi)}.
    \end{align}
    We can then expand this to get
    \begin{align}
        \alpha(t) &= \alpha(t_0) + \frac{2}{3\pi}\alpha^2(t_0)(t - t_0) + \frac{4}{9\pi}\alpha^3(t_0)(t - t_0)^2 + \dotsb\\
        &= \alpha(t_0) + \frac{2}{3\pi}\alpha^2(t_0) \log \frac{\mu}{\mu_0} + \frac{4}{9\pi} \alpha^3(t_0) \left( \log \frac{\mu}{\mu_0} \right)^2 + \dotsb\\
        &= 
        \tikzsetnextfilename{qed-beta-alpha-expansion}
        \begin{tikzpicture}[baseline=(plus.base)]
            \draw[photon] (-1.5, 0) -- (-0.5, 0);
            \draw[photon] (0.5, 0) -- (1.5, 0);
            \draw[electron=0.55] (-0.5, 0) arc (180:0:0.5);
            \draw[positron=0.45] (-0.5, 0) arc (180:360:0.5);
            \node (plus) at (1.8, 0) {\(+\)};
            \begin{scope}[xshift=3.6cm]
                \draw[photon] (-1.5, 0) -- (-0.5, 0);
                \draw[photon] (0.5, 0) -- (1.5, 0);
                \draw[photon] (2.5, 0) -- (3.5, 0);
                \draw[electron=0.55] (-0.5, 0) arc (180:0:0.5);
                \draw[positron=0.45] (-0.5, 0) arc (180:360:0.5);
                \draw[electron=0.55] (1.5, 0) arc (180:0:0.5);
                \draw[positron=0.45] (1.5, 0) arc (180:360:0.5);
            \end{scope}
            \node at (7.6, 0) {\(+ \dotsb.\)};
        \end{tikzpicture}
        \notag
    \end{align}
    Here we interpret this series as a series of diagrams, each order of \(\alpha\) giving an extra loop, which each come with two vertices, accounting for the two extra powers of \(e\).
    So we see that the divergence of the coupling constant can be understood as the result of resuming an infinite class of diagrams.
    
    There are other two loop corrections to this, such as
    \begin{equation}
        \tikzsetnextfilename{qed-beta-other-2-loop-corrections-to-alpha}
        \begin{tikzpicture}[baseline=-0.05cm]
            \draw[photon] (0, 0) -- (1, 0);
            \draw[photon] (2, 0) -- (3, 0);
            \draw[electron=0.55] (1, 0) arc (180:90:0.5);
            \draw[electron=0.55] (1.5, 0.5) arc (90:0:0.5);
            \draw[electron=0.55] (2, 0) arc (0:-90:0.5);
            \draw[electron=0.55] (1.5, -0.5) arc (270:180:0.5);
            \draw[photon] (1.5, 0.5) -- (1.5, -0.5);
        \end{tikzpicture}
        \sim \alpha^4(t_0) \log\left( \frac{\mu}{\mu_0} \right),
    \end{equation}
    but if \(\log(\mu/\mu_0)\) is large then the class of diagrams already accounted for is the most important.
    This is called resummation of large logs.
    
    \chapter{Magnetic Moment}
    \epigraph{This is a story where the 2s matter. I'm not going to get the 2s right.}{Donal O'Connell}
    \section{Goal}
    Consider a particle in an external magnetic field, \(\vv{B}\).
    The contribution to the Hamiltonian from this is
    \begin{equation}
        \delta H = -\vv{\mu} \cdot \vv{B},
    \end{equation}
    where \(\vv{\mu}\) is the particle's \defineindex{magnetic moment}.
    For a particle with spin \(\vv{S}\) and charge \(q\) the magnetic moment is
    \begin{equation}
        \vv{\mu} = \frac{q}{2m} g \vv{S}
    \end{equation}
    where \(g\), known as the \define{Land\'e \(\bm{g}\) factor}\index{Lande g factor@Land\'e \(g\) factor} is some value which we will compute in this chapter.
    
    For an electron, of charge \(-e\), or a muon, also of charge \(-e\) but a different mass, the magnetic moment can be written as
    \begin{equation}
        \vv{\mu} = - \frac{e}{2m}g \vv{S} = -\bohrMagneton g \vv{S}
    \end{equation}
    where
    \begin{equation}
        \bohrMagneton \coloneqq \frac{e}{2m}
    \end{equation}
    is the \defineindex{Bohr magneton}.
    
    A calculation of \(g\) not accounting for spin will give \(g = 1\).
    A calculation of \(g\) using the non-relativistic limit of the Dirac equation, which accounts for spin, will give \(g = 2\), we'll do this calculation in the next section.
    A calculation in QED predicts \(g\) as a power series, giving \(g = 2 + \order(\alpha)\).
    We will do a one loop QED calculation of \(g\).
    Calculations up to five loops have been performed which predict \(g\) for an electron to many significant figures, and align well with experimental results.
    
    \section{Non-Relativistic Calculation of the Land\'e \texorpdfstring{\(g\)}{g} Factor}
    In this section we will compute the magnetic moment of an electron using the non-relativistic limit of the Dirac equation.
    This is valid for energies small compared to the electron mass where there can be no pair production.
    Define the \defineindex{wave function} of the electron to be
    \begin{equation}
        \psi_{\symrm{WF}}(x) = \bra{0} \psi(x) \ket{\text{single particle state}}
    \end{equation}
    where \(\psi\) is the normal quantum fermion field.
    The motivation here is that a single particle state is of the form
    \begin{equation}
        \int \frac{\dhat{^3p}}{2E_p} \varphi(p) \ket{p}
    \end{equation}
    for some function \(\varphi\).
    The annihilation operator in \(\psi\) will then annihilate the single particle state, and then the creation operator acts as its conjugate on the vacuum state, so we only get the overlap between single particles, which is what we would expect in the non-relativistic limit.
    
    It follows from this definition that \(\psi_{\symrm{WF}}\) also satisfies the Dirac equation, since the only \(x\) dependence comes from the \(\psi\) field.
    That is,
    \begin{equation}
        (i\slashed{\covariantDerivative} - m)\psi_{\symrm{WF}} = 0.
    \end{equation}
    Applying \((-i\slashed{\covariantDerivative} + m)\) to this we must still get zero,
    \begin{equation}
        -(i\slashed{\covariantDerivative} + m)(i\slashed{\covariantDerivative} - m)\psi_{\symrm{WF}} = 0.
    \end{equation}
    Expanding this operator we get
    \begin{equation}
        -(i\slashed{\covariantDerivative} + m)(i\slashed{\covariantDerivative} - m) = (\slashed{\covariantDerivative}\slashed{\covariantDerivative} + m^2).
    \end{equation}
    So we need to compute \(\slashed{\covariantDerivative}\slashed{\covariantDerivative}\).
    To start off write this as
    \begin{align}
        \slashed{\covariantDerivative}\slashed{\covariantDerivative} &= \covariantDerivative_\mu \covariantDerivative_\nu \gamma^\mu \gamma^\nu\\
        &= \frac{1}{2}\anticommutator{\covariantDerivative_\mu}{\covariantDerivative_\nu}\gamma^\mu \gamma^\nu + \frac{1}{2}\commutator{\covariantDerivative_\mu}{\covariantDerivative_\nu}\gamma^\mu \gamma^\nu
    \end{align}
    where the last step is simply writing \(\covariantDerivative_\mu \covariantDerivative_\nu\) as a sum of its symmetric and antisymmetric part, expanding the (anti)commutators it should be clear that this holds.
    We can then use the fact that \(\anticommutator{\covariantDerivative_\mu}{\covariantDerivative_\nu}\) is symmetric to replace \(\gamma^\mu\gamma^\nu\) with its symmetric part, \(\anticommutator{\gamma^\mu}{\gamma^\nu}/2\), and similarly the antisymmetry of \(\commutator{\covariantDerivative_\mu}{\covariantDerivative_\nu}\) allows us to replace \(\gamma^\mu\gamma^\nu\) with its antisymmetric part, \(\commutator{\gamma^\mu}{\gamma^\nu}/2\).
    Doing so we get
    \begin{align}
        \slashed{\covariantDerivative}\slashed{\covariantDerivative} &= \frac{1}{4}\anticommutator{\covariantDerivative_\mu}{\covariantDerivative_\mu} \anticommutator{\gamma^\mu}{\gamma^\nu} + \frac{1}{4}\commutator{\covariantDerivative_\mu}{\covariantDerivative_\nu} \commutator{\gamma^\mu}{\gamma^\nu}\\
        &= \frac{1}{4}\anticommutator{\covariantDerivative_\mu}{\covariantDerivative_\nu} 2\minkowskiMetric^{\mu\nu} + \frac{1}{4}\left( -ieF_{\mu\nu} \right) ( -2i\sigma^{\mu\nu} )\\
        &= \frac{1}{2} \anticommutator{\covariantDerivative_\mu}{\covariantDerivative^\mu} - \frac{1}{2}eF_{\mu\nu} \sigma^{\mu\nu}\\
        &= \covariantDerivative^2 - \frac{1}{2}eF_{\mu\nu} \sigma^{\mu\nu}.
    \end{align}
    Here we've used the commutators
    \begin{equation}
        \commutator{\covariantDerivative_\mu}{\covariantDerivative_\nu} = -ieF_{\mu\nu}, \qqand \sigma^{\mu\nu} \coloneqq \frac{i}{2}\commutator{\gamma^\mu}{\gamma^\nu},
    \end{equation}
    the definition of the gamma matrices, \(\anticommutator{\gamma^\mu}{\gamma^\nu} \coloneqq 2\minkowskiMetric^{\mu\nu}\), and
    \begin{equation}
        \anticommutator{\covariantDerivative_\mu}{\covariantDerivative^\mu} = \covariantDerivative_\mu \covariantDerivative^\mu + \covariantDerivative^\mu \covariantDerivative_\mu = 2\covariantDerivative^2.
    \end{equation}
    Hence, the wave function must satisfy 
    \begin{equation}\label{eqn:qed klein gordon}
        \left[ \covariantDerivative^2 - \frac{1}{2}eF_{\mu\nu}\sigma^{\mu\nu} + m^2 \right] \psi_{\symrm{WF}} = 0.
    \end{equation}

    Now consider a magnetic field, \(\vv{B}\), in the \(z\) direction.
    Then \(F^{12} = -F^{21} = -B\), and all other components of \(F^{\mu\nu}\) vanish.
    With this we have
    \begin{equation}
        F^{\mu\nu}\sigma_{\mu\nu} = F^{12}\sigma_{12} + F^{21}\sigma_{21} = -B\sigma_{12} + B\sigma_{21} = 2B\sigma_{21}.
    \end{equation}
    In the Dirac representation we have
    \begin{align}
        \sigma_{21} &= \frac{i}{2}\commutator{\gamma_2}{\gamma_1}\\
        &= \frac{i}{2}
        \begin{pmatrix}
            0 & \sigma^2\\
            -\sigma^2 & 0
        \end{pmatrix}
        \begin{pmatrix}
            0 & \sigma^1\\
            -\sigma^1 & 0
        \end{pmatrix}
        - \frac{i}{2}
        \begin{pmatrix}
            0 & \sigma^1\\
            -\sigma^1 & 0
        \end{pmatrix}
        \begin{pmatrix}
            0 & \sigma^2\\
            -\sigma^2 & 0
        \end{pmatrix}
        \\
        &= 
        -
        \begin{pmatrix}
            \sigma^3 & 0\\
            0 & \sigma^3
        \end{pmatrix}
        .
    \end{align}
    Define
    \begin{equation}
        \vv{S} = \frac{1}{2}
        \begin{pmatrix}
            \vv{\sigma} & 0\\
            0 & \vv{\sigma}
        \end{pmatrix}
        ,
    \end{equation}
    which we can recognise as the spin operator\footnote{we've also called this \(\vv{\Sigma}\)}.
    and we can write this result as
    \begin{equation}
        F^{\mu\nu}\sigma_{\mu\nu} = 4BS^3 = 4 \vv{B} \cdot \vv{S},
    \end{equation}
    having used the fact that \(\vv{B} = (0, 0, B)\).
    Now, since we picked the \(z\) direction arbitrarily this result must hold for any uniform magnetic field, that is
    \begin{equation}
        -\frac{e}{2}F^{\mu\nu}\sigma_{\mu\nu} = 2e\vv{B} \cdot \vv{S}.
    \end{equation}
    Thus we have
    \begin{equation}
        \left[ \covariantDerivative^2 + 2e\vv{B} \cdot \vv{S} + m^2 \right] \psi_{\symrm{WF}} = 0.
    \end{equation}
    
    To pass to the non-relativistic limit consider plane wave solutions
    \begin{equation}
        \exp\{-ip \cdot x\} = \exp\{-iEt + i\vv{p} \cdot \vv{x}\}.
    \end{equation}
    Then using
    \begin{equation}
        E = \sqrt{\vv{p}^2 + m^2} \approx m + \frac{\vv{p}^2}{2m}
    \end{equation}
    we get
    \begin{equation}
        \e\{-ip\cdot x\} \approx \exp\{-imt\}\exp\left\{ -i\frac{\vv{p}^2}{2m}t + i\vv{p} \cdot \vv{x} \right\}.
    \end{equation}
    Recognising that \(\vv{p}^2/(2m)\) is the kinetic energy we can see that the second exponential is the plane wave we would make as an ansatz in a non-relativistic setting.
    This suggests that we should look for solutions of the form
    \begin{equation}
        \psi_{\symrm{WF}}(x) = \e^{-imt} \psi_{\symrm{NR}}(x)
    \end{equation}
    where \(\psi_{\symrm{NR}}\) is the non-relativistic wave function.
    
    Now consider the \(\covariantDerivative^2\) term, we have
    \begin{align}
        \covariantDerivative^2 f &= (\partial_\mu - ieA_\mu)(\partial^\mu - ieA^\mu) f\\
        &= \dalembertian f - \partial_\mu (ieA^\mu f) - ieA_\mu \partial^\mu f - e^2 A_\mu A^\mu f\\
        &= \dalembertian f - ie (\partial_\mu A^\mu) f - ie A^\mu \partial_\mu f - ieA_\mu \partial^\mu f - e^2 A_\mu A^\mu f\\
        &= \dalembertian f - ie(\partial_\mu A^\mu)f - 2ieA^\mu \partial_\mu f - e^2 A_\mu A^\mu f.
    \end{align}
    We can drop the \(e^2\) term, since it will be small in the non-relativistic limit.
    If \(\vv{E} = \vv{0}\) then we can take \(A^0 = 0\).
    Taking \(f = \psi_{\symrm{WF}} = \e^{-imt}\psi_{\symrm{NR}}\) we get
    \begin{align}
        \covariantDerivative^2\psi_{\symrm{WF}} &= \dalembertian \psi_{\symrm{WF}} - ie(\partial_\mu A^\mu)\psi_{\symrm{WF}} - 2ieA^\mu \partial_\mu(\e^{-imt}\psi_{\symrm{NR}})\\
        &= \dalembertian \psi_{\symrm{WF}} - ie(\partial_\mu A^\mu)\psi_{\symrm{WF}} \notag\\
        &\qquad - 2ie[A^0 \partial_0 (\e^{-imt} \psi_{\symrm{NR}}) - \vv{A} \cdot \grad (\e^{-imt} \psi_{\symrm{NR}})]\\
        &=  \dalembertian \psi_{\symrm{WF}} - ie(\partial_\mu A^\mu)\psi_{\symrm{WF}}\\
        &\qquad - 2ie[-im A^0\e^{-imt} \psi_{\symrm{NR}} + A^0\e^{-imt} \partial_t \psi_{\symrm{NR}} - \e^{-imt}\vv{A} \cdot \grad\psi_{\symrm{NR}}].\notag
    \end{align}
    Now suppose that we have a uniform, constant magnetic field, and no electric field, so \(\partial_\mu A^\mu = 0\).
    Reinstating factors of \(c\) we have \(A^\mu = (\varphi/c, \vv{A})\), so we can neglect \(A^0\) in the non-relativistic limit.
    Thus,
    \begin{equation}
        \covariantDerivative^2 \psi_{\symrm{WF}} = \dalembertian \psi_{\symrm{WF}} + 2ie\e^{-imt} \vv{A} \cdot \grad \psi_{\symrm{NR}}.
    \end{equation}
    
    We can expand the \(\dalembertian \psi_{\symrm{WF}}\) term as
    \begin{align}
        \dalembertian \psi_{\symrm{WF}} &= (\partial_t^2 - \laplacian)(\e^{-imt}\psi_{\symrm{NR}})\\
        &= \partial_t(-im\e^{-imt}\psi_{\symrm{NR}} - \e^{-imt} \partial_t \psi_{\symrm{NR}}) - \e^{-imt}\laplacian\psi_{\symrm{NR}}\\
        &= (-im)^2\e^{-imt}\psi_{\symrm{NR}} - im\e^{-imt}\partial_t\psi_{\symrm{NR}} - im\e^{-imt} \partial_t \psi_{\symrm{NR}} \notag\\
        &\qquad - \e^{-imt} \partial_t^2 \psi_{\symrm{NR}} - \e^{-imt}\laplacian\psi_{\symrm{NR}}\\
        &= \e^{-imt} \left[ -m^2\psi_{\symrm{NR}} - 2im\partial_t\psi_{\symrm{NR}} + \partial_t^2 \psi_{\symrm{NR}} - \laplacian \psi_{\symrm{NR}} \right].
    \end{align}
    Notice that the first term cancels with the term coming from \(m^2\) in \cref{eqn:qed klein gordon}.
    Thus we have
    \begin{align}
        0 &= \left[ \covariantDerivative^2 + 2e\vv{B} \cdot \vv{S} + m^2 \right] \psi_{\symrm{WF}}\\
        &= \e^{-imt} [ - 2im\partial_t + \partial_t^2 - \laplacian + 2ie\e^{-imt} \vv{A} \cdot \grad + 2e\vv{B} \cdot \vv{S} ] \psi_{\symrm{NR}}.
    \end{align}
    Dividing through by \(2m\e^{-imt}\) we get
    \begin{equation}
        0 = \left[ -i\partial_t + \frac{1}{2m} \partial_t^2 - \frac{1}{2m} \laplacian + \frac{e}{m} \vv{A} \cdot \grad + \frac{e}{2m} 2\vv{B} \cdot \vv{S} \right] \psi_{\symrm{NR}}.
    \end{equation}
    We can neglect the \(\partial_t^2\) term as reinstating factors of \(c\) we see it is suppressed by \(1/c^2\).
    Similarly, we drop the \(\vv{A} \cdot \grad\) term, leaving us with
    \begin{equation}
        \left[ -\frac{1}{2m} \laplacian + \frac{e}{2m}2\vv{B} \cdot \vv{S} \right] \psi_{\symrm{NR}} = -i\diffp*{\psi_{\symrm{NR}}}{t}.
    \end{equation}
    This is simply the Schrödinger equation.
    
    From this we can read off the change to the Hamiltonian due to the magnetic field, it is
    \begin{equation}
        \delta H = \frac{e}{2m} 2 \vv{B} \cdot \vv{S}.
    \end{equation}
    Comparing this with the definition of the magnetic moment, \(\delta H = -\vv{\mu} \cdot \vv{B}\) we see that
    \begin{equation}
        \vv{\mu} = -\frac{e}{2m}2 \vv{S},
    \end{equation}
    that is, 
    \begin{equation}
        g = 2.
    \end{equation}
    
    \section{As an Effective Field Theory}
    We can see this same result in QED by considering the three point tree amplitude
    \begin{equation}
        \tikzsetnextfilename{magnetic-moment-three-point-amplitude}
        \begin{tikzpicture}[baseline=-0.08cm]
            \draw[photon] (-1, 0) -- (0, 0);
            \draw[electron=0.6] (0, 0) -- (30:1);
            \draw[positron=0.4] (0, 0) -- (-30:1);
            \draw[yshift=-0.2cm, ->] (-0.8, 0) -- (-0.2, 0) node [midway, below] {\(k\)};
            \draw[yshift=0.2cm, xshift=-0.1cm, ->] (30:0.2) -- (30:0.8) node [midway, above] {\(p'\)};
            \draw[yshift=-0.2cm, xshift=-0.1cm, <-] (-30:0.2) -- (-30:0.8) node [midway, below, xshift=-0.1cm] {\(p\)};
        \end{tikzpicture}
        = ie\diracadjoint{u}(p') \gamma^\mu u(p) \varepsilon_\mu(k) = \amplitude_3.
    \end{equation}
    Note that \(k = p' - p\).
    We might think of this as probing an electron with a photon.
    We'll take the electron to be on-shell and the photon to be either on-shell or at least have \(k^2\) be negligible.
    
    Consider the following expression with \(k = p' - p\) and \(\sigma^{\mu\nu} = i\commutator{\gamma^\mu}{\gamma^\nu}/2\):
    \begingroup
    \allowdisplaybreaks
    \begin{align}
        &\diracadjoint{u}(p')\bigg[ \frac{p'^\mu + p^\mu}{2m} + \frac{i\sigma^{\mu\nu}k_\nu}{2m} \bigg]u(p)\\
        &\hspace{-1em}= \frac{\diracadjoint{u}(p')}{2m}\left[ p'^\mu + p^\mu - \frac{1}{2}(\gamma^\mu \gamma^\nu - \gamma^\nu \gamma^\mu)(p'_\nu - p_\nu) \right] u(p)\\
        &\hspace{-1em}= \frac{\diracadjoint{u}(p')}{2m}\left[ p'^\mu + p^\mu - \frac{1}{2}(\gamma^\mu \slashed{p}' - \gamma^\mu \slashed{p} - \slashed{p}' \gamma^\mu + \slashed{p}\gamma^\mu) \right] u(p)\\
        &\hspace{-1em}= \frac{\diracadjoint{u}(p')}{2m}\left[ p'^\mu + p^\mu - \frac{1}{2}(2\minkowskiMetric^{\mu\nu}p'_\nu - \slashed{p}'\gamma^\mu - \gamma^\mu\slashed{p} - \slashed{p}'\gamma^\mu + 2\minkowskiMetric^{\mu\nu}p_\nu - \gamma^\mu \slashed{p}) \right] u(p) \notag\\
        &\hspace{-1em}= \frac{\diracadjoint{u}(p')}{2m}\left[ p'^\mu + p^\mu - (p'^\mu - \slashed{p}'\gamma^\mu - \gamma^\mu \slashed{p} + p^\mu) \right] u(p)\\
        &\hspace{-1em}= \frac{\diracadjoint{u}(p')}{2m}\left[ p'^\mu + p^\mu - (p'^\mu - \slashed{p}'\gamma^\mu - \gamma^\mu \slashed{p} + p^\mu) \right] u(p)\\
        &\hspace{-1em}= \frac{\diracadjoint{u}(p')}{2m}\left[ p'^\mu + p^\mu - (p'^\mu - m\gamma^\mu - \gamma^\mu m + p^\mu) \right] u(p)\\
        &\hspace{-1em}= \frac{\diracadjoint{u}(p')}{2m}\left[ 2m\gamma^\mu \right] u(p)\\
        &\hspace{-1em}= \diracadjoint{u}(p') \gamma^\mu u(p).
    \end{align}
    \endgroup
    Here we've made use of the definition of \(u(p)\) as the momentum space solution to the Dirac equation, meaning \((\slashed{p} - m)u(p) = 0\), so \(\slashed{p}u(p) = mu(p)\), and the adjoint, \(\diracadjoint{u}(p)\slashed{p} = m\).
    This gives us the \defineindex{Gordon identity}:
    \begin{equation}
        \diracadjoint{u}(p')\gamma^\mu u(p) = \diracadjoint{u}(p')\left[ \frac{p'^\mu + p^\mu}{2m} + \frac{i\sigma^{\mu\nu}k_\nu}{2m} \right]u(p).
    \end{equation}
    
    Using this we can rewrite the three point amplitude as
    \begin{equation}
        \amplitude_3 = ie\diracadjoint{u}(p')\left[ \frac{p'^\mu + p^\mu}{2m} + \frac{i\sigma^{\mu\nu}k_\nu}{2m} \right] u(p) \varepsilon_\mu(k).
    \end{equation}
    We can think of the first term as being the result from scalar QED, the vertex term here being \(ie(p'^\mu + p^\mu)\varepsilon_\mu(k)\), and the second term being a spin effect.
    Taking this spin correction we can imagine that it has a prefactor of 1, and we'll see that this term corresponds to \(g/2\), meaning if this prefactor is changed to \(1 + a\) then \(g\) will change from \(2\) to \(2 + 2a\), this is essentially what we will compute.
    This quantity, \(a = (g - 2)/2\), is called the \defineindex{anomalous magnetic moment}.
    
    We can see how this relates to classical physics by comparing world lines.
    The action in this case is given by
    \begin{equation}
        S = -m \int \dl{\tau} - \int \dl{\tau} \, e u_\mu A^\mu(r(\tau)) + \int \dl{\tau} \, \frac{e}{2m} g \frac{1}{4}S_{\mu\nu}F^{\mu\nu}(r(\tau)).
    \end{equation}
    The first two terms here are the normal action\footnote{derived in \course{Classical Electrodynamics}} for a charged particle with four-velocity \(u^\mu\) moving in an external field, \(A^\mu\).
    The third term contains the spin tensor, \(S_{\mu\nu}\) \textcolor{red}{WHAT IS THIS????? Is it \(\sigma^{\mu\nu}\)? Is it antisymmetric?}, is needed for a particle with spin, this term is included as it is Lorentz invariant, and there is no reason to exclude it.
    
    In QFT we are interested in allowing for high energies where we have pair production.
    The action written down here is for an effective field theory at low energies which doesn't allow for pair production.
    We have to perform a matching process to compare this effective field theory to QED.
    To do this we have to consider some scenario where we have a reasonable idea of what the outcome will be.
    
    Consider the case of a particle moving at a constant speed starting at the origin at time \(\tau = 0\).
    It's position is given by \(r^\mu(\tau) = u^\mu \tau\).
    The interaction action in the effective field theory, simply the action of this effective theory without the first term, is then given by
    \begin{equation}
        S_{\interaction} = \int \dl{\tau} \left[ -e u_\mu A^\mu(u\tau) + \frac{e}{2m}g \frac{1}{4}S_{\mu\nu}F^{\mu\nu}(u\tau) \right].
    \end{equation}
    Now we can write this as the inverse Fourier transform of the Fourier transform, defining \(\varepsilon^\mu(k) = \fourierTransform\{A^\mu(u\tau)\}\).
    Note that this isn't a polarisation vector, but we pick the notation \(\varepsilon^\mu\) as it is \emph{like} a polarisation vector.
    Then the action is
    \begin{equation}
        S_{\interaction} = \int \dhat{^4k} \int \dl{\tau} \, \e^{-ik \cdot u\tau}\left[ -eu_\mu \varepsilon^\mu(k) - i\bohrMagneton g \frac{1}{4}S_{\mu\nu}(k^\mu \varepsilon^\nu - k^\nu \varepsilon^\mu) \right],
    \end{equation}
    where derivatives in \(F^{\mu\nu}\) have turned into \(k\)s and \(A\)s have turned into \(\varepsilon\)s.
    The spin tensor, \(S_{\mu\nu}\), is antisymmetric, so we have
    \begin{equation}
        S_{\mu\nu}k^\mu \varepsilon^\nu = S_{\mu\nu} k^{(\mu}\varepsilon^{\nu)} = \frac{1}{2}S_{\mu\nu}(k^\mu \varepsilon^\nu - k^\nu \varepsilon^\mu)
    \end{equation}
    allowing us to write the interaction action as
    \begin{equation}
        S_{\interaction} = \int\dhat{^4k} \int \dl{\tau} \, \e^{-ik\cdot u \tau}\left[ -e u_\mu \varepsilon^\mu(k) - i \bohrMagneton g \frac{1}{2}S_{\mu\nu}k^\mu \varepsilon^\nu(k) \right].
    \end{equation}
    We can also compute the \(\tau\) integral, since the only \(\tau\) dependence is in the exponential.
    Doing this, and assuming that the interaction is switched on at \(\tau = 0\), we get
    \begin{equation}
        \int \dl{\tau}_{0}^{\infty} \, \e^{-ik \cdot u \tau} = \left[ \frac{i}{k \cdot u} \e^{-ik \cdot u\tau} \right]_{\tau = 0}^{\tau = \infty}.
    \end{equation}
    This doesn't converge.
    However, if we modify it slightly so that \(k \cdot u \mapsto k \cdot u \mapsto k \cdot u - i\epsilon\) with \(\epsilon > 0\) then we get a term \(\e^{-\epsilon k \cdot u \tau}\), which goes to zero as \(\tau \to \infty\).
    We then have convergence and we get
    \begin{equation}
        \int \dl{\tau}_{0}^{\infty} \, \e^{-i(k \cdot u - i \epsilon) \tau} = \left[ \frac{i}{k \cdot u - i\epsilon} \e^{-i(k \cdot u - i \epsilon)\tau} \right]_{\tau = 0}^{\tau = \infty} = \frac{-i}{k\cdot u - i\epsilon},
    \end{equation}
    and we are then free to take \(\epsilon \to 0\).
    Then we have
    \begin{equation}\label{eqn:magnetic moment eft calculation}
        S_{\interaction} = \int \dhat{^4k} \, \frac{-i}{k \cdot u}\left[ -e u \cdot \varepsilon - i \bohrMagneton g \frac{1}{2}S^{\mu\nu}k_\mu \varepsilon_\nu \right].
    \end{equation}
    
    Consider a four point process occurring via this interaction,
    \begin{equation}
        \tikzsetnextfilename{magnetic-moment-four-point-interaction}
        \begin{tikzpicture}[baseline=(current bounding box)]
            \draw[blob] (0, 0) circle [radius = 0.4cm];
            \draw[electron=0.6] (135:1.4) -- (135:0.4);
            \draw[electron=0.6] (225:1.4) -- (225:0.4);
            \draw[electron=0.6] (45:0.4) -- (45:1.4);
            \draw[electron=0.6] (315:0.4) -- (315:1.4);
        \end{tikzpicture}
        .
    \end{equation}
    At time \(\tau = 0\) we can probe one of the particles emitted in this interaction with a photon,
    \begin{equation}
        \tikzsetnextfilename{magnetic-moment-four-point-interaction-probed}
        \begin{tikzpicture}[baseline=(current bounding box)]
            \draw[blob] (0, 0) circle [radius = 0.4cm];
            \draw[electron=0.6] (135:1.4) -- (135:0.4);
            \draw[electron=0.6] (225:1.4) -- (225:0.4);
            \draw[electron=0.6] (45:0.4) -- (45:1.4);
            \draw[electron=0.7] (315:0.4) -- (315:0.9);
            \draw[electron=0.7] (315:0.9) -- (315:1.4);
            \draw[photon] (315:0.9) -- ++ (45:1) node [right] {\(\varepsilon\)};
            \draw[->, font=\tiny, shift=(225:0.2)] (315:0.5) -- (315:0.8) node [left, xshift=-0.05cm] {\(p + k\)};
            \draw[->, font=\tiny, shift=(225:0.2)] (315:1) -- (315:1.3) node [left, xshift=-0.05cm] {\(p\)};
            \draw[->, font=\tiny, shift={(0.4, 0.1)}] (315:0.9) -- ++ (45:0.5) node [midway, below right, shift=(135:0.1)] {\(k\)};
        \end{tikzpicture}
        .
    \end{equation}
    
    On the QFT side we can consider this interaction followed by probing.
    Computing the amputated correlator of this probing part, where the only particle not amputated is the electron before probing, after throwing away factors like \(u(p)\) which aren't interesting we get
    \begin{equation}
        ie \frac{i}{(p + k)^2 + i\epsilon} \left( \frac{2p \cdot \varepsilon}{2m} + i2S^{\mu\nu}k_\mu \varepsilon_\mu \right)
    \end{equation}
    where we've used the Gordon identity on the \(\diracadjoint{u}(p) \gamma^\mu u(p + k)\) term before throwing away the spinor factors.
    We've also used \(k \cdot \varepsilon = 0\).
    
    We can rewrite this as
    \begin{equation}
        - \frac{e}{u \cdot k} \frac{1}{2m}(2u \cdot \varepsilon + i\bohrMagneton S^{\mu\nu}\varepsilon_\mu k_\nu).
    \end{equation}
    This can then be compared with \cref{eqn:magnetic moment eft calculation} and we see that we must have
    \begin{equation}
        g = 2.
    \end{equation}	
    
    \chapter{Anomalous Magnetic Moment}
    \section{Setup}
    The derivation towards the end of the last chapter suggests that we should get a more accurate value of \(g\) if we compute loop corrections to the QED vertex.
    At tree level we have
    \begin{equation}
        \tikzsetnextfilename{anomalous-magnetic-moment-tree-level-qed-vertex}
        i\amplitude_{\symrm{tree}}
        \begin{tikzpicture}[baseline=(current bounding box)]
            \draw[electron=0.6] (150:1) -- (0, 0);
            \draw[positron=0.4] (210:1) -- (0, 0);
            \draw[photon] (0, 0) -- (1, 0);
            \draw[->, font=\small, shift=(60:0.15)] (150:0.8) -- (150:0.2) node [midway, above right, shift={(-0.1, -0.05)}] {\(p\)};
            \draw[<-, font=\small, shift=(-60:0.15)] (210:0.8) -- (210:0.2) node [midway, below right, shift={(-0.1, 0.1)}] {\(p'\)};
            \draw[<-] (0.2, -0.2) -- (0.8, -0.2) node [below] {\(k, \varepsilon\)};
        \end{tikzpicture}
        = ie \diracadjoint{u}(p') \gamma^\mu u(p) \varepsilon_\mu(k).
    \end{equation}
    To arbitrary order this becomes
    \begin{equation}
        \tikzsetnextfilename{anomalous-magnetic-moment-all-orders-qed-vertex}
        i\amplitude = 
        \begin{tikzpicture}[baseline=(current bounding box)]
            \draw[blob] (0, 0) circle [radius=0.25cm];
            \draw[electron=0.7] (150:1) -- (150:0.25);
            \draw[positron=0.3] (210:1) -- (210:0.25);
            \draw[photon] (0.25, 0) -- (1, 0);
            \draw[->, font=\small, shift=(60:0.15)] (150:0.8) -- (150:0.45) node [midway, above right, shift={(-0.1, -0.05)}] {\(p\)};
            \draw[<-, font=\small, shift=(-60:0.15)] (210:0.8) -- (210:0.45) node [midway, below right, shift={(-0.1, 0.1)}] {\(p'\)};
            \draw[<-] (0.45, -0.2) -- (0.8, -0.2) node [midway, below] {\(k, \varepsilon\)};
        \end{tikzpicture}
        = ie\diracadjoint{u}(p') \Gamma^\mu(p', p) u(p) \varepsilon_\mu(k).
    \end{equation}
    Here \(\Gamma^\mu(p', p)\) generalises the gamma matrix, \(\gamma^\mu\), to include loop corrections.
    Our goal will be to compute \(\Gamma^\mu\).
    
    \section{The Structure of \texorpdfstring{\(\Gamma^\mu\)}{Gamma}}
    Fortunately the form of \(\Gamma^\mu\) is rather limited.
    It must be a four-vector, meaning it can only be formed from other four-vectors and Lorentz invariant scalars.
    The only four-vectors we have are \(p\), \(p'\), and \(k = p - p'\), as well as \(\gamma^\mu\).
    This means we only have two independent variable four vectors, \(p\) and \(p'\), but we can change these to \(p + p'\) and \(p - p'\), which are also independent.
    From this we see that \(\Gamma^\mu\) must be a linear combination of \(\gamma^\mu\), \(p^\mu + p'^\mu\), and \(p^\mu - p'^\mu\).
    The coefficients must be Lorentz scalars.
    We assume that the electrons are on shell, so \(p^2 = p'^2 = m^2\), which is just a constant.
    Since the photon is being used to probe the electron it must have been produced in some other process, so it is a virtual particle, but we assume it is almost on-shell, meaning that \(k^2 \approx 0\), but \(k^2\) is not necessarily exactly zero.
    This means that the only variable Lorentz scalar we have is \(k^2\), so the coefficients must be functions of \(k^2\).
    We also know that we can't have products of gamma matrices other than \(\gamma^\mu\) on its own, since either these reduce to Lorentz scalars, such as \(\gamma^\nu \gamma_\nu = D\), or we can use \(\slashed{p}u(p) = mu(p)\) to remove them, or they reduce to something which can written in terms of \(\gamma^5\), but this is not parity invariant, which is something that we require of QED.
    
    Combing this we see that \(\Gamma^\mu\) must be of the form
    \begin{equation}
        \Gamma^\mu(p', p) = A(k^2) \gamma^\mu + B(k^2)(p^\mu + p'^\mu) + C(k^2)(p^\mu - p'^\mu)
    \end{equation}
    where \(A\), \(B\), and \(C\) are scalar functions to be determined.
    
    The first observation we can make about this is that \(p^\mu - p'^\mu = k^\mu\), and so the final term is \(C(k^2) k^\mu\).
    However, the Ward identity means that we must have \(k_\mu \Gamma^\mu(p', p) = 0\), meaning we must have \(C(k^2) = 0\) so that \(C(k^2)k^2\) vanishes identically, not just when the photon is on-shell.
    
    \begin{ntn}{}{}
        We are about to make a lot of steps which don't quite leave the result the same.
        We will use the symbol \(\rightsquigarrow\), such as \(A \rightsquigarrow B\) to denote that for our purposes \(A\) and \(B\) are equivalent, if not equal.
        For example, since we know that our final result has no \(k^\mu\) term we have \(ak^\mu \rightsquigarrow 0\), since any \(k^\mu\) factor must eventually cancel out.
    \end{ntn}
    
    We can then rewrite \(\Gamma^\mu\) as
    \begin{equation}
        \Gamma^\mu(p', p) \rightsquigarrow \gamma^\mu F_1(k^2) + \frac{i\sigma^{\mu\nu}k_\nu}{2m}F_2(k^2).
    \end{equation}
    This is equivalent by again applying the Gordon identity since \(\Gamma^\mu(p', p)\) appears sandwiched between \(\diracadjoint{u}(p')\) and \(u(p)\).
    This allows us to replace \(p'^\mu + p^\mu\) with \(2m\gamma^\mu - i\sigma^{\mu\nu}k_\nu\).
    We then absorb various constants into the definitions of \(F_1\) and \(F_2\) to get the result above.
    This form of \(\Gamma^\mu(p', p)\) is the one which people work with, however we'll now invert what we've done and use the Gordon identity to write \(\Gamma^\mu\) as
    \begin{equation}
        \Gamma^\mu(p', p) = \frac{p'^\mu + p^\mu}{2m} F_1(k^2) + \frac{i\sigma^{\mu\nu}k_\nu}{2m}(F_1(k^2) + F_2(k^2)).
    \end{equation}
    
    Finally, we have to renormalise this, and as a renormalisation condition we use \(F_1(0) = 1\).
    This is a nice condition because it enforces \(e_{\measured} = e\).
    We could use another condition and get a result in terms of \(e\) which we then eliminate in preference of \(e_{\measured}\), but this is just more work.
    
    Comparing this result to \cref{eqn:magnetic moment eft calculation}, and working with \(k^2 \approx 0\), we see that the Land\'e \(g\) factor is
    \begin{equation}
        g = 2(F_1(0) + F_2(0)) = 2 + 2F_2(0).
    \end{equation}
    We can see this as the Dirac equation result, \(2\), plus a higher order correction, \(2F_2(0)\).
    So, we now just need to compute \(F_2(0)\).
    
    \section{One Loop Calculation of \texorpdfstring{\(F_2(0)\)}{F2(0)}}
    The one-loop correction to the amplitude is
    \begin{align}
        i\amplitude_{\symrm{one\ loop}} &=
        \tikzsetnextfilename{anomalous-magnetic-moment-one-loop-vertex-correction}
        \begin{tikzpicture}[baseline=(plus.base), font=\scriptsize]
            \draw[electron=0.6] (0, 0) -- ++ (-30:1) coordinate (A);
            \draw[positron=0.4] (0, -2) -- ++ (30:1) coordinate (B);
            \draw[electron=0.6] (A) -- ++ (-30:1) coordinate (C);
            \draw[positron=0.4] (B) -- ++ (30:1);
            \draw[photon] (A) -- (B);
            \draw[photon] (C) -- ++ (1, 0);
            \draw[->] ($(A)!0.2!(B) - (0.2, 0)$) -- ($(A)!0.8!(B) - (0.2, 0)$) node [midway, left] {\(q - p\)};
            \draw[->, shift={(0.2, 0.05)}] (0, 0) -- ++ (-30:0.6) node [midway, above right, shift={(-0.1, -0.1)}] {\(p\)};
            \draw[<-, shift={(0.2, -0.05)}] (0, -2) -- ++ (30:0.6) node [midway, below right, shift={(-0.1, 0.1)}] {\(p' = p + k\)};
            \draw[->] ($(A) + (0.2, 0.05)$) -- ++ (-30:0.6) node [midway, above right, shift={(-0.1, -0.1)}] {\(q\)};
            \draw[<-] ($(B) + (0.2, -0.05)$) -- ++ (30:0.6) node [midway, below right, shift={(-0.1, 0.1)}] {\(q' = q + k\)};
            \draw[<-] ($(C) + (0.2, 0.2)$) -- ++ (0.6, 0) node [midway, above] {\(k\)};
            \node (plus) at (3.2, -1) {\(+\)};
            \begin{scope}[xshift=4.4cm, yshift=-1cm]
                \foreach \angle in {45, 135, 225, 315} {
                    \draw (0, 0) -- (\angle:0.15);
                }
                \draw (0, 0) circle [radius=0.15cm];
                \draw[electron=0.6] (150:1) -- (150:0.15);
                \draw[positron=0.4] (210:1) -- (210:0.15);
                \draw[photon] (0.15, 0) -- (1, 0);
                \draw[->] (0.8, 0.15) -- (0.35, 0.15) node [midway, above] {\(k\)};
                \draw[shift={(0.08, 0.1)}, ->] (150:0.8) -- (150:0.35) node [midway, above right] {\(p\)};
                \draw[shift={(0.08, -0.1)}, <-] (210:0.8) -- (210:0.35) node [midway, below right, shift={(-0.1, 0.1)}] {\(p'\)};
                \node at (-60:0.3) {\(\delta_1\)};
            \end{scope}
        \end{tikzpicture}
        \\
        &= ie\diracadjoint{u}(p')\delta \Gamma^\mu(p', p) u(p) \varepsilon_\mu(k)
    \end{align}
    where
    \begin{equation}
        \Gamma^\mu(p', p) = \gamma^\mu + \delta\Gamma^\mu(p', p) + \dotsb
    \end{equation}
    with the one loop correction being \(\delta\Gamma^\mu(p', p)\).
    Then \(\delta\Gamma^\mu(p', p)\) is given by evaluating the above Feynman diagrams and stripping off the external states, so the spinors and the polarisation vector, as well as a factor of \(ie\).
    
    Applying the Feynman rules, stripping the external states and a factor of \(ie\) gives
    \begin{equation}
        \delta\Gamma^\mu(p', p) = (ie)^2\mu^{3\varepsilon} \int \dhat{^Dq} \frac{\gamma^\nu i(\slashed{q}' + m)\gamma^\mu i(\slashed{q} + m)\gamma_\nu}{(q'^2 - m^2 + i\epsilon)(q^2 - m^2 + i\epsilon)} \frac{-i}{(q - p)^2 + i\epsilon} + ie\mu^\varepsilon \delta_1 \gamma^\mu.
    \end{equation}
    We now need to compute this integral.
    
    Lots of terms in this integral will only contribute to \(F_1\), and we aren't interested in this term, so we will drop these terms as we go.
    These terms will be the ones which are proportional to \(\gamma^\mu\), including the counter term.
    Since there is no contribution of the counter term to \(F_2\) and we have regularised so everything is finite we know that this integral should be convergent, which allows us to drop the \(i\epsilon\)s.
    We can also drop the \(\varepsilon\) of dimensional regularisation, setting \(D = 4\), this means that \(\mu^{3\varepsilon} \rightsquigarrow 1\).
    We can also throw away \(k^\mu\) terms, since these must all cancel before we get to the final result.
    Note that this applies only to \(k^\mu\), with a free \(\mu\) index, not to any summed over \(k_\nu\) as this appears in \(\sigma^{\mu\nu}k_\nu\).
    We can also use the Gordon identity and the fact that \(\Gamma^\mu\) is sandwiched by \(\diracadjoint{u}(p')\) and \(u(p)\) to replace \(p'^\mu + p^\mu\) with \(-i\sigma^{\mu\nu}k_\nu\) wherever this term appears.
    
    Using these simplifications we have
    \begin{equation}\label{eqn:delta gamma}
        \delta\Gamma^\mu(p', p) \rightsquigarrow -ie^2 \int \dhat{^4D} \frac{\gamma^\nu(\slashed{q}' + m)\gamma^\mu(\slashed{q} + m)\gamma_\nu}{(q^2 - m^2)(q'^2 - m^2)(q - p)^2}.
    \end{equation}
    Note that this integral is still divergent, but we will drop any divergent terms as they arise since they don't contribute to \(F_2\).
    
    Now we have to use Feynman parametrisation, which for three variables takes the form
    \begin{align}
        \frac{1}{ABC} &= \int_0^1 \dl{x} \int_0^1 \dl{y} \int_0^1 \dl{z} \frac{\delta(1 - x - y - z)}{(xA + bY + cZ)^3} \frac{\Gamma(3)}{\Gamma(1)\Gamma(1)\Gamma(1)}\\
        &= 2 \int_{\mathrlap{[0, 1]^3}} \  \dl{x} \dd{y} \dd{z} \, \frac{\delta(1 - x - y - z)}{(xA + bY + cZ)^3}.
    \end{align}
    So we want to compute
    \begin{equation}
        \delta\Gamma^\mu(p', p) \rightsquigarrow -2ie^2 \int_{\mathrlap{[0, 1]^3}} \  \dl{x} \dd{y} \dd{z} \, \delta(1 - x - y - z) \int \dl{^4\ell} \frac{N^\mu}{d^3}
    \end{equation}
    where \(N^\mu\) is the numerator and denominator of \cref{eqn:delta gamma} r,eparametrised to be in terms of \(\ell\) and
    \begin{equation}
        d = x(q^2 - m^2) + y(q'^2 - m^2) + z(q - p)^2.
    \end{equation}

    At this point it's helpful to list things we can do to simplify the calculation:
    \begin{itemize}
        \item drop any \(\gamma^\mu\) term as these contribute only to \(F_1\);
        \item drop any \(k^\mu\) term as these must cancel in the final result;
        \item drop any \(k^2\) terms as we are assuming \(k^2 \approx 0\);
        \item \(p'^\mu + p^\mu \rightsquigarrow -i\sigma^{\mu\nu}k_\nu\), by the Gordon identity and dropping \(\gamma^\mu\) terms;
        \item use \(\delta(1 - x - y - z)\) at any time to replace, for example, \(x + y\) with \(1 - z\).
    \end{itemize}
    
    We can use \(q' = q + k\) to simplify \(d\).
    Expanding \(d\) we get
    \begin{equation}
        d = xq^2 + yq^2 + yk^2 + 2yq \cdot k - (x + y)m^2 + zq^2 + zp^2 - 2zq \cdot p.
    \end{equation}
    Now we can take \(k^2 \approx 0\) and \(p^2 = m^2\) giving
    \begin{equation}
        d = (x + y + z)q^2 + 2yq \cdot k - (x + y + z)m^2 - 2zq \cdot p.
    \end{equation}
    Now using the Dirac delta we can replace \(x + y + z \rightsquigarrow 1\) and \(x + y \rightsquigarrow 1 - z\), giving
    \begin{equation}
        d = q^2 + 2q \cdot (yk - zp) - (1 - 2z)m^2.
    \end{equation}
    Looking at this we see that if we set \(\ell = q + yk - zp\) then \(\ell^2 = q^2 + 2q\cdot (yk - zp) + (yk - zp)^2\) contains the first two terms here, so let's do that.
    The last term here is
    \begin{equation}
        (yk - zp)^2 = y^2k^2 + z^2p^2 - 2yz k \cdot p \rightsquigarrow z^2m^2
    \end{equation}
    having used \(k^2 \approx 0\), \(p^2 = m^2\), and
    \begin{equation}
        p' = p + k \implies p'^2 = m^2 = p^2 + k^2 + 2p \cdot k \approx m^2 + 2p\cdot k \implies p \cdot k \approx 0.
    \end{equation}
    This gives
    \begin{align}
        d &= q^2 + 2q \cdot (yk - zp) - (1 - 2z)m^2\\
        &= q^2 + 2q\cdot (yk - zp) + z^2m^2 - (1 - 2z)m^2 - z^2m^2\\
        &= \ell^2 - (1 - 2z)m^2 - z^2m^2\\
        &= \ell^2 - (z^2 - 2z + 1)m^2\\
        &= \ell^2 - (1 - z)^2m^2\\
        &= \ell^2 - \Delta
    \end{align}
    with \(\Delta = (1 - z)^2 m^2\).
    
    This takes us to
    \begin{equation}
        \delta\Gamma^\mu(p', p) \rightsquigarrow -2ie^2 \int_{\mathrlap{[0, 1]^3}} \  \dl{x} \dd{y} \dd{z} \, \delta(1 - x - y - z) \int \dhat{^4\ell} \frac{N^\mu}{(\ell^2 - \Delta)^3}.
    \end{equation}
    Now we need to simplify the numerator.
    For this we'll use the following identities which are valid in \(D = 4\) dimensions:
    \begin{align}
        \gamma^\nu \slashed{a} \gamma_\nu &= -2\slashed{a},\\
        \gamma^\nu \slashed{a} \slashed{b} \gamma_\nu &= 4 a \cdot b,\\
        \gamma^\nu \slashed{a} \slashed{b} \slashed{c} \gamma_\nu &= -2\slashed{c}\slashed{b}\slashed{a}.
    \end{align}
    The numerator is
    \begin{align}
        N^\mu &= \gamma^\nu(\slashed{q}' + m)\gamma^\mu(\slashed{q} + m)\gamma_\nu\\
        &= \gamma^\nu\slashed{q}'\gamma^\mu\slashed{q}\gamma_\nu + m\gamma^\nu\slashed{q}'\gamma^\mu\gamma_\nu + m\gamma^\nu\gamma^\mu\slashed{q}\gamma_\nu + m^2\gamma^\nu\gamma^\mu\gamma_\nu\\
        &= \slashed{q} \gamma^\mu \slashed{q}' + 4mq^\mu + 4mq'^\mu - 2m^2\gamma^\mu\\
        &\rightsquigarrow \slashed{q}\gamma^\mu\slashed{q}' + 4m(q^\mu + q'^\mu)
    \end{align}
    having dropped the \(\gamma^\mu\) term as it only contributes to \(F_1\).
    
    Using \(\ell = q + yk - zp\) we have \(q = \ell - yk + zp\), and then using \(q' = q + k\) we have \(q' = \ell + (1 - y)k + zp\).
    This gives
    \begin{equation}
        N^\mu \rightsquigarrow -2[(\slashed{\ell} - y\slashed{k} + z\slashed{p})\gamma^\mu(\slashed{\ell} + (1 - y)\slashed{k} + z\slashed{p}) - 2m(2\ell^\mu + (1 - 2y)k^\mu + 2zp^\mu)].
    \end{equation}
    This can be simplified by dropping the \(\ell^\mu\) term, since it vanishes in the integral by symmetry, and the \(k^\mu\) term, since it must cancel in the final result, and dropping the \(\slashed{\ell}\) terms since if we have just one of them in the product then the integral of that part vanishes by symmetry and if we have two of them then this gives a divergent integral, which doesn't contribute to \(F_2\).
    Applying these simplifications, as well as using \(p = p' - k\) we get
    \begin{equation}
        N^\mu \rightsquigarrow -2[(z\slashed{p}' - (y + z)\slashed{k})\gamma^\mu((1 - y)\slashed{k} + z\slashed{p}) - 4zmp^\mu].
    \end{equation}
    One further simplification is to use the Dirac delta to replace \(y + z\) with \(1 - x\), which just slightly increases the symmetry between the two factors and is slightly simpler:
    \begin{equation}
        N^\mu \rightsquigarrow -2[(z\slashed{p}' - (1 - x)\slashed{k})\gamma^\mu((1 - y)\slashed{k} + z\slashed{p}) - 4zmp^\mu].
    \end{equation}
    
    Since \(u(p)\) is a solution to the Dirac equation in momentum space we have \((\slashed{p} + m)u(p) = 0\) and \(\diracadjoint{u}(p)(\slashed{p} + m) = 0\).
    This gives \(\slashed{p}u(p) = mu(p)\) and \(\diracadjoint{u}(p)\slashed{p} = m\).
    Using the fact that \(\delta\Gamma^\mu\) is sandwiched between \(\diracadjoint{u}(p')\) and \(u(p)\) we get
    \begin{equation}
        N^\mu \rightsquigarrow -2[(zm - (1 - x)\slashed{k})\gamma^\mu((1 - y)\slashed{k} + zm) - 4zmp^\mu].
    \end{equation}
    Now we can use
    \begin{equation}
        \slashed{k} \gamma^\mu \slashed{k} = \slashed{k}\slashed{k}\gamma^\mu + 2\slashed{k}k^\mu \rightsquigarrow 0
    \end{equation}
    since \(\slashed{k}\slashed{k} \propto k^2 \approx 0\) and \(k^\mu\) terms can be dropped as they must all cancel in the final result.
    Using this we can expand the numerator to get
    \begin{align}
        N^\mu &\rightsquigarrow -2[zm\gamma^\mu(1 - y)\slashed{k} - (1 - x)\slashed{k}\gamma^\mu zm - 4zmp^\mu]\\
        &= -2[zm((1 - y)\gamma^\mu\slashed{k} - (1 - x)\slashed{k}\gamma^\mu) - 4zmp^\mu].
    \end{align}
    
    We can replace the terms with two gamma matrices with \(\sigma^{\mu\nu} = i\commutator{\gamma^\mu}{\gamma^\nu}/2\) using
    \begin{align}
        \gamma^\mu \slashed{k} &= \frac{1}{2}\commutator{\gamma^\mu}{\gamma^\nu}k_\nu + \frac{1}{2}\anticommutator{\gamma^\mu}{\gamma^\nu}k_\nu\\
        &= -i\sigma^{\mu\nu}k_\nu + k^\mu\\
        &\rightsquigarrow -i\sigma^{\mu\nu}k_\nu
    \end{align}
    having used \(\anticommutator{\gamma^\mu}{\gamma^\nu} \coloneqq 2\minkowskiMetric^{\mu\nu}\) and dropped the \(k^\mu\) term.
    
    We have
    \begin{equation}
        p^\mu = \frac{1}{2}(p^\mu + p'^\mu) - \frac{1}{2}k^\mu \rightsquigarrow -\frac{1}{2}i\sigma^{\mu\nu}k_\nu
    \end{equation}
    also which follows as previously mentioned by Gordon's identity.
    
    Thus we have
    \begin{align}
        N^\mu &\rightsquigarrow -2[zm(2 - x - y)(-i\sigma^{\mu\nu}k_\nu) - 2zm(-i\sigma^{\mu\nu}k_\nu)]\\
        &= -2zm(x + y)(i\sigma^{\mu\nu}k_\nu)\\
        &\rightsquigarrow -2imz(1 - z)\sigma^{\mu\nu}\\
        &= -\frac{i\sigma^{\mu\nu}k_\nu}{2m}4m^2z(1 - z).
    \end{align}
    This is the end of step 3: simplify the numerator, in computing loop integrals, now we need to do steps 4 and 5: do the integral.
    We want to evaluate
    \begin{multline}
        \delta\Gamma^{\mu}(p', p) \rightsquigarrow \\
        -2ie^2 \int_{\mathrlap{[0,1]^3}} \  \dl{x} \dd{y} \dd{z} \, \delta(1 - x - y - z) \int \dhat{^4\ell} \frac{1}{(\ell^2 - \Delta)^3} \left( -\frac{i\sigma^{\mu\nu}k_\nu}{2m} 4m^2 z(1 - z) \right).
    \end{multline}
    The momentum integral is particularly simple since we can take \(D = 4\), and using our formula then gives
    \begin{equation}
        \int \dhat{^4\ell} \frac{1}{(\ell^2 - \Delta)^3} = \frac{-i}{32\pi^2} \Delta^{-1}.
    \end{equation}
    Putting in \(\Delta = (1 - z)^2m^2\) we see that the \(m^2\) cancels with one in the integral and one \(1 - z\) cancels from \((1 - z)^2\) giving
    \begin{equation}
        \delta\Gamma^\mu(p', p) \rightsquigarrow ie^2 \int_{\mathrlap{[0,1]^3}} \  \dl{x} \dd{y} \dd{z} \, \delta(1 - x - y - z) \frac{\sigma^{\mu\nu}k_\nu}{2m} \frac{z}{4\pi^2(1 - z)}.
    \end{equation}
    We can perform the \(y\) integral, which just gives a factor of 1, but we have to be careful because the Dirac delta adds the restriction that \(x\) and \(z\) must then be such that \(1 - x - z\) can vanish, so \(x < 1 - z\), giving
    \begin{equation}
        \delta\Gamma^\mu(p', p) \rightsquigarrow \frac{e^2}{4\pi^2} \frac{i\sigma^{\mu\nu}k_\nu}{2m} \int_0^1 \dl{z} \int_0^{1 - z} \dl{z} \, \frac{z}{1 - z}.
    \end{equation}
    Doing the \(z\) \(x\) integral gives a factor of \(1 - z\), so we get
    \begin{equation}
        \delta\Gamma^\mu(p', p) \rightsquigarrow \frac{e^2}{4\pi^2} \frac{i\sigma^{\mu\nu}k_\nu}{2m} \int_0^1 \dl{z}  \dl{z} z.
    \end{equation}
    Finally, we can do the \(z\) integral giving \(1/2\) for the final result
    \begin{equation}
        \delta\Gamma^\mu(p', p) \rightsquigarrow \frac{e^2}{8\pi^2} \frac{i\sigma^{\mu\nu}k_\nu}{2m}.
    \end{equation}
    
    Using this we see that
    \begin{equation}
        F_2(0) = \frac{e^2}{8\pi^2} = \frac{\alpha}{2\pi}.
    \end{equation}	
    This gives
    \begin{equation}
        g - 2F_2(0) = \frac{\alpha}{\pi}.
    \end{equation}
    The correction to \(g\) then is quite small,
    \begin{equation}
        \frac{\alpha}{\pi} \approx 0.002323.
    \end{equation}
    So we now have
    \begin{equation}
        g = 2.002323.
    \end{equation}
    Each extra term has an added factor of \(\alpha/\pi \sim 1/400\), so the next term is a correction on the order of \(1/400^2 = 1/160000\).
    This is born out in how close this result is to experimental and higher order theoretical results.
    
    Recent experimental results give
    \begin{equation}
        g = \num{2.00231930436146(56)}.
    \end{equation}
    Precision calculations up to fifth order give
    \begin{equation}
        g = \num{2.002319304363286(1528)}.
    \end{equation}
    These agree up to 12 significant figures
    At this level of precision the calculations need to include other effects, such as non-perturbative QCD from quark-antiquark pairs which can produced in loops.
    This is even more the case for the calculation of the muon magnetic moment.
    
    Now days this calculation is used to define \(\alpha\), and hence to define \(e\).
    This means that it's no longer that interesting to compare the two results, since there's some circular logic, but the level of agreement is a large success of QED.
    A similar calculation and measurement can be performed for the muon, and disagreement between these caused some excitement in 2021 when it was suggested that experiment and theory weren't in perfect agreement.
    
    
    \part{Quantum Chromodynamics}
    \chapter{Yang--Mills Theory}
    In this part of the course we will attempt to understand quantum chromodynamics (QCD)\glossary[acronym]{QCD}{Quantum Chromodynamics}.
    To do so we actually start with a more general theory, which encompasses QED, QCD, and electroweak theory.
    This theory was first developed by Yang Chen-Ning and Robert Mills in 1954.
    Their aim was to generalise QED, as a \(\unitary(1)\) gauge theory, to \(\specialUnitary(N)\).
    We will follow the same steps as they did to develop Yang--Mills theory, but we shall at stages be more general, since they originally focused on developing the theory for the proton and neutron.
    
    \section{Non-Interacting Yang--Mills Theory}
    Consider a theory with two non-interacting fermions of equal mass, \(m\).
    We'll assume these are the proton and neutron, which have approximately equal mass.
    The Lagrangian for this theory is simply the Dirac Lagrangian for each particle:
    \begin{equation}
        \lagrangianDensity = \diracadjoint{\Pn}(i\slashed{\partial} - m)\Pn + \Pp(i\slashed{\partial} - m)\Pp = \sum_{i=1}^{2} \Pn_i(i\slashed{\partial} - m)\Pn_i
    \end{equation}
    where \(\Pn_1 = \Pn\) and \(\Pn_2 = \Pp\).
    Here we are using the symbols for particles to represent their spinor wave functions.
    Note that \(\diracadjoint{\Pn} = \Pn^\hermit \gamma^0\) here is the Dirac adjoint of \(\Pn\), not the antineutron.
    
    If we package the neutron and proton up into a single object,
    \begin{equation}
        N = 
        \begin{pmatrix}
            \Pn\\ \Pp
        \end{pmatrix}
        ,
    \end{equation}
    so that
    \begin{equation}
        \lagrangianDensity = \diracadjoint{N}(i\slashed{\partial} - m)N,
    \end{equation}
    then it becomes clear that this theory is invariant under rotations of these fields into each other.
    By this we mean that if we have \(U \in \specialUnitary(2)\) then \(N\) and \(UN\) both result in the same Lagrangian.
    That is, we have a symmetry
    \begin{equation}
        \psi \mapsto U\psi.
    \end{equation}
    This symmetry also gives
    \begin{equation}
        \diracadjoint{\psi} = \psi^\hermit \gamma^0 \mapsto (U\psi)^\hermit \gamma^0 = \psi^\hermit U^\hermit \gamma^0 = \psi^\hermit \gamma^0 U^\hermit = \diracadjoint{\psi}U^\hermit = \diracadjoint{\psi}U^{-1}.
    \end{equation}
    Note that \(U\) and \(\gamma^0\) commute since they act on different spaces, \(U\) is acting on the vector of spinors \(N\), whereas \(\gamma^0\) is acting on each spinor individually.
    Also recall that \(U^\hermit = U^{-1}\) as part of the definition of \(\specialUnitary(2)\).
    
    Note that we use \(\specialUnitary(2)\) rather than \(\specialOrthogonal(2)\) to allow for complex \enquote{rotations}.
    We use \(\specialUnitary(2)\) instead of \(\unitary(2)\) since \(\unitary(2)\) can be decomposed into \(\specialUnitary(2)\) and \(\unitary(1)\), and \(\unitary(1)\) isn't interesting right now, its just more QED.
    
    We want to generalise this theory to the case of \(N\) fermions\footnote{here \(N\) is just some positive integer, not the combined neutron and proton.} of equal mass, \(m\), which will be non-interacting for the moment.
    The Lagrangian is then
    \begin{equation}
        \lagrangianDensity = \sum_{i=1}^{N} \diracadjoint{\Pq}_i(i\slashed{\partial} - m)\Pq_i = \diracadjoint{\psi}(i \slashed{\partial} - m)\psi
    \end{equation}
    where \(\Pq_i\) are the individual particle spinor wave functions, we call them \(\Pq\) as we will think of them as quarks later, and
    \begin{equation}
        \psi = 
        \begin{pmatrix}
            \Pq_1\\ \vdots\\ \Pq_N
        \end{pmatrix}
        .
    \end{equation}
    We will then consider \(\specialUnitary(N)\) transformations applied to \(\specialUnitary(N)\).
    
    \section{Gauging Yang--Mills Theory}
    Yang and Mills didn't like that the \(\specialUnitary(N)\) symmetry was global and sought to find a local \(\specialUnitary(N)\) theory, such that the Lagrangian was invariant under the action of some \(U(x) \in \specialUnitary(N)\), now depending on position.
    This poses a problem, due to the derivative term, since we have
    \begin{multline}
        \diracadjoint{\psi}(x)(i\slashed{\partial})\psi(x) \mapsto (\diracadjoint{\psi}U^{\hermit}(x))(i\slashed{\partial})(U(x)\psi(x))\\
        = \diracadjoint{\psi}U^{\hermit}(x)(iU(x)\slashed{\partial}\psi(x) + i(\slashed{\partial}U(x))\psi(x)) \ne \diracadjoint{\psi}(x)(i\slashed{\partial})\psi(x).
    \end{multline}
    The mass term is fine, since we have
    \begin{multline}
        m\diracadjoint{\psi}(x)\psi(x) \mapsto \diracadjoint{\psi}(x)U^{\hermit}(x)U(x)\psi(x)\\
        = \diracadjoint{\psi}(x)U^{-1}(x)U(x)\psi(x) = \diracadjoint{\psi}(x)\psi(x).
    \end{multline}
    
    Inspired by QED we introduce a covariant derivative, \(\covariantDerivative_\mu\), and hence a gauge field, \(A_\mu\),
    \begin{equation}
        \covariantDerivative_\mu \coloneqq \partial_\mu - igA_\mu.
    \end{equation}
    Here \(g\) is a coupling constant, generalising \(e\) in the QED case.
    Our goal will be to figure out the properties of the gauge field from symmetry.
    
    By definition we want our covariant derivative to transform covariantly, by this we mean that \(\covariantDerivative_\mu \psi\) should transform exactly as \(\psi\) does, so we demand that under this \(\specialUnitary(N)\) local symmetry
    \begin{equation}
        \covariantDerivative_\mu\psi(x) \mapsto \covariantDerivative_\mu'\psi'(x) = U(x)\covariantDerivative_\mu\psi(x).
    \end{equation}
    We know that \(\psi'(x) = U(x)\psi(x)\), so we just need to work out what we need for \(\covariantDerivative_\mu'\) to make this statement true.
    
    To find the transformation law for the derivative we can expand the definitions, assuming that \(D_\mu' = \partial_\mu - igA_\mu'\), that is the only part of the covariant derivative that transforms under a gauge transformation is the gauge field, which seems right.
    Then we have
    \begin{align}
        \covariantDerivative_\mu'\psi'(x) &= (\partial_\mu - igA_\mu'(x))U(x)\psi(x)\\
        &= (\partial_\mu U(x))\psi(x) + U(x)\partial_\mu\psi(x) - igA_\mu'(x)U(x)\psi(x).
    \end{align}
    We also have
    \begin{equation}
        \covariantDerivative_\mu'\psi'(x) = U(x)\covariantDerivative_\mu\psi(x) = U(x)(\partial_\mu - igA_\mu(x))\psi(x) = U(x)\partial_\mu - igU(x)A_\mu(x),
    \end{equation}
    note that we are being careful not to commute anything here, since we're not sure exactly what commutativity properties our new fields, \(\psi\) and \(A_\mu\), will have.
    We want these quantities to be equal for all \(\psi\), so we can strip off the factors of \(\psi\) and demand that
    \begin{equation}
        (\partial_\mu U(x)) + U(x)\partial_\mu - igA_\mu'(x)U(x) = U(x)\partial_\mu - igU(x)A_\mu(x).
    \end{equation}
    Cancelling the \(U(x)\partial_\mu\) terms we have
    \begin{equation}
        \partial_\mu U(x) - igA_\mu'(x)U(x) = -igU(x)A_\mu(x).
    \end{equation}
    Solving for \(A_\mu'(x)\) we get
    \begin{equation}
        A_\mu'(x)U(x) = U(x)A_\mu(x)U^{-1}(x) - \frac{i}{g}(\partial_\mu U(x))U^{-1}(x).
    \end{equation}
    
    Note that for QED we have \(g = 3\) and \(U(x) \in \unitary(1)\), so \(U(x) = \e^{i\alpha(x)}\), then we have \(\partial_\mu U(x) = i(\partial_\mu\alpha(x))\e^{i\alpha(x)}\) and \(U^{-1}(x) = \e^{-i\alpha(x)}\), and both of these commute with \(A_\mu(x)\) so we get
    \begin{equation}
        A_\mu'(x) = A_\mu(x) + \frac{1}{e}\partial_\mu \alpha(x),
    \end{equation}
    as expected.
    
    Recall that\footnote{see \course{Symmetries of Quantum Mechanics} or \course{Symmetries of Particles and Fields}} that given some Lie group, \(G\), with Lie algebra \(\lie{g}\) we can write \(g \in G\) as\footnote{Conventions vary here, often the factor of \(i\) is left out by mathematicians or people working with real groups, but we like it here as it makes \(t^a\) Hermitian for \(\specialUnitary(N)\).}
    \begin{equation}
        g = \e^{i\vartheta^at_a}
    \end{equation}
    where \(t_a \in \lie{g}\) are the \define{generators}\index{generator} of \(\lie{g}\) and there is a sum over \(a = 1, \dotsc, \dim \lie{g}\).
    
    We can write \(U(x) \in \specialUnitary(N)\) as\footnote{Note that it doesn't matter for \(\specialUnitary(N)\) if the indices are raised or lowered, since we can impose the Kronecker delta as the metric, see \course{Symmetries of Particles and Fields}.}
    \begin{equation}
        U(x) = \e^{i\vartheta^a(x)t^a}.
    \end{equation}
    For \(\specialUnitary(2)\) we have \(t^a = \sigma^a/2\) with \(a = 1, 2, 3\), and for \(\specialUnitary(3)\) we have \(t^a = \lambda^a/2\) with \(a = 1, \dotsc, 8\) where \(\lambda^a\) are the Gell-Mann matrices.
    For \(\specialUnitary(N)\) there are \(N^2 - 1\) generators, so \(a = 1, \dotsc, N^2 - 1\).
    
    Consider a small gauge transformation, by this we mean that \(\vartheta^a(x)\) is small, or rather that we can parametrise things such that \(U(x) = \exp\{i\varepsilon\vartheta^a(x)t^a\}\) and \(\varepsilon\) is small.
    Then we can expand to linear order in \(\varepsilon\), this means that terms like \(\varepsilon\vartheta^a(x)\partial_\mu \varepsilon\vartheta^b(x)\) are small, since they are quadratic in \(\varepsilon\).
    However, we already have so many \(\varepsilon\)s floating around that we absorb \(\varepsilon\) into the definition of \(\vartheta\) and simply remember that terms like \(\vartheta^a(x)\partial_\mu\vartheta^b(x)\) are of quadratic order and so are small.
    
    For a small gauge transformation we have
    \begin{align}
        U(x) &= \e^{i\vartheta^a(x)t^a} \approx 1 + i\vartheta^a(x)t^a,\\
        U^{-1}(x) &= \e^{-i\vartheta^a(x)t^a} \approx 1 - i\vartheta^a(x)t^a.
    \end{align}
    Note that \(\vartheta^a(x)\) is just a number so commutes with everything except position derivatives, whereas \(t^a \in \specialUnitaryLie(N)\) so we cannot assume it commutes with anything except position derivatives, since \(t^a\) is constant in spacetime.
    For these small gauge transformations the gauge field transforms as
    \begin{align}
        A_\mu'(x) &= (1 + i\vartheta^a(x)t^a)A_\mu(x)(1 - i\vartheta^b(x)t^b) \notag\\
        &\qquad\qquad - \frac{i}{g}(\partial_\mu(1 + i\vartheta^a(x)t^a))(1 - i\vartheta^b(x)t^b)\\
        &= A_\mu(x) + i\vartheta^a(x)t^aA_\mu(x) - i\underbrace{\vartheta^b(x)A_\mu(x)t^b}_{b \to a} + \underbrace{\vartheta^a(x)\vartheta^b(x)}_{\to 0}t^at^b \notag\\
        &\qquad\qquad - \frac{i}{g}it^a(\partial_\mu \vartheta^a(x))(1 - i\vartheta^b(x)t^b)\\
        &= A_\mu(x) + i\vartheta^a(x)(t^a A_\mu(x) - A_\mu(x)t^a)\\
        &\qquad\qquad + \frac{1}{g}(\partial_\mu\vartheta^a(x))t^a - \frac{1}{g} \underbrace{(\partial_\mu\vartheta^a(x))t^a\vartheta^b(x)}_{\to 0} t^b\\
        &= A_\mu(x) + i\vartheta^a(x)\commutator{t^a}{A_\mu(x)} + \frac{1}{g}(\partial_\mu \vartheta^a(x))t^a.
    \end{align}
    
    From this we see that \(A_\mu(x)\) is a linear combination of the generators, \(t^a\).
    This means that \(A_\mu(x)\) is Lie algebra valued, and so does not, in general, commute with \(t^a\).
    In QED we happen to have an Abelian gauge group and so \(\commutator{t^a}{A_\mu(x)} = 0\), and in fact there is only one generator, \(t = 1\), so we have
    \begin{equation}
        A'_\mu(x) = A_\mu(x) + \frac{1}{e}\partial_\mu\alpha(x),
    \end{equation}
    again recovering the expected transformation law.
    
    \section{Gauge Field Lagrangian}
    In order to fully specify Yang--Mills theory we also need a Lagrangian for the gauge fields.
    We look to generalise
    \begin{equation}
        \lagrangianDensity = -\frac{1}{4}F^{\mu\nu}F_{\mu\nu}
    \end{equation}
    from QED.
    
    To do so we need to write \(F^{\mu\nu}\) in such a way that its transformation is understood with the Yang--Mills theory we have developed so far.
    The QED option of \(F^{\mu\nu} = \partial^\mu A^\nu - \partial^\nu A^\mu\) is not appropriate here, since it isn't gauge invariant and the transformation of \(A_\mu\) is not simple.
    Instead, we use the fact that in QED we have
    \begin{equation}
        \commutator{\covariantDerivative_\mu}{\covariantDerivative_\nu} = -ieF_{\mu\nu}
    \end{equation}
    to generalise our definition of \(F_{\mu\nu}\).
    
    Define \(F_{\mu\nu}\) in Yang--Mills theory to be such that
    \begin{equation}
        \commutator{\covariantDerivative_\mu}{\covariantDerivative_\nu}\psi(x) = -igF_{\mu\nu}\psi(x)
    \end{equation}
    where \(\psi\) is a spinor valued test function to aid in computing the commutator, and we assume that \(\psi\) has everywhere continuous second partial derivatives, so that \(\partial_\mu \partial_\nu \psi = \partial_\nu \partial_\mu \psi\).
    Expanding this all out we get
    \begin{align}
        \commutator{\covariantDerivative_\mu}{\covariantDerivative_\nu} &= \commutator{\partial_\mu - igA_\mu}{\partial_\nu - igA_\nu}\psi\\
        &= \underbrace{\commutator{\partial_\mu}{\partial_\nu}\psi}_{= 0} - ig\commutator{\partial_\mu}{A_\nu}\psi - ig\commutator{A_\mu}{\partial_\nu}\psi + (ig)^2\commutator{A_\mu}{A_\nu}\psi\\
        &= -ig(\partial_\mu(A_\nu \psi) - A_\nu\partial_\mu\psi) \notag\\
        &\qquad\qquad - ig(A_\mu\partial_\nu\psi - \partial_\nu(A_\mu\psi)) + (ig)^2\commutator{A_\mu}{A_\nu}\psi\\
        &= -ig[(\partial_\mu A_\nu)\psi + A_\nu\partial_\mu\psi - A_\nu\partial_\mu\psi] \notag\\
        &\qquad\qquad- ig[A_\mu\partial_\nu\psi - (\partial_\nu A_\mu)\psi - A_\mu\partial_\nu\psi] + (ig)^2\commutator{A_\mu}{A_\nu}\\
        &= -ig(\partial_\mu A_\nu)\psi + ig(\partial_\nu A_\mu)\psi + (ig)^2\commutator{A_\mu}{A_\nu}\\
        &= -ig[\partial_\mu A_\nu - \partial_\nu A_\mu - ig\commutator{A_\mu}{A_\nu}]\psi.
    \end{align}
    So, we can define \(F_{\mu\nu}\) according to
    \begin{equation}
        F_{\mu\nu}(x) \coloneqq \partial_\mu A_\nu(x) - \partial_\nu A_\mu(x) - ig\commutator{A_\mu(x)}{A_\nu(x)}.
    \end{equation}
    
    Notice that in QED we have \(\commutator{A_\mu}{A_\nu} = 0\) so this reduces to the usual \(F_{\mu\nu} = \partial_\mu A_\nu - \partial_\nu A_\mu\).
    Notice also that \(F_{\mu\nu}\) is still antisymmetric, since the commutator is antisymmetric, \(\commutator{x}{y} = -\commutator{y}{x}\).
    
    Since \(A_\mu\) is Lie algebra valued we can define \(A^a_\mu\) to be such that
    \begin{equation}
        A_\mu(x) = A^a_\mu(x)t^a.
    \end{equation}
    Then we have
    \begin{equation}
        \commutator{A_\mu}{A_\nu} = \commutator{A^a_\mu t^a}{A^b_\nu t^b} = A^a_\mu A^b_\nu \commutator{t^a}{t^b} = iA^a_\mu A^b_\nu f^{cba}t^c
    \end{equation}
    where \(f^{abc}\) are the \define{structure constants}\index{structure constant} defined by\footnote{conventions vary here, commonly mathematicians, and those working with real groups, leave out the factor of \(i\), but it is due to our inclusion of \(i\) in \(U = \exp\{i\vartheta^at^a\}\), which we like as it makes \(t^a\) Hermitian for \(\specialUnitary(N)\). Conventions also vary over which of the indices of the structure constant is summed over in the definition.}
    \begin{equation}
        \commutator{t^a}{t^b} = if^{cba}t^c.
    \end{equation}
    Similarly, we can define \(F^a_{\mu\nu}\) to be such that
    \begin{equation}
        F_{\mu\nu} = F^a_{\mu\nu}t^a.
    \end{equation}
    Using this we have
    \begin{equation}
        F^a_{\mu\nu}(x) = \partial_\mu A^a_\nu(x) - \partial_\nu A^a_\mu(x) + gf^{abc}A^b_\mu A^c_\nu.
    \end{equation}
    
    
    We can now explore how \(F_{\mu\nu}\) transforms under a gauge transformation by studying the gauge transformation
    \begin{equation}
        F_{\mu\nu}\psi \mapsto F'_{\mu\nu}\psi'.
    \end{equation}
    Covariance of the covariant derivative means that
    \begin{equation}
        \commutator{\covariantDerivative_\mu}{\covariantDerivative_\nu}\psi \mapsto U(x)\commutator{\covariantDerivative_\mu}{\covariantDerivative_\nu}\psi,
    \end{equation}
    and so we have
    \begin{equation}
        F'_{\mu\nu}\psi' = -ig\commutator{\covariantDerivative_\mu'}{\covariantDerivative_\nu'}\psi' = -igU(x)\commutator{\covariantDerivative_\mu}{\covariantDerivative_\nu}\psi = U(x)F_{\mu\nu}\psi.
    \end{equation}
    On the other hand, we have \(\psi(x) \mapsto U(x)\psi(x)\), so
    \begin{equation}
        F'_{\mu\nu}\psi' = F'_{\mu\nu}U(x)\psi.
    \end{equation}
    Comparing these two results we can see that
    \begin{equation}
        U(x)F_{\mu\nu} = F'_{\mu\nu}U(x) \implies F'_{\mu\nu} = U(x)F_{\mu\nu}U^{-1}(x).
    \end{equation}
    In group theory terms we see that \(F\) is a rank two tensor, since it transforms as \(T \mapsto UTU^{-1}\), or \(T'_{ij} = U_{ik}U_{lj}^*T_{kl}\).
    Fundamentally, this is because \(A_\mu \in \specialUnitaryLie(2)\), so \(A_\mu\), and hence \(F_{\mu\nu}\), transform in the adjoint representation of \(\specialUnitaryLie(2)\).
    
    We are now in a position to write down a Lagrangian for the gauge field.
    We want this Lagrangian to be invariant under both Lorentz transformations and gauge transformations.
    The Lorentz invariance is taken care of by contracting the Lorentz indices.
    The gauge invariance can be taken care of using the trace, since we have
    \begin{multline}
        \tr(F_{\mu\nu}F^{\mu\nu}) \mapsto \tr(F'_{\mu\nu}F'^{\mu\nu}) = \tr(UF_{\mu\nu}U^{-1}UF^{\mu\nu}U)\\
        = \tr(UF_{\mu\nu}F^{\mu\nu}U^{-1}) = \tr(F_{\mu\nu}F^{\mu\nu}U^{-1}U) = \tr(F_{\mu\nu}F^{\mu\nu}),
    \end{multline}
    using the cyclic property of the trace.
    Note that the trace here is taken over \(\specialUnitaryLie(N)\), since \(F_{\mu\nu}\) is Lie algebra valued.
    We can define \(F^a_{\mu\nu}\) to be such that
    \begin{equation}
        F_{\mu\nu} = F^a_{\mu\nu}t^a,
    \end{equation}
    then we have
    \begin{multline}
        \tr(F_{\mu\nu}F^{\mu\nu}) = \tr(F^a_{\mu\nu}t^aF^{b\mu\nu}t^b) = F^a_{\mu\nu}F^{b\mu\nu}\tr(t^at^b)\\
        = F^a_{\mu\nu}F^{b\mu\nu}\frac{1}{2}\delta^{ab} = F^a_{\mu\nu}F^{a\mu\nu}.
    \end{multline}
    Here we have used
    \begin{equation}
        \tr(t^at^b) = \frac{1}{2}\delta^{ab},
    \end{equation}
    which is the Killing form of \(\specialUnitaryLie(N)\).
    
    Putting this together we define the Lagrangian
    \begin{equation}
        \lagrangianDensity = -\frac{1}{4}F^a_{\mu\nu}F^{a\mu\nu} = -\frac{1}{2}\tr(F_{\mu\nu}F^{\mu\nu}).
    \end{equation}
    We can then define the full Lagrangian for Yang--Mills theory:
    \begin{equation}
        \lagrangianDensity = -\frac{1}{2}\tr(F_{\mu\nu}F^{\mu\nu}) + \diracadjoint{\psi}(i\slashed{\covariantDerivative} - m)\psi.
    \end{equation}
    
    Studying this for a while we see that it generates all the same terms as QED, such as \(m\diracadjoint{\psi}\psi\), \((\partial_\mu A_\nu)(\partial^\nu A^\mu)\), and \(\diracadjoint{\psi}A_\mu\psi\), but it also generates new terms, such as \(\partial_\mu(A_\nu^a)f^{abc}A_\mu^bA_\nu^c\), which corresponds to a three-point interaction between gauge bosons, and \((A_\mu^b A_\nu^c)(A^{b\mu}A^{c\nu})\), which corresponds to a four-point interaction between gauge bosons.
    Interactions between gauge bosons is not something we see in QED, which is fundamentally due to \(\unitary(1)\) being Abelian and so these terms cancel out.
    
    
    \chapter{Group Theory}
    For more details see \course{Symmetries of Quantum Mechanics} and \course{Symmetries of Particles and Fields}.
    
    \section{Basics of \texorpdfstring{\(\specialUnitary(N)\)}{SU(N)}}
    Yang--Mills theory is defined by the existence of a gauge symmetry.
    We will restrict ourselves to \(\specialUnitary(N)\) gauge symmetries, since this is the case for QCD, which is the \(\specialUnitary(3)\) case.
    
    Our first definition is \define{\(\specialUnitary(N)\)}\index{SU(N)@\(\specialUnitary(N)\)} itself, this is the group of \(N \times N\) unitary matrices with unit determinant.
    Strictly this is only one representation of \(\specialUnitary(N)\), called the defining or fundamental representation.
    Any \(U \in \specialUnitary(N)\) can be written as\footnote{mathematicians follow a different convention and write \(U = \e^{\vartheta^a t^a}\)} \(U = \e^{i\vartheta^a t^a}\) for \(\vartheta^a \in \complex\) where \(t^a\) are the \define{generators}\index{generator} of \define{\(\specialUnitaryLie(N)\)}\index{su(N)@\(\specialUnitaryLie(N)\)}, the Lie algebra of \(\specialUnitary(N)\).
    In the defining representation \(t^a\) are \(N \times N\) matrices over \(\complex\).
    The conditions upon \(\specialUnitary(N)\) enforce conditions at the level of the Lie algebra:
    \begin{itemize}
        \item Unitarity, \(U^\hermit = U^{-1}\) forces the generators to be Hermitian\footnote{in the mathematics convention the generators are instead anti-Hermitian}, \((t^a)^\hermit = t^a\), to see this expand \((\e^{i\vartheta^a t^a})^\hermit = \e^{-i\vartheta^a t^{a\hermit}} = \e^{-i\vartheta^a t^a} = (\e^{-i\vartheta^a t^a})^{-1}\) to linear order.
        \item Unit determinant, \(\det U = 1\), enforces \(\tr t^a = 0\), through the identity \(\det(\e^A) = \e^{\tr A}\).
    \end{itemize}
    
    There are \(2N^2\) linearly independent complex \(N \times N\) matrices, since there are \(N^2\) entries each with two degrees of freedom.
    Imposing Hermiticity restricts the entries on the diagonal to be real, removing \(N\) degrees of freedom, and fixes the entries below the diagonal to be complex conjugates of those above the diagonal, fixing \(N(N - 1)/2\) complex values, so removing \(N(N - 1)\) degrees of freedom.
    We are therefore left with \(2N^2 - N - N(N - 1) = N^2\) degrees of freedom.
    So there are \(N^2\) linearly independent Hermitian \(N \times N\) matrices.
    Imposing tracelessness fixes one more degree of freedom, since given all but the first element in the diagonal we can deduce this value by requiring that the diagonal sums to zero, and recalling that all values on the diagonal are real for a Hermitian matrix.
    Thus there are \(N^2 - 1\) generators of \(\specialUnitaryLie(N)\).
    This means that \(\dim \specialUnitary(N) = N^2 - 1\) as a manifold also.
    
    The generators of \(\specialUnitaryLie(2)\) are the Pauli matrices, of which there are \(2^2 - 1 = 3\).
    The generators of \(\specialUnitaryLie(3)\) are the Gell-Mann matrices, of which there are \(3^2 - 1 = 8\).
    The standard choice in both cases is actually to include an extra factor of \(1/2\), for reasons we'll see later.
    
    \section{Structure Constants}
    The \define{structure constants}\index{structure constant}, \(f^{abc}\), are defined by\footnote{the \(i\) here is only present in the physics convention of \(U = \e^{i\vartheta^a t^a}\).}
    \begin{equation}
        \commutator{t^a}{t^b} = if^{abc}t^c.
    \end{equation}
    Here \(\commutator{-}{-}\) is the commutator, or more generally a Lie bracket.
    It is simple to show that the following holds for commutators, and it holds by definition for a Lie bracket:
    \begin{equation}
        \commutator{\commutator{t^a}{t^b}}{t^c} + \commutator{\commutator{t^b}{t^c}}{t^a} + \commutator{\commutator{t^c}{t^a}}{t^b} = 0.
    \end{equation}
    From this it is possible to show that
    \begin{equation}
        f^{abd}f^{dce} + f^{bcd}f^{dae} + f^{cad}f^{dbe} = 0.
    \end{equation}
    Both of these results are known as the \defineindex{Jacobi identity}.
    
    \section{Other Representations}
    So far we have thought of \(\specialUnitary(N)\) as acting on \(\complex^N\) through the group action
    \begin{equation}
        U \action v = Uv
    \end{equation}
    for \(U \in \specialUnitary(N)\) in the fundamental representation and \(v \in \complex^N\).
    
    Other representations act on different representation spaces.
    We'll often work with an arbitrary representation, \(R\).
    In this representation we'll denote the generators by \(t_R^a\).
    Note that the structure constants are representation independent, since the Lie bracket is representation independent as part of the definition of a Lie group.
    So we have
    \begin{equation}
        \commutator{t_R^a}{t_R^b} = if^{abc}t_R^c
    \end{equation}
    in any representation.
    It is common to label a representation by its dimension, so the fundamental representation is \(N\).
    
    One particularly important representation is the \defineindex{adjoint representation} in which \(\specialUnitaryLie(N)\) acts on itself by taking the generators, \(t_{\adj}^b\), to have components
    \begin{equation}
        (t_{\adj}^b)^{ac} = if^{abc}.
    \end{equation}
    
    In an arbitrary representation, \(R\), the group action on some vector \(v\) is
    \begin{equation}
        g \action v = \e^{i\vartheta^at_R^a}v = (1 + i\vartheta^at_R^a)v
    \end{equation}
    for infinitesimal \(\vartheta^a\).
    We can define the conjugate representation, \(\overline{R}\), through the action
    \begin{equation}
        g \action v^* = (1 - i\vartheta^a(t_R^a)^*)v^*,
    \end{equation}
    so \(t_{\overline{R}}^a = -(t_R^a)^*\).
    
    Two relations are \defineindex{equivalent} if they are related by a change of basis of the representation space.
    That is, two representation, \(R_1\) and \(R_2\), are equivalent if
    \begin{equation}
        t_{R_1}^a = Xt_{R_2}^aX^\hermit
    \end{equation}
    for some fixed matrix \(X\).
    A representation which is equivalent to its conjugate is called a \defineindex{real representation}.
    The adjoint representation is always real.
    So is the fundamental representation of \(\specialUnitary(2)\), but the fundamental representation of \(\specialUnitary(3)\) is not.
    
    \section{Casimirs}
    In the fundamental representation, writing \(t^a\) for the generators, we are free to choose the normalisation of the generators such that
    \begin{equation}
        \tr(t^at^b) = \frac{1}{2}\delta^{ab}.
    \end{equation}
    In a general representation, we can define the trace, \(T(R)\), through
    \begin{equation}
        \tr(t_R^at_R^b) = T(R)\delta^{ab}.
    \end{equation}
    We cannot in general take \(T(R) = 1/2\), since this is in conflict with \(\commutator{t_R^a}{t_R^b} = if^{abc}t_R^c\), where the normalisation of the structure constants is fixed by taking \(T(N) = 1/2\).
    
    Now consider the operator \(t_R^a t_R^a\).
    This commutes with all of the generators, which we can show by commuting each \(t_R^a\) with the generator individually, picking up a commutator term, and then using the structure constants:
    \begin{align}
        t_R^at_R^at_R^b &= t_R^at_R^bt_R^a + t_R^a\commutator{t_R^a}{t_R^b} = t_R^bt_R^at_R^a + \commutator{t_R^a}{t_R^b}t_R^a + t_R^a\commutator{t_R^a}{t_R^b}\\
        &= t_R^bt_R^at_R^a + if^{abc}t_R^ct_R^a + if^{abc}t_R^at_R^c = t_R^bt_R^at_R^a + if^{abc}(t_R^ct_R^a + t_R^at_R^c) = t_R^bt_R^at_R^a \notag
    \end{align}
    having used the fact that \(f^{abc}\) is totally antisymmetric and \((t_R^ct_R^a + t_R^at_R^c)\) is symmetric in \(a\) and \(c\), so their product vanishes.
    
    In group theory language we say that \(t_R^at_R^a\) is a \defineindex{Casimir operator}, it commutes with all generators, and hence all elements of \(\specialUnitaryLie(N)\).
    More specifically, \(t_R^at_R^a\) is the \defineindex{quadratic Casimir}, since it is quadratic in the generators, but it is common to refer to it simply as \emph{the} Casimir, since we aren't interested in any of the other Casimirs, apart from trivial things like scalars which commute with everything.
    A basic result of Lie theory, Schur's lemma, says that all Casimir operators are simply a (representation dependent) scalar times the identity, so we have
    \begin{equation}
        (t_R^a)_{ik}(t_R^a)_{kj} = C_2(R)\delta_{ij} \iff t_R^at_R^a = C_2(R)1
    \end{equation}
    where the \(i\) and \(j\) indices run from \(1\) to \(d(R)\), where \(d(R)\) is the dimension of representation \(R\).
    Note that \(a\) runs from \(1\) to \(N^2 - 1\), regardless of the representation.
    Taking traces we have
    \begin{equation}
        \tr_R(t_R^at_R^a) = C_2(R)\delta_{ii} = C_2(R)d(R) = T(R)\delta^{aa} = T(R)\dim G.
    \end{equation}
    Using \(T(N) = 1/2\) we have
    \begin{equation}
        C_2(N)N = \frac{1}{2}(N^2 - 1) \implies C_2(N) = \frac{1}{2N}(N^2 - 1).
    \end{equation}
    
    We denote the adjoint representation by \(\adjointRep\), and it can be shown that \(T(\adjointRep) = C_2(\adjointRep) = N\).
    
    \section{Representations and Yang--Mills Theory}
    Recall that the gauge transformation of \(A_\mu\) is
    \begin{equation}
        A'^a_\mu = A^a_\mu + i\vartheta^a \commutator{t^a}{t^b}A^b_\mu + \frac{1}{g}(\partial_\mu \vartheta^a)t^a.
    \end{equation}
    If we consider a global transformation, where \(\vartheta^a\) is constant, then the derivative vanishes and we are left with
    \begin{align}
        A'^a_\mu &= A^a_\mu + i\vartheta^a \commutator{t^a}{t^b}A^b_\mu\\
        &= A^a_\mu - f^{abc}\vartheta^b A^c_\mu\\
        &= (\delta^{ca} - f^{abc}\vartheta^b)A^c_\mu.
    \end{align}
    This is the condition for the gauge fields to transform in the adjoint representation for global transformations.
    In turn this means that the gauge fields are elements of the Lie algebra, \(\specialUnitaryLie(N)\), or \enquote{Lie algebra valued}.
    This is quite hand wavy and will annoy mathematicians, who will insist on talking about fibre bundles here.
    
    Under a global transformation we get an extra term from the derivative called an \defineindex{affine transformation}.
    This extra term cancels in \(F^{a\mu\nu}\), so \(F^{a\mu\nu}\) is always in the adjoint representation for both global and local transformations.
    
    Recall that for matter we have the Lagrangian
    \begin{equation}
        \lagrangianDensity = \diracadjoint{\psi} (i\slashed{\covariantDerivative} - m) \psi
    \end{equation}
    where now \(\psi\) and \(\diracadjoint{\psi}\) transform under some representation, \(R\), of \(\specialUnitary(N)\).
    Note that
    \begin{equation}
        \covariantDerivative_\mu = \partial_\mu - igA^a_\mu t_R^a
    \end{equation}
    is representation dependent.
    Note that here \(t_R^a\) are operators on spinor space.
    If we introduce spinor indices, \(i\) and \(j\), then we have
    \begin{equation}
        (\covariantDerivative_\mu\psi)_i = (\delta_{ij}\partial_\mu - igA^a_\mu(t_R^a)_{ij})\psi_j.
    \end{equation}
    
    Including the gauge fields we get the Lagrangian
    \begin{equation}
        \lagrangianDensity = -\frac{1}{2}\tr(F^{\mu\nu}F_{\mu\nu}) + \diracadjoint{\psi}(i \slashed{\covariantDerivative} - m) \psi = -\frac{1}{4}F^a_{\mu\nu}F^{a\mu\nu} + \diracadjoint{\psi}(i\slashed{\covariantDerivative} - m)\psi.
    \end{equation}
    
    We can also add in scalar matter, again transforming under an arbitrary, potentially different, representation.
    We arrange these scalars into a column matrix, \(\varphi\), of size \(d(R)\).
    The most general renormalisable Lagrangian including a single fermion and and a single scalar, without interactions between them, is
    \begin{equation}
        \lagrangianDensity = -\frac{1}{2}\tr(F^{\mu\nu}F_{\mu\nu}) + \diracadjoint{\psi}(i \slashed{\covariantDerivative} - m) \psi + (\covariantDerivative_\mu \varphi)^\hermit (\covariantDerivative^\mu \varphi) - M^2 \varphi^\hermit \varphi + \frac{\lambda}{4}(\varphi^\hermit \varphi)^2.
    \end{equation}
    
    A Yang--Mills theory is completely specified by the following information:
    \begin{itemize}
        \item the gauge group, \(G\);
        \item the matter fields, \(\psi\) and/or \(\varphi\);
        \item the interactions of the matter fields, terms like \(\lambda(\varphi^\hermit \varphi)^2/4\) or \(g\varphi\diracadjoint{\psi}\psi\);
        \item the masses of the particles;
    \end{itemize}
    
    QCD is a Yang--Mills theory with the gauge group \(G = \specialUnitary(3)\).
    There are no scalar fields and six fermion fields, the quarks, \(\Pq_{f}\) with \(f = 1, \dotsc, 6\), which we take to all be the same mass.
    The Lagrangian is then
    \begin{equation}
        \lagrangianDensity_{\symrm{QCD}} = -\frac{1}{4}F^a_{\mu\nu}F^{a\mu\nu} + \sum_{f=1}^{6} \diracadjoint{\Pq}_f(i\slashed{\covariantDerivative} - m) \Pq_f.
    \end{equation}
    
    \section{Feynman Rules for Yang--Mills Theory}
    Given how similar the Yang--Mills Lagrangian is to the QED Lagrangian it should not be too surprising that the Feynman rules for Yang--Mills theory are similar to the Feynman rules for QED.
    We won't derive most of them here, the process is very similar to the process for deriving the QED Feynman rules.
    We'll state the rules, and derive one example.
    
    First, we have the fermion propagator, this is exactly the same as in QED, although now the mass is a quark mass:
    \begin{equation}
        \tikzsetnextfilename{yang-mills-feynman-rules-fermion-propagator}
        \begin{tikzpicture}[baseline=(current bounding box), font=\scriptsize]
            \draw[electron] (0, 0) -- (1, 0);
            \draw[->] (0.2, -0.2) -- (0.8, -0.2) node [midway, below, yshift=0.05cm] {\(p\)};
        \end{tikzpicture}
        = \frac{i(\slashed{p} + m)}{p^2 - m^2 + i\epsilon}.
    \end{equation}
    
    Next, we have the gluon propagator.
    This is very similar to the photon propagator, but we have these extra \(a\) indices hanging about.
    Since the gluon doesn't change colour when propagating the colour at either end of a gluon propagator must be the same, so we just have an extra \(\delta^{ab}\) to include:
    \begin{equation}
        \tikzsetnextfilename{yang-mills-feynman-rules-gluon-propagator}
        \begin{tikzpicture}[baseline=(current bounding box), font=\scriptsize]
            \draw[gluon] (0, 0) node [left] {\(\mu, a\)} -- (1, 0) node [right] {\(\nu, b\)};
            \draw[->] (0.2, -0.2) -- (0.8, -0.2) node [midway, below, yshift=0.05cm] {\(k\)};
        \end{tikzpicture}
        = -\frac{i\delta^{ab}}{k^2 + i\epsilon}\left( \minkowskiMetric^{\mu\nu} - (1 - \xi) \frac{k^\mu k^\nu}{k^2} \right).
    \end{equation}
    Here, as in QED, \(\xi\) is a gauge parameter, which we'll often take to be 1.
    
    We'll see at the end of the next chapter that in the gauge fixing process we are forced to introduce another particle, and that these are massless scalars with a colour index, \(a\), which we call Faddeev--Popov ghosts.
    They enter the Lagrangian through the term \(-\diracadjoint{c}^a\partial_\mu\covariantDerivative^\mu c^a\).
    The propagator for these ghosts, coming from the \(-\diracadjoint{c}^a\partial_\mu\partial^\mu c^a\) term, is very similar to the usual scalar propagator, but with an extra \(\delta^{ab}\), since the ghosts don't change colour as they propagate.
    We then have the Feynman rule
    \begin{equation}
        \tikzsetnextfilename{yang-mills-feynman-rules-ghost-propagator}
        \begin{tikzpicture}[baseline=(current bounding box), font=\scriptsize]
            \draw[gluon] (0, 0) node [left] {\(a\)} -- (1, 0) node [right] {\(b\)};
            \draw[->] (0.2, -0.2) -- (0.8, -0.2) node [midway, below, yshift=0.05cm] {\(k\)};
        \end{tikzpicture}
        = \frac{i\delta^{ab}}{k^2 + i\epsilon}.
    \end{equation}
    
    Now we get to the interactions.
    The most familiar is the three point quark and gluon vertex, which is very similar to the QED vertex, and indeed differs only by a colour factor, \(t_R^a\):
    \begin{equation}
        \tikzsetnextfilename{yang-mills-feynman-rules-quark-gluon-vertex}
        \begin{tikzpicture}[baseline=(current bounding box), font=\scriptsize]
            \draw[gluon] (-1, 0) node [left] {\(\mu, a\)} -- (0, 0);
            \draw[electron=0.6] (0, 0) -- (40:1);
            \draw[positron=0.4] (0, 0) -- (-40:1);
        \end{tikzpicture}
        = ig\mu^\varepsilon \gamma^\mu t_R^a.
    \end{equation}
    Note that this depends on the representation under which our fermions transform.
    The generators included through this vertex usually reduce to \(T(R)\) and \(C(R)\) or \(T(\adjointRep)\) and \(C(\adjointRep)\).
    These are known as \define{colour factors}\index{colour factor}.
    
    As well as this familiar interaction we also have an interaction coming from the cubic and quartic terms in \(A^a_\mu\) contained in \(F^a_{\mu\nu}F^{a\mu\nu}\).
    These give three-point and four-point interactions between gluons, for which the Feynman rules are
    \begin{equation}
        \tikzsetnextfilename{yang-mills-feynman-rules-gluon-3-point-vertex}
        \begin{tikzpicture}[baseline=(current bounding box), font=\scriptsize]
            \draw[gluon] (0, 0) -- (90:1) node [above] {\(\mu, a\)};
            \draw[gluon] (0, 0) -- (330:1) node [right] {\(\nu, b\)};
            \draw[gluon] (0, 0) -- (210:1) node [left] {\(\rho, c\)};
            \draw[->, shift={(0.15, 0.15)}] (330:0.8) -- (330:0.2) node [midway, above right] {\(p_2\)};
            \draw[->, shift={(0.15, -0.15)}] (210:0.8) -- (210:0.2) node [midway, below right] {\(p_3\)};
            \draw[->, xshift=-0.21cm] (90:0.8) -- (90:0.2) node [midway, left] {\(p_1\)};
        \end{tikzpicture}
        =
        \begin{aligned}[t]
            gf^{abc}[&(p_{1\rho} - p_{2\rho})\minkowskiMetric_{\mu\nu}\\
            &+ (p_{2\mu} - p_{3\mu})\minkowskiMetric_{\nu\rho}\\
            &+ (p_{3\nu} - p_{1\nu})\minkowskiMetric_{\rho\mu}],
        \end{aligned}
    \end{equation}
    and
    \begin{equation}
        \tikzsetnextfilename{yang-mills-feynman-rules-gluon-4-point-vertex}
        \begin{tikzpicture}[baseline=(current bounding box), font=\scriptsize]
            \draw[gluon] (0, 0) -- (135:1) node [above] {\(\mu, a\)};
            \draw[gluon] (0, 0) -- (45:1) node [right] {\(\nu, b\)};
            \draw[gluon] (0, 0) -- (225:1) node [left] {\(\rho, c\)};
            \draw[gluon] (0, 0) -- (315:1) node [left] {\(\sigma, c\)};
        \end{tikzpicture}
        =
        \begin{aligned}[t]
            -ig^2[&f^{abe}f^{cde}(\minkowskiMetric_{\mu\rho}\minkowskiMetric_{\nu\sigma} - \minkowskiMetric_{\mu\sigma}\minkowskiMetric_{\nu\rho})\\
            &+ f^{ace}f^{dbe}(\minkowskiMetric_{\mu\sigma}\minkowskiMetric_{\rho\nu} - \minkowskiMetric_{\mu\nu}\minkowskiMetric_{\rho\sigma})\\
            &+ f^{ade}f^{bce}(\minkowskiMetric_{\mu\nu}\minkowskiMetric_{\sigma\rho} - \minkowskiMetric_{\mu\rho}\minkowskiMetric_{\sigma\nu})].
        \end{aligned}
    \end{equation}
    
    The final interaction is the interaction of gluons with the ghosts.
    This arises from the \(-\diracadjoint{c}^a\partial_\mu (igA^{b\mu}t_R^ac^a)\) term, which is very similar to the \(\varphi A^\mu \varphi\) term appearing in scalar QED\footnote{see tutorials} and the Feynman rule for this term is
    \begin{equation}
        \tikzsetnextfilename{yant-mills-feynman-rules-ghost-interaction}
        \begin{tikzpicture}[baseline=(current bounding box), font=\scriptsize]
            \draw[gluon] (-1, 0) node [left] {\(\mu, b\)} -- (0, 0);
            \draw[ghost=0.6] (0, 0) -- (40:1) node [right] {\(a\)};
            \draw[antighost=0.4] (0, 0) -- (-40:1) node [right] {\(c\)};
            \draw[->, shift={(130:0.2)}] (40:0.2) -- (40:0.8) node [midway, above left] {\(p'\)};
            \draw[<-, shift={(-130:0.2)}] (-40:0.2) -- (-40:0.8) node [midway, below left] {\(p\)};
        \end{tikzpicture}
        = -gf^{abc}p'_\mu.
    \end{equation}
    Note that it is the momentum of the outgoing ghost that appears on the right hand side.
    
    \subsection{Derivation of the Three Gluon Interaction}
    Consider the \(F^a_{\mu\nu}F^{a\mu\nu}\) term in the Lagrangian.
    Expanding this we have
    \begin{equation*}
        -\frac{1}{4}F^a_{\mu\nu}F^{a\mu\nu} = -\frac{1}{4}(\partial_\mu A^a_\nu - \partial_\nu A^a_\mu + gf^{abc}A^b_\mu A^c_\nu)(\partial^\mu A^{a\nu} - \partial^\mu A^{a\mu} + gf^{ade}A^{d\mu}A^{e\nu}).
    \end{equation*}
    Partially expanding this one of the terms we get is
    \begin{equation}
        -\frac{1}{4}(\partial_\mu A^a_\nu - \partial_\nu A^a_\mu)gf^{ade}A^{d\mu}A^{e\nu}.
    \end{equation}
    Notice that \(f^{ade}A^{d\mu}A^{e\nu}\) is antisymmetric in \(\mu\) and \(\nu\), since we can swap \(\mu\) and \(\nu\) to get
    \begin{equation}
        f^{ade}A^{d\nu}A^{e\mu} = f^{ade}A^{e\mu}A^{d\nu} = -f^{aed}A^{e\mu}A^{d\nu} = -f^{ade}A^{d\mu}A^{e\nu},
    \end{equation}
    where the last step is just renaming the indices \(d \leftrightarrow e\).
    We can therefore use
    \begin{equation}
        \partial_\mu A^a_\nu - \partial_\nu A^a_\mu = 2 \partial_{(\mu} A^a_{\nu)}
    \end{equation}
    to replace \(\partial_\mu A^a_\nu - \partial_\nu A^a_\mu\) with \(\partial_\mu A^a_\nu\), which when antisymmetrised over \(\mu\) and \(\nu\) reproduces \(\partial_\mu A^a_\nu - \partial_\nu A^a_\mu\), and this antisymmetrising is enforced by the antisymmetry of \(f^{ade}A^{e\mu}A^{d\nu}\), which kills the symmetric part of \(\partial_\mu A^a_\nu\).
    Therefore we have the term
    \begin{equation}
        -\frac{1}{2} g f^{abc} (\partial_\mu A^a_\nu)A^{b\mu}A^{c\nu},
    \end{equation}
    having renamed the colour indices \(d \to b\) and \(e \to c\).
    
    The Feynman rule corresponding to this, which is cubic in the gauge field so corresponds to a three-point gluon interaction, is given by considering an amputated correlator for three incoming particles, which expanded to first order gives
    \begin{equation}
        \bra{0} i \int \dl{^4x} \, (-g)f^{ABC}(\partial_\mu A^A_\nu)A^{B\mu}A^{C\nu}\ket{p_1 \, \alpha \, a, p_2 \, \beta \, b, p_3 \, \gamma \, c}.
    \end{equation}
    The other factor of 2 is removed by the antisymmetry of exchanging the two gauge fields which we aren't taking the derivative of.
    Note that in the initial state we place the momenta, colour indices, and also the Lorentz indices which are attached to the external polarisation states.
    This means that a contraction between, for example, \(A^{B\mu}\) and \(p_1 \, \alpha \, a\) gives \(\varepsilon^\mu \varepsilon^\alpha\), and a contraction between \(\partial_\mu A^A_\nu\) and \(p_1 \, \alpha \, A\) gives \(-ip_1 \varepsilon_\nu \varepsilon^\alpha\), as the derivative brings down a factor of \(-i\) from the plane wave in the mode expansion.
    
    One particular contraction which we have to consider is
    \begin{equation}
        \wick{\\bra{0} i \int \dl{^4x} \, (-g)f^{ABC}(\partial_\mu \c3{A}^A_\nu)\c2{A}^{B\mu}\c1{A}^{C\nu}\ket{\c3{p}_1 \, \alpha \, a, \c2{p}_2 \, \beta \, b, \c1{p}_3 \, \gamma \, c}}.
    \end{equation}
    The contractions are the same as in QED, but with an extra Kronecker delta on the colour indices, so this contraction sets \(A = a\), \(B = b\), and \(C = c\).
    The derivative brings down an \(-ip_{1\mu}\), but then since the \(\mu\) index is contracted (as in indices match across) with \(A^{B\mu}\) and \(A^{B\mu}\) is contracted (as in Wick contractions) with \(p_2 \, \beta \, b\), so this \(\mu\) becomes a \(\beta\).
    The other Lorentz indices are contracted together and there is only really one reasonable result, \(\minkowskiMetric_{\gamma\alpha}\).
    Overall this contraction gives the term
    \begin{equation}
        (-ig) f^{abc}(-ip_{1\beta}\minkowskiMetric_{\gamma\alpha})
    \end{equation}
    There are then two other contractions which preserve the cyclic order of \(f^{ABC}\), and these are just cyclic permutations of the above.
    There are then three terms which change the cyclic order of \(f^{ABC}\), these are just the contraction we've considered swapping which of the two contractions goes into \(p_2 \, \beta \, b\) and which goes into \(p_3 \, \gamma \, c\), so they're quite similar, and these can then be made negative to return to the same cyclic order of \(f^{ABC}\).
    Overall this gives us six terms which give the result quoted above.
    
    
    
    
    \chapter{Faddeev--Popov Gauge Fixing}
    In this chapter we discuss gauge fixing through the Faddeev--Popov procedure.
    We followed this same procedure in \course{Quantum Field Theory} for QED, here we follow it for the more general Yang--Mills theory, but the idea is the same.
    
    \section{Setup and the Problem}
    The pure Yang--Mills action is
    \begin{equation}
        S_{\symrm{YM}} = \int \dl{^Dx} \left[ -\frac{1}{4} F^a_{\mu\nu}F^{a\mu\nu} \right]
    \end{equation}
    where
    \begin{equation}
        F^a_{\mu\nu}(x) = \partial_\mu A^a_\nu(x) - \partial_\nu A^a_\mu(x) + g f^{abc}A^b_\mu(x)A^c_\nu(x).
    \end{equation}
    The gauge transformation law for the gauge field under an infinitesimal transformation is
    \begin{align}
        A'^a_\mu(x) &= A^a_\mu(x) + f^{abc}A^b_\mu(x)\vartheta^c(x) + \frac{1}{g}\partial_\mu \vartheta^a(x)\\
        &= A_\mu^a(x) + \frac{1}{g}\covariantDerivative_\mu \vartheta^a(x).
    \end{align}
    
    The problem we face is that the naive Yang--Mills path integral with this action is undefined.
    This is because it includes an integral over all gauge fields, which includes an integral over \(A^a_\mu\) and also every \(\tensor[^\vartheta]{A}{^a_\mu}\) related to \(A^a_\mu\) by a gauge transformation parametrised by \(\vartheta^a(x)\).
    
    In the language of group theory we are integrating over all gauge fields, and also the result of any \(\specialUnitary(N)\) group action on a gauge field.
    Given some gauge field \(A^a_\mu\) and some \(g \in \specialUnitary(N)\) we can construct the \defineindex{orbit} of \(A^a_\mu\) to be 
    \begin{equation}
        \{g^n \mathbin{.} A^a_\mu \mid n \in \naturals\}
    \end{equation}
    which consists of all fields related to \(A^a_\mu\) by the gauge transformation \(g\).
    The solution is to somehow limit our integral to include only one gauge field from each orbit.
    
    The problem is evident on the level of the free action, with \(g = 0\), where we have
    \begin{align}
        \frac{1}{2}F^a_{\mu\nu}F^a_{\mu\nu} &= (\partial_\mu A^a_\nu)(\partial^\mu A^{a\nu} - \partial^\nu A^{a\mu})\\
        &= (\partial_\mu A^a_\nu)(\partial^\mu A^{a\nu}) - (\partial_\mu A^a_\nu)(\partial^\nu A^{a\mu})\\
        &= \partial_\mu (A^a_\nu \partial^\mu A^{a\nu}) - A^a_\nu \partial_\mu \partial^\mu A^{a\nu} - \partial_\mu(A^a_\nu \partial^\nu A^{a\mu}) + A^a_\nu(\partial_\mu \partial^\nu A^{a\mu}) \notag\\
        &= \partial_\mu(A^a_\nu \partial^\mu A^{a\nu} - A^a_\nu \partial^\nu A^{a\mu}) - A^a_\nu \dalembertian A^{a\nu} + A^a_\nu \partial_\mu \partial^\nu A^{a\mu}\\
        &= \partial_\mu(A^a_\nu \partial^\mu A^{a\nu} - A^a_\nu \partial^\nu A^{a\mu}) - A^a_\mu [\minkowskiMetric^{\mu\nu}\dalembertian - \partial^\mu \partial^\nu] A^a_\mu.
    \end{align}
    The first term here is just a surface term, so doesn't contribute to the action.
    So the free action is
    \begin{equation}
        S_0 = \int \dl{^Dx} \, \frac{1}{2}A^a_\mu(x)[\minkowskiMetric^{\mu\nu}\dalembertian - \partial^\mu \partial^\nu] A^a_\nu(x).
    \end{equation}
    The problem comes when trying to compute path integrals of \(\e^{iS_0[A]}\), since the general answer requires the inverse of the kernel, which here is \(\minkowskiMetric^{\mu\nu}\dalembertian - \partial^\mu\partial^\nu\), but this isn't invertible.
    
    Geometrically, if we fix some value of \(x\) then we can consider a volume defined by all possible values of \(A^a_\mu(x)\) for fixed \(a\) and \(\mu\).
    For simplicity we'll assume that this defines a two-dimensional surface.
    The orbits are then lines on this surface generated by acting with some fixed gauge transformation on \(A^a_\mu\) repeatedly.
    We want to impose a gauge fixing constraint which picks out one value of \(A^a_\mu\) for each orbit, that is one point on each line.
    These orbit lines are dense in the surface, and so the collection of points chosen after gauge fixing define another line.
    We assume that there is a single solution to the gauge fixing condition for each orbit.
    
    \begin{figure}
        \tikzsetnextfilename{gauge-fixing-condition}
        \begin{tikzpicture}[baseline=(current bounding box), font=\scriptsize]
            \draw[thick] (0, 0) .. controls (2, 1) and (3, 0.75)  .. (4, 0.5) .. controls (3.8, 2) .. (4, 4.8) .. controls (3, 5.05) and (2, 5.3) .. (0, 4.3) .. controls (0, 4.3) and (-0.2, 2) .. cycle;
            \node[below left] at (4, 4.8) {\(A^a_\mu(x)\)};
            \foreach \y in {0.5, 1, ..., 4} {
                \draw[thick, Purple] (0.5, \y) .. controls (1, \y+0.25+0.3*rand) and (2, \y+1+0.3*rand) .. (3, \y+0.5);
            }
            \draw[Blue, thick] (1, 0.5) -- (2.5, 5);
        \end{tikzpicture}
        \caption{All possible values of \(A^a_\mu(x)\) related by gauge transformations at fixed \(x\), \(a\), and \(\mu\). The purple lines represent orbits, all points on the line are given by acting on one point of the line with a fixed gauge transformation. The other gauge transformations will move points between lines. The blue line is a gauge fixing condition, we assume that it crosses each orbit once, and we choose the point it crosses the orbit to be the one included in the path integral.}
    \end{figure}
    
    \section{The Faddeev--Popov Procedure}
    We use the gauge fixing constraint
    \begin{equation}
        G^a(A) = \partial_\mu \tensor[^\vartheta]{A}{^{a\mu}}(x) - \omega^a(x) = 0
    \end{equation}
    for some function \(\omega^a\).
    This is a generalisation of the Lorenz gauge condition, \(\partial_\mu A^{a\mu} = 0\), and indeed choosing \(\omega^a(x) = 0\) recovers the Lorenz gauge.
    
    Now consider the correlator of some gauge invariant observable \(\symcal{O}\).
    This is given by
    \begin{equation}
        \correlator{\symcal{O}} = \frac{\int \! \DL{A} \, \symcal{O}\e^{iS[A]}}{\int \! \DL{A} \, \e^{iS[A]}}.
    \end{equation}
    The denominator is just the normalisation, which normally we leave off, but we want it here as it will simplify things later.
    
    As with QED in \course{Quantum Field Theory} we aim to get this in the form
    \begin{equation}
        \frac{\int \! \DL{A} \, \delta(G(A)) \symcal{O} \e^{iS[A]}}{\int \! \DL{A} \, \delta(G(A)) \e^{iS[A]}}.
    \end{equation}
    We won't quite succeed but we'll get close enough.
    Here \(\delta\) is a functional Dirac delta given by
    \begin{equation}
        \delta(G(A)) = \prod_{x, a} \delta(G(A^a_\mu(x)))
    \end{equation}
    where on the right \(\delta\) is just a normal Dirac delta.
    We use this to impose that \(\partial^\mu A^a_\mu(x) = \omega^a(x)\) for all \(a\) and all \(x\).
    
    To proceed we insert unity.
    First notice that for a function \(g\) with \(g(\vartheta) > 0\) we have
    \begin{equation}
        1 = \int \dl{g} \, \delta(g) = \int \dl{\vartheta} \, \diffp{g}{\vartheta} \delta(g(\vartheta)).
    \end{equation}
    If \(g(\vartheta) < 0\) we instead get \(-1\), but the sign will cancel out in the normalisation, so it doesn't really matter.
    We can generalise this to multiple dimensions with
    \begin{equation}
        1 = \int \dl{^ng} \, \delta(g) = \int \dl{^n\vartheta} \, \det\left( \diffp{g^i}{\vartheta^j} \right) \delta(g(\vartheta)).
    \end{equation}
    Further generalising to a functional identity we get
    \begin{equation}
        1 = \int \DL{\vartheta} \, \det\left( \diffd{G}{\vartheta} \right) \delta(G(\vartheta)).
    \end{equation}
    
    We will use
    \begin{align}
        G(A^a(x)) &= \partial_\mu \tensor[^\vartheta]{A}{^{a\mu}}(x) - \omega^a(x)\\
        &= \partial_\mu A^{a\mu}(x) + \frac{1}{g} \partial_\mu \covariantDerivative^\mu \vartheta^a(x) - \omega^a(x).
    \end{align}
    We then have
    \begin{equation}
        \diffd{G(A^a(x))}{\vartheta^b(y)} = \frac{1}{g} \partial_\mu \covariantDerivative^\mu \delta^{ab} \delta(x - y).
    \end{equation}
    Importantly this value depends on \(A^{a\mu}\) through \(\covariantDerivative^\mu\).
    This wasn't the case in QED.
    Expanding the definition of \(\covariantDerivative^\mu\) we have
    \begin{equation}
        \diffd{G(A^a(x))}{\vartheta^b(y)} = \frac{1}{g}\partial_\mu (\partial^\mu \delta^{ab} - gf^{abc}A^{c\mu}(x)) \delta(x - y) \eqqcolon \symcal{F}[A^{a\mu}(x)].
    \end{equation}
    
    Intuitively, the difference between Yang--Mills theory and QED in this respect is that in QED the orbits are all parallel, in the sense that if a gauge transformation takes you between two orbits then it doesn't matter where you start on one of those orbits, you'll always move to the same orbit.
    In Yang--Mills this is not the case, to move from orbit one to orbit two at one point requires a different gauge transformation than moving between orbit one and orbit two at another point.
    
    We can proceed to insert unity into the path integral:
    \begin{align}
        \int \DL{A} \, 1 \symcal{O} \e^{iS[A]} &= \int \DL{A} \int \DL{\vartheta} \, \det(\symcal{F}[A]) \delta(G(\tensor[^\vartheta]{A}{})) \symcal{O} \e^{iS[A]}\\
        &= \int \DL{\vartheta} \int \DL{A} \, \det(\symcal{F}[A]) \delta(G(\tensor[^\vartheta]{A}{})) \symcal{O} \e^{iS[A]}.
    \end{align}
    We can swap the integrals because we aren't worrying about convergence and both \(\vartheta\) and \(A\) are bosonic, so commute.
    
    Most gauge fields in this integral don't contribute, because of the \(\delta(G(\tensor[^\vartheta]{A}{}))\) term.
    A given gauge field only contributes if it satisfies \(G(\tensor[^\vartheta]{A}{}) = 0\), that is if
    \begin{equation}
        \partial_\mu A^{a\mu}(x) + \frac{1}{g} \partial_\mu \covariantDerivative^\mu \vartheta^a(x) - \omega^a(x) = 0.
    \end{equation}
    Now define \(A'\) to be
    \begin{equation}
        A'^{a\mu}(x) = A^{a\mu}(x) + \frac{1}{g}\covariantDerivative^\mu \vartheta^a(x).
    \end{equation}
    We can change \(A\) to \(A'\) in the integral, assuming that \(\DL{A} = \DL{A'}\), and using the gauge invariance of the observable, \(\symcal{O}\), and the action.
    We then have
    \begin{equation}
        \int\DL{A} \, \symcal{O} \e^{iS[A]} = \int \DL{\vartheta} \int \DL{A'} \det(\symcal{F}[A']) \delta(G(A')) \symcal{O} \e^{iS[A']}.
    \end{equation}
    Now the only \(\vartheta\) dependence is in the \(\int\DL{\vartheta}\) part.
    Doing exactly the same with the denominator of \(\correlator{\symcal{O}}\) we can cancel these two integrals with \(\vartheta\) dependence.
    Dropping the primes we then have
    \begin{equation}
        \correlator{\symcal{O}} = \frac{\int \! \DL{A} \, \det(\symcal{F}[A]) \delta(G(A)) \e^{iS[A]}}{\int \! \DL{A} \, \det(\symcal{F}[A]) \delta(G(A)) \e^{iS[A]}}.
    \end{equation}
    In QED \(\symcal{F}[A]\) is actually independent of \(A\), and so we can bring it outside of the integral and it cancels.
    This isn't the case in Yang--Mills, so we won't quite get the desired form, but we continue anyway.
    
    Now use
    \begin{equation}
        1 = \frac{1}{N} \int \DL{\omega} \, \exp\left\{ -i \int \dl{^Dx} \frac{1}{2\xi} \omega^a(x)\omega^a(x) \right\}
    \end{equation}
    where \(N\) is a normalisation factor chosen to make this true.
    This is a Gaussian integral.
    We can use it, as well as \(G(A^a(x)) = \partial_\mu A^{a\mu}(x) - \omega^a(x)\), to rewrite our result as
    \begin{align}
        \int \DL{A} \, &\det(\symcal{F}[A]) \delta(\partial_\mu A^{a\mu}(x) - \omega^a(x)) \symcal{O} \e^{iS[A]}\\
        &= \frac{1}{N} \int \DL{A} \int \DL{\omega} \, \det(\symcal{F}[A]) \delta(\partial_\mu A^{a\mu}(x) - \omega^a(x)) \notag\\
        &\qquad\qquad \times \exp\left\{ -i\int \dl{^Dx} \frac{1}{2\xi} \omega^b(x)\omega^b(x) \right\} \symcal{O} \e^{iS[A]}\\
        &= \frac{1}{N} \int \DL{A} \, \det(\symcal{F}[A]) \symcal{O} \exp\left\{ iS[A] - i\int \dl{^Dx} \frac{1}{2\xi} (\partial_\mu A^{b\mu})(\partial_\nu A^{b\nu}) \right\} \notag
    \end{align}
    where in the last step we've used the \(\delta\) to perform the \(\omega\) integral.
    
    We can interpret this as modifying the original Lagrangian to contain a gauge fixing term
    \begin{equation}
        \lagrangianDensity = -\frac{1}{4}F^a_{\mu\nu}F^{a\mu\nu} - \frac{1}{2\xi}(\partial_\mu A^{a\mu})(\partial_\nu A^{a\nu}).
    \end{equation}
    What this tells us is that the gluon propagator is exactly the photon propagator with an extra \(\delta^{ab}\) factor coming from this term.
    
    We still have to deal with the \(\det(\symcal{F}[A])\) term.
    To do this we can recognise that determinants can arise from Gaussian path integrals.
    In particular, we know that if \(M\) is an operator on spinors we have
    \begin{equation}
        \int \DL{\psi} \DD{\diracadjoint{\psi}} \, \exp\left\{ -i\int\dl{^Dx} \, \diracadjoint{\psi} M \psi \right\} = N' \det M
    \end{equation}
    where \(N'\) is another not-important normalisation constant which will cancel in the ratio.
    Note that we need anticommuting variables in order to get the correct power of the determinant.
    We can thus get an appearance of the desired determinant by taking 
    \begin{equation}
        M = \frac{1}{g}\partial_\mu \covariantDerivative^\mu.
    \end{equation}
    Then the Lagrangian can be written as
    \begin{equation}
        \lagrangianDensity = -\frac{1}{4}F^a_{\mu\nu}F^{a\mu\nu} - \frac{1}{2\xi}(\partial_\mu A^{a\mu})(\partial_\nu A^{a\nu}) - \diracadjoint{c}^a \partial_\mu \covariantDerivative^\mu c^a.
    \end{equation}
    Here \(c\) replaces \(\psi\) in the above.
    First note that we just absorbed the \(1/g\) factor into the two \(c\)s, changing the normalisation \(N'\) appropriately, but \(N'\) will cancel in the ratio so we don't worry about it.
    Next note that \(c\)s must carry an adjoint index, since \(\partial_\mu \covariantDerivative^\mu\) is an operator in the adjoint representation.
    Now notice that the \(c\)s don't carry a Lorentz index, since the Lorentz index in the operator is already contracted.
    This means that the \(c\)s are scalars.
    But they anticommute.
    This seems to violate the spin-statistics theorem, which says that fermionic states are the anticommuting ones.
    The solution is that the \(c\)s aren't physical states.
    In fact it is possible to show that there is no process starting without \(c\)s which can produces \(c\)s in the final state.
    For this reason the \(c\)s are called \defineindex{Faddeev--Popov ghosts}.
    
    We can't see the ghosts in external states, but have to allow for ghost loops in our expansions.
    The purpose of the ghosts is to remove a degree of freedom, since for each \(a\) we have two helicities, but four degrees of freedom in \(A^a_\mu\).
    In QED we fixed two degrees of freedom by requiring that \(\varepsilon \cdot k = 0\) and the gauge symmetry \(\varepsilon \mapsto \varepsilon' = \varepsilon + \alpha k\), which satisfies \(\varepsilon' \cdot k = \varepsilon \cdot k + \alpha k^2 = 0 + \alpha k^2\), and \(k^2 = 0\) is just the Fourier transformed equation of motion, \(\dalembertian A^\mu = 0\).
    In Yang--Mills the equation of motion is more complicated, so we can't fix the degrees of freedom as simply and are forced to introduce these ghost particles.
    
    \chapter{Renormalising Yang--Mills}
    As with QED Yang--Mills theory at one loops is plagued by infinities.
    We will remove these following an almost identical procedure to QED.
    In particular, we will use
    \begin{itemize}
        \item dimensional regularisation;
        \item the \MSbar{} scheme;
        \item renormalised perturbation theory,
    \end{itemize}
    and we will make use of results from QED, as many results in Yang--Mills theory differ only by factors of the \(\specialUnitary(N)\) generators, \(t_R^a\), which we call \defineindex{colour factors}, taking this name from QCD and applying it to general Yang--Mills theory.
    
    Our ultimate goal will be to compute the beta function of Yang--Mills theory:
    \begin{equation}
        \beta(g) \coloneqq \mu\diff{g}{\mu}
    \end{equation}
    which is defined when \(D = 4\), so after taking \(\varepsilon \to 0\).
    
    \section{Regulating the Lagrangian}
    The Yang--Mills Lagrangian, now in terms of bare fields, is
    \begin{equation}
        \lagrangianDensity = -\frac{1}{4}F^a_{\bare\mu\nu}F^{a\mu\nu}_{\bare} + \diracadjoint{\psi}_{\bare}(i\slashed{\covariantDerivative}_{\bare} - m_{\bare})\psi_{\bare} - \diracadjoint{c}_{\bare}^a \partial_\mu \covariantDerivative_{\bare}^\mu c^a.
    \end{equation}
    where
    \begin{equation}
        F_{\bare}^{\mu\nu} = \partial^\mu A_{\bare}^\nu - \partial^\nu A_{\bare}^\mu + g_{\bare} f^{abc}A_{\bare}^{b\mu} A_{\bare}^{c\nu}
    \end{equation}
    and
    \begin{equation}
        \covariantDerivative_{\bare}^\mu = \partial^\mu - ig_{\bare} t_R^a A_{\bare}^{a\mu}.
    \end{equation}
    We can now introduce renormalised fields by defining
    \begin{align}
        A_{\bare\mu}^{a} &= \sqrt{Z_3} A^a_\mu, \quad &\psi_{\bare} &= \sqrt{Z_2} \psi, \quad& c_{\bare}^a &= \sqrt{Z_{3c}}c^a,\\
        \quad g_{\bare} &= Z_g \mu^\varepsilon g, &&\mathrm{and} & m_{\bare} &= Z_mm.
    \end{align}
    Note that \(Z_i\) are dimensionless, and \(\mu\) carries dimensions of energy, with \(\varepsilon\) being the \(\varepsilon\) of dimensional regularisation, \(D = 4 - 2\varepsilon\).
    
    Substituting these into the Lagrangian we get
    \begin{align}
        \lagrangianDensity = -\frac{1}{4}&Z_3(\partial^\mu A^{a\nu} - \partial^\nu A^{a\mu})(\partial_\mu A_\nu^a - \partial_\nu A_\mu^a) + Z_2\diracadjoint{\psi}(i\slashed{\partial} - Z_mm)\psi \\
        &- Z_{3c}\diracadjoint{c}^a\dalembertian c^a - \underbrace{Z_gZ_3^{3/2}}_{Z_{A^3}} g\mu^\varepsilon f^{abc}(\partial^\mu A^{a\nu})A^b_\mu A^c_\nu \notag\\
        &- \frac{1}{4}\underbrace{Z_g^2 Z_3^2}_{Z_{A^4}} g^2 \mu^2 \varepsilon f^{abc}f^{ade}A^b_\mu A^c_\nu A^{d\mu} A^{e\nu} + \underbrace{Z_2\sqrt{Z_3}Z_g}_{Z_1} g\mu^\varepsilon \diracadjoint{\psi} \slashed{A}^a t_R^a \psi \notag\\
        &+ \underbrace{Z_gZ_{3c}\sqrt{Z_3}}_{Z_{1c}} g\mu^\varepsilon (\partial_\mu \diracadjoint{c}^a) f^{abc}A^{b\mu}c^c. \notag
    \end{align}
    
    This is getting out of hand, so we'll focus only on the fermions for simplicity:
    \begin{align}
        \lagrangianDensity &= \diracadjoint{\psi}_{\bare}(i\slashed{\covariantDerivative}_{\bare} - m_{\bare})\psi_{\bare}\\
        &= Z_2\diracadjoint{\psi}(i\slashed{\partial} - Z_m m)\psi + Z_1\mu^\varepsilon g\diracadjoint{\psi}\slashed{A}^a t_R^a \psi.
    \end{align}
    Note that in QED we had \(Z_1 = Z_2\), this is not the case in Yang--Mills theory.
    Now for each \(Z_i\) write \(Z_i = 1 + \delta_i\), then we have
    \begin{multline}
        \lagrangianDensity = \diracadjoint{\psi}(i\slashed{\partial} - m)\psi + g\mu^\varepsilon \diracadjoint{\psi}\slashed{A}^a t_R^a \psi\\
        + \delta_2 \diracadjoint{\psi}(i\slashed{\partial} + m)\psi - \delta_m m \diracadjoint{\psi} \diracadjoint{\psi}\psi + \delta_1 g\mu^\varepsilon \diracadjoint{\psi}\slashed{A}^a t_R^a \psi - \delta_2 \delta_m m\diracadjoint{\psi}\psi.
    \end{multline}
    As with QED the \(\delta_2 \delta_m\) term only contributes at lowest order if we have two or more loops, so we neglect it.
    
    \section{Feynman Rules for Counterterms}
    We have three counterterms, and so three new, albeit familiar, Feynman rules.
    The first corresponds to the \(\delta_1\) counterterm.
    This has the form of the QCD gluon-gluon-fermion vertex, but with an extra factor of \(\delta_1\), so we have
    \begin{equation}
        \tikzsetnextfilename{yang-mills-counterterm-1}
        \begin{tikzpicture}[baseline=(current bounding box), font=\scriptsize]
            \draw[gluon] (-1, 0) node [left] {\(\mu, a\)} -- (-0.15, 0);
            \draw[quark=0.6] (40:0.15) -- (40:1);
            \draw[antiquark=0.4] (-40:0.15) -- (-40:1);
            \draw (0, 0) circle [radius = 0.15];
            \foreach \angle in {45, 135, 225, 315} {
                \draw (0, 0) -- (\angle:0.15);
            }
        \end{tikzpicture}
        = i\delta_1 g\mu^\varepsilon \gamma^\mu t_R^a.
    \end{equation}
    
    The \(\delta_2\) term is very similar to the QED result:
    \begin{equation}
        \tikzsetnextfilename{yang-mills-counterterm-2}
        \begin{tikzpicture}[baseline=-2.5pt, font=\scriptsize]
            \draw[quark=0.6] (-1, 0) -- (-0.15, 0);
            \draw[antiquark=0.6] (0.15, 0) -- (1, 0);
            \draw[->] (-0.8, -0.2) -- (-0.35, -0.2) node [midway, below] {\(p\)};
            \draw[->] (0.35, -0.2) -- (0.8, -0.2) node [midway, below] {\(p\)};
            \draw (0, 0) circle [radius = 0.15];
            \foreach \angle in {45, 135, 225, 315} {
                \draw (0, 0) -- (\angle:0.15);
            }
        \end{tikzpicture}
        = i\delta_2(\slashed{p} - m) - i\delta_m m.
    \end{equation}
    
    The \(\delta_3\) term is also very similar to the QED result, but as each gluon propagator carries a \(\delta^{ab}\) this factor is carried through to the counterterm:
    \begin{equation}
        \tikzsetnextfilename{yang-mills-counter-term-3}
        \begin{tikzpicture}[baseline=-2.5pt, font=\scriptsize]
            \draw[gluon] (-1, 0) node [left] {\(\nu, b\)} -- (-0.15, 0);
            \draw[gluon] (0.15, 0) -- (1, 0) node [right] {\(\mu, a\)};
            \draw[->] (-0.8, -0.2) -- (-0.35, -0.2) node [midway, below] {\(k\)};
            \draw[->] (0.35, -0.2) -- (0.8, -0.2) node [midway, below] {\(k\)};
            \draw (0, 0) circle [radius = 0.15];
            \foreach \angle in {45, 135, 225, 315} {
                \draw (0, 0) -- (\angle:0.15);
            }
        \end{tikzpicture}
        = i\delta_3 \delta^{ab} (k^\mu k^\nu - \minkowskiMetric^{\mu\nu} k^2).
    \end{equation}
    
    \section{Computing The \texorpdfstring{\(\delta_3\)}{delta 3} Counterterm}
    To compute the beta function we will need to find the values of \(\delta_1\), \(\delta_2\), and \(\delta_3\).
    We'll start by computing \(\delta_3\).
    We do so in a theory of \(\numberFermions\) fermions of mass \(m\), all in some representation, \(R\), of \(\specialUnitary(N)\).
    If we want to add in scalars or fermions in some other representation this can be done by following the same process and then just adding all the contributions to \(\delta_3\) together.
    
    Recall that the \MSbar{} scheme is defined by taking the counterterm to contain only divergent terms, as well as rescaling \(\mu\).
    For this reason we need only keep divergent terms in our calculations.
    To compute \(\delta_3\) we consider one loop corrections to the gluon propagator, which is itself finite.
    Thus, we need to consider the following
    \begin{equation}
        \tikzsetnextfilename{yang-mills-delta-3-finite-sum}
        \begin{tikzpicture}[baseline=(plus.base)]
            \draw[gluon] (0, 0) -- (2, 0);
            \node (plus) at (2.25, 0) {\(+\)};
            \draw[gluon] (2.5, 0) -- (3.35, 0);
            \draw (3.5, 0) circle [radius = 0.15];
            \foreach \angle in {45, 135, 225, 315} {
                \draw (3.5, 0) -- ++ (\angle:0.15);
            }
            \draw[gluon] (3.65, 0) -- (4.5, 0);
            \node at (4.75, 0) {\(+\)};
            \draw[gluon] (5, 0) -- (6, 0);
            \draw[gluon] (7, 0) -- (8, 0);
            \draw[antiquark=0.45] (6, 0) arc (180:0:0.5);
            \draw[quark=0.55] (6, 0) arc (180:360:0.5);
            \node at (0.75, -2) {\(+\)};
            \draw[gluon] (1, -2) -- (2, -2);
            \draw[gluon] (2, -2) -- (3, -2);
            \draw[mirror gluon, looseness=1.25] (2, -2) to[out=135, in=180] (2, -1) to [out=0, in=45] (2, -2);
            \node at (3.25, -2) {\(+\)};
            \draw[gluon] (3.5, -2) -- (4.5, -2);
            \draw[gluon] (5.5, -2) -- (6.5, -2);
            \draw[gluon] (4.5, -2) arc (180:0:0.5);
            \draw[mirror gluon] (4.5, -2) arc (180:360:0.5);
            \node at (6.75, -2) {\(+\)};
            \draw[gluon] (7, -2) -- (8, -2);
            \draw[gluon] (9, -2) -- (10, -2);
            \draw[antighost=0.45] (8, -2) arc (180:0:0.5);
            \draw[ghost=0.55] (8, -2) arc (180:360:0.5);
            \node at (10.25, -2) {.};
        \end{tikzpicture}
    \end{equation}
    We demand that this sum is finite, and define \(\delta_3\) to make this true.
    We can neglect the first diagram, since it is finite.
    We'll compute the rest of the diagrams one at a time, then combine the results to find \(\delta_3\).
    
    \subsection{Fermion Loop}
    Consider the diagram
    \begin{equation}
        \tikzsetnextfilename{yang-mills-delta-3-fermion-loop}
        \begin{tikzpicture}[baseline=(current bounding box), font=\small]
            \draw[gluon] (0, 0) node [left] {\(\mu, a\)} -- (1, 0);
            \draw[gluon] (2, 0) -- (3, 0) node [right] {\(\nu, b\){\normalsize.}};
            \draw[antiquark=0.45] (1, 0) arc (180:0:0.5);
            \draw[quark=0.55] (1, 0) arc (180:360:0.5);
            \draw[->] (0.2, -0.2) -- (0.8, -0.2) node [midway, below] {\(k\)};
            \draw[->] (2.2, -0.2) -- (2.8, -0.2) node [midway, below] {\(k\)};
            \draw[->, xshift=1.5cm] (45:0.7) arc (45:135:0.7) node [midway, above] {\(q - k\)};
            \draw[->, xshift=1.5cm] (225:0.7) arc (225:315:0.7) node [midway, below] {\(q\)};
        \end{tikzpicture}
    \end{equation}
    This is straightforward to turn into an integral:
    \begin{itemize}
        \item The closed fermion loop gives a factor of \(-1\).
        \item The left most vertex gives a factor of
        \begin{equation}
            ig\mu^\varepsilon\gamma^\mu t_R^a.
        \end{equation}
        \item Following the fermion propagator backwards we have a fermion of momentum \(q - k\), contributing
        \begin{equation}
            \frac{i(\slashed{q} - \slashed{k} + m)}{(q - k)^2 - m^2 + i\epsilon}.
        \end{equation}
        \item Next we reach the other vertex, giving a factor of
        \begin{equation}
            ig\mu^\varepsilon\gamma^\nu t_R^b.
        \end{equation}
        \item Then we have the second fermion propagator,
        \begin{equation}
            \frac{i(\slashed{q} + m)}{q^2 - m^2 + i\epsilon}.
        \end{equation}
        \item We have to integrate over the undetermined momentum,
        \begin{equation}
            \int \dhat{^Dq}.
        \end{equation}
    \end{itemize}
    
    The generators of the \(\specialUnitaryLie(N)\) algebra, \(t_R^a\), and the gamma matrices, \(\gamma^\mu\), commute, since they're in different spaces, and act on different spaces.
    Just as there is a trace over gamma matrices we have a trace over the \(\specialUnitaryLie(N)\) generators.
    Thus, this diagram corresponds to
    \begin{equation}
        -(ig\mu^\varepsilon)^2 \tr(t_R^a t_R^b) \int \dhat{^Dq} \frac{\tr[\gamma^\mu i(\slashed{q} - \slashed{k} + m) \gamma^\nu i(\slashed{q} + m)]}{[(q - k)^2 - m^2 + i\epsilon][q^2 - m^2 + i\epsilon]}.
    \end{equation}
    
    Fortunately, we've already done most of the work to compute this in QED in \cref{sec:vacuum polarisation}.
    The only new part in Yang--Mills is
    \begin{equation}
        \tr(t_R^a t_R^b) = T(R)\delta^{ab}.
    \end{equation}
    Using the result of \cref{eqn:delta 3 QED} we can evaluate this integral, and then drop the terms which are finite as \(\varepsilon \to 0\), and we find that the divergent contribution is
    \begin{equation}
        \frac{ig^2}{16\pi^2\varepsilon} \frac{4}{3}T(R)\delta^{ab}(p^\mu p^\nu - \minkowskiMetric^{\mu\nu}p^2).
    \end{equation}
    We choose to include the factor of \(4/3\), as we will later see that all of the terms come with a factor of \(ig^2/(16\pi^2\varepsilon)\).
    Notice that this has the same tensor structure as the counterterm, a good sign.
    
    \subsection{Single Gluon Loop}
    Now consider the diagram
    \begin{equation}
        \tikzsetnextfilename{yang-mills-delta-3-single-gluon-loop}
        \begin{tikzpicture}[baseline=(current bounding box)]
            \draw[gluon] (-1, 0) node [left] {\(\mu, a\)} -- (0, 0);
            \draw[gluon] (0, 0) -- (1, 0) node [right] {\(\nu, b\)};
            \draw[mirror gluon, looseness=1.25] (0, 0) to[out=135, in=180] (0, 1) to [out=0, in=45] (0, 0);
            \draw[->] (70:1.2) arc (70:110:1.2) node [midway, above] {\(\ell\)};
        \end{tikzpicture}
        .
    \end{equation}
    This integral has the form
    \begin{equation}
        \int \dhat{^D\ell} \frac{N^{ab\mu\nu}}{\ell^2},
    \end{equation}
    where the numerator is independent of the loop momenta since the four-gluon vertex is independent of momentum.
    Using the general loop formula,
    \begin{equation}
        \int \dhat{^D\ell} \, \frac{1}{\ell^2 - \Delta} = -\frac{i}{(4\pi)^{D/2}} \Gamma(1 - D/2) \Delta^{D/2 - 1},
    \end{equation}
    with \(\Delta = 0\) taking \(\varepsilon\) to be small we have \(D/2 - 1 > 0\), and so this vanishes.
    This is known as \defineindex{Veltman's formula}.
    
    This integral vanishing is a property of dimensional regularisation.
    If instead we worked with a momentum cut-off then we would find that the integral is quadratically divergent.
    This quadratic divergence would cancel in the final sum of diagrams, since there is no counterterm to absorb it and the result must be finite.
    
    \subsection{Double Gluon Loop}
    Now consider the diagram
    \begin{equation}
        \tikzsetnextfilename{yang-mills-delta-3-double-gluon-loop}
        \begin{tikzpicture}[font=\scriptsize]
            \draw[gluon] (-3.5, 0) node [left] {\(\mu, a\)} -- (-1.5, 0);
            \draw[gluon] (1.5, 0) -- (3.5, 0) node [right] {\(\nu, b\)};
            \draw[gluon] (-1.5, 0) arc (180:0:1.5);
            \draw[mirror gluon] (-1.5, 0) arc (180:360:1.5);
            \draw[->] (-2.2, 0.2) -- (-1.8, 0.2) node [midway, above] {\(k\)};
            \draw[->] (140:1.3) arc (140:175:1.3) node [midway, right, shift={(-0.1, -0.05)}] {\(q\)};
            \draw[->] (220:1.3) arc (220:185:1.3) node [midway, right, shift={(-0.1, 0.05)}] {\(-k - q\)};
            \draw[->] (2.2, 0.2) -- (1.8, 0.2) node [midway, above] {\(-k\)};
            \draw[->] (40:1.3) arc (40:5:1.3) node [midway, left, shift={(0.1, -0.05)}] {\(-q\)};
            \draw[->] (-40:1.3) arc (-40:-5:1.3) node [midway, left, shift={(0.1, 0.05)}] {\(k + q\)};
            \draw[->] (-3, -0.3) -- (-2, -0.3) node [midway, below] {\(k\)};
            \draw[->] (2, -0.3) -- (3, -0.3) node [midway, below] {\(k\)};
            \draw[->] (60:1.8) arc (60:120:1.8) node [midway, above] {\(q\)};
            \draw[->] (-120:1.8) arc (-120:-60:1.8) node [midway, below] {\(k + q\)};
            \node[font=\tiny] at (-1.9, -0.2) {\(\mu, a\)};
            \node[font=\tiny] at (195:1.8) {\(\sigma, d\)};
            \node[font=\tiny] at (165:1.8) {\(\rho, c\)};
            \node[font=\tiny] at (1.9, -0.2) {\(\nu, b\)};
            \node[font=\tiny] at (-15:1.8) {\(\alpha, e\)};
            \node[font=\tiny] at (17:1.8) {\(\beta, f\)};
        \end{tikzpicture}
    \end{equation}
    Here we label each end of the internal gluons with different colour and Lorentz indices.
    Usually people don't do this, since the gluon propagators carry factors of \(\delta^{ab}\) and \(\minkowskiMetric^{\mu\nu}\) (in the Feynman gauge, \(\xi = 1\)) and so in practice we can treat the indices at each end as being the same, and just make sure that all indices are contracted over and upper/lower Lorentz indices match.
    For this diagram, and this diagram only, we'll do the full calculation without using this.
    
    This diagram has a symmetry factor of \(2\), we can swap the upper and lower gluon.
    this means we need to include an extra factor of \(1/2\).
    For the vertex terms the Feynman rule assumes all momenta are incoming, so we've labelled each vertex with incoming momenta.
    The cyclic nature of the Feynman rule means it doesn't matter which gluon we start with, but we must always go in the same direction around the vertex.
    In this case we'll go clockwise.
    The first vertex gives us a factor of
    \begin{align}
        gf^{dac}&[(-k - q - k)_\rho\minkowskiMetric_{\sigma\mu} + (k - q)_{\sigma}\minkowskiMetric_{\mu\rho} + (q - (- k - q))_{\mu}\minkowskiMetric_{\rho\sigma}] \notag\\
        &= gf^{dac}[(-2k - q)_\rho\minkowskiMetric_{\sigma\mu} + (k - q)_\sigma\minkowskiMetric_{\mu\rho} + (k + 2q)_\mu\minkowskiMetric_{\rho\sigma}].
    \end{align}
    The second vertex gives us a factor of
    \begin{align}
        gf^{fbe}&[(-q - (-k))_\alpha\minkowskiMetric_{\beta\nu} + (-k - (k + q))_{\beta}\minkowskiMetric_{\nu\alpha} + (k + q - (-q))_{\nu}\minkowskiMetric_{\alpha\beta}] \notag\\
        &= gf^{fbe}[(-q - k)_\alpha\minkowskiMetric_{\beta\nu} + (-2k + q)_\beta\minkowskiMetric_{\nu\alpha} + (k + 2q)_\nu\minkowskiMetric_{\alpha\beta}].
    \end{align}
    We also have the two propagators,
    \begin{equation}
        \frac{-i}{p^2 + i\epsilon}\delta^{cf}\minkowskiMetric^{\rho\beta}, \qand \frac{-i}{(k + q)^2 + i\epsilon}\delta^{de}\minkowskiMetric^{\sigma\alpha}.
    \end{equation}
    
    The diagram is then given by integrating the product of these terms over \(q\).
    Considering just the colour indices we have
    \begin{multline}
        f^{dac}f^{fbe}\delta^{cf}\delta^{de} = f^{dac}f^{cbd} = (t_{\adj}^a)^{dc}(t_{\adj}^{b})^{cd}\\
        = (t_{\adj}^a t_{\adj}^b)^{cc} = \tr(t_{\adj}^at_{\adj}^b) = T(\adjointRep)\delta^{ab}.
    \end{multline}
    Now considering the Lorentz structure, we have
    \begin{align}
        &[(-2k - q)_\rho\minkowskiMetric_{\sigma\mu} + (k - q)_\sigma\minkowskiMetric_{\mu\rho} + (k + 2q)_\mu\minkowskiMetric_{\rho\sigma}]\\
        &\quad\times
        [(-q - k)_\alpha\minkowskiMetric_{\beta\nu} + (-2k + q)_\beta\minkowskiMetric_{\nu\alpha} + (k + 2q)_\nu\minkowskiMetric_{\alpha\beta}]
        \minkowskiMetric^{\rho\beta} \minkowskiMetric^{\sigma\alpha}\\
        &= [(-2k - q)_\rho\minkowskiMetric_{\sigma\mu} + (k - q)_\sigma\minkowskiMetric_{\mu\rho} + (k + 2q)_\mu\minkowskiMetric_{\rho\sigma}]\\
        &\qquad\times [(-q - k)^\sigma \delta^\rho_\nu + (-2k + q)^\rho \delta^\sigma_\nu + (k + 2q)_\nu \minkowskiMetric^{\rho\sigma}].
    \end{align}
    We could have gone straight to this result if we chose to just label both ends of the propagator with the same indices and ensured that all indices were summed over properly.
    Expanding this out we get
    \begin{align}
        &(-2k - q)_\nu(-q - k)_\mu + (-2k - q) \cdot (-2k + q)\minkowskiMetric_{\mu\nu} + (-2k - q)_\mu (k + 2q)_\nu \notag\\
        &\quad + (k - q) \cdot (-q - k) \minkowskiMetric_{\mu\nu} + (k - q)_\nu (-2k + q)_\mu + (k - q)_\mu (k + 2q)_\nu \notag\\
        &\quad + (k + 2q)_\mu (-q - k)_\nu + (k + 2q)_\mu (-2k + q)_\nu + (k + 2q)_\mu (k + 2q)_\nu 4.
    \end{align}
    Here we've used \(\minkowskiMetric^{\rho\sigma}\minkowskiMetric_{\rho\sigma} = D \rightsquigarrow 4\) dropping the \(\varepsilon\) term as we're only interested in divergent terms when \(\varepsilon \to 0\).
    Further expanding this we get
    \begin{equation}
        -2k^\mu k^\nu + 10q^\mu q^\nu + 5(k^\mu q^\nu + k^\nu q^\mu) + \minkowskiMetric^{\mu\nu}[(q - k)^2 + (q + 2k)^2] = N^{\mu\nu}.
    \end{equation}
    
    Putting this together we can write the term corresponding to this diagram as
    \begin{equation}
        \frac{1}{2}g^2 T(\adjointRep)\delta^{ab} \int \dhat{q} \frac{N^{\mu\nu}}{q^2(k + q)^2}.
    \end{equation}
    We can now use Feynman parametrisation to write this as
    \begin{equation}
        \frac{1}{2}g^2 T(\adjointRep)\delta^{ab} \int_0^1 \dl{x} \int \dhat{q} \frac{N^{\mu\nu}}{[q^2 + x(k + q)^2]^2}.
    \end{equation}
    Expanding the denominator we have
    \begin{equation}
        q^2 + xk^2 + 2xk \cdot q + x^2q^2 = q^2 + 2xk \cdot q + x^2q^2
    \end{equation}
    since the external gluons are on-shell, so \(k^2 = 0\).
    
    We can then shift the loop momentum, defining \(\ell = q + xk\), and the result is very similar to \cref{eqn:photon-self-energy}, but with \(m = 0\), replacing \(q\) with \(k\) and \(-k\) with \(q\).
    We then have the denominator \((\ell^2 - \Delta)^2\) with \(\Delta = -x(1 - x)k^2\).
    
    The next step is to simplify the numerator using \(q = \ell - xk\).
    We can then drop any terms which are linear in \(\ell\), since these vanish in the integral.
    We can also replace \(\ell^\mu \ell^\nu\) with \(\ell^2\minkowskiMetric^{\mu\nu}/D\rightsquigarrow \ell^2\minkowskiMetric^{\mu\nu}/4\).
    Thus,
    \begin{align}
        N^{\mu\nu} &= - 2k^\mu k^\nu + 10(\ell^\mu - xk^\mu)(\ell^\nu - xk^\nu) + 5(k^\mu(\ell^\nu - xk^\nu) \notag\\
        &\qquad + k^\nu(\ell^\mu - xk^\mu)) + \minkowskiMetric^{\mu\nu}[(\ell - (x - 1)k)^2 + (\ell - (2 - x)k)^2]\\
        &\rightsquigarrow -2k^\mu k^\nu + \frac{5}{2}\ell^2\minkowskiMetric^{\mu\nu} + 10x^2 k^\mu k^\nu - 10xk^\mu k^\nu \notag\\
        &\qquad +\minkowskiMetric^{\mu\nu}[2\ell^2 + (x - 1)^2k^2 + (2 - x)^2k^2]\\
        &= (10x^2 - 10x - 2)k^\mu k^\nu + \minkowskiMetric^{\mu\nu} \left[ \frac{9}{2}\ell^2 + (5 - 2x + 2x^2)k^2 \right].
    \end{align}
    This is, unfortunately, as simple as it gets.
    
    Next we have to perform the integrals.
    To do this we use
    \begin{alignat*}{2}
        \int \dhat{^D\ell} \frac{1}{(\ell^2 - \Delta)^2} &= \frac{i}{(4\pi)^{D/2}} \Gamma(2 - D/2) \Delta^{D/2 - 2} &&= \frac{i}{16\pi^2\varepsilon} + \text{finite},\\
        \int \dhat{^D\ell} \frac{\ell^2}{(\ell^2 - \Delta)^2} &= \frac{-i}{(4\pi)^{D/2}} \frac{D}{2} \Gamma(1 - \Delta/2)\Delta^{D/2 - 2} &&= \frac{i}{16\pi^2\varepsilon} 2\Delta + \text{finite}.
    \end{alignat*}
    We get these by setting \(D = 4 - 2\varepsilon\), expanding the gamma functions, then keeping only the divergent terms as \(\varepsilon \to 0\).
    
    This gives the result
    \begin{align}
        \frac{ig^2}{16\pi^2\varepsilon} \frac{1}{2}T(\adjointRep)\delta^{ab} \int_0^1 \dl{x} \, [ \minkowskiMetric^{\mu\nu}[&(5 - 2x + 2x + 2x^2) - 9x(1 - x)]k^2 \notag\\
        &\quad+ (10x^2 - 10x - 2)k^\mu k^\nu ].
    \end{align}
    Performing the \(x\) integral we get the final result
    \begin{equation}
        \frac{ig^2}{16\pi^2\varepsilon} T(\adjointRep)\delta^{ab}\left[ \frac{19}{12}k^2\minkowskiMetric^{\mu\nu} - \frac{11}{6} k^\mu k^\nu \right].
    \end{equation}
    This doesn't have the correct tensor structure to give the counterterm.
    The solution to this problem is that we need to include the ghosts, only when combined do we get the required tensor structure.
    
    
    %%%%%%%%%%%%%%%%%%%%%%%%%%%%%%%%%%%%%%%%%%%%%%%%%%%%%%%%%%%%%%%%%%%%%%%%%%%%%%%%%%%%%%%%%
    
    
    
    
    
    
    
    \renewcommand{\amplitude}{\symcal{M}}    
    
    
    
    
    
    
    
    
    
    
    
    
    
    
    %%%%%%%%%%%%%%%%%%%%%%%%%%%%%%%%%%%%%%%%%%%%%%%%%%%%%%%%%%%%%%%%%%%%%%%%%%%%%%%%%%%%%%%%%%%%%%%%%%%%%%%%%%%%%%
    
    \part{Electroweak}
    \chapter{Overview}
    There are four fundamental forces,
    \begin{itemize}
        \item electromagnetism;
        \item the strong force;
        \item the weak force;
        \item gravity.
    \end{itemize}
    In this course we are interested in the first three.
    To each force we can associate a type of radiation,
    \begin{itemize}
        \item gamma radiation;
        \item alpha radiation;
        \item beta radiation;
        \item gravitational waves.
    \end{itemize}
    These forces all have different ranges, reflecting the different mechanisms behind the difference forces, at long ranges these forces scale as
    \begin{itemize}
        \item \(1/r\);
        \item \(\e^{-\mu r}/r\) with \(\mu \approx \qty{1}{\giga\electronvolt}\);
        \item \(\e^{-\mu r}\) with \(\mu \approx \qty{100}{\giga\electronvolt}\);
        \item \(1/r\).
    \end{itemize}
    Each of these forces can be described as a gauge theory with a different gauge group,
    \begin{itemize}
        \item \(\unitary(1)\);
        \item \(\specialUnitary(3)\);
        \item \(\specialUnitary(2)\);
        \item diffeomorphisms of spacetime\footnote{some don't consider this to be a gauge theory, for one thing the group of diffeomorphisms is non-compact, and the theory doesn't work in quite the same way as the other examples.}.
    \end{itemize}
    No one is quite sure why these particular gauge groups are the ones which apply in each case, and the choice of gauge group pretty much completely determines the physics.
    For example, the fact that \(\specialUnitary(3)\) and \(\specialUnitary(2)\) are non-Abelian is responsible for the short range interactions of the strong and weak force, which is ultimately due to the self interactions of exchange particles, which doesn't occur in the Abelian \(\unitary(1)\) case.
    
    One of the main tools in our arsenal is perturbation theory, although this doesn't work in all cases
    \begin{itemize}
        \item perturbation theory works at sufficiently low energies, including most cases of real world interest;
        \item perturbation theory works at high energies, but not low energies;
        \item perturbation theory works but is \enquote{weird} and \enquote{interesting};
        \item perturbation theory works for sufficiently small perturbations.
    \end{itemize}
    
    Symmetry is important in physics, these forces posses, or don't posses, various symmetries.
    One symmetry being preserving flavour, that is the type of fermion involved in interactions,
    \begin{itemize}
        \item flavour stays the same;
        \item flavour stays the same;
        \item flavours can change;
        \item flavours stay the same.
    \end{itemize}
    Another set of symmetries is parity (\(\parity\)), charge conjugation (\(\chargeConjugation\)), and time reversal (\(\timeReversal\)), although often due to the \(\chargeConjugation\parity\timeReversal\) theorem, which states that the combination of all three of these symmetries is respected, we replace time reversal, which is weird involving anti-unitary operations, with charge conjugation and parity transformation (\(\chargeConjugation\parity\)).
    \begin{itemize}
        \item \(\parity\), \(\chargeConjugation\), and \(\timeReversal\) are all respected;
        \item \(\parity\), \(\chargeConjugation\), and \(\timeReversal\) are all respected;
        \item \(\parity\) and \(\chargeConjugation\) are maximally violated, \(\chargeConjugation\parity\) is violated a very small amount;
        \item GR respects these symmetries, other theories of gravity do not.
    \end{itemize}
    
    Each force allows for different interactions,
    \begin{itemize}
        \item
        \tikzsetnextfilename{overview-vertex-qed}
        \begin{tikzpicture}[baseline=(base.base)]
            \draw[electron] (-1, 0) node [left] (base) {\(\Pe\)}-- (0, 0);
            \draw[electron] (0, 0) -- (1, 0) node [right] {\(\Pe\)};
            \draw[photon] (0, 0) -- (0, 1) node [above] {\(\Pphoton\)};
        \end{tikzpicture}
        \kern-0.6em, and the same with \(\Pmu\) and \(\Ptau\)
        \item
        \tikzsetnextfilename{overview-vertex-qcd-qqg}
        \begin{tikzpicture}[baseline=(base.base)]
            \draw[quark] (-1, 0) node [left] {\(\Pq\)} -- (0, 0);
            \draw[quark] (0, 0) -- (1, 0) node [right] (base) {\(\Pq\)};
            \draw[gluon] (0, 0) -- (0, 1) node [above] {\(\Pg\)};
        \end{tikzpicture}
        \kern-0.6em,
        \tikzsetnextfilename{overview-vertex-qcd-ggg}
        \begin{tikzpicture}[baseline=(base.base)]
            \draw[gluon] (0, 0) -- (210:0.7) node [left] {\(\Pg\)};
            \draw[gluon] (0, 0) -- (330:0.7) node [right] (base) {\(\Pg\)};
            \draw[gluon] (0, 0) -- (90:0.7) node [above] {\(\Pg\)};
        \end{tikzpicture}
        \kern-0.6em,
        \tikzsetnextfilename{overview-vertex-qcd-gggg}
        \begin{tikzpicture}[baseline=(base.base)]
            \draw[gluon] (0, 0) -- (45:0.7) node [right] {\(\Pg\)};
            \draw[gluon] (0, 0) -- (135:0.7) node [left] {\(\Pg\)};
            \draw[gluon] (0, 0) -- (225:0.7) node [left] (base) {\(\Pg\)};
            \draw[gluon] (0, 0) -- (315:0.7) node [right] {\(\Pg\)};
        \end{tikzpicture}
        \item \begingroup\raggedright
        \tikzsetnextfilename{overview-vertex-ew-enuw}
        \begin{tikzpicture}[baseline=(base.base)]
            \draw[electron] (-1, 0) node [left] {\(\PeLeft\)}-- (0, 0);
            \draw[electron] (0, 0) -- (1, 0) node [right] (base) {\(\PnueLeft\)};
            \draw[WZ boson] (0, 0) -- (0, 1) node [above] {\(\PW\)};
        \end{tikzpicture}
        \kern-0.55em, and the same with \(\PmuLeft\) and \(\PnumuLeft\), and \(\PtauLeft\) and \(\PnutauLeft\),\\
        \tikzsetnextfilename{overview-vertex-ew-eez}
        { % don't line break before the comma
        \begin{tikzpicture}[baseline=(base.base)]
            \draw[electron] (-1, 0) node [left] (base) {\(\Pe\)}-- (0, 0);
            \draw[electron] (0, 0) -- (1, 0) node [right] {\(\Pe\)};
            \draw[photon] (0, 0) -- (0, 1) node [above] {\(\PZ\)};
        \end{tikzpicture}
        \kern-0.6em,} and the same with \(\Pmu\) and \(\Ptau\),\\
        { % don't line break before the comma
        \tikzsetnextfilename{overview-vertex-ew-udw}
        \begin{tikzpicture}[baseline=(base.base)]
            \draw[quark] (-1, 0) node [left] {\(\Pu_{\Left}\)}-- (0, 0);
            \draw[quark] (0, 0) -- (1, 0) node [right] (base) {\(\Pd_{\Left}\)};
            \draw[WZ boson] (0, 0) -- (0, 1) node [above] {\(\PW\)};
        \end{tikzpicture}
        \kern-0.6em,} and the same with \(\Pc_{\Left}\) and \(\Ps_{\Left}\), and \(\Pt_{\Left}\) and \(\Pb_{\Left}\),\\
        \tikzsetnextfilename{overview-vertex-ew-qqz}
        \begin{tikzpicture}[baseline=(base.base)]
            \draw[electron] (-1, 0) node [left] (base) {\(\Pq\)}-- (0, 0);
            \draw[electron] (0, 0) -- (1, 0) node [right] {\(\Pq\)};
            \draw[photon] (0, 0) -- (0, 1) node [above] {\(\PZ\)};
        \end{tikzpicture}
        \kern-0.55em,
        \tikzsetnextfilename{overview-vertex-ew-www}
        \begin{tikzpicture}[baseline=-0.31cm]
            \draw[WZ boson] (0, 0) -- (210:0.7);
            \draw[WZ boson] (0, 0) -- (330:0.7);
            \draw[WZ boson] (0, 0) -- (90:0.7);
        \end{tikzpicture}
        and
        \tikzsetnextfilename{overview-vertex-wwww}
        \begin{tikzpicture}[baseline=-0.5cm]
            \draw[WZ boson] (0, 0) -- (45:0.7);
            \draw[WZ boson] (0, 0) -- (135:0.7);
            \draw[WZ boson] (0, 0) -- (225:0.7);
            \draw[WZ boson] (0, 0) -- (315:0.7);
        \end{tikzpicture}
        with the particles being any of \(\PW\), \(\PZ\), or \(\Pphoton\), so long as charge is conserved, 
        \tikzsetnextfilename{overview-vertex-ew-ffh}
        \begin{tikzpicture}[baseline=(base.base)]
            \draw[electron] (-1, 0) node [left] (base) {\(\Pf\)}-- (0, 0);
            \draw[electron] (0, 0) -- (1, 0) node [right] {\(\Pf\)};
            \draw[higgs] (0, 0) -- (0, 1) node [above] {\(\Phiggs\)};
        \end{tikzpicture}
        \kern-0.5em,\\
        \tikzsetnextfilename{overview-vertex-ew-wwh}
        \begin{tikzpicture}[baseline=(base.base)]
            \draw[WZ boson] (-1, 0) node [left] (base) {\(\PW\)/\(\PZ\)}-- (0, 0);
            \draw[WZ boson] (0, 0) -- (1, 0) node [right] {\(\PW\)/\(\PZ\)};
            \draw[higgs] (0, 0) -- (0, 1) node [above] {\(\Phiggs\)};
        \end{tikzpicture}
        \kern-0.5em,
        \tikzsetnextfilename{overview-vertex-wwhh}
        { % don't line break before the comma
        \begin{tikzpicture}[baseline=(base.base)]
            \draw[higgs] (0, 0) -- (45:0.7) node [right] {\(\Phiggs\)};
            \draw[higgs] (0, 0) -- (135:0.7) node [left] {\(\Phiggs\)};
            \draw[WZ boson] (0, 0) -- (225:0.7) node [left] (base) {\(\PW\)/\(\PZ\)};
            \draw[WZ boson] (0, 0) -- (315:0.7) node [right] {\(\PW\)/\(\PZ\)};
        \end{tikzpicture}
        \kern-0.55em,}
        \tikzsetnextfilename{overview-vertex-hhh}
        \begin{tikzpicture}[baseline=(base.base)]
            \draw[higgs] (0, 0) -- (210:0.7) node [left] {\(\Phiggs\)};
            \draw[higgs] (0, 0) -- (330:0.7) node [right] (base) {\(\Phiggs\)};
            \draw[higgs] (0, 0) -- (90:0.7) node [above] {\(\Phiggs\)};
        \end{tikzpicture}
        \kern-0.6em,\\
        and
        \tikzsetnextfilename{overview-vertex-hhhh}
        \begin{tikzpicture}[baseline=(base.base)]
            \draw[higgs] (0, 0) -- (45:0.7) node [right] {\(\Phiggs\)};
            \draw[higgs] (0, 0) -- (135:0.7) node [left] {\(\Phiggs\)};
            \draw[higgs] (0, 0) -- (225:0.7) node [left] (base) {\(\Phiggs\)};
            \draw[higgs] (0, 0) -- (315:0.7) node [right] {\(\Phiggs\)};
        \end{tikzpicture}
        , note that interactions involving a photon or Higgs boson are due to the electroweak interaction, rather than just the weak interaction;
        \endgroup
        \item depends on your theory of gravity.
    \end{itemize}
    
    \section{The Weak Force}
    As we saw in the previous section the weak force is quite different from electromagnetism and the strong force, which really only differ due to \(\unitary(1)\) being Abelian and \(\specialUnitary(2)\) being non-Abelian.
    The weak force allows for flavour change, and includes many more interactions than either electromagnetism or the strong interaction.
    
    Examples of processes mediated by the weak force are \defineindex{beta decay},
    \begin{equation}
        \Pneutron \to \Pproton \Pe \APnue, \qquad \Pu\Pd\Pd \to \Pu\Pu\Pd \Pe \APnue,
    \end{equation}
    \define{pion decays}\index{pion decay}, such as
    \begin{equation}
        \Ppim \to \Pmu\APnumu, \qquad \Ppim \to \Pe\APnue, \qquad \Ppizero \to \Pphoton\Pphoton, \qquad \Ppizero \to \Pphoton \Pe \APe,
    \end{equation}
    and \define{kaon decays}\index{kaon decay}, such as
    \begin{equation}
        \PKm \to \Ppim \Ppizero, \qquad \PKm \to \Ppim \Ppim \Ppip.
    \end{equation}
    
    The violation of symmetries and the ability of the weak force to change the flavour of particles makes the world a much more interesting place.
    Without out this the universe would be static, up to uninteresting dynamical changes, with lepton and quark number remaining constant.
    The fact that the weak force allows for decays also means that if we reverse time, despite the symmetry breaking, the weak force allows for creation of particles as well, and this is one of the major avenues of study for early universe physics.
    
    In this course we will follow a historical approach, piecing together first the theory of the weak interaction, and then combining this with electromagnetism to get electroweak theory.
    As such there will be points where we study theories which later turned out to be wrong, or at least not the complete picture.
    
    The theory of weak interactions begins with Pauli in 1930 who was studying the kinematics of beta decay and realised that seemingly missing energy could be explained if the decay was producing another particle, the neutrino.
    The neutrino's existence was the confirmed experimentally in 1956 by Frederick Reines and Clyde Cowan\footnote{Nobel prize for (electro)weak theory count: 1}.
    In the time between positing the existence of the neutrino and its experimental confirmation the theory of the weak interaction had progressed a lot.
    In the same year Chien-Shiung Wu\footnote{Nobel prize for (electro)weak theory count: 2} observed parity violation for the first time, which was a big surprise, since the prevailing wisdom was that parity was always conserved.
    
    Before we can discus the weak force further we need to recap chirality, helicity, and parity.
    
    \chapter{Left and Right}
    \section{Chirality}
    Recall that the fifth gamma matrix is defined as
    \begin{equation}
        \gamma^5 = i\gamma^0 \gamma^1 \gamma^2 \gamma^3,
    \end{equation}
    and is such that \((\gamma^5)^2 = 1\), \((\gamma^5)^\hermit = \gamma^5\), and \(\anticommutator{\gamma^5}{\gamma^\mu} = 0\).
    Using this we can define a complete set of Hermitian, orthogonal projection operators:
    \begin{equation}
        P_{\Left} \coloneqq \frac{1}{2}(1 - \gamma^5), \qqand P_{\Right} \coloneqq \frac{1}{2}(1 + \gamma^5).
    \end{equation}
    We call these the left and right projection operators respectively, we'll justify this name later.
    Breaking it down \enquote{complete set of Hermitian, orthogonal projection operators} tells us the following:
    \begin{itemize}
        \item Projectors: \(P_{\Left}^2 = (1 - \gamma^5)^2/4 = (1 - 2\gamma^5 + 1)/4 = (1 - \gamma^5)/2 = P_{\Left}\), and similarly \(P_{\Right}^2 = P_{\Right}\);
        \item Orthogonal: \(P_{\Left}P_{\Right} = (1 - \gamma^5)(1 + \gamma^5)/2 = (1 - (\gamma^5)^2)/2 = (1 - 1)/2 = 0\), and similarly \(P_{\Right}P_{\Left} = 0\);
        \item Complete: \(P_{\Left} + P_{\Right} = (1 - \gamma^5)/2 + (1 + \gamma_5)/2 = 1\);
        \item Hermitian: \(P_{\Left}^\hermit = [(1 - \gamma^5)/2]^{\hermit} = (1 - (\gamma^5)^\hermit)/2 = (1 - \gamma^5)/2 = P_{\Left}\), and similarly \(P_{\Right}^\hermit = P_{\Right}\).
    \end{itemize}
    Note that we have \(P_{\Right} - P_{\Left} = (1 + \gamma^5)/2 - (1 - \gamma^5)/2 = \gamma^5\).
    Another useful property is that commuting a \(\gamma^\mu\) with a projector turns left into right and vice versa, so \(\gamma^\mu P_{\Left} = \gamma^\mu(1 - \gamma^5)/2 = (\gamma^\mu - \gamma^\mu \gamma^5)/2 = (\gamma^\mu + \gamma^5 \gamma^\mu)/2 = (1 + \gamma^5)\gamma^\mu/2 = P_{\Right}\gamma^\mu\), and similarly \(\gamma^\mu P_{\Right} = P_{\Left}\gamma^\mu\).
    
    Given a (Dirac) spinor, \(\psi\), we can decompose it into left and right components using
    \begin{equation}
        \psi = 1\psi = (P_{\Left} + P_{\Right})\psi = P_{\Left}\psi + P_{\Right}\psi = \psi_{\Left} + \psi_{\Right},
    \end{equation}
    where
    \begin{equation}
        \psi_{\Left} \coloneqq P_{\Left}\psi, \qqand \psi_{\Right} \coloneqq P_{\Right}\psi
    \end{equation}
    are the left and right components of \(\psi\).
    
    For the adjoint spinor left and right are swapped around, that is
    \begin{equation}
        \diracadjoint{\psi}P_{\Left} = \psi^\hermit \gamma^0 P_{\Left} = \psi^\hermit P_{\Right} \gamma^0 = (P_{\Right} \psi)^\hermit \gamma^0 = \psi_{\Right}^\hermit \gamma^0 = \overline{\psi_{\Right}}.
    \end{equation}
    Note that this is the adjoint of the right component of \(\psi\), rather than the right component of the adjoint, which we might write as \(\diracadjoint{\psi}_{\Right}\).
    However, this notation is confusing and so we assume that \(\Right\) or \(\Left\) is always applied to the spinor before we take the adjoint, and write this as \(\diracadjoint{\psi}_{\Right}\).
    Some brackets may help here, we have
    \begin{equation}
        \diracadjoint{\psi}P_{\Left} = (\diracadjoint{\psi})_{\Left} = \diracadjoint{\psi}_{\Right}.
    \end{equation}
    Similarly, we have \(\diracadjoint{\psi}P_{\Right} = (\diracadjoint{\psi})_{\Right} = \diracadjoint{\psi}_{\Left}\).
    
    Now consider the mass term in the Dirac Lagrangian, \(\diracadjoint{\psi}\psi\).
    We have
    \begin{alignat}{3}
        \diracadjoint{\psi} \psi &= \diracadjoint{\psi}(\psi_{\Right} + \psi_{\Left}) \qquad\qquad && \text{completeness}\\
        &= \diracadjoint{\psi}\psi_{\Right} + \diracadjoint{\psi}\psi_{\Left} \qquad\qquad && \\
        &= \diracadjoint{\psi}P_{\Right}\psi + \diracadjoint{\psi}P_{\Left}\psi \qquad\qquad && \\
        &= \diracadjoint{\psi}P_{\Right}^2\psi + \diracadjoint{\psi}P_{\Left}^2\psi \qquad\qquad && \text{idempotency}\\
        &= \diracadjoint{\psi}_{\Left}\psi_{\Right} + \diracadjoint{\psi}_{\Right}\psi_{\Left}. \qquad &&
    \end{alignat}
    From this we can see that the mass term mixes left and right spinors.
    We'll see that this is a general property of massive particles, they have some mixing of parities.
    
    Consider a fermion current, \(\diracadjoint{\psi}\gamma^\mu \psi\), we have
    \begin{alignat}{3}
        \diracadjoint{\psi} \gamma^\mu \psi &= \diracadjoint{\psi}\gamma^\mu(\psi_{\Right} + \psi_{\Left}) \qquad && \text{completeness}\\
        &= \diracadjoint{\psi} \gamma^\mu P_{\Right}\psi + \diracadjoint{\psi} \gamma^\mu P_{\Left} \psi \qquad &&\\
        &= \diracadjoint{\psi} \gamma^\mu P_{\Right}^2\psi + \diracadjoint{\psi} \gamma^\mu P_{\Left}^2 \psi \qquad && \text{idempotency}\\
        &= \diracadjoint{\psi} P_{\Left} \gamma^\mu P_{\Right} \psi + \diracadjoint{\psi} P_{\Right} \gamma^\mu P_{\Left} \psi \qquad && \gamma^\mu P_{\Left} = P_{\Right}\gamma^\mu \text{ and } \gamma^\mu P_{\Right} = P_{\Left}\gamma^\mu\\
        &= \diracadjoint{\psi}_{\Right} \gamma^\mu \psi_{\Right} + \psi_{\Left} \gamma^\mu \psi_{\Left}. \qquad &&
    \end{alignat}
    From this we see that the current doesn't mix left and right.
    
    The Dirac Lagrangian is then
    \begin{align}
        \lagrangianDensity_{\symrm{D}} &= \diracadjoint{\psi}(i\slashed{\partial} - m)\psi\\
        &= \diracadjoint{\psi}_{\Right}i\slashed{\partial}\psi_{\Right} + m \diracadjoint{\psi}_{\Left}\psi_{\Right} + \diracadjoint{\psi}_{\Left} i\slashed{\partial} \psi_{\Left} + m \diracadjoint{\psi}_{\Right}\psi_{\Left}.
    \end{align}
    Again, we see that the mass term mixes the left and right.
    The Dirac Lagrangian gives rise to the Dirac equation, \((i\slashed{\partial} - m)\psi\).
    For a massless particle this gives two independent equations for the left and right components, 
    \begin{equation}
        i\slashed{\partial}\psi_{\Left} = 0, \qqand i\slashed{\partial}\psi_{\Right} = 0.
    \end{equation}
    These are known as the \define{Weyl equations}\index{Weyl equation}.
    Importantly for massless particles the left and right components evolve separately.
    
    Currently we are treating \(\psi_{\Left}\) and \(\psi_{\Right}\) as four component Dirac spinors projected onto a two-dimensional subspace.
    Instead we can treat them as two component Weyl spinors.
    We won't do this.
    
    The operators \(P_{\Left}\) and \(P_{\Right}\) define what we call the \defineindex{chirality} of the particle, for example, a right handed chiral state is an eigenstate of \(P_{\Right}\) with eigenvalue 1.
    
    \section{Helicity}
    For a massive fermion the \defineindex{spin operator} is\footnote{note that this differs by a sign from the definition in \course{Quantum Field Theory}, by a factor of 2 from the definition in the notes, and a factor of 4 from the value given in the lecture, I've checked it works as written here in the Dirac representation.}
    \begin{align}
        \Sigma^i &\coloneqq \frac{i}{4}\varepsilon^{ijk}\commutator{\gamma^j}{\gamma^k}\\
        &= \frac{i}{4}\varepsilon^{ijk}(\gamma^j\gamma^k - \gamma^k\gamma^j)\\
        &= \frac{i}{4}(\varepsilon^{ijk}\gamma^j\gamma^k - \varepsilon^{ijk}\gamma^k\gamma^j)\\
        &= \frac{i}{4}(\varepsilon^{ijk}\gamma^j\gamma^k - \varepsilon^{ikj}\gamma^j\gamma^k)\\
        &= \frac{i}{4}(\varepsilon^{ijk}\gamma^j\gamma^k + \varepsilon^{ijk}\gamma^j\gamma^k)\\
        &= \frac{i}{2}\varepsilon^{ijk}\gamma^j\gamma^k.
    \end{align}
    Alternatively we could have used the antisymmetry of \(\varepsilon^{ijk}\) to replace \(\commutator{\gamma^j}{\gamma^k}\) with \(\commutator{\gamma^j}{\gamma^k} + \anticommutator{\gamma^j}{\gamma^k} = 2\gamma^j\gamma^k\).
    Either way, \(\Sigma^1 = i\varepsilon^{1jk}\gamma^j\gamma^k/2 = (i\gamma^2\gamma^3 - i\gamma^3\gamma^2)/2 = i\gamma^2\gamma^3\), and similarly \(\Sigma^2 = i\gamma^3\gamma^1\) and \(\Sigma^3 = i\gamma^1\gamma^2\).
    We can write all of this as \(\Sigma^i = \gamma^5\gamma^0\gamma^i\), which follows by expanding \(\gamma^5\):
    \begin{equation}
        \gamma^5\gamma^0\gamma^i = i\gamma^0\gamma^1\gamma^2\gamma^3\gamma^0\gamma^i = -i(\gamma^0)^2\gamma^1\gamma^2\gamma^3\gamma^i = -i\gamma^1\gamma^2\gamma^3\gamma^i,
    \end{equation}
    and so we have \(\gamma^5\gamma^0\gamma^1 = -i\gamma^1\gamma^2\gamma^3\gamma^1 = -i(\gamma^1)^2\gamma^2\gamma^3 = i\gamma^2\gamma^3\), and similarly \(\gamma^5\gamma^0\gamma^2 = -i\gamma^1\gamma^3\) and \(\gamma^5\gamma^0\gamma^3 = i\gamma^1\gamma^2\), which shows that
    \begin{equation}
        \Sigma^i = \frac{i}{4}\varepsilon^{ijk}\commutator{\gamma^j}{\gamma^k} = \gamma^5\gamma^0\gamma^i.
    \end{equation}
    
    Now consider the commutator
    \begin{align}
        \commutator{P_{\Left}}{\Sigma^i} &= \frac{1}{2}(\commutator{1}{\Sigma^i} - \commutator{\gamma^5}{\Sigma^i})\\
        &= -\frac{1}{2}\commutator{\gamma^5}{\gamma^5\gamma^0\gamma^i}\\
        &= -\frac{1}{2}((\gamma^5)^2\gamma^0\gamma^i - \gamma^5\gamma^0\gamma^i\gamma^5)\\
        &= -\frac{1}{2}(\gamma^0\gamma^i - (\gamma^5)^2\gamma^0\gamma^i)\\
        &= -\frac{1}{2}(\gamma^0\gamma^i - \gamma^0\gamma^i)\\
        &= 0.
    \end{align}
    Similarly, \(\commutator{P_{\Right}}{\Sigma^i} = 0\).
    This means that spin and chirality commute, and so we can measure both at once.
    
    For massless particles instead of spin we have \defineindex{helicity}, given by the helicity operator\footnote{note that this differs from the value given in the notes by a factor of 2.}
    \begin{equation}
        h = \frac{\vv{\Sigma} \cdot \vv{p}}{\abs{\vv{p}}}.
    \end{equation}
    This has eigenvalues \(\pm 1\), which can be shown by considering the case where \(\vv{p} = (0, 0, p)\), in which case \(h = \Sigma^3 = \diag(1, -1, 1, -1)\), and so the eigenvalues are \(\pm 1\), each with multiplicity 2.
    States with helicity \(+1\) are called \defineindex{right handed}, because their spin aligns with their momentum.
    Interpreting the spin as the angular momentum of the particle this means that the particle \enquote{spins} around the momentum axis in accordance with the right hand grip rule.
    Similarly states with helicity \(-1\) are called \defineindex{left handed}, as their spin goes in the opposite direction around the momentum, so is instead given by a left hand grip rule.
    This is shown in \cref{fig:helicity}.
    Importantly if momentum and spin are aligned then it is a right handed helicity state, and if they are antialigned it is a left handed helicity state.
    
    \begin{figure}
        \tikzsetnextfilename{left-right-helicity}
        \begin{tikzpicture}
            \draw[very thick, ->] (0, 0) -- (0, 3) node [above] {\(\vv{p}\)};
            \draw[line width=1mm, white, yshift=1.6cm, xshift=-0.075cm, yscale=0.4] (100:0.1) arc (100:440:0.5);
            \draw[->, thick, yshift=1.6cm, xshift=-0.075cm, yscale=0.4] (100:0.1) arc (100:440:0.5);
            \draw[->, thick] (0.7, 1.2) -- (0.7, 1.8) node [above] {\(\vv{S}\)};
            \node[below] at (0, 0) {Right Handed};
            \begin{scope}[xshift=3cm]
                \draw[very thick, ->] (0, 0) -- (0, 3) node [above] {\(\vv{p}\)};
                \draw[line width=1mm, white, yshift=1.6cm, xshift=-0.075cm, yscale=0.4] (100:0.1) arc (100:440:0.5);
                \draw[<-, thick, yshift=1.6cm, xshift=-0.075cm, yscale=0.4] (100:0.1) arc (100:440:0.5);
                \draw[<-, thick] (0.7, 1.2) node [below] {\(\vv{S}\)} -- (0.7, 1.8);
                \node[below] at (0, 0) {Left Handed};
            \end{scope}
        \end{tikzpicture}
        \caption{Left and right helicities showing the relation between spin and momentum.}
        \label{fig:helicity}
    \end{figure}
    
    Now consider a spinor, \(u^{\pm}(p)\), in momentum space which is a positive energy solution to the Dirac equation, that is
    \begin{equation}
        (\slashed{p} - m)u^{\pm}(p) = 0 \implies \slashed{p}u^{\pm}(p) = mu^{\pm}(p),
    \end{equation}
    with helicity \(\pm 1\), that is \(hu^{\pm1}(p) = \pm u^{\pm}(p)\).
    
    Note that we have
    \begin{equation}
        \slashed{p} = \gamma^0E - \vv{\gamma} \cdot \vv{p} \implies \vv{\gamma} \cdot \vv{p} = \gamma^0 E - \slashed{p},
    \end{equation}
    and so we have
    \begin{equation}
        h = \frac{1}{\abs{\vv{p}}}\Sigma^ip^i = \frac{1}{\abs{\vv{p}}}\gamma^5\gamma^0\gamma^ip^i = \frac{1}{\abs{\vv{p}}}\gamma^5\gamma^0 \vv{\gamma} \cdot \vv{p} = \frac{\gamma^5\gamma^0}{\abs{\vv{p}}}(\gamma^0E - \slashed{p}).
    \end{equation}
    Multiplying through the factor of \(\gamma^0\) and using \((\gamma^0)^2 = 1\) this becomes
    \begin{equation}
        h = \frac{\gamma^5}{\abs{\vv{p}}} (E - \gamma^0\slashed{p}).
    \end{equation}
    Acting with this on \(u^{\pm}(p)\) on the one hand should give \(\pm u^{\pm}(p)/\abs{\vv{p}}\), and on the other hand gives
    \begin{equation}
        \frac{\gamma^5}{\abs{\vv{p}}}(E - \gamma^0\slashed{p})u^{\pm}(p) = \frac{\gamma^5}{\abs{\vv{p}}}(-E - \gamma^0m)u^{\pm}(p).
    \end{equation}
    That is,
    \begin{equation}
        \gamma^5(E - \gamma^0m)u^{\pm}(p) = \pm\abs{\vv{p}}u^{\pm}(p).
    \end{equation}
    Now use \(\gamma^5 = P_{\Right} - P_{\Left}\) and insert a factor of \(1 = P_{\Right} + P_{\Left}\) on the right to get
    \begin{equation}
        (P_{\Right} - P_{\Left})(E - \gamma^0m)u^{\pm}(p) = \pm\abs{\vv{p}}(P_{\Right} + P_{\Left})u^{\pm}(p).
    \end{equation}
    Since the projectors are orthogonal we can read off the left and right components from this separately, for example taking just the right handed component gives
    \begin{equation}
        P_{\Right}(E - \gamma^0m)u^{\pm}(p) = \pm\abs{\vv{p}}P_{\Right}u^{\pm}(p).
    \end{equation}
    Commuting the \(P_{\Right}\) through \(\gamma^0\) turns it into \(P_{\Left}\) and then acting on the spinor we have
    \begin{equation}
        Eu^{\pm}_{\Right}(p) - \gamma^0mu^{\pm}_{\Left}(p) = \pm\abs{\vv{p}}u^{\pm}_{\Right}(p).
    \end{equation}
    Similarly, reading off the left handed component we have
    \begin{equation}
        -P_{\Left}(E - \gamma^0m)u^{\pm}(p) = \pm\abs{\vv{p}}P_{\Left}u^{\pm}(p),
    \end{equation}
    which gives
    \begin{equation}
        -Eu^{\pm}_{\Left}(p) + \gamma^0mu^{\pm}_{\Right}(p) = \pm\abs{\vv{p}}u^{\pm}_{\Left}(p).
    \end{equation}
    Rearranging these two results gives
    \begin{align}
        (E \mp \abs{\vv{p}})u^{\pm}_{\Right}(p) &= m\gamma^0u^{\pm}_{\Left}(p),\\
        (E \pm \abs{\vv{p}})u^{\pm}_{\Left}(p) &= m\gamma^0u^{\pm}_{\Right}(p).
    \end{align}
    Again, we see that if \(m \ne 0\) then we get mixing of left and right handed states.
    
    Now consider the \(m \to 0\) limit, then we have
    \begin{equation}
        E^2 = \abs{\vv{p}}^2 + m^2 \implies \abs{\vv{p}} = \sqrt{E^2 - m^2} \approx E + \order\left( \frac{m^2}{E^2} \right).
    \end{equation}
    So when we choose \(+\) in the sum on the left hand side of these two equations we get \(2E\), and the right hand side vanishes as \(m \to 0\), giving
    \begin{equation}
        2Eu^{-}_{\Right}(p) = 0, \qqand 2Eu^{+}_{\Left} = 0.
    \end{equation}
    That is, \(u^{-}_{\Right} = u^{+}_{\Left} = 0\).
    
    This means that the spinor \(u_{\Right} = u^+_{\Right} + u^-_{\Right} = u^+_{\Right}\) has helicity \(+1\), so is right handed.
    Similarly, \(u_{\Left} = u^+_{\Left} + u^-_{\Left} = u^-_{\Left}\) has helicity \(-1\), so is left handed.
    This justifies our choice to call the chirality states left and right, it's because in the case of massless particles left and right chiral states coincide with left and right helicity states.
    
    For \(m \ne 0\) but still \(m \ll E\) we have
    \begin{equation}
        u^-_{\Right} = \frac{m\gamma^0}{E + \abs{\vv{p}}}u^-_{\Left} \approx \frac{m\gamma^0}{2E}u^-_{\Left},
    \end{equation}
    which means that the right handed chiral component of a left handed helicity eigenstate, that is \(u^-_{\Right}\), is small.
    Similarly the left handed chiral component of a right handed helicity eigenstate, that is \(u^+_{\Left}\), is small.
    
    \section{Discrete Symmetries}
    \subsection{Parity}
    Consider the parity transformation, \(\parity\), defined on a spinor as
    \begin{equation}
        \psi \xmapsto{\parity} \psi_{\parity} \coloneqq \gamma^0\psi.
    \end{equation}
    Thus, we have
    \begin{equation}
        \psi_{\Left} \xmapsto{\parity} \gamma^0\psi_{\Left} = \gamma^0 P_{\Left} \psi = P_{\Right} \gamma^0 \psi = (\psi_{\parity})_{\Right}.
    \end{equation}
    Similarly,
    \begin{equation}
        \psi_{\Right} \xmapsto{\parity} (\psi_{\parity})_{\Left}.
    \end{equation}
    That is, \((\psi_{\Left})_{\parity} = (\psi_{\parity})_{\Right}\) and \((\psi_{\Right})_{\parity} = (\psi_{\Left})_{\parity}\).
    This shows that parity swaps left and right, further justifying us naming these projectors left and right in the first place.
    
    Since
    \begin{equation}
        \commutator{\gamma^0}{\Sigma^i} = \gamma^0\gamma^5\gamma^0\gamma^i - \gamma^5\gamma^0\gamma^i\gamma^0 = -\gamma^5(\gamma^0)^2\gamma^i + \gamma^5(\gamma^0)^2\gamma^i = 0
    \end{equation}
    parity doesn't change spin.
    Under parity the three-momentum is reversed,
    \begin{equation}
        \vv{p} \xmapsto{\parity} -\vv{p}.
    \end{equation}
    Thus, the helicity is reversed,
    \begin{equation}
        h = \frac{\vv{\Sigma} \cdot \vv{p}}{\abs{\vv{p}}} \xmapsto{\parity} \frac{\vv{\Sigma} \cdot (-\vv{p})}{\abs{-\vv{p}}} = -\frac{\vv{\Sigma} \cdot \vv{p}}{\abs{\vv{p}}} = -h.
    \end{equation}
    So under a parity transformation
    \begin{equation}
        u^+_{\Right} \xmapsto{\parity} u^-_{\Left}.
    \end{equation}
    That is, parity transformation swap left and right chiralities and left and right helicities, all while leaving spin unchanged.

    \subsection{Charge Conjugation}
    Consider charge conjugation, \(\chargeConjugation\), defined on a spinor as
    \begin{equation}
         \psi \xmapsto{\chargeConjugation} \psi_{\chargeConjugation} \coloneqq C \diracadjoint{\psi}^\trans
    \end{equation}
    where \(C = i \gamma^2 \gamma^0\).
    We then have
    \begin{equation}
        P_{\Left}C = iP_{\Left}\gamma^2\gamma^0 = i\gamma^2P_{\Right}\gamma^0 = i\gamma^2\gamma^0P_{\Left} = CP_{\Left}.
    \end{equation}
    Therefore
    \begin{equation}
        \psi_{\Left} \xmapsto{\chargeConjugation} C(\diracadjoint{\psi})_{\Left}^{\trans}.
    \end{equation}
    So charge conjugation leaves chirality unchanged.
    Similarly, the spin and momentum, and hence momentum, are not changed by charge conjugation.
    The only thing that changes is particles become antiparticles and vice versa.
    
    \subsection{Time Reversal}
    Consider time reversal, \(\timeReversal\), defined on a spinor as
    \begin{equation}
        \psi \xmapsto{\timeReversal} \psi_{\timeReversal} = T\psi
    \end{equation}
    with \(T = i\gamma^1\gamma^3 = -i\gamma^5C\) and \(T^{\hermit} = T = T^{-1}\).
    We then again have \(P_{\Left}T = TP_{\Left}\), and so
    \begin{equation}
        \psi_{\Left} \xmapsto{\timeReversal} T\psi_{\Left}.
    \end{equation}
    Time reversal reverses momentum, and also spin, since both are dynamical and so reverse if we do things backwards.
    This means helicity is left unchanged, since both spin and momentum are reversed and the two minus signs cancel.
    
    \section{Chiral Representation}
    In the \defineindex{chiral representation} the gamma matrices are
    \begin{equation}
        \gamma^0 = 
        \begin{pmatrix}
            0 & 1\\
            1 & 0
        \end{pmatrix}
        , \qquad \gamma^i = 
        \begin{pmatrix}
            0 & -\sigma^i\\
            \sigma^i & 0
        \end{pmatrix}
        , \qqand \gamma^5 = 
        \begin{pmatrix}
            1 & 0\\
            0 & -1
        \end{pmatrix}
        .
    \end{equation}
    Then the projection operators are
    \begin{equation}
        P_{\Right} = 
        \begin{pmatrix}
            1 & 0\\
            0 & 0
        \end{pmatrix}
        , \qqand P_{\Left} = 
        \begin{pmatrix}
            0 & 0\\
            0 & 1
        \end{pmatrix}
        .
    \end{equation}
    These act on the spinor
    \begin{equation}
        \psi = 
        \begin{pmatrix}
            \psi_{\Right}\\ \psi_{\Left}
        \end{pmatrix}
        .
    \end{equation}
    
    The spin operator is
    \begin{equation}
        \Sigma^i = \gamma^5\gamma^0\gamma^i = 
        \begin{pmatrix}
            \sigma^i & 0\\
            0 & \sigma^i
        \end{pmatrix}
        ,
    \end{equation}
    which gives the four helicity eigenstates
    \begin{equation}
        \begin{pmatrix}
            \psi_{\Right}\\ 0
        \end{pmatrix}
        , \qqand 
        \begin{pmatrix}
            0\\ \psi^{\pm}_{\Left}
        \end{pmatrix}
        .
    \end{equation}
    
    Consider the positive and negative energy solutions to the momentum space Dirac equation, \((\slashed{p} - m)u = 0\), these are such that
    \begin{equation}
        \begin{pmatrix}
            -m & E + \vv{\sigma} \cdot \vv{p}\\
            E - \vv{\sigma} \cdot \vv{p} & -m
        \end{pmatrix}
        \begin{pmatrix}
            u_{\Right}\\ u_{\Left}
        \end{pmatrix}
        = 0.
    \end{equation}
    We can then use \((\vv{\sigma} \cdot \vv{p})u_{\Left,\Right} = \pm pu_{\Left,\Right}^{\pm}\), so \((E \pm p)u_{\Left}^{\pm} = mu_{\Right}^{\pm}\).
    Since \(E^2 = p^2 + m^2\) we have
    \begin{equation}
        (E \pm p) u_{\Left}^{\pm} = \sqrt{(E + p)(E - p)}u_{\Right}^{\pm}
    \end{equation}
    which gives
    \begin{equation}
        \sqrt{E \pm p}u_{\Left}^{\pm} = \sqrt{E \mp p}u_{\Right}^{\pm}.
    \end{equation}
    So we can write
    \begin{equation}
        u^{\pm} = 
        \begin{pmatrix}
            \sqrt{E \pm p}\xi^{\pm}\\
            \sqrt{E \mp p}\xi^{\pm}
        \end{pmatrix}
    \end{equation}
    where
    \begin{equation}
        \xi^{+} = 
        \begin{pmatrix}
            1\\ 0
        \end{pmatrix}
        , \qqand \xi^{0} = 
        \begin{pmatrix}
            0\\ 1
        \end{pmatrix}
        .
    \end{equation}
    These are such that \(\diracadjoint{u}u = 2m\) and \(u^\hermit u = 2E\) whenever all spinors are evaluated at the same momentum.
    
    Similarly, for the negative energy solutions, \((\slashed{p} + m)v = 0\), we have \((\vv{\sigma} \cdot \vv{p})v^{\pm} = \mp pv^{\pm}\) and so
    \begin{equation}
        (E \mp p)v_{\Left}^{\pm} = -mv_{\Right}^{\pm}
    \end{equation}
    which gives
    \begin{equation}
        \sqrt{E \mp p}v_{\Left}^{\pm} = -\sqrt{E \pm p}v_{\Right}^{\pm}.
    \end{equation}
    Then we can write
    \begin{equation}
        v^{\pm} = 
        \begin{pmatrix}
            \sqrt{E \mp p}\xi^{\pm}\\
            -\sqrt{E \pm p}\xi^{\pm}
        \end{pmatrix}
        .
    \end{equation}
    
    Recall that we used
    \begin{equation}
        \chi^{+} =
        \begin{pmatrix}
            0\\ 1
        \end{pmatrix}
        , \qqand \chi^{-} = 
        \begin{pmatrix}
            -1\\ 0
        \end{pmatrix}
        ,
    \end{equation}
    as a basis for the negative energy solutions in the Dirac representation.
    This is related to \(\xi^{\pm}\) by charge conjugation, using \(v^{\pm} = i\gamma^2 u^{\pm *}\).
   
    \chapter{Charged Current Weak Interactions}
    \section{Fermi Theory}
    After Pauli theorised the existence of the neutrino to explain seeming violations of conservation of energy and momentum a new theory was needed to explain an interaction involving four particles.
    The first attempt was \defineindex{Fermi theory}.
    Fermi hypothesised a four point interaction, which we now call the four-Fermi interaction, involving four fermions.
    The assumption was that this interaction must be similar to QED.
   
   For the case of beta decay, \(\Pn \to \Pp \Pe \APnue\), the four-Fermi interaction takes the form
    \begin{equation}
        \fermiConst (\diracadjoint{\Pn} \gamma_\mu \Pp)(\diracadjoint{\Pnu} \gamma^\mu \Pex)
    \end{equation}
    where we are using the symbols for the particles to represent their spinors, so \(\Pex\) is a spinor corresponding to the electron, \(\diracadjoint{\Pn}\) is the adjoint of a spinor corresponding to the neutron and so on.
    This helps to declutter the notation, and avoids confusing subscripts like \(\psi_{\Pmux}\) vs.\@ \(\psi_{\mu}\), the first being a spinor for a muon, the second a four--vector with index \(\mu\).
    Here \(\fermiConst\) is a constant called the \defineindex{Fermi constant}.
    It is a coupling constant.
    
    We can generalise this to an interaction
    \begin{equation}
        \fermiConst J_\mu^\hermit J^\mu
    \end{equation}
    where \(J_\mu\) is the \defineindex{weak current} given by
    \begin{equation}
        J_\mu \coloneqq \diracadjoint{\Pnu} \gamma_\mu \Pex + \diracadjoint{\Pp} \gamma_\mu \Pn + \dotsb.
    \end{equation}
    Here the first term corresponds to a current of electrons and electron neutrinos, the second to a current of protons and neutrons, and subsequent terms correspond to currents of other similar pairs of particles.
    Note that we can split \(J_\mu\) into a leptonic term and a hadronic term.
    
    Notice that \(J_\mu\) has \(\Delta Q = 1\), by this we mean that if we interpret a term in the current, for example \(\diracadjoint{\Pnu} \gamma_\mu \Pex\), as an incoming electron and outgoing neutrino then the change in charge is \(\Delta Q = 1\).
    Since the Hermitian conjugate reverses the order of factors we have terms like \(\diracadjoint{\Pex} \gamma_\mu \Pnu\) appearing in \(J_\mu^\hermit\), so \(J_\mu^\hermit\) has \(\Delta Q = -1\).
    This means that overall the interaction conserves charge.
    
    Write \([X]\) for the mass dimension of \(X\), so for a mass \([m] = 1\), for the charge \([e] = 0\), and so on.
    We know that the action is dimensionless, \([S] = 0\), and working in four dimensions we have that \([\dl{^4x}] = -4\).
    So, we must have \([\lagrangianDensity] = 4\).
    Hence all terms appearing in the Lagrangian have mass dimension \(4\), then \([m\diracadjoint{\psi}\psi] = 4\), but \([m] = 1\), so we have \([m\diracadjoint{\psi}\psi] = 1 + [\diracadjoint{\psi}\psi]\), so \([\diracadjoint{\psi}\psi] = 3\), and since the gamma matrices are dimensionless we have \([\diracadjoint{\psi}] = [\psi] = 3/2\).
    This means that \([J_\mu] = [\diracadjoint{\Pnu}\Pex] = 3\), and so \([J_\mu^\hermit J^\mu] = 6\).
    We need to have \([\fermiConst J_\mu^\hermit J^\mu] = [\fermiConst] + 6 = 4\), so \([\fermiConst] = -2\).
    This means that the coupling is \emph{not} dimensionless, in contrast to QED where \(e\) is dimensionless.
    
    We therefore expect that the Fermi constant is of the form \(1/\text{mass}^2\) for some mass scale.
    It turns out that the mass scale is on the order of \qty{300}{\giga\electronvolt}.
    This makes the interaction very weak, for comparison the proton and neutron have masses of less than \qty{1}{\giga\electronvolt}.
    
    As stated here this interaction conserves parity, as well as \(\chargeConjugation\) and \(\chargeConjugation\parity\), for exactly the same reasons that QED does.
    But we know that this is wrong, having been disproven in 1956 by Reins and Cowan, and experimentally by Wu.
    In charged current weak interactions, that is weak interactions where the flavour and charge of the fermions changes, both parity and charge conjugation conservation are violated.
    Later on we'll see that even \(\chargeConjugation\parity\) is not a good symmetry.
    For now though, we just look for a theory which allows parity violations.
    
    \section{\texorpdfstring{\(V - A\)}{V - A} Currents}
    In 1957 Robert Marshak and George Sudarshan formulated a theory, later popularised by Richard Feynman\footnote{Nobel prize for (electro)weak theory count: 3} and Murray Gell-Mann\footnote{Nobel prize for (electro)weak theory count: 4}, which allowed for parity violations.
    
    We want the interaction term to be a Lorentz invariant and to depend on the spinors for our fermions.
    To this end we consider Dirac bilinears.
    There are only five such types of object:
    \begin{itemize}
        \item scalars;
        \item pseudoscalars;
        \item vectors;
        \item axial vectors; and
        \item tensors.
    \end{itemize}
    We can just consider combinations of these until we find one that we works.
    Fortunately this has been done already and the solution is a combination of a vector current,
    \begin{equation}
        V_\mu = \diracadjoint{\Pnu}\gamma_\mu \Pex + \diracadjoint{\Pp} \gamma_\mu \Pn + \dotsb,
    \end{equation}
    and an axial current,
    \begin{equation}
        A_\mu = \diracadjoint{\Pnu} \gamma_\mu \gamma^5 \Pex + \diracadjoint{p} \gamma_\mu \gamma^5 \Pn + \dotsb.
    \end{equation}
    Combining these we get what is called a \define{\(\bm{V - A}\) current}\index{V - A current@\(V - A\) current},
    \begin{align}
        \frac{1}{2}J_\mu &= \frac{1}{2}(V_\mu - A_\mu)\\
        &= \diracadjoint{\Pnu}\gamma_\mu \frac{1}{2}(1 - \gamma^5) \Pex + \diracadjoint{\Pp}\gamma_\mu \frac{1}{2}(1 - \gamma^5)\Pn + \dotsb\\
        &= \diracadjoint{\Pnu}\gamma_\mu P_{\Left} \Pex + \diracadjoint{\Pp}\gamma_\mu P_{\Left} \Pn + \dotsb\\
        &= \diracadjoint{\Pnu}_{\Left} \gamma_\mu \Pex_{\Left} + \diracadjoint{\Pp}_{\Left} \gamma_\mu \Pn_{\Left} + \dotsb.
    \end{align}
    We see that using this current the weak interaction involves only left handed fields.
    This means that not only is parity violated, but it is maximally violated.
    
    Under a parity transformation
    \begin{equation}
        V^\mu \xmapsto{\parity} V_\mu, \qqand A^\mu \xmapsto{\parity} -A_\mu
    \end{equation}
    so
    \begin{equation}
        V^\mu - A^\mu \xmapsto{\parity} V_\mu + A_\mu.
    \end{equation}
    Under charge conjugation
    \begin{equation}
        V^\mu \xmapsto{\chargeConjugation} -V_\mu, \qqand A^\mu \xmapsto{\chargeConjugation} -A_\mu
    \end{equation}
    so
    \begin{equation}
        V^\mu - A^\mu \xmapsto{\parity} -V_\mu - A_\mu.
    \end{equation}
    This means that charge conjugation symmetry is also violated.
    Combining these under \(\chargeConjugation\parity\) transformation
    \begin{equation}
        V^\mu \xmapsto{\chargeConjugation\parity} -V^\mu, \qqand A^\mu \xmapsto{\chargeConjugation\parity} A^\mu
    \end{equation}
    so
    \begin{equation}
        V^\mu - A^\mu \xmapsto{\chargeConjugation\parity} -V_\mu + A_\mu = -(V_\mu - A_\mu).
    \end{equation}
    This means that
    \begin{equation}
        (V^\mu - A^\mu)^\hermit (V_\mu - A_\mu)
    \end{equation}
    is invariant under \(\chargeConjugation\parity\), and hence under time reversal.
    
    Since neutrinos only interact weakly (ignoring gravity) right handed neutrinos don't act at all, so it's as if they don't exist\footnote{not ignoring gravity, right handed neutrinos are a candidate for dark matter}.
    If the neutrino is massless then left and right handed neutrinos are independent so we don't need to consider right handed neutrinos.
    
    From first being theorised in 1930 all the way to 1998 it was assumed that neutrinos were massless.
    Evidence since then shows that this isn't the case.
    While the exact mass of the neutrinos is not known it is known that it is nonzero, but less than \(\qty{0.120}{\electronvolt}\), for comparison the electron, the next lightest massive particle, has a mass of about \(\qty{511}{\kilo\electronvolt}\), more than a million times larger.
    
    We will assume that neutrinos are massless for two reasons
    \begin{itemize}
        \item their mass is negligible for kinematics;
        \item it makes things significantly easier.
    \end{itemize}
    This means that all neutrinos will be left handed, and all antineutrinos are right handed, which we will also assume and so don't bother with left and right labels, which is good as it makes space for the flavour labels like \(\Pex\) on the electron neutrino \(\Pnue\), which would otherwise be \(\PnueLeft\).
    
    So, in the \(V - A\) theory we have the four-Fermi interaction
    \begin{equation}
        \lagrangianDensity_{4\symrm{F}} = -\frac{\fermiConst}{\sqrt{2}} J_\mu^\hermit J^\mu.
    \end{equation}
    Here the minus sign is a matter of convention, the factor of \(\sqrt{2}\) is due to historical reasons and arises when we go from Fermi theory to \(V - A\) theory, and the current splits as
    \begin{equation}
        J_\mu = J_\mu^{\symrm{l}} + J_\mu^{\symrm{h}}
    \end{equation}
    where
    \begin{equation}
        \frac{1}{2}J_\mu^{\symrm{l}} = \diracadjoint{\Pnu}_{\Pex} \gamma_\mu \Pex_{\Left} + \diracadjoint{\Pnu}_{\Pmux} \gamma_\mu \Pmux_{\Left} + \diracadjoint{\Pnu}_{\Ptaux} \gamma_\mu \Ptaux_{\Left}
    \end{equation}
    is the leptonic part of the current, now including all three generations, and
    \begin{equation}
        \frac{1}{2}J_\mu^{\symrm{h}} = \diracadjoint{\Pu}_{\Left} \gamma_\mu \Pd'_{\Left} + \diracadjoint{\Pc}_{\Left} \gamma_\mu \Ps'_{\Left} + \diracadjoint{\Pt}_{\Left} \gamma_\mu \Pb'_{\Left}
    \end{equation}
    is the hadronic part of the current.
    Note that the down type quarks are given a prime here for reasons we will discuss later, for now just ignore the prime.
    
    The \defineindex{lepton number} is defined for electrons as
    \begin{equation}
        L_{\Pex} = N_{\Pe} - N_{\APe} + N_{\Pnu} - N_{\APnue}
    \end{equation}
    where \(N_{\Pp}\) is the number of particles \(\Pp\), and similarly for muons and tau.
    The lepton number is conserved\footnote{assuming that neutrinos are massless, if they aren't then much like quarks we have to put primes on all of the neutrinos and the reason for this leads to non-conservation of lepton number, in which case \(L_{\Pex} + L_{\Pmux} + L_{\Ptaux}\) is still conserved}.
    
    The \defineindex{baryon number} is defined as
    \begin{equation}
        B = \sum_{\mathclap{\text{generations}}} (N_{\Pu} - N_{\APu} + N_{\Pd} - N_{\APd}).
    \end{equation}
    This is conserved, but individual baryon numbers for each generation are not conserved.
    
    There are lots of cross terms which all give different interactions.
    There is one simplifying feature, which is that there is only one coupling constant, \(\fermiConst\).
    This is called \defineindex{universality} and is an important feature meaning that all interactions are the same strength regardless of the particles involved\footnote{kinematics and the primes placed on the down-type quarks make a difference though}.
    
    There are three types of charged current weak interaction, they are as follows.
    \begin{itemize}
        \item \defineindex{leptonic} processes involving only muons, such as muon decay: \(\Pmu \to \Pe \Pnumu \APnue\).
        These are typically the easiest processes to work with.
        \item \defineindex{semi-leptonic} processes occur due to the cross terms and involve a lepton and a quark, for example, beta decay, \(\Pd \to \Pu \Pe \APnue\), or pion decay \(\Pd\APu \to \Pmu\APnumu\).
        These processes are harder to work with due to the strongly interacting particles.
        \item \defineindex{non-leptonic} processes involving only quarks, such as kaon decay \(\Ps\APu \to \Pu\APd \Pu\APd\).
        These are very hard to work with, and we won't do any calculations with them, since these calculations require us to consider lots of strong effects which cannot easily be separated from the from the weak effects.
    \end{itemize}
    
    All of these violate parity and charge conjugation symmetry, but leave \(\chargeConjugation\parity\) invariant\footnote{except that non-leptonic interactions don't \emph{quite} leave \(\chargeConjugation\parity\) invariant, again this is to do with the primes.}
    Only left handed particles (and right handed antiparticles) are involved in any of these decays.
    Since the coupling is the same in all cases these interactions are all the same if we ignore QCD effects.
    
    Consider a decay of the positive pion, \(\Ppip\), which has quark content \(\Pu\APd\), into an antimuon and a muon neutrino,
    \begin{equation}
        \Ppip \approx \Pu \APd \to \APmu \Pnumu.
    \end{equation}
    We use \(\approx\) here to denote the quark content, rather than \(=\), as the quark content is not quite the whole picture of what makes up a pion, but for our purposes right now it's enough.
    Working in the rest frame of the pion the antimuon and neutrino must go in opposite directions to conserve momentum.
    The neutrino must also be left handed, so its spin is antialigned with its momentum.
    We can depict this as
    \begin{equation}
        \tikzsetnextfilename{V-A-pion-decay}
        \begin{tikzpicture}[baseline=(mu.base)]
            \fill (0, 0) circle [radius = 0.1cm];
            \draw[->] (0.2, 0) -- (1, 0) node [right] (mu) {\(\APmu\)};
            \draw[->] (-0.2, 0) -- (-1, 0) node [left] {\(\Pnumu\)};
            \draw[->] (-0.8, -0.2) -- (-0.3, -0.2) node [midway, below] {\(\scriptstyle S\)};
        \end{tikzpicture}
        .
    \end{equation}
    
    Now apply a parity transformation to this.
    The particles momenta will be reversed, so they go in the other direction.
    The particles spins will be unchanged.
    We can write the final state as
    \begin{equation}
        \tikzsetnextfilename{V-A-pion-decay-parity}
        \begin{tikzpicture}[baseline=(mu.base)]
            \fill (0, 0) circle [radius = 0.1cm];
            \draw[->] (-0.2, 0) -- (-1, 0) node [left] (mu) {\(\APmu\)};
            \draw[->] (0.2, 0) -- (1, 0) node [right] {\(\Pnumu\)};
            \draw[->] (0.3, -0.2) -- (0.8, -0.2) node [midway, below] {\(\scriptstyle S\)};
        \end{tikzpicture}
        .
    \end{equation}
    From this we can see that the neutrino spin and momentum are aligned, the neutrino is right handed.
    This is not allowed, so the parity inverted process cannot occur.
    
    Instead consider charge conjugation.
    This doesn't change the momentum or spin, it just changes the antimuon into a muon and the neutrino into an antineutrino.
    We can depict this as
    \begin{equation}
        \tikzsetnextfilename{V-A-pion-decay-charge-conjugation}
        \begin{tikzpicture}[baseline=(mu.base)]
            \fill (0, 0) circle [radius = 0.1cm];
            \draw[->] (0.2, 0) -- (1, 0) node [right] (mu) {\(\Pmu\)};
            \draw[->] (-0.2, 0) -- (-1, 0) node [left] {\(\APnumu\)};
            \draw[->] (-0.8, -0.2) -- (-0.3, -0.2) node [midway, below] {\(\scriptstyle S\)};
        \end{tikzpicture}
        .
    \end{equation}
    From this we can see that the antineutrino spin and momentum are antialigned, the antineutrino is left handed.
    This is not allowed, so the charge conjugated process cannot occur.
    
    If we apply both charge conjugation and a parity transformation then the momenta reverse and the particles become antiparticles and vice versa, giving
    \begin{equation}
        \tikzsetnextfilename{V-A-pion-decay-charge-conjugation-parity}
        \begin{tikzpicture}[baseline=(mu.base)]
            \fill (0, 0) circle [radius = 0.1cm];
            \draw[->] (-0.2, 0) -- (-1, 0) node [left] (mu) {\(\Pmu\)};
            \draw[->] (0.2, 0) -- (1, 0) node [right] {\(\APnumu\)};
            \draw[->] (0.3, -0.2) -- (0.8, -0.2) node [midway, below] {\(\scriptstyle S\)};
        \end{tikzpicture}
        .
    \end{equation}
    From this we can see that the antineutrino spin and momentum are aligned, the antineutrino is right handed, which is allowed, so this the charge conjugated \emph{and} parity transformed version of this process can occur.
    This is summarised in \cref{fig:pion decay symmetries}.
    
    \begin{figure}
        \tikzsetnextfilename{V-A-pion-decay-symmetries}
        \begin{tikzpicture}
            \fill (0, 0) circle [radius = 0.1cm];
            \draw[->] (0.2, 0) -- (1, 0) node [right] (B) {\(\APmu\)};
            \draw[->] (-0.2, 0) -- (-1, 0) node [left] (A) {\(\Pnumu\)};
            \draw[->] (-0.8, -0.2) -- (-0.3, -0.2) node [midway, below] {\(\scriptstyle S\)};
            \coordinate (E) at (0, -0.3);
            \begin{scope}[xshift=-3cm, yshift=-2cm];
                \fill (0, 0) circle [radius = 0.1cm];
                \draw[->] (-0.2, 0) -- (-1, 0) node [left] {\(\APmu\)};
                \draw[->] (0.2, 0) -- (1, 0) node [right] {\(\Pnumu\)};
                \draw[->] (0.3, -0.2) -- (0.8, -0.2) node [midway, below] {\(\scriptstyle S\)};
                \draw[{-Latex}] (A) -- (0, 0.3) node [midway, left] {\(\parity\)};
                \coordinate (C1) at (0, -0.3);
                \draw[Red, line width=1mm, opacity=0.8] (-1, 0.3) -- (1, -0.3);
                \draw[Red, line width=1mm, opacity=0.8] (-1, -0.3) -- (1, 0.3);
            \end{scope}
            \begin{scope}[xshift=3cm, yshift=-2cm];
                \fill (0, 0) circle [radius = 0.1cm];
                \draw[->] (0.2, 0) -- (1, 0) node [right] {\(\Pmu\)};
                \draw[->] (-0.2, 0) -- (-1, 0) node [left] {\(\APnumu\)};
                \draw[->] (-0.8, -0.2) -- (-0.3, -0.2) node [midway, below] {\(\scriptstyle S\)};
                \draw[{-Latex}] (B) -- (0, 0.3) node [midway, right] {\(\chargeConjugation\)};
                \coordinate (D1) at (0, -0.3);
                \draw[Red, line width=1mm, opacity=0.8] (-1, 0.3) -- (1, -0.3);
                \draw[Red, line width=1mm, opacity=0.8] (-1, -0.3) -- (1, 0.3);
            \end{scope}
            \begin{scope}[yshift=-4cm];
                \fill (0, 0) circle [radius = 0.1cm];
                \draw[->] (-0.2, 0) -- (-1, 0) node [left] (C2) {\(\Pmu\)};
                \draw[->] (0.2, 0) -- (1, 0) node [right] (D2) {\(\APnumu\)};
                \draw[->] (0.3, -0.2) -- (0.8, -0.2) node [midway, below] {\(\scriptstyle S\)};
                \draw[{-Latex}] (C1) -- (C2) node [midway, left] {\(\chargeConjugation\)};
                \draw[{-Latex}] (D1) -- (D2) node [midway, right] {\(\parity\)};
                \draw[{-Latex}] (E) -- (0, 0.3) node [midway, left] {\(\chargeConjugation\parity\)};
            \end{scope}
        \end{tikzpicture}
        \caption{While charge conjugation and parity transformations don't yield valid process when applied to pion decay when combined they do.}
        \label{fig:pion decay symmetries}
    \end{figure}
    
    \chapter{Leptonic Decay}
    \section{Setup}
    Consider the decay of a muon into an electron:
    \begin{equation}
        \Pmu \to \Pe \APnue \Pmu.
    \end{equation}
    This is the first example of a decay we've seen, since decays can only occur through the weak interaction as they require flavours to change.
    
    We can draw this decay as
    \begin{equation}
        \tikzsetnextfilename{decays-muon-decay-1}
        \begin{tikzpicture}
            \draw[electron=0.6] (-1, 0) node [left] {\(\Pmu\)} -- (0, 0);
            \draw[electron=0.6] (0, 0) -- (-40:1) node [right] {\(\Pnumu\)};
            \draw[electron=0.6] (0, 0) -- (50:1) node [right] {\(\Pe\)};
            \draw[positron=0.4] (0, 0) -- (15:1) node [right] {\(\APnue\)};
        \end{tikzpicture}
        .
    \end{equation}
    It can be helpful to separate out the two currents, instead drawing
    \begin{equation}
        \tikzsetnextfilename{decays-muon-decay-2}
        \begin{tikzpicture}
            \draw[electron=0.6] (-1, 0) node [left] {\(\Pmu\)} -- (0, 0);
            \draw[electron=0.6] (0, 0) -- (-40:1) node [right] {\(\Pnumu\)};
            \begin{scope}[shift={(0.1, 0.1)}]
                \draw[electron=0.6] (0, 0) -- (50:1) node [right] {\(\Pe\)};
                \draw[positron=0.4] (0, 0) -- (15:1) node [right] {\(\APnue\)};
            \end{scope}
        \end{tikzpicture}
        .
    \end{equation}
    
    Note that we have a crossing symmetry.
    If we cross the muon neutrino over to become an antimuon neutrino this same calculation can be applied to the scattering \(\Pmu\APnumu \to \Pe\APnue\).
    
    Adding in labels for momenta we get
    \begin{equation}
    \tikzsetnextfilename{decays-muon-decay-3}
        \begin{tikzpicture}
            \draw[electron=0.6] (-1, 0) node [left] {\(\Pmu\)} -- (0, 0);
            \draw[electron=0.6] (0, 0) -- (-40:1) node [right] {\(\Pnumu\)};
            \draw[electron=0.6] (0, 0) -- (50:1) node [right] {\(\Pe\)};
            \draw[positron=0.4] (0, 0) -- (15:1) node [right] {\(\APnue\)};
            \draw[->] (-0.8, -0.15) -- (-0.2, -0.15) node [midway, below] {\(\scriptstyle p\)};
            \draw[->, yshift=-0.1cm, xshift=-0.1cm] (-40:0.2) -- (-40:0.8) node [midway, below left] {\(\scriptstyle q\)};
            \draw[->, yshift=0.1cm, xshift=-0.1cm] (50:0.2) -- (50:0.8) node [midway, above left] {\(\scriptstyle k\)};
            \draw[->, yshift=-0.1cm, xshift=0.1cm] (15:0.2) -- (15:0.8) node [pos=0.6, below] {\(\scriptstyle q'\)};
        \end{tikzpicture}
        .
    \end{equation}
    The amplitude for this process at tree level is given by the matrix element of \(i\lagrangianDensity_{\interaction} = i\lagrangianDensity_{4\symrm{F}}\), which means that
    \begin{equation}
        \amplitude = \bra{\Pe(k), \Pnue(q'), \Pnumu(q)} \lagrangianDensity_{4\symrm{F}} \ket{\Pmu(p)}.
    \end{equation}
    
    To proceed we need the Feynman rule for the four fermion vertex.
    This can be found through the usual method, consider the vertex and compute the amputated correlator stripping off external states.
    Doing so one finds that the Feynman rule gives the four fermion vertex
    \begin{equation}
        -i\frac{\fermiConst}{\sqrt{2}} [\gamma_\mu(1 - \gamma_5)]_{ab}[\gamma^\mu(1 - \gamma^5)]_{cd}
    \end{equation}
    where \(a\), \(b\), \(c\), and \(d\) are spinor indices.
    In particular, notice that these two spinor matrices are not multiplied together, and so amplitudes with the four fermion vertex factorise.
    
    Applying this to our case
    \begin{equation}
        \amplitude = -i \frac{\fermiConst}{\sqrt{2}} [ \diracadjoint{u}(k)\gamma^\mu(1 - \gamma^5)v(q') ] [ \diracadjoint{u}(q) \gamma_\mu (1 - \gamma^5) u(p) ]
    \end{equation}
    where \(\diracadjoint{u}(k)\), \(v(q')\), \(\diracadjoint{u}(q)\), and \(u(p)\) are refer to the electron, antielectron neutrino, muon neutrino, and muon respectively.
    
    We will follow an approach of using Dirac spinors for everything and enforcing left handed interactions through factors of \(1 - \gamma^5\).
    This allows us to use the same machinery we developed for QED.
    It is also possible to use two-component Weyl spinors to automatically enforce the left handed interactions, but then we need to develop new machinery.
    
    \subsection{Spin Averaged Amplitude}
    As with QED we now want to compute the spin averaged amplitude squared, so we want to compute \(\amplitude^\hermit \amplitude\).
    To do this we need to compute \(\overline{\gamma^\mu(1 - \gamma^5)}\).
    This is the adjoint of a spinor matrix, not a spinor, and so it is slightly different.
    Given a matrix, \(A\), we can define its Dirac adjoint to be \(\diracadjoint{A} = \gamma^0M^\hermit\gamma^0\), which is compatible with the definition of the Dirac adjoint of a spinor, since
    \begin{equation}
        \overline{A \psi} = (A \psi)^\hermit \gamma^0 = \psi^\hermit A^\hermit \gamma^0 = \psi^\hermit \gamma^0 \gamma^0 A^\hermit \gamma^0 = \diracadjoint{\psi}\diracadjoint{A}.
    \end{equation}
    Then we have
    \begin{equation}
        \diracadjoint{\gamma}^\mu = \gamma^0 (\gamma^\mu)^\hermit \gamma^0.
    \end{equation}
    If \(\mu = 0\) then \((\gamma^0)^\hermit = \gamma^0\), and so we have \((\gamma^0)^3 = \gamma^0\) as \((\gamma^0)^2 = 1\).
    If \(\mu = i\) then \((\gamma^i)^\hermit = -\gamma^i\), and so we have \(-\gamma^0\gamma^i\gamma^0 = (\gamma^0)^2\gamma^i = \gamma^i\).
    Either way, \(\diracadjoint{\gamma}^\mu = \gamma^\mu\).
    This definition respects the usual rule that the adjoint of a product is the reversed product of the adjoints.
    Thus
    \begin{equation*}
        \overline{\gamma^\mu(1 - \gamma^5)} = \overline{(1 - \gamma^5)} \diracadjoint{\gamma}^\mu = \gamma^0(1 - \gamma^5)^\hermit \gamma^0 \gamma^\mu = (\gamma^0)^2 (1 + \gamma^5) \gamma^\mu = \gamma^\mu(1 - \gamma^5).
    \end{equation*}
    Here we've used \((\gamma^5)^\hermit = \gamma^5\) and \(\anticommutator{\gamma^\mu}{\gamma^5} = 0\).
    The two spinor matrices in the vertex, and hence two independent terms of the form \(\diracadjoint{u}(\gamma)u\) both give traces, just like in QED.
    Then terms like \(u\diracadjoint{u}\) become \(\slashed{p} + m\), for appropriate momenta and masses.
    We need to average over the initial spins, which there are two of since the muon is the only particle in the initial state.
    So we have
    \begin{multline}
        \frac{1}{2}\sum_{\text{spins}} \abs{\amplitude}^2 = \frac{1}{4}\fermiConst^2 \tr[(\slashed{k} + m_{\Pex})\gamma^\mu(1 - \gamma^5)\slashed{q}'\gamma^\nu(1 - \gamma^5)]\\
        \times \tr[\slashed{q}\gamma_\mu(1 - \gamma^5)(\slashed{p} + m_{\Pmux})\gamma_\nu(1 - \gamma^5)].
    \end{multline}
    Note that we've already assumed that the neutrino is massless here since we haven't included its mass in the \(\slashed{q}\) or \(\slashed{q}'\) terms.
    
    Now consider the first trace
    \begin{equation}
        \amplitude_{\mu\nu}(q, p) = \tr[\slashed{q}\gamma_\mu(1 - \gamma^5)(\slashed{p} + m_{\Pmux})\gamma_\nu(1 - \gamma^5)].
    \end{equation}
    Often the first useful thing to do is use the idempotency of the projectors to reduce the number of \(\gamma^5\) matrices.
    Commuting the first \(1 - \gamma^5\) factor past both \(\gamma_\mu\) and the gamma matrix implicit in \(\slashed{q}\) changes the sign and changes it back, so we can bring this factor to the front
    \begin{equation}
        \amplitude_{\mu\nu}(p, q) = \tr[(1 - \gamma^5)\slashed{q}\gamma_\mu(\slashed{p} + m_{\Pmux})\gamma_\nu(1 - \gamma^5)].
    \end{equation}
    Now we can use the cyclic property of the trace and \([(1 - \gamma^5)/2]^2 = (1 - \gamma^5)/2\) to get
    \begin{equation}
        \amplitude_{\mu\nu}(p, q) = \tr[\slashed{q}\gamma_\mu(\slashed{p} + m_{\Pmux})\gamma_\nu(1 - \gamma^5)^2] = 2\tr[\slashed{q}\gamma_\mu(\slashed{p} + m_{\Pmux})\gamma_\nu(1 - \gamma^5)].
    \end{equation}
    
    Expanding the brackets inside the trace we can throw away any term with an odd number of gamma matrices in the product, since these all vanish in the trace.
    Note that \(\gamma^5 = i\gamma^0\gamma^1\gamma^2\gamma^3\) is an even number of gamma matrices, so a product of an odd number of gamma matrices and a \(\gamma^5\) can also be thrown out.
    This gives
    \begin{equation}
        \amplitude_{\mu\nu}(p, q) = 2\tr[\slashed{q}\gamma_\mu\slashed{p}\gamma_\nu(1 - \gamma^5)],
    \end{equation}
    since the term with the muon mass includes an odd number of gamma matrices.
    
    Now we can use the identity
    \begin{equation}
        \tr[\gamma^\mu\gamma^\nu\gamma^\rho\gamma^\sigma] = 4(\minkowskiMetric^{\mu\nu}\minkowskiMetric^{\rho\sigma} - \minkowskiMetric^{\mu\rho}\minkowskiMetric^{\nu\sigma} + \minkowskiMetric^{\mu\sigma}\minkowskiMetric^{\rho\nu})
    \end{equation}
    to write the term with \(1\) as 
    \begin{align}
        \tr[\slashed{q}\gamma_\mu\slashed{p}\gamma_\nu] &= q^\rho p^\sigma\tr[\gamma_\mu\gamma_\rho\gamma_\nu\gamma_\sigma]\\
        &= q^\rho p^\sigma 4(\minkowskiMetric_{\mu\rho}\minkowskiMetric_{\nu\sigma} - \minkowskiMetric_{\mu\nu}\minkowskiMetric_{\rho\sigma} + \minkowskiMetric_{\mu\sigma}\minkowskiMetric_{\rho\nu})\\
        &= 4(q_\mu p_\nu - \minkowskiMetric_{\mu\nu}(q \cdot p) + p_\mu q_\nu).
    \end{align}
    We also have the identity
    \begin{equation}
        \tr[\gamma^5\gamma^\mu\gamma^\nu\gamma^\alpha\gamma^\beta] = 4i\varepsilon^{\mu\nu\alpha\beta}.
    \end{equation}
    Using this
    \begin{equation}
        \tr[\slashed{q}\gamma_\mu\slashed{p}\gamma_\nu\gamma^5] = q^\alpha p^\beta \tr[\gamma^5\gamma_\alpha\gamma_\mu\gamma_\beta\gamma_\nu] = 4i q^\alpha p^\beta \varepsilon_{\alpha\mu\beta\nu} = -4iq^\alpha p^\beta \varepsilon_{\mu\nu\alpha\beta}.
    \end{equation}
    
    Combining these we have
    \begin{equation}
        \amplitude_{\mu\nu}(p, q) = 8[q_\mu p_\nu - \minkowskiMetric_{\mu\nu}(q \cdot p) + p_\mu q_\nu + i\varepsilon_{\mu\nu\alpha\beta}q^\alpha p^\beta].
    \end{equation}

    Since the mass term didn't contribute the other trace works exactly the same and so we have
    \begin{multline}
        \frac{1}{2}\sum_{\text{spins}} \abs{\amplitude}^2 = 16 \fermiConst^2(k^\mu q'^\nu + q'^\mu k^\nu - \minkowskiMetric^{\mu\nu} (k \cdot q') - i\varepsilon^{\mu\nu\alpha\beta}q'_\alpha k_\beta)\\
        \times (q_\mu p_\nu + p_\mu q_\nu - \minkowskiMetric_{\mu\nu}(p \cdot q) + i\varepsilon_{\mu\nu\rho\sigma}q^\rho p^\sigma).
    \end{multline}
    If we expand these brackets then we get
    \begin{gather}
        (k \cdot q)(q' \cdot p) + (k \cdot p)(q' \cdot q) - (k \cdot q')(p \cdot q)\\
        + (q' \cdot q)(k \cdot p) + (q' \cdot p)(k \cdot q) - (q' \cdot k)(p \cdot q)\\
        - (q \cdot p)(k \cdot q') - (p \cdot q)(k \cdot q') + 4(k \cdot q')(p \cdot q)\\
        + i\varepsilon_{\mu\nu\rho\sigma} (k^\mu q'^\nu + q'^\mu k^\nu) i\varepsilon^{\mu\nu\alpha\beta}q'_\alpha k_\beta(q_\mu p_\nu + p_\mu q_\nu)\\
        + \varepsilon^{\mu\nu\alpha\beta}\varepsilon_{\mu\nu\rho\sigma}q'_\alpha k_\beta q^\rho p^\sigma.
    \end{gather}
    The first three lines of terms come from the first three terms in each bracket, the term with a 4 coming from \(\minkowskiMetric^{\mu\nu}\minkowskiMetric_{\mu\nu} = 4\).
    The single Levi-Civita terms come from the first two terms in one bracket and the Levi-Civita in the other.
    Products which would contain \(\minkowskiMetric^{\mu\nu}\varepsilon_{\mu\nu\rho\sigma}\) or \(\minkowskiMetric_{\mu\nu}\varepsilon^{\mu\nu\alpha\beta}\) vanish as the metric is symmetric and the Levi-Civita is antisymmetric.
    Now notice that \(k^\mu q'^\nu + q'^\mu k^\nu\) is symmetric in \(\mu\) and \(\nu\), so its product with \(\varepsilon_{\mu\nu\rho\sigma}\) vanishes.
    The other single Levi-Civita term vanishes for the same reason.
    Thus, we are left with
    \begin{equation}
        2(k \cdot q)(q' \cdot p) + 2(k \cdot p)(q' \cdot q) + \varepsilon^{\mu\nu\alpha\beta} \varepsilon_{\mu\nu\rho\sigma} q'_\alpha k_\beta q^\rho p^\sigma.
    \end{equation}
    
    We can further simplify this with the identity
    \begin{equation}
        \varepsilon^{\mu\nu\alpha\beta}\varepsilon_{\mu\nu\rho\sigma} = -2(\delta^\alpha_\rho \delta^\beta_\sigma - \delta^\alpha_\sigma \delta^\beta_\rho).
    \end{equation}
    We have
    \begin{align}
        \varepsilon_{\mu\nu\rho\sigma} q'_\alpha k_\beta q^\rho p^\sigma &= -2(\delta^\alpha_\rho \delta^\beta_\sigma - \delta^\alpha_\sigma \delta^\beta_\rho) q'_\alpha k_\beta q^\rho p^\sigma\\
        &= -2(q'_\rho k_\sigma q^\rho p^\sigma - q'_\sigma k_\rho q^\rho p^\sigma)\\
        &= -2((k \cdot p)(q' \cdot q) - (k \cdot q)(q' \cdot p)).
    \end{align}
    Then we are left with just
    \begin{equation}
        4(k \cdot q)(q' \cdot p).
    \end{equation}
    Giving
    \begin{equation}
        \frac{1}{2}\sum_{\text{spins}} \abs{\amplitude}^2 = 64 \fermiConst^2 (k \cdot q)(p \cdot q').
    \end{equation}
    
    \subsection{Decay Rate}
    \epigraph{No one knows how to do phase space integrals, \ldots, you just keep trying until it works.}{Richard Ball}
    The differential decay rate in the lab frame of the muon is given by
    \begin{equation}
        \dl{\Gamma} = \frac{1}{2m_{\Pmux}} \frac{1}{2}\sum_{\text{spins}} \abs{\amplitude}^2 \phaseSpaceMeasure[3].
    \end{equation}
    Computing the phase space measure, \(\phaseSpaceMeasure[3]\), is no easy task.
    Expanding the definition of the phase space measure we are looking to compute
    \begin{align}
        \dl{\Gamma} &= \frac{1}{2m_{\Pmux}} \int \frac{\dl{^3k}}{(2\pi)^3 2k^0} \frac{\dl{^3q}}{(2\pi)^3 2q^0} \frac{\dl{^3q'}}{(2\pi)^3 2q'^0}\\
        &\qquad\qquad \times (2\pi)^4\delta(p - k - q - q') \frac{1}{2} \sum_{\text{spins}} \abs{\amplitude}^2\\
        &= \frac{1}{8 m_{\Pmux} \pi^5} \int \frac{\dl{^3k}}{(2\pi)^3 2k^0} \frac{\dl{^3q}}{(2\pi)^3 2q^0} \frac{\dl{^3q'}}{(2\pi)^3 2q'^0}\\
        &\qquad\qquad (2\pi)^4\delta(p - k - q - q') (k \cdot q)(p \cdot q').
    \end{align}
    
    The first thing we have to do is decide which variable to integrate out first.
    There is an argument, of questionable validity as it doesn't hold for beta decay, that we should start with the momenta of the neutrinos, since we don't usually see the neutrinos in the final state, we just deduce their existence from kinematics.
    This happens to work in this case, so we'll start by integrating out the neutrino momenta \(q\) and \(q'\).
    
    Set \(Q = q + q' = p - k\).
    Then we have
    \begin{equation}
        Q^2 = q^2 + q'^2 + 2q \cdot q' \approx 2 q \cdot q'
    \end{equation}
    since the neutrinos are on shell so \(q^2 = q'^2 = m_{\Pnu}^2 \approx 0\).
    We can write this as
    \begin{equation}
        Q^2 = 2(q^0 q'^0 - \vv{q} \cdot \vv{q'}) \ge 0
    \end{equation}
    since \(q^0 = \abs{\vv{q}}\) and \(q'^0 = \abs{\vv{q'}}\) for massless (or essentially massless) particles and the dot product comes with a factor of \(\cos\vartheta\) and \(\abs{\cos \vartheta} \le 1\).
    
    Pulling out just the \(q\) and \(q'\) integrals we want to compute
    \begin{equation}
        I_{\mu\nu}(Q) = \int \frac{\dl{^3q}}{q^0} \frac{\dl{^3q'}}{q'^0} \delta(Q - q - q') q_\mu q'_\nu.
    \end{equation}
    This isn't straightforward on its own.
    Fortunately, we know that the decay rate is Lorentz invariant, so this must be a Lorentz tensor, since the decay rate is proportional to
    \begin{equation}
        \int \frac{\dl{^3k}}{k^0} I_{\mu\nu} p^\mu k^\nu
    \end{equation}
    which must be a Lorentz scalar.
    Using this requirement we notice that we can write
    \begin{equation}
        I_{\mu\nu}(Q) = aQ_\mu Q_\nu + b\minkowskiMetric_{\mu\nu}Q^2
    \end{equation}
    since this is the only tensor we can form with the given data.
    
    All that is left is to find the values of \(a\) and \(b\).
    To do this we can contract this result with \(\minkowskiMetric^{\mu\nu}\), giving
    \begin{equation}
        I_{\mu\nu}(Q)\minkowskiMetric^{\mu\nu} = \int \frac{\dl{^3q}}{q^0} \frac{\dl{^3q'}}{q'^0} \delta(Q - q - q') \underbrace{(q \cdot q')}_{Q^2/2} = aQ^2 + 4bQ^2,
    \end{equation}
    so
    \begin{equation}
        a + 4b = \frac{1}{2}I
    \end{equation}
    where
    \begin{equation}
        I = \int \frac{\dl{^3q}}{q^0} \frac{\dl{^3q'}}{q'^0} \delta(Q - q - q').
    \end{equation}
    Contracting  with \(Q^\mu Q^\nu\) gives
    \begin{equation}
        I_{\mu\nu}(Q)Q^\mu Q^\nu = \int \frac{\dl{^3q}}{q^0} \frac{\dl{^3q'}}{q'^0} \delta(Q - q - q') q_\mu q'_\nu Q^\mu Q^\nu = aQ^4 + bQ^4,
    \end{equation}
    so
    \begin{equation}
        a + b = \frac{1}{4}I
    \end{equation}
    having used
    \begin{align}
        q_\mu q'_\nu Q^\mu Q^\mu &= q_\mu q'_\nu (q^\mu + q'^\mu)(q^\nu + q'^\nu)\\
        &= (q^2q'_\nu + (q \cdot q')q'_\nu)(q^\nu + q'^\nu)\\
        &= \underbrace{q^2}_{=0}(q' \cdot q) + \underbrace{q^2q'^2}_{=0} + \underbrace{(q \cdot q')^2}_{=Q^4/4} + (q \cdot q')\underbrace{q'^2}_{=0}.\\
    \end{align}
    
    To compute \(I\), which is a Lorentz scalar, we can simply pick a frame.
    An easy one to do the computation in is the frame in which \(Q^\mu = (Q^0, \vv{0})\), so \(\vv{q} = -\vv{q}'\) and \(q^0 = q'^0 = \abs{\vv{q}} = \abs{\vv{q}'}\).
    We then have the fairly easy integral
    \begin{equation}
        I = \int \frac{\dl{^3q}}{\abs{\vv{q}}^2} \delta(Q - 2\abs{\vv{q}}) = 4\pi \int_0^\infty \dl{q} \, \delta(Q - 2q)
    \end{equation}
    having transformed to spherical polar coordinates and performed the angular integrals.
    Note that the \(\abs{\vv{q}}^2\) in the denominator cancels with the \(q^2\) in the Jacobian.
    To move on we need the following identity for Dirac deltas, which holds under an integral,
    \begin{equation}
        \delta(f(x)) = \sum_{i} \frac{\delta(x - x_i)}{\abs{f'(x_i)}}
    \end{equation}
    where \(x_0\) are the zeros of \(f\).
    In our case we have \(f(q) = Q - 2q\), which has a single zero at \(q = Q/2\), and \(f'(q) = -2\), giving
    \begin{equation}
        I = 4\pi \int_0^{\infty} \dl{q} \, \frac{1}{2} \delta(q - Q/2) = 2\pi.
    \end{equation}
    
    Solving the simultaneous equations then gives us
    \begin{equation}
        a = \frac{\pi}{3}, \qqand b = \frac{\pi}{6}.
    \end{equation}
    Using this we kahve
    \begin{equation}
        p^\mu k^\nu I_{\mu\nu}(Q) = \frac{\pi}{6}(2(Q \cdot p)(Q \cdot k) + (p \cdot k)Q^2).
    \end{equation}
    We can simplify this by substituting in \(Q = p - k\), giving
    \begin{align}
        p^\mu k^\nu I_{\mu\nu}(Q) &= \frac{\pi}{6}(2(p^2 - p \cdot k)(p \cdot k - k^2) + (p \cdot k)(p^2 + k^2 - 2p\cdot k))\notag\\
        &= \frac{\pi}{6}(2(p^2(p \cdot k) - p^2k^2 - (p \cdot k)^2 + k^2(p \cdot k))\\
        &\qquad\qquad + p^2(p\cdot k) + k^2(p \cdot k) - 2(p \cdot k)^2)\\
        &= \frac{\pi}{6}(-2p^2 k^2 + 3(p^2 + k^2)(p \cdot k) - 4(p \cdot k)^2)\\
        &= \frac{\pi}{6}(-2m_{\Pmux}^2m_{\Pex}^2 + 3(m_{\Pmux}^2 + m_{\Pex}^2)(p \cdot k) - 4(p \cdot k)^2).
    \end{align}
    
    In the rest frame of the muon \(p^\mu = (m_{\Pmux}, \vv{0})\) and \(k^\mu = (E, \vv{k})\) with \(E = k^0\) and \(E^2 = m_{\Pex} + \vv{k}^2\).
    Then \(p \cdot k = m_{\Pmux}E\).
    We also have \(E\dd{E} = \abs{\vv{k}} \dd{\abs{\vv{k}}}\).
    We can transform to spherical coordinates for the \(k\) integral, and perform the angular integrals.
    This allows us to substitute
    \begin{equation}
        \frac{\dl{^3k}}{k^0} \to 4\pi \frac{\abs{\vv{k}^2} \dd{\abs{\vv{k}}}}{E} = 4\pi \abs{\vv{k}} \dd{E}.
    \end{equation}
    Since we're looking for the differential decay rate it's fine to stop here with the integrals, and just give the differential cross section as a differential in \(\dl{E}\).
    
    The electron is much lighter than the muon, so we have \(m_{\Pex} \ll E\), meaning that \(\abs{\vv{k}} \approx E\).
    Note that this approximation is not valid for beta decay as up and down quarks don't have this large difference in masses.
    After neglecting the \(m_{\Pex}\) terms we finally have the differential decay rate
    \begin{equation}
        \dl{\Gamma} = \frac{2\fermiConst^2}{3(2\pi)^3} m_{\Pmux} E^2(2m_{\Pmux} - 4E) \dd{E}.
    \end{equation}
    
    Since \(Q^2 \ge 0\) we must have \(m_{\Pmux}^2 - 2p \cdot k = m_{\Pmux}(m_{\Pmux} - 2E) \ge 0\), so \(E \le m_{\Pmux}/2\).
    This means that the neutrinos have to carry away at least half the energy in order to conserve momentum.
    The differential cross section rises from \(0\) at \(E = 0\) to a maximum at \(E = m_{\Pmux}/2\).
    Above this it drops off again but this region is not physical as we need \(E \le m_{\Pmux}/2\).
    
    The interpretation for why the differential cross section is smaller at lower energies comes from considering the spins of the neutrinos.
    If \(E\) is small then the neutrinos must come out almost back to back to conserve momentum.
    Both neutrinos when they're produced must be left handed.
    This means that there spins are antialigned with their momenta, and hence their spins are antialigned with each other.
    This makes it \enquote{hard} to conserve angular momentum so decays like this are suppressed.
    
    Computing the total cross section we then have
    \begin{equation}
        \Gamma = \frac{\fermiConst^2}{12\pi^3}m_{\Pmux} \int_0^{m_{\Pmux}/2} \dl{E} \, (3m_{\Pmux} E^2 - 4E^3) = \frac{\fermiConst^2 m_{\Pmux}^5}{192\pi^3}.
    \end{equation}
    The first thing to notice is that \(\Gamma \propto m_{\Pmux}^5\), so the decay rate is very sensitive to the mass.
    
    We can use this equation to determine the value of \(\fermiConst\).
    It is known that the lifetime of a muon is \(\tau_{\Pmux} = \qty{2.2}{\micro\second}\).
    This gives \(\Gamma_{\Pmux} = \hbar/\tau_{\Pmux} = \qty{3.0e-19}{\giga\electronvolt}\).
    So the value of Fermi's constant is \(\fermiConst \approx \qty{1.2e-5}{\per\giga\electronvolt\squared}\).
    Note that we are neglecting other decay modes of the muon here, which are possible, for example we can add in extra neutrinos, but these are much less likely.
    
    The exact same calculation works for the tau decays \(\Ptau \to \Pe\APnue\Pnutau\) and \(\Ptau \to \Pmu\APnumu\Pnutau\).
    However, there are multiple non-negligible decay modes for tau, and it turns out that both of these occur about \qty{18}{\percent} of the time so the branching ratio is \(B = 0.18\), we then expect that
    \begin{equation}
        \Gamma(\Ptau \to \Pe\APnue\Pnutau) = \Gamma(\Ptau \to \Pmu\APnumu\Pnutau) = 0.18 = B\Gamma_{\Ptau} = \frac{m_{\Ptaux}^5}{m_{\Pmux}^5} \Gamma_{\Pmu}
    \end{equation}
    where \(\Gamma_{\Ptaux}\) is the total decay width for the tau.
    Measurements of this ratio give
    \begin{equation}
        \frac{m_{\Ptaux}^5}{m_{\Pmux}^5} \Gamma_{\Pmu} \approx \qty{4e-13}{\giga\electronvolt}
    \end{equation}
    and measurements of the masses give
    \begin{equation}
        \frac{m_{\Ptaux}^5}{m_{\Pmux}^5} \approx \num{1.4e6}.
    \end{equation}
    Combining these gives
    \begin{equation}
        \tau_{\Ptaux} = \frac{\hbar}{\Gamma_{\Ptaux}} = \frac{Bm_{\Ptaux}^5\hbar}{m_{\Pmux}^5 \Gamma_{\Pmux}} = \qty{2.8e-13}{\second}
    \end{equation}
    which compare well to the measured tau lifetime of \(\tau_{\Ptaux} = \qty{2.9e-13}{\second}\).
    This is evidence for the universality of the weak interaction, it wouldn't work if the coupling was different for different particles, since then we couldn't cancel out \(\fermiConst\) like we have here.
    
    \chapter{Semileptonic Decays}
    \section{Pion Decay}
    Consider the example of a charged pion decay, such as
    \begin{equation}
        \Ppip \to \APmu \Pnumu, \qqor \Ppim \to \Pmu \APnumu
    \end{equation}
    The case of the neutral pion is slightly different as the dominant decay mode is \(\Ppizero \to 2\Pphoton\), which will require us to discuss combined electromagnetic and weak interactions, which we aren't ready for yet.
    For simplicity we'll just consider the \(\Ppip\) case.
    The quark content of the \(\Ppip\) is \(\Pu\APd\).
    We can draw this decay as
    \begin{equation}
        \tikzsetnextfilename{semileptonic-decay-pion-decay-1}
        \begin{tikzpicture}
            \draw[blob] (0, 0) circle [radius = 0.3cm];
            \draw[electron] (-1.5, 0.2) -- (-0.23, 0.2);
            \draw[electron] (-1.5, -0.2) -- (-0.23, -0.2);
            \draw[thick, decoration={calligraphic brace}, decorate] (-1.6, -0.3) -- (-1.6, 0.3) node [midway, left] {\(\Ppip\)};
            \draw[electron=0.47] (45:0.3) to[out=45, in=135] (2, 0);
            \node at (1, 0.75) {\(\Pu\)};
            \draw[positron=0.37] (-45:0.3) to[out=-45, in=225] (2, 0);
            \node at (1, -0.75) {\(\APd\)};
            \draw[positron=0.4] (2, 0) -- ++ (45:1) node [right] {\(\APmu\)};
            \draw[electron=0.6] (2, 0) -- ++ (-45:1) node [right] {\(\Pnumu\)};
            \draw[->] (-1.3, -0.5) -- (-0.5, -0.5) node [midway, below] {\(p\)};
            \begin{scope}[xshift=2cm]
                \draw[->, shift={(-0.1, 0.1)}] (45:0.2) -- (45:0.8) node [midway, above left] {\(k\)};
                \draw[->, shift={(-0.1, -0.1)}] (-45:0.2) -- (-45:0.8) node [midway, below left] {\(q\)};
            \end{scope}
        \end{tikzpicture}
        .
    \end{equation}
    Here the blob represents some strong interactions which we won't deal with, resulting in a quark content of \(\Pu\APd\) which then interacts through the weak interaction.
    
    As with the last example we can separate the two currents and draw this as
    \begin{equation}
        \tikzsetnextfilename{semileptonic-decay-pion-decay-2}
        \begin{tikzpicture}
            \draw[blob] (0, 0) circle [radius = 0.3cm];
            \draw[electron] (-1.5, 0.2) -- (-0.23, 0.2);
            \draw[electron] (-1.5, -0.2) -- (-0.23, -0.2);
            \draw[thick, decoration={calligraphic brace}, decorate] (-1.6, -0.3) -- (-1.6, 0.3) node [midway, left] {\(\Ppip\)};
            \draw[electron=0.47] (45:0.3) to[out=45, in=135] (2, 0);
            \node at (1, 0.75) {\(\Pu\)};
            \draw[positron=0.37] (-45:0.3) to[out=-45, in=225] (2, 0);
            \node at (1, -0.75) {\(\APd\)};
            \draw[positron=0.4] (2.15, 0) -- ++ (45:1) node [right] {\(\APmu\)};
            \draw[electron=0.6] (2.15, 0) -- ++ (-45:1) node [right] {\(\Pnumu\)};
            \draw[->] (-1.3, -0.5) -- (-0.5, -0.5) node [midway, below] {\(p\)};
            \begin{scope}[xshift=2.15cm]
                \draw[->, shift={(-0.1, 0.1)}] (45:0.2) -- (45:0.8) node [midway, above left] {\(k\)};
                \draw[->, shift={(-0.1, -0.1)}] (-45:0.2) -- (-45:0.8) node [midway, below left] {\(q\)};
            \end{scope}
        \end{tikzpicture}
        .
    \end{equation}
    The amplitude then factors into two parts, one for each current:
    \begin{align}
        \amplitude &= \bra{\APmu(k) \Pnumu(q)} \lagrangianDensity_{4\symrm{F}} \ket{\Ppip(p)}\\
        &= -i\frac{\fermiConst}{\sqrt{2}} \diracadjoint{u}(q)\gamma^\mu(1 - \gamma^5)v(k) \bra{0} \APd \gamma_\mu(1 - \gamma^5)\Pu \ket{\Ppip(p)}.
    \end{align}
    Here \(\diracadjoint{u}(q)\) corresponds to the neutrino and \(v(k)\) to the antimuon.
    We have then accounted for all of the particles in the final state, so we're left with just the vacuum, \(\ket{0}\).
    We leave the strong part of the interaction as a \(V - A\) current with the up quark, antidown quark, and pion.
    
    We can use the fact that the pion is a pseudoscalar, so has negative parity, to impose that \(\bra{0}V_\mu{\Ppip} = 0\) for a vector \(V_\mu\), since this product is an axial vector and we need a vector quantity for \(\amplitude\) to be a scalar.
    We are then left with \(-\bra{0}A_\mu\ket{\Ppip} = if_{\Ppi}p_\mu\), where \(f_{\Ppi}\) is some constant to be determined from experiments or difficult non-perturbative QCD calculations, and the \(p_\mu\) dependence is because this result must be a vector, and this is the only vector it could possibly be.
    Thus the amplitude is
    \begin{equation}
        \amplitude = \frac{\fermiConst f_{\Ppi}}{\sqrt{2}} \diracadjoint{u}(q) \slashed{p}(1 - \gamma^5)v(k).
    \end{equation}
    
    Now we can use momentum conservation to impose that \(p = k + q\), and so we have
    \begin{equation}
        \diracadjoint{u}(q)(\slashed{q} + \slashed{k})(1 - \gamma^5)v(k) = \diracadjoint{u}(q)\slashed{q}(1 - \gamma^5)v(k) + \diracadjoint{u}(q)(1 + \gamma^5)\slashed{k}v(k) = -m_{\Pmux}\diracadjoint{u}(q)(1 + \gamma^5)v(k)
    \end{equation}
    having used the fact that \(u\) and \(v\) are solutions to the Dirac equation with positive and negative energy respectively, meaning that \(\slashed{q}u(q) = 0\), for massless neutrinos, and \((\slashed{k} + m_{\Pmux})v(k) = 0\).
    
    We want to evaluate the spin-averaged matrix element.
    Since the pion has spin zero there are no initial spins to average over, so we just sum over final spins and we get
    \begin{equation}
        \sum_{\text{spins}} \abs{\amplitude}^2 = \frac{\fermiConst^2f_{\Ppi}^2}{2} m_{\Pmux}^2 \tr[\slashed{q}(1 + \gamma^5)\slashed{k}(1 - \gamma^5)]
    \end{equation}
    where we've used the orthogonality of the projectors, \((1 + \gamma^5)(1 - \gamma^5) = 0\), and done the usual manipulations of spin indices to get a trace.
    Commuting the \(\slashed{k}\) through the \(1 - \gamma^5\) we get \((1 + \gamma^5)\slashed{k}\), and then we can use \((1 + \gamma^5)^2 = 2(1 + \gamma^5)\) to simplify this result:
    \begin{equation}
        \sum_{\text{spins}} \abs{\amplitude}^2 = \fermiConst^2 f_{\Ppi}^2 m_{\Pmux}^2 \tr[\slashed{q}(1 + \gamma^5)\slashed{k}].
    \end{equation}
    We can then use the identities
    \begin{equation}
        \tr[\slashed{q}\slashed{k}] = 4q \cdot k, \qqand \tr[\slashed{q}\gamma^5\slashed{k}] = 0
    \end{equation}
    to get
    \begin{equation}
        \sum_{\text{spins}} \abs{\amplitude}^2 = 4\fermiConst^2 f_{\Ppi}^2 m_{\Pmux}^2 (q \cdot k).
    \end{equation}
    
    Consider \(p^2 = m_{\Ppi}^2\) and \(p^2 = (q + k)^2 = q^2 + k^2 + 2q \cdot k \approx m_{\Pmux}^2 + 2q\cdot k\) having used \(q^2 = m_{\Pnu}^2 \approx 0\).
    We therefore have \(k \cdot q = (m_{\Ppi}^2 - m_{\Pmux}^2)/2\), giving
    \begin{equation}
        \sum_{\text{spins}} \abs{\amplitude}^2 = 2\fermiConst^2 f_{\Ppi}^2 m_{\Pmux}^2(m_{\Ppi}^2 - m_{\Pmux}).
    \end{equation}
    
    The differential decay rate in the lab frame is
    \begin{equation}
        \dl{\Gamma} = \frac{1}{2m_{\Ppi}} \sum_{\text{spins}} \abs{\amplitude}^2 \phaseSpaceMeasure[2].
    \end{equation}
    In \course{Quantum Field Theory} we computed the phase space measure for two particles to be
    \begin{equation}
        \phaseSpaceMeasure[2] = \frac{1}{(4\pi)^2} \frac{\abs{\vv{k}}}{m_{\Ppi}} \int \dl{\Omega} = \frac{1}{4\pi} \abs{\vv{k}}.
    \end{equation}
    Working in the rest frame of the pion we have \(p^\mu = (m_{\Ppi}, \vv{0})\), \(k^\mu = (E_{\Pmux}, \vv{k})\), and \(q^\mu = (E_{\Pnu}, \vv{q})\).
    Using \(q = p - k\) we get \(q^2 = 0 = p^2 + k^2 - 2p\cdot k = m_{\Ppi}^2 + m_{\Pmux}^2 - 2m_{\Ppi}E_{\Pmux}\).
    Setting \(E_{\Pmux} = \sqrt{m_{\Pmux}^2 + \abs{\vv{k}}^2}\), and enforcing \(\abs{\vv{k}} > 0\) we get
    \begin{equation}
        \abs{\vv{k}} = \frac{m_{\Ppi}^2 - m_{\Pmux}^2}{2m_{\Ppi}}.
    \end{equation}
    Thus we have
    \begin{equation}
        \Gamma = \frac{1}{8\pi} \fermiConst^2 f_{\Ppi}^2 \frac{m_{\Pmux}^2}{m_{\Ppi}^3}(m_{\Ppi}^2 - m_{\Pmux}^2)^2.
    \end{equation}
    Note that \(\dl{\Gamma}\) is differential in \(\dl{\Omega}\) here, and we've already performed this integral since there is no angular dependence.
    
    The measured pion lifetime, \(\tau_{\Ppi} = \qty{2.6e-8}{\second}\), can be used to calculate \(f_{\Ppi} \approx \qty{130}{\mega\electronvolt}\).
    
    We can use the same calculation to compute the decay rate for \(\Ppip \to \APe\Pnue\).
    Universality tells us that the result should be the same, apart from masses, so if we take the ratio most of the constants should cancel and we should find that
    \begin{equation}
        \frac{\Gamma(\Ppip \to \APe\Pnue)}{\Gamma(\Ppip \to \APmu\Pnumu)} = \frac{m_{\Pex}^2(m_{\Ppi}^2 - m_{\Pex}^2)^2}{m_{\Pmux}^2(m_{\Ppi}^2 - m_{\Pmux}^2)^2} \approx 10^{-4}
    \end{equation}
    and this is the same order of magnitude as measurements.
    
    Notice that as \(m_{\Pex} \to 0\) this result vanishes.
    This is due to helicity conservation.
    A neutrino must be left handed, and in the rest frame of the pion the neutrino and positron emerge back to back.
    Since the pion has no spin the positron and neutrino must have opposite spins, and so the positron must also be left handed.
    If the positron was massless then this would be impossible, just as it is for antineutrinos.
    As it is the decay must produce a right handed positron, since this is a weak interaction, so the positron must almost immediately switch to being left handed, which is only possible if it is massive.
    
    \section{Kaon Decay}
    By 1963 kaons, which contain strange quarks, had been discovered.
    The quark content of a positive kaon is, approximately, \(\Pu\APs\).
    We can consider the decay \(\PKp \to \APmu\Pmu\).
    The weak part of this interaction is completely the same as the decay of the \(\Ppip\), since the quantum numbers of \(\APs\) and \(\APd\) relevant for the weak interaction are all the same.
    The strong stuff does change, so we just replace \(f_{\Ppi}\) with \(f_{\PK}\).
    
    Experiments show that
    \begin{equation}
        \frac{\Gamma(\PKp \to \APmu\Pnumu)}{\Gamma(\Ppip \to \APmu\Pnumu)} \approx 1.3,
    \end{equation}
    so the kaon and pion aren't that different.
    
    The problem is that this interaction is not possible with the four fermion interaction we have been working with, since there is no way to get a current of up quarks and strange quarks, there is no \(\diracadjoint{\Pu}_{\Left} \gamma_\mu \Ps_{\Left}\) term.
    Suppose that there was.
    Then we would expect that
    \begin{equation}
        \frac{\Gamma(\PKp \to \APmu\Pnumu)}{\Gamma(\Ppip \to \APmu\Pnumu)} = \frac{f_{\PK}^2}{F_{\Ppi}^2} \frac{m_{\Ppi}^3(m_{\PK}^2 - m_{\Pmux}^2)^2}{m_{\PK}^3(m_{\Ppi}^2 - m_{\Pmux}^2)^2} \approx 24
    \end{equation}
    and experimental measurements put \(f_{\PK}/f_{\Ppi} \approx 1.2\), so this is off by an order of magnitude.
    
    The solution to this problem is due to Nicola Cabbibo, and finally explains the primes in the current.
    He posited that the correct hadronic current was instead
    \begin{equation}
        \frac{1}{2}J_\mu^{\symrm{h}} = \diracadjoint{\Pu}_{\Left}\gamma_\mu(\Pd_{\Left} \cos \cabibboangle + \Ps_{\Left} \sin \cabibboangle) + \dotsb.
    \end{equation}
    We allowed for this before by including primes, so we have
    \begin{equation}
        \Pd_{\Left}' = \cos \cabibboangle \Pd_{\Left} + \sin \cabibboangle \Ps_{\Left}.
    \end{equation}
    The interpretation is that the down quark which interacts through weak interactions is a mixture of down and strange quarks interacting through other means, this is called \defineindex{quark mixing}.
    The exact mix is controlled by a dimensionless parameter, \(\cabibboangle\), called the \defineindex{Cabibbo angle}, which is \(\cabibboangle \approx \ang{13}\), meaning \(\cos\cabibboangle = 0.97\) and \(\sin\cabibboangle = 0.22\).
    Since \(\cos\cabibboangle\) is so close to one this tells us that \(\Pd_{\Left}'\) is mostly formed from \(\Pd_{\Left}\) and only a small amount of \(\Ps_{\Left}\).
    
    Including this we see that
    \begin{equation}
        \Gamma(\Ppip \to \APmu\Pnumu) = \frac{\fermiConst^2f_{\Ppi^2}}{4\pi} \frac{m_{\Pmux}^2}{m_{\Ppi}^3}(m_{\Ppi}^2 - m_{\Pmux})^2 \cos^2\cabibboangle,
    \end{equation}
    which is \(0.94\) times our previous result, so the effects of quark mixing is pretty negligible here.
    On the other hand
    \begin{equation}
        \Gamma(\Ppip \to \APmu\Pnumu) = \frac{\fermiConst^2f_{\PK^2}}{4\pi} \frac{m_{\Pmux}^2}{m_{\PK}^3}(m_{\PK}^2 - m_{\Pmux})^2 \sin^2\cabibboangle,
    \end{equation}
    which is \(0.045 \approx 1/20\) times the previous result, which brings the theoretical result inline with the experimental result.
    
    Note that in 1963 only two generations of quarks were known, this has since been extended to include all three generations.
    We'll come back to this later.
    If we have massive neutrinos we also get a similar effect.
    
    \section{Beta Decay}
    Consider the example of beta decay,
    \begin{equation}
        \Pn \to \Pp\Pe\APnue,
    \end{equation}
    or on the quark level
    \begin{equation}
        \Pd \to \Pu\Pe\APnue,
    \end{equation}
    with the two other quarks making up the neutron as spectators, not contributing to the weak interaction, although they are important for the strong interactions occurring.
    We can draw this as
    \begin{equation}
        \tikzsetnextfilename{semileptonic-beta-decay}
        \begin{tikzpicture}
            \draw[electron=0.6] (-2, 0) node [left] {\(\Pd\)} -- ++ (2, 0);
            \draw[electron=0.6] (-2, -0.5) node [left] {\(\Pd\)} -- ++ (2, 0);
            \draw[electron=0.6] (-2, -1) node [left] {\(\Pu\)} -- ++ (2, 0);
            \draw[electron=0.6] (0, 0) -- ++ (-30:2) node [right] {\(\Pu\)};
            \draw[electron=0.6] (0, -0.5) -- ++ (-30:2) node [right] {\(\Pd\)};
            \draw[electron=0.6] (0, -1) -- ++ (-30:2) node [right] {\(\Pu\)};
            \draw[electron=0.55] (0, 0) -- ++ (40:2) node [right] {\(\Pe\)};
            \draw[positron=0.45] (0, 0) -- ++ (20:2) node [right] {\(\APnue\)};
            \draw[->] (-1.7, 0.2) -- (-0.3, 0.2) node [midway, above] {\(p\)};
            \draw[->, shift={(-0.15, 0.15)}] (40:0.5) -- (40:1.7) node [pos=0.4, above] {\(k\)};
            \draw[->, shift={(0.15, -0.15)}] (20:0.5) -- (20:1.7) node [pos=0.85, below] {\(q\)};
            \draw[->, shift={(0.15, 0.15)}] (-30:0.5) -- (-30:1.7) node [pos=0.9, above] {\(p'\)};
        \end{tikzpicture}
        .
    \end{equation}
    
    As with pion decay we can factorise the matrix element into a weak interaction and some strong interaction, giving
    \begin{align}
        \amplitude &= \bra{\Pp(p') \Pe(k) \APnue(q)} \lagrangianDensity_{4\symrm{F}} \ket{\Pn(p)}\\
        &= -i\frac{\fermiConst}{\sqrt{2}} \diracadjoint{u}(k) \gamma^\mu(1 - \gamma^5)v(q) \bra{\Pp(p')} \diracadjoint{\Pd}\gamma_\mu(1 - \gamma^5)\Pu\ket{\Pn(p)} \cos\cabibboangle.
    \end{align}
    Here \(\diracadjoint{u}(k)\) and \(v(p)\) represent the electron and neutrino.
    
    Now consider the hadronic part, which is a \(V - A\) current, so takes the form
    \begin{equation}
        \bra{\Pp(p')} (V_\mu - A_\mu) \ket{\Pn(p)} = \cos(\cabibboangle) \diracadjoint{u}(p')\gamma_\mu(g_V - g_A\gamma^5) u(p)
    \end{equation}
    for some constants \(g_V\) and \(g_A\).
    Here \(\diracadjoint{u}(p')\) and \(u(p)\) represent the proton and neutron respectively.
    Conservation of the vector current implies that \(g_V = 1\), whereas the axial current is only partially conserved and experimental measurements give \(g_{A} \approx 1.25\).
    
    We can then compute the spin-averaged matrix element squared,
    \begin{align}
        \frac{1}{2} \sum_{\text{spins}} \abs{\amplitude}^2 &= \frac{1}{4}\fermiConst^2\cos^2(\cabibboangle) \tr[(\slashed{k} + m_{\Pex})\gamma^\mu(1 - \gamma^5)\slashed{q}\gamma^\mu(1 - \gamma^5)]\\
        &\quad\qquad \times \tr[(\slashed{p}' + m_{\Pp})\gamma_\mu(1 - g_A\gamma^5)(\slashed{p} + m_{\Pn})\gamma_\nu(1 - g_A\gamma^5)] \notag
    \end{align}
    Consider the second trace:
    \begin{align}
        \amplitude_{\mu\nu} &= \tr[(\slashed{p}' + m_{\Pp})\gamma_\mu(1 - g_A\gamma^5)(\slashed{p} + m_{\Pn})\gamma_\nu(1 - g_A\gamma^5)]\\
        &= \tr[(\slashed{p}' + m_{\Pp})\gamma_\mu(1 - g_A\gamma^5)\slashed{p}\gamma_\nu(1 - g_A\gamma^5)]\\
        &\quad\qquad + m_{\Pn}\tr[(\slashed{p}' + m_{\Pp})\gamma_\mu(1 - g_A\gamma^5)\gamma_\nu(1 - g_A\gamma^5)]\\
        &= \tr[(\slashed{p}' + m_{\Pp})\gamma_\mu\slashed{p}\gamma_\nu(1 - g_A\gamma^5)^2]\\
        &\quad\qquad m_{\Pn}\tr[(\slashed{p}' + m_{\Pp})\gamma_\mu\gamma_\nu(1 + g_A\gamma^5)(1 - g_A\gamma^5)]
    \end{align}
    having anticommuted \(1 - g_A\gamma^5\) through \(\slashed{p}\gamma_\nu\) and \(\gamma_\nu\), changing the sign as appropriate.
    We then can use
    \begin{align}
        (1 - g_A\gamma^5)^2 &= 1 + g_A^2(\gamma^5)^2 + 2g_A\gamma^5 = 1 + g_A^2 + 2g_A\gamma^5,\\
        (1 + g_A\gamma^5)(1 - g_A\gamma^5) &= 1 - g_A^2(\gamma^5)^2 = 1 - g_A^2.
    \end{align}
    So we get
    \begin{align}
        \amplitude_{\mu\nu} &= \tr[(\slashed{p}' + m_{\Pp})\gamma_\mu\slashed{p}\gamma_\nu(1 + g_A^2 + 2g_A\gamma^5)]\\
        &\quad\qquad + m_{\Pn}(1 - g_A^2)\tr[(\slashed{p}' + m_{\Pp})\gamma_\mu\gamma_\nu].
    \end{align}
    We can now expand the arguments of the trace, and we need only keep terms with an even number of gamma matrices, remembering that \(\gamma^5\) is formed from a product of four gamma matrices.
    We then have
    \begin{equation*}
        \amplitude_{\mu\nu} = 2g_A\tr[\slashed{p}'\gamma_\mu\slashed{p}\gamma_\nu\gamma^5] + (1 + g_A^2)\tr[\slashed{p}'\gamma_\mu\slashed{p}\gamma_\nu] + m_{\Pn}m_{\Pp}(1 - g_A^2)\tr[\gamma_\mu\gamma_\nu].
    \end{equation*}
    Now we can apply the identities
    \begin{align}
        \tr(\gamma_\mu \gamma_\nu) &= 4\minkowskiMetric_{\mu\nu},\\
        \tr(\gamma_\mu\gamma_\nu\gamma_\rho\gamma_\sigma) &= 4(\minkowskiMetric_{\mu\nu}\minkowskiMetric_{\rho\sigma} - \minkowskiMetric_{\mu\rho}\minkowskiMetric_{\nu\sigma} + \minkowskiMetric_{\mu\sigma}\minkowskiMetric_{\nu\rho}),\\
        \tr(\gamma_\mu\gamma_\nu\gamma_\rho\gamma_\sigma\gamma^5) &= 4i\varepsilon_{\mu\nu\rho\sigma}.
    \end{align}
    The first trace then gives us
    \begin{align}
        \tr[\slashed{p}' \gamma_\mu \slashed{p} \gamma_\nu \gamma^5] &= p'^\rho p^\sigma \tr[\gamma_\rho \gamma_\mu \gamma_\sigma \gamma_\nu \gamma^5]\\
        &= p'^\rho p^\sigma 4i\varepsilon_{\rho\mu\sigma\nu}.
    \end{align}
    The second trace gives
    \begin{align}
        \tr[\slashed{p}'\gamma_\mu\slashed{p}\gamma_\nu] &= p'^\rho p^\sigma \tr[\gamma_\rho \gamma_\mu \gamma_\sigma \gamma_\nu]\\
        &= p'^\rho p^\sigma4(\minkowskiMetric_{\rho\mu}\minkowskiMetric_{\sigma\nu} - \minkowskiMetric_{\rho\sigma}\minkowskiMetric_{\mu\nu} + \minkowskiMetric_{\rho\nu}\minkowskiMetric_{\mu\sigma})\\
        &= 4(p'_\mu p_\nu - (p' \cdot p)\minkowskiMetric_{\mu\nu} + p'_\nu p_\mu).
    \end{align}
    Hence,
    \begin{multline}
        \amplitude_{\mu\nu} = 8ig_A \varepsilon_{\rho\mu\sigma\nu}p'^\rho p^\sigma + 4(1 + g_A^2)(p'_\mu p_\nu - (p' \cdot p)\minkowskiMetric_{\mu\nu} + p'_\nu p_\mu)\\
        + 4m_{\Pn}m_{\Pp}(1 - g_A^2)\minkowskiMetric_{\mu\nu}.
    \end{multline}
    
    The other trace is almost identical, just replace \(p'\) with \(k\), \(p\) with \(q\), \(m_{\Pp}\) with \(m_{\Pex}\), and \(m_{\Pn}\) with \(m_{\Pnu} = 0\), and swap which indices are raised and lowered.
    The result is
    \begin{multline}
        \tr[(\slashed{k} + m_{\Pe})\gamma^\mu(1 - \gamma^5)\slashed{q}\gamma^\mu(1 - \gamma^5)]\\
        = 8i\varepsilon^{\rho\mu\sigma\nu}k_\rho p_\sigma + 4(k^\mu q^\nu - (k \cdot q)\minkowskiMetric^{\mu\nu} + k^\nu q^\mu).
    \end{multline}
    
    The product required for the spin-averaged amplitude is quite large so we'll break it down.
    First,
    \begin{align}
        (k^\mu q^\nu - &(k \cdot q)\minkowskiMetric^{\mu\nu} + k^\nu q^\mu)(p'_\mu p_\nu - (p' \cdot p)\minkowskiMetric_{\mu\nu} + p'_\nu p_\mu)\notag\\
        &= (k \cdot p')(q \cdot p) - (k \cdot q)(p' \cdot p) + (k \cdot p)(q \cdot p')\\
        &\quad - (k \cdot q)(p \cdot p' - 4p\cdot p' + p \cdot p') - 2p\cdot p' \notag\\
        &\quad + (k \cdot p)(1 \cdot p') - (k \cdot q)(p' \cdot p) + (k \cdot p')(q \cdot p) \notag\\
        &= 2[(k \cdot p')(q \cdot p) + (k\cdot p)(q \cdot p')].
    \end{align}
    Second, the terms with a single Levi-Civita vanish by symmetry, for example taking the Levi-Civita from the first trace the single Levi-Civita term we get is
    \begin{equation}
        \varepsilon^{\alpha\mu\beta\nu}k_\alpha q_\beta(p_\mu'p_\nu - (p' \cdot p)\minkowskiMetric_{\mu\nu} + p'_\nu p_\mu) = 0
    \end{equation}
    as the term in brackets is symmetric in \(\mu\) and \(\nu\), while \(\varepsilon^{\alpha\mu\beta\nu}\) is antisymmetric in \(\mu\) and \(\nu\).
    The double Levi-Civita terms can be simplified using the identity
    \begin{equation}
        \varepsilon^{\alpha\mu\beta\nu}\varepsilon_{\rho\mu\sigma\nu} = \varepsilon^{\mu\nu\alpha\beta}\varepsilon_{\mu\nu\rho\sigma} = -2(\delta^\mu_\rho \delta^\beta_\sigma - \delta^\alpha_\rho \delta^\beta_\rho).
    \end{equation}
    We then get
    \begin{multline}
        \frac{1}{2}\sum_{\text{spins}} \abs{\amplitude}^2 = 16\fermiConst^2\cos^2(\cabibboangle) [ (1 - g_A^2)(p \cdot k)(p' \cdot q) + (1 + g_A^2)(p\cdot q)(p' \cdot k)\\
        - (1 - g_A^2)m_{\Pn}m_{\Pp}(k \cdot q) ].
    \end{multline}
    
    The decay rate is then given by
    \begin{align}
        \dl{\Gamma}
        &= \frac{1}{2m_{\Pn}} \int \frac{\dl{^3k}}{(2\pi)^3 2k^0} \int \frac{\dl{^3q}}{(2\pi)^3 2q^0} \int \frac{\dl{^3p'}}{(2\pi)^3 2p'^0}\\
        &\qquad\qquad\times(2\pi)^4 \delta(p - p' - k - q) \frac{1}{2} \sum_{\text{spins}} \abs{\amplitude}^2\\
        &= \frac{1}{16m_{\Pn}(2\pi)^5} \int \frac{\dl{^3k}}{k^0} \int \frac{\dl{^3q}}{q^0} \frac{1}{p'^0} \delta(p^0 - p'^0 - k^0 - q^0) \frac{1}{2}\sum_{\text{spins}} \abs{\amplitude}^2,
    \end{align}
    having used three of the Dirac deltas to perform the \(p'\) integral.
    We can now write \(p^0 = E_{\Pn}\), \(p'^0 = E_{\Pp}\), \(k^0 = E_{\Pex}\), and \(q^0 = E_{\Pnu}\), as well as changing to spherical coordinates, so \(\dl{^3k} = p_{\Pex}^2 \dd{p_{\Pex}} \dd{\Omega_{\Pex}} = p_{\Pex}E_{\Pex} \dd{E_{\Pex}} \dd{\Omega_{\Pex}}\), having used \(p_{\Pex}^2 = E_{\Pex}^2 - m_{\Pex}^2\) so \(p_{\Pex} \dd{p_{\Pex}} = E_{\Pex} \dd{E_{\Pex}}\), and \(\dl{^3q} = E_{\Pnu}^2 \dd{E_{\Pnu}} \dd{\Omega_{\Pnu}}\), having used \(p_{\Pnu} = E_{\Pnu}\) for massless neutrinos.
    This gives
    \begin{equation*}
        \dl{\Gamma} = \frac{1}{16m_{\Pn}(2\pi)^5} \int \dl{E_{\Pex}} \dd{\Omega_{\Pex}} \dd{E_{\Pnu}} \dd{\Omega_{\Pnu}} \, \frac{p_{\Pex}E_{\Pnu}}{E_{\Pp}} \delta(E_{\Pn} - E_{\Pp} - E_{\Pex} - E_{\Pnu}) \frac{1}{2}\sum_{\text{spins}} \abs{\amplitude}^2.
    \end{equation*}
    Using the Dirac delta to perform the \(E_{\Pnu}\) integral, fixing \(E_{\Pnu} = E_{\Pn} - E_{\Pp} - E_{\Pex}\), we get
    \begin{equation}
        \diff{\Gamma}{E_{\Pex}} = \frac{1}{16m_{\Pn}(2\pi)^5} \int \dl{\Omega_{\Pex}} \dd{\Omega_{\Pnu}} \, \frac{p_{\Pex}E_{\Pnu}}{E_{\Pp}} \frac{1}{2}\sum_{\text{spins}}\abs{\amplitude}^2.
    \end{equation}
    
    In the lab frame \(E_{\Pn} = m_{\Pn}\), and \(\vv{p}_{\Pn} = \vv{0}\).
    We can define \(\Delta \coloneqq m_{\Pn} - m_{\Pp} = \qty{1.29}{\mega\electronvolt} \ll m_{\Pn}\).
    The Born--Oppenheimer approximation, that nuclear recoil is negligible, so \(\vv{p}_{\Pp} \approx \vv{0}\) and \(E_{\Pp} \approx m_{\Pp}\), giving \(\Delta \approx E_{\Pn} - E_{\Pp} = E_{\Pex} + E_{\Pnu}\).
    Notice that \(\Delta\) is of the same order as \(m_{\Pex} = \qty{0.51}{\mega\electronvolt}\), so we must keep both.
    Then we have
    \begin{align}
        (p \cdot k)(p' \cdot q) &\approx m_{\Pn}E_{\Pex}m_{\Pp}E_{\Pnu},\\
        (p \cdot q)(p' \cdot k) &\approx m_{\Pn}E_{\Pnu}m_{\Pp}E_{\Pex},\\
        (k \cdot q) &= E_{\Pex}E_{\Pnu} - \vv{p}_{\Pex} \cdot \vv{p}_{\Pnu}.
    \end{align}
    Giving
    \begin{align}
        \diff{\Gamma}{E_{\Pex}} &= \frac{\fermiConst^2\cos^2\cabibboangle}{m_{\Pn}(2\pi)^5} \int \dl{\Omega_{\Pex}} \dd{\Omega_{\Pnu}} \, \frac{p_{\Pex}E_{\Pnu}}{E_{\Pp}} m_{\Pn} m_{\Pp}\\
        &\qquad \times \biggl[ E_{\Pex}E_{\Pnu}(1 - g_A^2) + E_{\Pnu}E_{\Pex}(1 + g_A^2) - E_{\Pex}E_{\Pnu}(1 - g_A^2)\left( 1 - \frac{\vv{p}_{\Pex} \cdot \vv{p}_{\Pnu}}{E_{\Pex} E_{\Pnu}} \right) \biggr] \notag\\
        &= \frac{\fermiConst^2\cos^2\cabibboangle}{2\pi^3}(1 + 3g_A^2)p_{\Pex}E_{\Pex}(\Delta - E_{\Pex})^2
    \end{align}
    with \(m_{\Pex} < E_{\Pex} < \Delta\).
    Note that the \(\vv{p}_{\Pex}\cdot \vv{p}_{\Pnu}\) term integrates to zero by symmetry, and we use \(p_{\Pex} = \sqrt{E_{\Pex}^2 - m_{\Pex}^2}\).
    
    Integrating over \(E_{\Pex}\) gives the total decay rate, we won't do this integral, but the result is
    \begin{equation}
        \Gamma \propto \frac{\fermiConst^{2} \cos^2\cabibboangle}{2\pi^3}(1 + 3g_A^2)\Delta^5
    \end{equation}
    with the constant of proportionality turning out to be around \(0.47\).
    
    \section{Other Semileptonic Decays}
    Many other semileptonic decays are very similar kinematically to beta decay.
    For example,
    \begin{equation}
        \Ppip \to \Ppizero\APe\Pnue,
    \end{equation}
    which has \(\Delta S = 0\), that is strangeness is conserved, and
    \begin{equation}
        \PKp \to \Ppizero\APmu\Pnumu, \qqand \PKzero \to \Ppim\APmu\Pnumu,
    \end{equation}
    which both have \(\Delta S = 1\), strangeness is not conserved as a strange quark becomes some other type of quark.
    Note that these last two examples are Cabibbo suppressed, since they come with a factor of \(\sin\cabibboangle\).
    
    Other semileptonic decays which one might want to consider are hyperon decays of \(\PSigmap \approx \Pu\Pu\Ps\), \(\PLambdazero \approx \Pu\Pd\Ps\), \(\PXizero \approx \Pu\Ps\Ps\), and \(\POmegam \approx \Ps\Ps\Ps\).
    Some examples of these decaying are the strangeness conserving
    \begin{equation}
        \PSigmap \to \PLambdazero \Pe\APnue,
    \end{equation}
    and the strangeness changing
    \begin{equation}
        \PLambdazero \to \Pp\Pe\APnu, \quad \PSigmap \to \Pn\Pe\APnu, \quad \PXizero \to \PSigmap\Pe\APnue, \qand \POmegam \to \PXizero\Pe\APnue.
    \end{equation}
    All of which have \(\Delta S = 1\).
    
    \section{Non-Leptonic Decays}
    Non-leptonic deecays, such as
    \begin{equation}
        \PKp \to \Ppip\Ppizero, \quad \PKp \to \Ppip\Ppip\Ppim, \quad \PLambdazero \to \Pp\Ppim, \qand \POmegam \to \PXizero\Ppim,
    \end{equation}
    are much harder.
    The factorisation of the amplitude no longer works, everything interacts strongly.
    Generally we have to resort to non-perturbative methods, such as lattice QCD.
    We won't do much with these types of decays.
    
    \chapter{Problems with Fermi Theory}
    Fermi theory works well at low energies, but there are problems when we consider high energy processes.
    We'll investigate a few of these problems before introducing the fix in the next chapter.
    
    \section{Divergent Cross Sections}
    Consider the interaction \(\Pnumu\Pe \to \Pmu\Pnue\), known as \defineindex{inverse muon decay}:
    \begin{equation}
        \tikzsetnextfilename{IVB-inverse-muon-decay}
        \begin{tikzpicture}[baseline=(current bounding box)]
            \draw[electron] (-135:1) node [left] {\(\Pe\)} -- (0, 0);
            \draw[electron] (135:1) node [left] {\(\Pnumu\)} -- (0, 0);
            \draw[electron] (0, 0) -- (45:1) node [right] {\(\Pmu\)};
            \draw[electron] (0, 0) -- (-45:1) node [right] {\(\Pnue\)};
            \draw[->, shift={(-0.1, 0.1)}] (-135:0.8) -- (-135:0.2) node [midway, above left, shift={(0.1, -0.1)}, font=\scriptsize] {\(p\)};
            \draw[->, shift={(0.1, 0.1)}] (135:0.8) -- (135:0.2) node [midway, above right, shift={(-0.1, -0.1)}, font=\scriptsize] {\(q\)};
            \draw[<-, shift={(0.1, -0.1)}] (45:0.8) -- (45:0.2) node [midway, below right, shift={(-0.1, 0.1)}, font=\scriptsize] {\(p'\)};
            \draw[<-, shift={(-0.1, -0.1)}] (-45:0.8) -- (-45:0.2) node [midway, below left, shift={(0.1, 0.1)}, font=\scriptsize] {\(q'\)};
        \end{tikzpicture}
    \end{equation}
    The matrix element for this is
    \begin{equation}
        \amplitude = -i\frac{\fermiConst}{2}\diracadjoint{v}(q')\gamma_\mu(1 - \gamma^5)u(p)\diracadjoint{u}(p')\gamma^\mu(1 - \gamma^5)u(q).
    \end{equation}
    To compute the spin-averaged matrix element we need only average over the electrons spin, since the neutrinos spin is fixed by the requirement that it is left handed.
    So we have
    \begin{align}
        \frac{1}{2}\sum_{\text{spins}}\abs{\amplitude}^2 &= \frac{1}{4} \fermiConst^2 \tr[\slashed{q}'(1 - \gamma^5)(\slashed{p} + m_{\Pex})\gamma_\nu(1 - \gamma^5)]\\
        &\qquad\times \tr[(\slashed{p}' + m_{\Pmux})\gamma^\mu(1 - \gamma^5)\slashed{q}\gamma^\nu(1 - \gamma^5)]\\
        &= 64\fermiConst^2(p \cdot q)(p' \cdot q')\\
        &= 16\fermiConst^2(s - m_{\Pex}^2)(s - m_{\Pmux}^2)
    \end{align}
    where \(s = (p + q)^2 = (p' + q')^2 = m_{\Pex}^2 + 2p\cdot q = m_{\Pmux} + 2p' \cdot q'\).
    We were able to write this down so quickly since we already computed these traces for the case of muon decay, \(\Pmu \to \Pe\APnue\Pnumu\).
    
    In the centre of mass frame at high energies, so masses are negligible, we have
    \begin{equation}
        \left( \diff{\sigma}{\Omega} \right)_{\text{CoM}} = \frac{1}{64\pi^2 s} \frac{1}{2}\sum_{\text{spins}}\abs{\amplitude}^2 = \frac{1}{4\pi}\fermiConst^2 s.
    \end{equation}
    Now using \(s = 4E^2\) and \(\dl{\Omega} = 2\pi\dd{(\cos\vartheta)}\) we ahve
    \begin{equation}
        \diff{\sigma}{\cos\vartheta} = \frac{2\fermiConst^2}{\pi}E^2.
    \end{equation}
    Contrast this with the QED process \(\Pe\APe\to\Pmu\APmu\):
    \begin{equation}
        \diff{\sigma}{\cos\vartheta} = \frac{\alpha^2}{4\pi E^2}\left( \cos^4\frac{\vartheta}{2} + \sin^4\frac{\vartheta}{2} \right).
    \end{equation}
    The QED cross section scales as \(1/E^2\), so vanishes as \(E \to \infty\).
    On the other hand the Fermi theory cross-section scales as \(E^2\), so diverges as \(E \to \infty\).
    This is a problem since the probability of an interaction occurring is proportional to the cross section, so it must be bounded.
    
    More generally any cross section can be decomposed into a sum of partial waves,
    \begin{equation}
        \sigma = \frac{4\pi}{E^2} \sum_j (2j + 1)\abs{f_j}^2,
    \end{equation}
    where \(f_j\) is the partial amplitude for some subprocess.
    Unitarity of the \(S\)-matrix, \(S^\hermit S = 1\), requires that \(\abs{f_j}^2 \le 1\).
    This means that the partial wave contributions to the cross-section must satisfy
    \begin{equation}
        \sigma_j \le \frac{4\pi}{E^2}(2j + 1),
    \end{equation}
    where \(\sigma_j\) is the contribution of the \(j\)th term in the sum above.
    
    Such divergent quantities are inevitable from dimensional analysis.
    The mass dimension of the Fermi constant is \([\fermiConst] = -2\), and similarly the cross section is an area, so \([\sigma] = -2\).
    Interactions in Fermi-theory come with \(\fermiConst^2\), and since at high energies the only energy scale is \(E\) we must have \(\sigma \propto \fermiConst^2 E^2\) to get the correct dimensions.
    Clearly this will diverge as \(E \to \infty\).
    This argument means that we can only satisfy the above unitarity conditions for couplings with non-negative mass dimension.
    Ideally, couplings would be dimensionless, as is the case with \(e\) in QED.
    
    \section{Renormalisability}
    Consider the process of neutrino scattering, \(\Pnue\APnumu \to \Pnue\APnumu\).
    When computing the full amplitude we must include all possible Feynman diagrams, including
    \begin{equation}
        \tikzsetnextfilename{IVB-renormalisability}
        \begin{tikzpicture}[baseline=(current bounding box)]
            \draw[electron=0.55] (-135:1) node [left] {\(\Pe\)} -- (0, 0);
            \draw[electron=0.55] (135:1) node [left] {\(\Pnumu\)} -- (0, 0);
            \draw[electron=0.6, xshift=1cm] (0, 0) -- (45:1) node [right] {\(\Pnumu\)};
            \draw[electron=0.6, xshift=1cm] (0, 0) -- (-45:1) node [right] {\(\Pnue\)};
            \draw[electron=0.55] (0, 0) arc (180:0:0.5) node [pos=0.55, above] {\(\Pe\)};
            \draw[electron=0.55] (1, 0) arc (360:180:0.5) node [pos=0.45, below] {\(\APmu\)};
        \end{tikzpicture}
        .
    \end{equation}
    Suppose that the loop momentum of this diagram is \(k\).
    That is, the electron has momentum \(k\) and the muon momentum \(p - k\), where \(p\) is the combined momentum of both the incoming neutrinos,
    Then when computing the amplitude we come across divergent integrals:
    \begin{equation}
        \amplitude \sim \int \dl{^4k} \frac{\slashed{k}}{k^2} \frac{\slashed{k}}{k^2} \sim \int \frac{\dl{^4k}}{k^2}.
    \end{equation}
    This is a problem since the Lagrangian for Fermi theory does not contain an appropriate \(\diracadjoint{\Pnu}\Pnu\diracadjoint{\Pnu}\Pnu\) counterterm to cancel this divergence.
    
    \subsection{Power Counting in QED}
    Consider QED as a simple example.
    A Feynman diagram is just a graph with \(L\) loops, \(I_{\symrm{F}}\) internal fermion lines, and \(I_{\symrm{B}}\) internal boson lines.
    The \defineindex{superficial degree of divergence} is
    \begin{equation}
        D = 4L - I_{\symrm{F}} - 2I_{\symrm{B}}.
    \end{equation}
    The logic being that a diagram with superficial degree of divergence \(D\) is given by a momentum integral with \(D\) powers of momentum, which we expect to be divergent if \(D \ge 0\).
    The \(4L\) term comes from counting the number of loop integrals, each of which comes with four powers of momentum, \(\dl{^4k}\).
    The \(i_{\symrm{F}}\) term comes from the \(\slashed{k}/k^2 \sim k^{-1}\) terms from fermion propagators, and the \(I_{\symrm{B}}\) term from counting the \(1/k^2\) terms from boson propagators.
    
    A result from graph theory tells us that a graph with \(L\) loops, \(I\) internal lines, and \(V\) vertices satisfies
    \begin{equation}
        L = I - V + 1.
    \end{equation}
    Hence we have
    \begin{equation}
        D = 2I_{\symrm{B}} + 3I_{\symrm{F}} - 4V + 4,
    \end{equation}
    where the number of vertices now counts the number of interactions plus the number of external particles.
    
    Since in QED we have only one type of interaction, involving two fermion lines and one photon, we can make some more deductions, including
    \begin{equation}
        2I_{\symrm{B}} + E_{\symrm{B}} = V, \qqand 2I_{\symrm{F}} + E_{\symrm{F}} = 2V,
    \end{equation}
    where \(E_{\symrm{B}}\) and \(E_{\symrm{F}}\) are the number of external boson and fermion lines respectively.
    Using this we have
    \begin{equation}
        D = V - E_{\symrm{B}} + \frac{3}{2}(2V - E_{\symrm{F}}) - 4V + 4 = 4 - E_{\symrm{B}} - \frac{3}{2}E_{\symrm{F}}.
    \end{equation}
    This means that \(V\) is independent of \(V\).
    The only divergent graphs in QED are ones with appropriate counterterms to regulate the divergences.
    We can list all potentially divergent graphs in QED, based solely on external particles, and they either turn out to be finite by symmetry or are regulated.
    
    The first potentially divergent graph is
    \begin{equation}
        \tikzsetnextfilename{IVB-potentially-divergent-graph-1}
        \begin{tikzpicture}[baseline=(current bounding box)]
            \draw[blob] (0, 0) circle [radius=0.3];
            \draw[photon] (-1.5, 0) -- (-0.3, 0);
            \draw[photon] (1.5, 0) -- (0.3, 0);
        \end{tikzpicture}
        .
    \end{equation}
    This diagram comes from \(F_{\mu\nu}F^{\mu\nu}\) and is regulated by \(Z_3\).
    
    The second potentially divergent graph is
    \begin{equation}
        \tikzsetnextfilename{IVB-potentially-divergent-graph-2}
        \begin{tikzpicture}[baseline=(current bounding box)]
            \draw[blob] (0, 0) circle [radius=0.3];
            \draw[photon] (-30:1.5) -- (-30:0.3);
            \draw[photon] (210:1.5) -- (210:0.3);
            \draw[photon ](90:1.5) -- (90:0.3);
        \end{tikzpicture}
        .
    \end{equation}
    This diagram with three external photons picks up a minus sign under charge conjugation, since the photon is odd under charge conjugation, and since QED respects charge conjugation as a symmetry diagrams like this must always vanish.
    
    The third potentially divergent graph is
    \begin{equation}
        \tikzsetnextfilename{IVB-potentially-divergent-graph-3}
        \begin{tikzpicture}[baseline=(current bounding box)]
            \draw[blob] (0, 0) circle [radius=0.3];
            \draw[photon] (45:1.5) -- (45:0.3);
            \draw[photon] (-45:1.5) -- (-45:0.3);
            \draw[photon] (135:1.5) -- (135:0.3);
            \draw[photon] (-135:1.5) -- (-135:0.3);
        \end{tikzpicture}
        .
    \end{equation}
    This turns out to actually be finite due to gauge invariance.
    
    The fourth potentially divergent diagram is
    \begin{equation}
        \tikzsetnextfilename{IVB-potentially-divergent-graph-4}
        \begin{tikzpicture}[baseline=(current bounding box)]
            \draw[blob] (0, 0) circle [radius=0.3];
            \draw[positron] (0:1.5) -- (0:0.3);
            \draw[electron] (180:1.5) -- (180:0.3);
        \end{tikzpicture}
        .
    \end{equation}
    This comes from the term \(\diracadjoint{\psi}(\slashed{\partial} + m)\psi\), which is regulated by \(Z_2\) and \(Z_m\).
    
    The fifth potentially divergent diagram is
    \begin{equation}
        \tikzsetnextfilename{IVB-potentially-divergent-graphi-5}
        \begin{tikzpicture}[baseline=(current bounding box)]
            \draw[blob] (0, 0) circle [radius=0.3];
            \draw[positron] (-30:1.5) -- (-30:0.3);
            \draw[electron] (210:1.5) -- (210:0.3);
            \draw[photon ](90:1.5) -- (90:0.3);
        \end{tikzpicture}
        .
    \end{equation}
    This comes from the term \(\diracadjoint{\psi}\slashed{A}\psi\), which is regulated by \(Z_1\).
    
    Since the Lagrangian contains all the necessary counterterms we say that QED is power-counting renormalisable.
    
    \subsection{Power Counting in Fermi Theory}
    Now consider our four fermion interaction in Fermi theory.
    We have
    \begin{equation}
        2I_{\symrm{F}} + E_{\symrm{F}} = 4V,
    \end{equation}
    since every internal fermion goes between two vertices and every external fermion goes into/comes out of one vertex, and each vertex has four fermions.
    In an interaction with only fermions we then have
    \begin{equation}
        D = \frac{3}{2}(4V - E_{\symrm{F}}) - 4V + 4 = 4 + 2V - \frac{3}{2}E_{\symrm{F}}.
    \end{equation}
    This means that every time we add a vertex the graph becomes more divergent.
    This means that processes with any number of external fermions will eventually diverge at high enough order.
    For example, if we have six external fermions, then we have divergences at third order in \(\fermiConst\), when \(V = 3\).
    
    \subsection{Power-Counting Renormalisable Interactions}
    \begin{thm}{}{}
        A theory is power-counting renormalisable if all the couplings have non-negative mass dimension.
        \begin{proof}
            Consider a vertex, with \(b\) boson lines, \(f\) fermion lines, and \(p\) derivatives, suppose we have \(V_{bf}^p\) such vertices.
            The corresponding term in the Lagrangian is, in full generality, \(g \varphi^b \psi^f \partial^p\), where \(\varphi\) are bosons (scalar or vector, or even tensor), \(\psi\) are fermions, and include adjoints like \(\diracadjoint{\psi}\) as required, and the derivatives act on some combination of the fields.
            The mass dimension of the coupling is then 
            \begin{equation}
                d_{bf}^p = 4 - b - \frac{3}{2}f - p,
            \end{equation}
            which we know since \([\lagrangianDensity] = 4\), \([\varphi] = [A] = 1\), \([\psi] = 3/2\), and \([\partial] = 1\).
            
            Counting the ends of the lines in any given diagram we have
            \begin{equation}
                2I_{\symrm{B}} + E_{\symrm{B}} = \sum b V_{bf}^{p}, \qqand 2I_{\symrm{F}} + E_{\symrm{F}} = \sum f V_{bf}^p.
            \end{equation}
            Hence, the superficial degree of divergence is
            \begin{align}
                D &= \sum bV_{bf}^p - E_{\symrm{B}} + \frac{3}{2}\left( \sum fV_{bf}^p - E_{\symrm{F}} \right) - 4\sum V_{bf}^p + 4 + \sum pV_{bf}^p\\
                &= \sum\left( b + \frac{3}{2}f + p - 4 \right)V_{bf}^p - E_{\symrm{B}} - \frac{3}{2}E_{\symrm{F}} + 4\\
                &= -\sum d_{bf}^p V_{bf}^p - E_{\symrm{B}} - \frac{3}{2}E_{\symrm{F}} + 4.
            \end{align}
            Thus, for renormalisability we require that \(d_{bf}^p \ge 0\) in order for \(D\) to not increase as the number of vertices increases.
        \end{proof}
    \end{thm}
    
    This theorem, combined with Lorentz invariance, is actually very restricting.
    In fact, between this course and \course{Quantum Field Theory}, we have already seen examples of every possible interaction this theorem allows.
    
    \begin{crl}{}{crl:allowed interactions}
        For a theory with scalar, fermion, and vector fields the only vertices allowed in a power-counting renormalisable Lorentz invariant Lagrangian are
        \begin{tabular}{lll}
            \textbullet & \(\varphi^3\), \(\varphi^4\) & pure scalars\\
            \textbullet & \(\partial_\mu A_\nu A^\mu A^\nu\), \(A_\mu A_\nu A^\mu A^\nu\) & Yang--Mills\\
            \textbullet & \(\varphi A_\mu A^\mu\), \(\varphi^2 A_\mu A^\mu\), \(\varphi (\partial_\mu \varphi) A^\mu\) & gauge-scalar interactions\\
            \textbullet & \(\varphi\diracadjoint{\psi}\psi\) & Yukawa interaction\\
            \textbullet & \(\diracadjoint{\psi}\slashed{A}\psi\) & gauge-fermion interactions
        \end{tabular}
        \begin{proof}
            We need at least three fields for an interaction.
            Lorentz invariance requires that if we have \(\psi\) we also have \(\diracadjoint{\psi}\).
            One can check that adding an extra field to any of the above results in a coupling with negative mass dimension and/or breaks Lorentz invariance.
            The only other terms with appropriate couplings such as, \(A_\mu A^\mu A^\nu\), \(\varphi^2 \partial_\mu \varphi\), and \(\varphi^2A_\mu\), are not Lorentz invariant.
        \end{proof}
    \end{crl}
    
    Not only are these the only allowed interactions, they are also exactly the interactions needed to formulate a consistent theory of weak interactions.
    It seems like if an interaction is possible then it occurs.
    
    \chapter{Intermediate Vector Bosons}
    \section{New Bosons}
    The arguments of the previous section suggest that we need a new theory which reduces to Fermi theory in particular, low energy, cases.
    Such a theory must be renormalisable, and so must contain only interactions of the form of those listed in \cref{crl:allowed interactions}.
    Further, this interaction must act only on left handed fermions.
    This rules out Yukawa interactions, since \(\varphi\diracadjoint{\psi}\psi = \varphi(\diracadjoint{\psi}_{\Left} \psi_{\Right} + \diracadjoint{\psi}_{\Right}\psi_{\Left})\) necessarily includes right handed fermions.
    
    We also know that we have two charged currents, since we have
    \begin{equation}
        J_\mu = 2(\diracadjoint{\Pnu}_{\Pex\Left} \gamma_\mu \Pex_{\Left} + \dotsb + \diracadjoint{\Pu}_{\Left} \gamma_\mu \Pd'_{\Left} + \dotsb),
    \end{equation}
    which has \(\Delta Q = 1\), following the fermion line through a vertex increases the charge by 1, and we also have its adjoint
    \begin{equation}
        J_\mu^\hermit = 2(\diracadjoint{\Pex}_{\Left} \gamma_\mu \PnueLeft + \dotsb + \diracadjoint{d}'_{\Left} \gamma_\mu \Pu_{\Left} + \dotsb),
    \end{equation}
    which has \(\Delta Q = -1\).
    Contrast this with QED, where we have a single current, \(\diracadjoint{\psi}\gamma_\mu\psi\), with \(\Delta Q = 0\).
    
    Because of this we need two bosons, and they need to be charged.
    That's the physics, translated to maths we need to introduce two complex vector fields, and we call them
    \begin{equation}
        W_{\pm}^\mu = \frac{1}{\sqrt{2}}(W_1^\mu \pm i W_2^\mu),
    \end{equation}
    where \(W_1^\mu\) and \(W_2^\mu\) are real vector fields.
    Note that we can equivalently define a single complex vector field, \(W^\mu\), and then take \(W_+^\mu = W^\mu\) and \(W_-^\mu = W^{\mu\hermit}\), which is analogous to our treatment of charged scalars and fermions.
    
    We can construct a Lagrangian for a charged current interaction:
    \begin{equation}
        \lagrangianDensity_{\symrm{CC}} = \frac{g}{2\sqrt{2}}(J_\mu^\hermit W^\mu + W_\mu^\hermit J^\mu)
    \end{equation}
    where the factor of \(1/2\) is the usual factor of 2 for complex fields, and the \(\sqrt{2}\) is convention so that our results align with Fermi theory.
    The coupling constant, \(g\), is dimensionless, and is the same for every term appearing in \(J_\mu\), another example of universality.
    
    The new bosons, \(W^\mu\), differ from \(A^\mu\) in electromagnetism in two important ways:
    \begin{itemize}
        \item \(W^\mu\) is charged, and so complex;
        \item \(W^\mu\) is massive.
    \end{itemize}
    An argument for the second point is that charged current interactions are short range, and so the exchange boson must be massive, but this isn't particularly compelling as QCD interactions are also often short range even though gluons are massive, the short range comes from gluons interacting with each other.
    
    For a full theory it is not enough to specify the interactions, we also need a Lagrangian for the new, massive, \(W^\mu\) field.
    The appropriate choice is the \defineindex{Proca Lagrangian}, which modifies the QED-like \(F^{\mu\nu}F_{\mu\nu}\) to allow for massive bosons:
    \begin{align}
        \lagrangianDensity_{W} &= -\frac{1}{2}(\partial_\mu W_\nu - \partial_\nu W_\mu)^\hermit (\partial^\mu W^\nu - \partial^\nu W^\mu) + m_{\PW}^2 W_\mu^\hermit W^\mu\\
        &= -\frac{1}{2}W_{\mu\nu}^\hermit W^{\mu\nu} + m_{\PW}^2W_\mu^\hermit W^\mu.
    \end{align}
    Here
    \begin{equation}
        W_{\mu\nu} \coloneqq \partial_\mu W_\nu - \partial_\nu W_\mu
    \end{equation}
    generalises the field strength tensor, making use of the same symbol and distinguishing it simply by the number of indices.
    Notice that the typical factor of \(1/4\) becomes \(1/2\) in this case because of the reduced symmetry of having a Hermitian conjugate of one term.
    If we compared the mass term to the mass term of a scalar we would also see that the sign is different, in a charged scalar theory we might have \(-\tfrac{1}{2}m\varphi^\hermit \varphi\).
    This is because the physical degrees of freedom of a massive vector field are the three spatial components, and in our metric convention these carry an extra minus sign.
    
    The equations of motion which one derives from this are a modification on the Klein--Gordon equation:
    \begin{equation}
        (\dalembertian + m_{\PW}^2) W^\mu - \partial^\mu \partial^\nu W_\nu = 0.
    \end{equation}
    Notice that if we assume that \(W^\mu\) also satisfies the Klein--Gordon equation, \((\dalembertian + m_{\PW}^2)W^\mu = 0\), and \(m_{\PW} \ne 0\) then we must have \(\partial^\nu \partial^\mu W_\mu = 0\), which we can assume means \(\partial^\mu W_\mu = 0\).
    
    As usual we look for plane wave solutions, with some polarisation vector, \(\varepsilon^\mu\), to give the correct Lorentz transformation of our field:
    \begin{equation}
        W^\mu = \varepsilon^\mu\e^{-ik \cdot x}.
    \end{equation}
    Substituting this into the equations of motion give
    \begin{equation}
        (-k^2 + m_{\PW}^2)\varepsilon^\mu \e^{-ik\cdot x} - k^\mu k^\nu \varepsilon_\nu \e^{-ik\cdot x} = 0
    \end{equation}
    which implies that
    \begin{equation}
        m_{\PW}^2 k_\mu \varepsilon^\mu = 0,
    \end{equation}
    meaning that on-shell solutions, where \(m_{\PW} \ne 0\), must have \(k \cdot \varepsilon = 0\).
    
    Since \(W^\mu\) is massive it has three spin states.
    In the rest frame of the boson we can take these to be
    \begin{equation}
        \varepsilon_1^\mu = (0, 1, 0, 0), \quad \varepsilon_2^\mu = (0, 0, 1, 0), \qand \varepsilon_3^\mu = (0, 0, 0, 1).
    \end{equation}
    Note that we can also swap \(\varepsilon_1\) and \(\varepsilon_2\) for the circular polarisations \(\varepsilon_{\pm} = (0, 1, \pm i, 0)/\sqrt{2}\).
    
    For momentum in the \(z\)-direction, \(k^\mu = (E, 0, 0, k)\), we have \(\varepsilon_3^\mu = (k, 0, 0, E)/m_{\PW}\).
    This means that at high energies, when \(E \gg m_{\PW}\), we have
    \begin{equation}
        \varepsilon_3^\mu \approx \frac{k^\mu}{m_{\PW}} + \order\left( \frac{m_{\PW}}{E} \right).
    \end{equation}
    This is somewhat strange, \(\varepsilon_3 \cdot k = 0\), so \(\varepsilon_3\) is perpendicular to the momentum, but at high energies \(\varepsilon_3\) is parallel to the momentum to leading order.
    
    As with the massless case we have polarisation sums.
    We can define \(\varepsilon_0^\mu = k^\mu/m_{\PW}\), so that in the rest frame \(\varepsilon_0^\mu = (1, 0, 0, 0)\), which is the timelike, nonphysical degree of freedom.
    Completeness of these states as a basis for Minkowski space tells us that
    \begin{equation}
        \sum_{r = 0}^{3} \minkowskiMetric^{rr} \varepsilon_r^\mu \varepsilon_r^{\nu*} = \minkowskiMetric^{\mu\nu},
    \end{equation}
    where the complex conjugate allows for exchanging the 1 and 2 states for the circular polarisations.
    Summing over only the physical modes, and using \(\minkowskiMetric^{ii} = -1\) (no sum no \(i\)) we then have
    \begin{equation}
        \sum_{r = 1}^{3} \varepsilon_r^\mu \varepsilon_r^{\nu*} = -\minkowskiMetric^{\mu\nu} + \varepsilon_0^\mu\varepsilon_0^{\nu} = -\minkowskiMetric^{\mu\nu} + \frac{k^\mu k^\nu}{m_{\PW}^2}.
    \end{equation}
    
    The result of this is that the tensor structure of the propagator is modified to
    \begin{equation}
        \Delta_{\PW}^{\mu\nu} = \frac{i}{k^2 - m_{\PW^2} + i\epsilon} \left( -\minkowskiMetric^{\mu\nu} + \frac{k^\mu k^\nu}{m_{\PW}^2} \right).
    \end{equation}
    We can check that this is correct by recalling that the propagator is the inverse in Fourier space to the equations of motion, and expanding everything shows that
    \begin{equation}
        [(-k^2 + m_{\PW}^2)\minkowskiMetric_{\mu\nu} + k_\mu k_\nu]\Delta_{\PW}^{\nu\rho} = -i\delta^\rho_\mu.
    \end{equation}
    
    %   Appdendix
%    \appendixpage
%    \begin{appendices}
%        
%    \end{appendices}
    
    \backmatter
%    \renewcommand{\glossaryname}{Acronyms}
%    \printglossary[acronym]
    \printindex
\end{document}