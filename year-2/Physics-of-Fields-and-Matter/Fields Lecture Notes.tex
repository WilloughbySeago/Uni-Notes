\documentclass{article}

\usepackage{NotesPackage}

\usepackage{esint}
\usepackage[european]{circuitikz}
\usepackage[version=4]{mhchem}

\newcommand{\emf}{\mathcal{E}}

\newcommand{\notesVersion}{1.0}
\newcommand{\notesDate}{04/01/2021}

\author{Willoughby Seago}
\date{January 13, 2020}  % ADD THE DATE
\title{Physics of Fields}
\begin{document}
    \maketitle
    These are my notes for the \textit{fields} part of the \textit{physics of fields and matter} course from the University of Edinburgh as part of the second year of the theoretical physics degree.
    When I took this course in the 2019/20 academic year it was taught by Dr Ross Galloway \footnote{\url{https://www.ph.ed.ac.uk/people/ross-galloway}}.
    These notes are based on the lectures delivered as part of this course, and the notes provided as part of this course.
    The content within is correct to the best of my knowledge but if you find a mistake or just disagree with something or think it could be improved please let me know.
    
    These notes were produced using \LaTeX\footnote{\url{https://www.latex-project.org/}}.
    Graphs where plotted using Python\footnote{\url{https://www.python.org/}}, Matplotlib\footnote{\url{https://matplotlib.org/}}, NumPy\footnote{\url{https://numpy.org/}}, and SciPy\footnote{\url{https://scipy.org/scipylib/}}.
    Diagrams were drawn with tikz\footnote{\url{https://www.ctan.org/pkg/pgf}}.
    
    This is version \notesVersion~of these notes, which is up to date as of \notesDate.
    \begin{flushright}
        Willoughby Seago
        
        s1824487@ed.ac.uk
    \end{flushright}
    \clearpage
    \tableofcontents
    \listoffigures
    \listoftables
    \clearpage
    
    \section{The Basics}
    Coulomb's law between two point charges \(q_1\) and \(q_2\) with positions \(\vv r_1\) and \(\vv r_2\) respectively gives the force \(\vv F_{12}\) on charge \(q_1\) due to charge \(q_2\) as:
    \[\vv F_{12} = \frac{q_1q_2}{4\pi\varepsilon_0}\frac{\vv r_1 - \vv r_2}{|\vv r_1 - \vv r_2|^3}\]
    Calculating the force \(\vv F_{21}\) on \(q_2\) due to \(q_1\) is the same and it is trivial to show that \(\vv F_{12} = - \vv F_{21}\) in accordance with Newton's third law.
    
    \section{Electric Field}
    We define the electric field \(\vv E\) as such that a particle of charge \(q\) will experience a force \(\vv F = q\vv E\).
    This means that for a point charge \(q\) at the origin the electric field that it creates at \(\vv r\) is
    \[\vv E = \frac{q}{4\pi\varepsilon_0}\frac{\vh r}{r^2}\]
    Like forces we can use the superposition principle with multiple charges \(q_1,\dotsc,q_n\)  creating electric fields \(\vv{E_1},\dotsc,\vv {E_n}\) to get the total electric field at a point:
    \[\vv E = \sum_i\vv{E_i}\qquad\text{where }\vv{E_i} = \frac{q_i}{4\pi\varepsilon_0}\frac{\vh{r_i}}{r^2}\]
    The units of the electric field are \(\si{NC^{-1}}\).
    
    Electric field lines point away from a positive charge and towards a negative.
    
    \subsection{Electric Dipole}
    \begin{figure}[ht]
        \centering
        \begin{tikzpicture}
            %\draw[lightgray] (0, 0) grid (4, 6);
            \draw[fill=blue, color=blue] (0, 0) circle (0.075cm);
            \draw[fill=red, color=red] (4, 0) circle (0.075cm);
            \node[below] at (0, 0) {\(-q\)};
            \node[below] at (4, 0) {\(q\)};
            \draw[fill=black] (2, 4) circle (0.03cm);
            \draw[<->] (0.075cm, 0) -- (3.925cm, 0);
            \node[below] at (2, 0) {\(d\)};
            \draw[->] (0, 0) -- (1, 2);
            \draw[->] (4, 0) -- (3, 2);
            \node[left] at (0.5, 1) {\(\vv r_-\)};
            \node[right] at (3.5, 1) {\(\vv r_+\)};
            \draw[->] (2, 4) -- (1.5, 5);
            \draw[->] (2, 4) -- (1.5, 3);
            \draw[->] (2, 4) -- (1, 4);
            \node[below] at (1.5, 3) {\(\vv E_-\)};
            \node[above] at (1.5, 5) {\(\vv E_+\)};
            \node[left] at (1, 4) {\(\vv E\)};
            \draw[<->] (2, 0) -- (2, 4);
            \node[right] at (2, 2) {\(a\)};
        \end{tikzpicture}
        \caption{An electric dipole}
        \label{fig:electric dipole}
    \end{figure}
    The setup in figure \ref{fig:electric dipole} is called an electric dipole.
    It is a positive and negative charge of equal magnitude separated by a distance \(d\).
    Directly between the charges but \(a\) units up the electric field is \(\vv E = \vv E_+ + \vv E_-\) where \(\vv E_+\) and \(\vv E_-\) are the electric fields due to the positive and negative charges respectively.
    \(\vv r_+\) and \(\vv r_-\) are the positions vectors from the positive and negative charges respectively to the point.
    \[\vv r_+ = r_+\vh r_+,\qquad r_+^2 = a^2 + \frac{d^2}{4},\qquad \vh r_+ = -\frac{d}{2r_+}\vi + \frac{a}{r_+}\vj\]
    \[\vv E_+ = \frac{1}{4\pi\varepsilon_0}\frac{1}{r_+^2}\left[-\frac{d}{2r_+}\vi + \frac{a}{r_+}\vj\right]\]
    \[\vv r_- = r_-\vh r_-,\qquad r_-^2 = a^2 + \frac{d^2}{4},\qquad \vh r_- = \frac{d}{2r_-}\vi + \frac{a}{r_-}\vj\]
    \[\vv E_- = -\frac{1}{r\pi\varepsilon_0}\frac{q}{r_-^2}\left[\frac{d}{2r_-}\vi + \frac{a}{r_-}\vj\right]\]
    Adding these two fields together and noting that the \(\vj\) components cancel and \(r_+ = r_- = r\) we get
    \[\vv E = -\frac{1}{4\pi\varepsilon_0}\frac{qd}{r^3}\vi\]
    The maximum electric field occurs at \(a = 0\):
    \[\vv E = -\frac{1}{4\pi\varepsilon_0}\frac{8q}{d^2}\vi\]
    The electric dipole can be characterised by the electric dipole moment \(\vv p = qd\vi\) which points from negative to positive.
    Using this we can write the electric field as
    \[\vv E = -\frac{1}{4\pi\varepsilon_0}\frac{\vv p}{r^3}\]
    In the case that \(|a|\gg d\) (either very far away or \(d\) is very small) we get
    \[r^2 = a^2 + \frac{d^2}{4}\approx a^2 \implies r\approx a\]
    This gives us the far field approximation
    \[\vv E \approx -\frac{1}{4\pi\varepsilon_0}\frac{\vv p}{a^3}\]
    This reduces as \(1/a^3\) which is characteristic of a dipole.
    
    \subsection{Continuous Charge Distribution}\label{sec:plane field}
    For a continuous distribution of charge the sum in the superposition principle becomes an integral over all charge.
    This gives
    \[\vv E = \int_\text{all charge}d\vv E\]
    For an infinitesimal charge element \(dq(\vv R)\), which is at position \(\vv R\), the electric field at \(\vv r\) is given by
    \[\vv E(\vv r) = \int_\text{all charge}d\vv E(\vv r) = \frac{1}{4\pi\varepsilon_0}\int_\text{all charge}\frac{\vh a}{a^2}\,dq(\vv R)\]
    where \(\vv a = \vv r - \vv R\).
    In other words
    \[d\vv E(\vv r) = \frac{1}{4\pi\varepsilon_0}\frac{\vh a}{a^2}dq(\vv R)\]
    
    There are three different charge densities that are normally considered.
    They are
    \begin{enumerate}
        \item Volume charge density \(\rho(\vv R)\)
        \begin{itemize}
            \item Charge spread through volume
            \item \([\rho] = \si{Cm^{-3}}\)
            \item \(dq = \rho(\vv R)dV\) where \(dV\) is a small volume element at \(\vv R\)
        \end{itemize}
        \item Surface charge density \(\sigma(\vv R)\)
        \begin{itemize}
            \item Charge spread over a surface
            \item \([\sigma] = \si{Cm^{-2}}\)
            \item \(dq = \sigma(x, y)dA\) where \(dA\) is a small area element at \((x, y)\)
        \end{itemize}
        \item Linear charge density \(\lambda(x)\)
        \begin{itemize}
            \item Charge spread along a line
            \item \([\lambda] = \si{Cm^{-1}}\)
            \item \(dq = \lambda(x)ds\) where \(ds\) is a small line element at \(x\)
        \end{itemize}
    \end{enumerate}
    
    \example
    Consider the electric field due to a uniformly charged rod of length \(L\) and charge \(Q\) at a position\(P\) halfway along the rod and distance \(R\) from the rod.
    Since the charge distribution is uniform we can see that
    \[\lambda = \frac{Q}{L}\]
    We take the length of the rod to be in the \(x\) direction, the point \(P\) to be away from the rod in only the \(z\) direction and the origin at the centre of the rod directly below \(P\).
    It is clear that any \(x\) or \(y\) contributions to the field cancel due to the symmetry of the situation.
    This means that the electric field is purely in the \(z\) direction.
    We consider the contribution to the electric field of a line element \(dx\) at position \(x\).
    This line element has charge \(dq = \lambda dx\).
    Take the angle \(\vartheta\) to be between the vertical and the line connecting \(P\) and \(dx\).
    Then the \(z\) component \(dE_z\) of the electric field is given by
    \[dE_z = dE \cos\vartheta\]
    From basic trig and Pythagoras we can see that
    \[\cos\vartheta = \frac{R}{r},\qquad r^2 = R^2 + x^2\]
    where \(r\) is the distance from \(P\) to \(dx\).
    This gives us
    \begin{align*}
        dE_z &= dE\frac{R}{\left(R^2 + x^2\right)^{1/2}}\\
        &= \frac{dq}{4\pi\varepsilon_0r^2}\frac{R}{\left(R^2 + x^2\right)^{1/2}}\\
        &= \frac{dq}{4\pi\varepsilon_0}\frac{R}{\left(R^2 + x^2\right)^{3/2}}\\
        &= \frac{R\lambda}{4\pi\varepsilon_0}\frac{1}{\left(R^2 + x^2\right)^{3/2}}dx\\
        E_z &= \int dE_z\\
        &= \frac{R\lambda}{4\pi\varepsilon_0}\int_{-L/2}^{L/2}\frac{1}{\left(R^2 + x^2\right)^{3/2}}\,dx\\
        &= \frac{R\lambda}{2\pi\varepsilon_0}\int_{0}^{L/2}\frac{1}{\left(R^2 + x^2\right)^{3/2}}\,dx\\
        \intertext{Given that \(\int\left(R^2 + x^2\right)^{-3/2}dx = xR^{-2}\left(R^2 + x^2\right)^{-1/2}\)}
        E &= E_z\\
        &= \frac{\lambda RL}{4\pi\varepsilon_0}\frac{1}{R^2\sqrt{R^2 + L^2/4}}\\
        &= \frac{Q}{4\pi\varepsilon_0R}\frac{1}{\sqrt{R^2 + L^2/4}}
    \end{align*}
    We can do a sanity check and investigate the extremes of the function:
    \begin{itemize}
        \item \(L\to 0\) and/or \(R \gg L\) \(E\to Q/(4\pi\varepsilon_0R)\) as we would expect for a point charge, which far enough away or for short enough rod the rod approximates.
        \item \(R\to 0\) \(E\to 0\) as we would expect.
        \item \(L\to \infty\) and/or \(R \ll L\):
        \[E\to \frac{Q}{4\pi\varepsilon_0R}\frac{1}{\sqrt{l^2/4}} = \frac{Q}{4\pi\varepsilon_0R}\frac{2}{L} = \frac{\lambda L}{2\pi\varepsilon_0 RL} = \frac{\lambda}{2\pi\varepsilon_0R}\propto\frac{1}{R}\]
        so for an infinitely long rod the electric field drops with \(1/R\) not \(1/R^2\).
    \end{itemize}
    
    \example
    What is the electric field strength \(\vv E\) at a point \(P\) above the centre of a ring radius \(R\) of uniform charge distribution and charge \(q\)?
    \begin{figure}[ht]
        \centering
        \begin{tikzpicture}
            %\draw[lightgray] (0, 0) grid (6, 6);
            \draw (3, 0) ellipse (3cm and 1.5cm);
            \draw (3, 0) ellipse (2.9cm and 1.4cm);
            \draw[fill=black] (3, 4) circle (0.05cm);
            \draw (3, 4) -- (6, 0);
            \draw (3, 4) -- (3, 0);
            \draw (3, 0) -- (6, 0);
            \node at (3.2, 3.4) {\(\vartheta\)};
            \node[left] at (3, 2) {\(z\)};
            \node at (4.5, 2.2) {\(r\)};
            \node[below] at (4.5, 0) {\(R\)};
            \node[above] at (3, 4) {\(P\)};
        \end{tikzpicture}
        \caption{Charged ring}
        \label{fig:charged ring}
    \end{figure}
    Since the charge distribution is uniform then the linear charge density \(\lambda\) is given by
    \[\lambda = \frac{q}{2\pi R}\]
    The charge of an infinitesimal line segment \(ds\) is then
    \[dq = \lambda ds\]
    and this causes an electric field at \(P\) of \(d\vv E\).
    By the symmetry of the system we can see that the horizontal components of the electric field must cancel with those from the diametrically opposite element.
    This means that the strength of the field \(E = E_z\) and the field is purely in the \(\vh k\) direction so we expect
    \[\vv E = E_z\vh k\]
    Each element contributes \(dE_z\) to the field:
    \[dE_z = dE\cos\vartheta = \frac{1}{4\pi\varepsilon_0} \frac{dq}{r^2}\cos\vartheta\]
    \[r = \sqrt{R^2 + z^2},\qquad \cos\vartheta = \frac{z}{r} = \frac{z}{\sqrt{R^2 + z^2}}\]
    \[dE_z = \frac{1}{4\pi\varepsilon_0}\frac{z\lambda}{\left(R^2 + z^2\right)^{3/2}}ds\]
    \[E_z = \int_{\text{ring}}dE_z = \frac{1}{4\pi\varepsilon_0} \frac{z\lambda}{\left(R^2 + z^2\right)^{3/2}}\int_{\text{ring}}ds\]
    \[\int_{\text{ring}}ds = \text{path length} = 2\pi R\]
    \[\vv E(z) = E_z\vh k = \frac{\lambda z R}{2\varepsilon_0\left(R^2 + z^2\right)^{3/2}}\vh k = \frac{qz}{4\pi\varepsilon_0\left(R^2 + z^2\right)^{3/2}}\vh k\]
    
    \example
    What is the electric field at a point \(P\) above the centre of a disc radius \(R\) of uniform charge distribution with total charge \(q\)?
    \begin{figure}[ht]
        \centering
        \begin{tikzpicture}
        %\draw[lightgray] (0, 0) grid (6, 6);
        \draw (3, 0) ellipse (3.5cm and 2cm);
        \draw (3, 0) ellipse (3cm and 1.5cm);
        \draw (3, 0) ellipse (2.9cm and 1.4cm);
        \draw[fill=black] (3, 4) circle (0.05cm);
        \draw (3, 4) -- (3, 0);
        \draw (3, 0) -- (6, 0);
        \node at (3.2, 3.4) {\(\vartheta\)};
        \node[left] at (3, 2) {\(z\)};
        \node[below] at (4.5, 0) {\(r\)};
        \node[above] at (3, 4) {\(P\)};
        \draw (3, 0) -- (3, -2);
        \node[right] at (3, -1) {\(R\)};
        \end{tikzpicture}
        \caption{Charged Disc}
    \end{figure}
    Since the charge is distributed evenly the surface charge density \(\sigma\) is given by
    \[\sigma = \frac{q}{\pi R^2}\]
    A ring of infinitesimal width \(dr\) at a distance \(r\) from the centre has a charge
    \[dq = 2\pi r\sigma dr\]
    Like with the ring we get all horizontal components cancelling.
    \[dE_z(r) = \frac{\sigma z}{2\epsilon_0}\frac{r}{\left(r^2 + z^2\right)^{3/2}}dr\]
    \[E_z = \int_0^R dE_z(r) = \frac{\sigma z}{2\varepsilon_0}\int_0^R\frac{r}{\left(r^2 + z^2\right)^{3/2}}\,dr\]
    Let \(u = r^2 + z^2\), hence \(dr = du/2r\). \(0\to z^2\), \(R\to R^2 + z^2\)
    \[E_z = \frac{\sigma z}{2\varepsilon_0}\int_{z^2}^{R^2 + z^2}\frac{r}{u^{3/2}}\frac{du}{2r} = \frac{\sigma z}{4\varepsilon_0}\int_{z^2}^{R^2 + z^2}u^{-3/2}\,du = \frac{\sigma z}{2\varepsilon_0}\left[-2u^{-1/2}\right]_{z^2}^{R^2 + z^2}\]
    \[E_z = \frac{\sigma z}{2\varepsilon_0}\left[-\frac{1}{\sqrt{R^2 + z^2}} + \frac{1}{z}\right] = \frac{\sigma}{2\varepsilon}\left[1 - \frac{z}{\sqrt{R^2 + z^2}}\right]\]
    \[\vv E(z) = E_z\vh k = \frac{\sigma}{2\varepsilon_0}\left(1 - \frac{z}{\sqrt{R^2 + z^2}}\right)\vh k = \frac{1}{2\pi\varepsilon_0R^2}\left(1 - \frac{z}{\sqrt{R^2 + z^2}}\right)\vh k\]
    At \(z = 0\) or \(z \ll R\) or \(R\to\infty\) the electric field is \(\vv E = \sigma\vh k/2\varepsilon_0\).
    This means that for an infinitely large plate the electric field strength is
    \[E = \frac{\sigma}{2\varepsilon_0}\]
    This doesn't depend on the distance from the plate.
    
    At \(z\gg R\) we get
    \[\vv E = \frac{\sigma R^2}{r\varepsilon_0z^2}\vh k = \frac{4}{\pi\varepsilon_0}\frac{q}{z^2}\vh k\]
    so a small, far away disc looks like a point charge.
    
    One common set up is two parallel infinite (or at least relatively large) plates of opposite charge distributions \(\sigma_+\) and \(\sigma_-\).
    Drawing on the field lines it is clear to see what happens. In figure \ref{fig:parallel plates} we see how the electric fields outside of the plates cancel and inside the field strength is double that of a single plate:
    \[E = \frac{\sigma}{\varepsilon_0}\]
    This is the context where \(\varepsilon_0\) was originally defined and is why the constant \(4\pi\) in Coulomb's law is not included in \(\varepsilon_0\).
    \begin{figure}[ht]
        \centering
        \begin{tikzpicture}
%            \draw[lightgray] (0, 0) grid (8, 4);
            \draw (1, 0) -- (1, 4);
            \node[above] at (1, 4) {\(\sigma_+\)};
            \draw[->] (1, 0.5) -- (2, 0.5);
            \draw[->] (1, 1.5) -- (2, 1.5);
            \draw[->] (1, 2.5) -- (2, 2.5);
            \draw[->] (1, 3.5) -- (2, 3.5);
            \draw[->] (1, 0.5) -- (0, 0.5);
            \draw[->] (1, 1.5) -- (0, 1.5);
            \draw[->] (1, 2.5) -- (0, 2.5);
            \draw[->] (1, 3.5) -- (0, 3.5);
            \node[below] at (1, 0) {Single Plate};
            
            \draw[dashed] (3, 0) -- (3, 4);
            
            \draw (5, 0) -- (5, 4);
            \node[above] at (5, 4) {\(\sigma_+\)};
            \draw[->] (5, 0.5) -- (6, 0.5);
            \draw[->] (5, 1.5) -- (6, 1.5);
            \draw[->] (5, 2.5) -- (6, 2.5);
            \draw[->] (5, 3.5) -- (6, 3.5);
            \draw[->] (5, 0.5) -- (4, 0.5);
            \draw[->] (5, 1.5) -- (4, 1.5);
            \draw[->] (5, 2.5) -- (4, 2.5);
            \draw[->] (5, 3.5) -- (4, 3.5);
            
            \draw (7, 0) -- (7, 4);
            \node[above] at (7, 4) {\(\sigma_-\)};
            \draw[<-] (7, 0.5) -- (8, 0.5);
            \draw[<-] (7, 1.5) -- (8, 1.5);
            \draw[<-] (7, 2.5) -- (8, 2.5);
            \draw[<-] (7, 3.5) -- (8, 3.5);
            \draw[<-] (7, 0.5) -- (6, 0.5);
            \draw[<-] (7, 1.5) -- (6, 1.5);
            \draw[<-] (7, 2.5) -- (6, 2.5);
            \draw[<-] (7, 3.5) -- (6, 3.5);
            
            \node[below] at (6, 0) {Two Plates};
            
            \draw[thick, ->] (8.2, 2) -- (8.8, 2);
            \draw (9, 0) -- (9, 4);
            \draw (11, 0) -- (11, 4);
            \draw[->] (9, 0.5) -- (11, 0.5);
            \draw[->] (9, 1.5) -- (11, 1.5);
            \draw[->] (9, 2.5) -- (11, 2.5);
            \draw[->] (9, 3.5) -- (11, 3.5);
            
            \node[below] at (10, 0) {Resultant Field};
        \end{tikzpicture}
        \caption{Electric field of infinite plates}
        \label{fig:parallel plates}
    \end{figure}
    
    \section{Gauss's Law}
    Consider a point charge \(q\) at the centre of a spherical surface.
    The electric field is radially away from \(q\) so at the surface \(\vv E\) is normal to the surface and of constant strength
    \[E = \frac{1}{4\pi\varepsilon_0}\frac{q}{r^2}\]
    where \(r\) is the radius of the sphere.
    We define the electric flux \(\Phi_E\) as the electric field strength passing normal to the surface.
    In the case of the sphere:
    \[\Phi_E = \left[\frac{1}{4\pi\varepsilon_0}\frac{q}{r^2}\right]\left[4\pi r^2\right] = \frac{q}{\varepsilon_0}\]
    
    In a more complicated case \(\vv E\) isn't normal to the surface.
    We define \(d\vv A\) to be a vector normal to a surface element of infinitesimal area \(dA\).
    The strength of \(\vv E\) perpendicular to this surface times the surface area (ie the flux) is then
    \[d\Phi_E = \vv E\cdot d\vv A\]
    We define next a general closed surface with all surface elements pointing outwards.
    We call this a Gaussian surface.
    For a Gaussian surface composed of \(n\) surface elements \(d\vv A_n\) we get
    \[\Phi_E = \lim_{n\to \infty, dA\to 0}\sum_{i=1}^{n}\vv E\cdot d\vv A_n = \oint_{\text{surface}} \vv E\cdot d\vv A\]
    Where \(\oint\) indicates an integral over a closed surface.
    Sometimes this is written with \(\oiint\) to indicate that this is a surface integral.
    The limits must be such that the integral is taken over the whole surface.
    
    Gauss's law states that for any closed Gaussian surface the electric flux through the surface is given by the enclosed charge \(q\) such that
    \[\Phi_E = \oint \vv E\cdot d\vv A = \frac{q}{\varepsilon_0}\]
    
    \subsection{Derivation of Gauss's Law}
    Take a point charge \(q\).
    We can form any Gaussian surface that we like that also contains this charge distribution.
    The flux through the Gaussian surface is
    \[\Phi_E = \oint \vv E\cdot d\vv A = \oint \frac{1}{4\pi\varepsilon_0}\frac{q}{r^2}\cos\vartheta dA\]
    where \(\vartheta\) is the angle between \(\vv E\) and \(d\vv A\) and \(r\) is the distance from the charge to the Gaussian surface.
    We can project the term \(\cos\vartheta dA\) onto a unit sphere.
    This projection simply scales \(\cos\vartheta dA\) by \(1/r^2\).
    If the surface area of the sphere is \(\Omega = 4\pi\) then we define the area that the surface projects onto as
    \[d\Omega = \frac{\cos\vartheta dA}{r^2}\]
    We can do this for all surface elements \(d\vv A\) and take the surface integral over the unit sphere:
    \[\Phi_E = \oint \vv E\cdot d\vv A = \oint\frac{1}{4\pi\varepsilon_0}\frac{q}{r^2}\cos\vartheta\,dA = \oint\frac{q}{4\pi\varepsilon_0}\,d\Omega = \frac{q}{4\pi\varepsilon_0}\oint d\Omega = \frac{q}{4\pi\varepsilon_0}4\pi = \frac{q}{\varepsilon_0}\]
    Hence Gauss's law holds for all Gaussian surfaces.
    This doesn't depend at all on the shape of the initial Gaussian surface or the position of the charge in the surface, only on the \(1/r^2\) dependence of electric field strength.
    We will see that there is a similar law for all fields with this \(1/r^2\) dependence.
    
    We can extend this easily to \(n\) charges \(q_1,\dotsc,q_n\) each contributing \(\vv E_i\) to the electric field.
    Hence \(\vv E = \sum_i \vv E_i\) and the flux is given by
    \[\Phi_E = \oint\vv E\cdot d\vv A = \oint \left[\sum_i \vv E_i\right]\cdot d\vv A = \sum_i\left[\oint \vv E_i\cdot d\vv A\right] = \sum_i\frac{q_i}{\varepsilon_0} = \frac{q}{\varepsilon_0}\]
    Where \(q = \sum_i q_i\) is the total charge.
    A similar argument can be made for a continuous charge distribution represented by charge elements \(dq\) such that the total charge \(q = \int dq\).
    If all of these charge elements are in the Gaussian surface then
    \[\Phi_E = \oint\vv E\cdot d\vv A = \frac{1}{\varepsilon_0}\int dq = \frac{q}{\varepsilon_0}\]
    
    \example
    What is the electric field from a long, thin rod of constant linear charge density \(\lambda\)?
    
    We choose a cylindrical Gaussian surface radius \(r\) and length \(l\) centred on the rod.
    At the ends of the cylinder \(\vv E\) is perpendicular to \(d\vv A\) so \(\vv E\cdot d\vv A = 0\).
    On the curved surface of the cylinder \(\vv E\) is parallel to \(d\vv A\) so \(\vv E\cdot d\vv A = EdA\).
    Over a length \(l\) the total charge is \(Q = \lambda l\).
    The total surface area for which \(\vv E\cdot d\vv A\ne 0\) is the curved surface area of the cylinder which is \(2\pi rl\).
    Hence the flux is:
    \[\Phi_E = \oint\vv E\cdot d\vv A = \oint E\,dA = E\oint dA = 2E\pi rl = \frac{Q}{\varepsilon} = \frac{\lambda l}{\varepsilon_0}\]
    \[E = \frac{\lambda}{2\pi\epsilon_0r}\]
    
    \example
    What is the electric field from an infinite plane of surface charge density \(\sigma\)?
    
    By symmetry \(\vv E\) must be perpendicular to the surface.
    We choose a cylindrical Gaussian surface of radius \(r\) such that the sides of the cylinder are perpendicular to the surface.
    On the curved sides of the cylinder \(\vv E\) is perpendicular to \(d\vv A\) so \(\vv E\cdot d\vv A = 0\).
    On the ends of the cylinder \(\vv E\) is parallel to \(d\vv A\) so \(\vv E \cdot d\vv A = EdA\).
    Thence the flux is
    \[\Phi_E = \oint \vv E\cdot d\vv A = \oint E\,dA = E\oint dA = 2\pi r^2 E = \frac{q}{\varepsilon_0} = \frac{\sigma\pi r^2}{\varepsilon_0}\]
    \[E = \frac{\sigma}{2\varepsilon_0}\]
    as we previously derived by a much more arduous process in section \ref{sec:plane field}.
    
    \example
    What is the electric field from a sphere of radius \(R\) and constant charge density \(\rho\)?
    
    We choose a spherical Gaussian surface radius \(r\).
    We have two cases \(r \ge R\) and \(r < R\).
    First we shall deal with the case \(r \ge R\)
    The total charge is \(Q = \frac{4}{3}\pi R^3\rho\).
    \(\vv E\) is parallel to \(d\vv A\) so \(\vv E\cdot d\vv A = EdA\).
    \[\Phi_E = \oint \vv E\cdot d\vv A = E\oint dA = 4E\pi r^2 = \frac{Q}{\varepsilon_0} = \frac{4\pi R^3\rho}{3\epsilon_0}\]
    \[E = \frac{1}{4\pi\varepsilon_0}\frac{Q}{r^2}\]
    so for \(r \ge R\) the sphere looks like a point charge.
    
    We now consider the case \(r < R\).
    The total charge in the Gaussian surface is \(Q = \frac{4}{3}\pi r^3\).
    \(\vv E\) is parallel to \(d\vv A\) so \(\vv E\cdot d\vv A = EdA\).
    \[\Phi_E = \oint \vv E\cdot d\vv A = E\oint dA = 4E\pi r^2 = \frac{Q}{\varepsilon_0} = \frac{4\pi r^3\rho}{3\epsilon_0}\]
    \[E = \frac{\rho r}{3\varepsilon_0}\]
    This gives us an electric field that increase linearly with \(r\) up to the edge of the sphere, \(r = R\) and then decreases with \(1/r^2\).
    
    \subsection{Charged Conductors}
    Up to now we have only considered charges in fixed positions.
    This works well for insulators but in conductors the charges can move.
    
    If we consider an isolated conductor we can assume that it is in equilibrium with no charges moving.
    This must mean that inside the conductor \(\vv E = \vv 0\) since if it wasn't the charges would move to redistribute until it was.
    Thus taking a Gaussian surface just inside the conductor there can't be any flux since \(\vv E = \vv 0 \implies \vv E\cdot d\vv A = 0\) so
    \[\Phi_E = \oint \vv E\cdot d\vv A = 0 = \frac{q}{\varepsilon_0}\]
    hence the total charge inside the conductor must be 0.
    This means that all of the charge must be on the outer surface of the conductor.
    
    \example
    Now consider a conducting sphere of charge \(Q\).
    Outside of the sphere \(r \ge R\) the situation is the same as for an insulating sphere so \(E \propto 1/r^2\).
    Inside the sphere \(E = 0\).
    The surface charge density is
    \[\sigma = \frac{Q}{4\pi R^2}\]
    
    \subsection{Faraday Cages}
    If we now place this sphere (or any conductor) in an external electric field \(\vv E_\text{ext}\) the field in the conductor must be \(0\).
    This means that the surface charges must move to cancel \(\vv E_\text{ext}\).
    The resultant electric field \(\vv E\) is normal to the surface of the conductor everywhere.
    
    This must still be true if we hollow out the conductor.
    This is the science behind a Faraday cage which protects it contents from electric fields.
    
    \section{Potential}
    If the work done to move from point \(A\) to point \(B\) is the same by any path then we say that the force that causes the motion is conservative.
    If a force is conservative then the work done around any closed path (starting and ending in the same place) is zero.
    The electrostatic force depends on the distance as \(1/r^2\) but not on the path taken so it is conservative since there are no dissipative components of the force.
    
    Consider a charge moving around the closed loop in an external field \(\vv E\) shown in figure \ref{fig:charge around a loop}.
    \begin{figure}[ht]
        \centering
        \begin{tikzpicture}
            \draw[lightgray, ->] (0, 0) -- (5, 0);
            \draw[lightgray, ->] (0, 1) -- (5, 1);
            \draw[lightgray, ->] (0, 2) -- (5, 2);
            \draw[lightgray, ->] (0, 3) -- (5, 3);
            \draw[lightgray, ->] (0, 4) -- (5, 4);
            \draw[lightgray, ->] (0, 5) -- (5, 5);
            \draw[->] (0.5, 4.5) -- (4.5, 4.5);
            \draw[->] (4.5, 4.5) -- (4.5, 0.5);
            \draw[->] (4.5, 0.5) -- (0.5, 0.5);
            \draw[->] (0.5, 0.5) -- (0.5, 4.5);
            \node[above] at (2.5, 4.5) {1};
            \node[right] at (4.5, 2.5) {2};
            \node[below] at (2.5, 0.5) {3};
            \node[left] at (0.5, 2.5) {4};
            \node[right, lightgray] at (5, 5) {\(\vv E\)};
        \end{tikzpicture}
        \caption{Charge moving around a closed loop}
        \label{fig:charge around a loop}
    \end{figure}
    The force on a charge \(q\) is given by \(\vv F = q\vv E\).
    The work done \(W\) is related to the force by
    \[W = -\int_{\text{path}} \vv F\cdot d\vv r\]
    For section 1 of the path \(\vv F\) is parallel to \(d\vv r\) so \(\vv F\cdot d\vv r = Fdr\) so the work done is  \(-Fr = W_1\).
    For section 2 and 3 of the path \(\vv F\) is perpendicular to \(d\vv r\) so \(\vv F\cdot d\vv r = 0\) so the work done is \(W_2 = W_3 = 0\).
    For section 4 of the path \(\vv F\) is antiparallel to \(d\vv r\) so \(\vv F\cdot d\vv r = -Fdr\) so the work done is \(Fr = W_4\).
    The total work done is then the sum of all of these which is
    \[W = \sum_i W_i = -Fr + 0 + 0 + Fr = 0\]
    which is what we would expect for a conservative force.
    
    Since the path is irrelevant we can assign a value of work done per unit charge to move between points \(A\) and \(B\).
    We define the potential difference between these two points as
    \[\Delta V = V_B - V_A = \frac{W_{AB}}{q}\]
    where \(W_{AB}\) is the work done moving from \(A\) to \(B\) and \(V_A\) and \(V_B\) are the potentials at points \(A\) and \(B\) respectively.
    The units of work are joules so the units of potential difference are joules per coulomb, also known as volts.
    This definition of potential difference doesn't say anything about the values of \(V_A\) and \(V_B\) only the difference between them so we can apply an offset to both without changing anything.
    By convention we define the potential at infinity \(V_\infty = 0\).
    So the potential at point \(A\) is
    \[V_A = \frac{W_{\infty A}}{q} = - \int_{\infty}^A\frac{1}{q}\vv F\cdot d\vv r = -\int_\infty^A\vv E\cdot d\vv r\]
    It is often easier to find the potential \(V\) and use it to calculate \(\vv E\) using
    \[-\grad V = \vv E\]
    For a point charge we know \(\vv E\) so we can find \(V\).
    At a distance \(r\) from the charge the potential is
    \[V(r) = -\int_\infty^r\vv E\cdot d\vv r = \int_r^\infty \vv E\cdot d\vv r\]
    Since the field is radially away from the point charge \(\vv E\) is parallel to \(d\vv r\) so \(\vv E\cdot d\vv r = Edr\):
    \[V = \int_r^\infty E\,dr = \frac{q}{4\pi\varepsilon_0}\int_r^\infty\frac{1}{r^2}\,dr = \frac{1}{4\pi\varepsilon_0}\frac{q}{r}\]
    Consider a charged conductive sphere of charge \(q\) and radius \(R\).
    Outside the sphere we know the electric field is the same as for a point charge \(q\) at the centre of the sphere.
    which gives for \(r \ge R\):
    \[V(r) = \frac{1}{4\pi\varepsilon_0}\frac{q}{r}\]
    Inside the sphere the electric field is zero so moving between points must result in a potential difference of zero.
    This means that the potential in the sphere is the same as the potential at the surface of the sphere so for \(r < R\):
    \[V(r) = \frac{1}{4\pi\varepsilon_0}\frac{q}{R}\]
    The charge is distributed evenly over the surface for a conductor so
    \[q = 4\pi R^2\sigma \implies V(R) = \frac{\sigma R}{\varepsilon_0}\]
    
    \subsection{Superposition of Potentials}
    Consider \(n\) charges \(q_1,\dotsc,q_n\).
    The electric field due to charge \(q_i\) is \(\vv E_i\).
    The total electric field is
    \[\vv E = \sum_i \vv E_i\]
    The potential is then
    \[V(\vv r) = -\int_\infty^r\left(\sum_i \vv E_i\right)\cdot d\vv r = \sum_i\left(-\int_\infty^r \vv E_i\cdot d\vv r\right) = \sum_i V_i\]
    where \(V_i\) is the potential due to charge \(q_i\).
    The total potential is then
    \[V = \sum_i V_i = \frac{1}{4\pi\varepsilon_0}\sum_i\frac{q_i}{r_i}\]
    This allows us to solve many problems by finding the potential due to each individual charge and summing and taking the negative gradient to find \(\vv E\).
    
    \subsection{Dipole Potential}
    \begin{figure}[ht]
        \centering
        \begin{tikzpicture}
            %\draw[lightgray] (0, -1) grid (6, 5);
            \draw[fill=blue, color=blue] (0, 2) circle (0.05);
            \draw[fill=red, color=red] (4, 2) circle (0.05);
            \node[above] at (0, 2) {\(-q\)};
            \node[above] at (4, 2) {\(+q\)};
            \draw[fill=black] (2, 2) circle (0.03);
            \draw[fill=black] (2, 4) circle (0.03);
            \draw[fill=black] (6, -1) circle (0.03);
            \draw (0, 2) -- (6, -1);
            \draw (4, 2) -- (6, -1);
            \draw[dashed] (2, 2) -- (6, -1);
            \node[below left] at (3, 0.5) {\(r_-\)};
            \node at (5, 1) {\(r_+\)};
            \node at (3.25, 1.25) {\(r\)};
            \draw[dotted] (2, 2) -- (3, 2);
            \node at (2.4, 1.9) {\tiny \(\vartheta\)};
            \node[left] at (2, 2) {\(P_0\)};
            \node[above] at (2, 4) {\(P_1\)};
            \node[below right] at (6, -1) {\(P_2\)};
            \draw[<->] (5, 4) -- (5, 3) -- (6, 3);
            \node[above] at (5, 4) {\(\vh y\)};
            \node[right] at (6, 3) {\(\vh x\)};
        \end{tikzpicture}
        \caption{Dipole}
        \label{fig:dipole}
    \end{figure}
    The dipole in figure \ref{fig:dipole} consists of two charges \(+q\) and \(-q\) separated by a distance \(d\)
    We define \(P_0\) directly between the two charges as \((0, 0)\).
    At this point both charges are equidistant with \(r = d/2\).
    The potential at this point is
    \[V_0 = \frac{1}{4\pi\varepsilon_0}\left[\frac{2q}{d} - \frac{2q}{d}\right] = 0\]
    At point \(P_1\) with coordinates \((0, a)\) the distance from the charges is again equidistant but now equal to
    \[r = \left[a^2 + \frac{d^2}{4}\right]\]
    so the potential at this point is
    \[V_1 = \frac{1}{4\pi\varepsilon_0}[r - r] = 0\]
    Extending the symmetry of this we can see that there is a plane of zero potential between the two charges normal to \(\vv p\).
    This plane means that we can do no work to move a charge from infinity to directly between the charges.
    This is because all the field lines are parallel to this plane so the only component of force needed is to keep the particle on the plane so is perpendicular to motion along the plane meaning that it does no work.
    
    At point \(P_2\) the potential is
    \[V_2 = \frac{1}{4\pi\varepsilon_0}\left[\frac{q}{r_+} - \frac{q}{r_-}\right] = \frac{q}{4\pi\varepsilon_0}\frac{r_- - r_+}{r_-r_+}\]
    Far away from the dipole we can make the approximation that \(r_-r_+ \approx r^2\) and \(r_- - r_+\approx d\cos\vartheta\).
    This gives a potential of approximately
    \[V = \frac{1}{4\pi\varepsilon_0}\frac{qd\cos\vartheta}{r^2} = \frac{1}{4\pi\varepsilon_0}\frac{p\cos\vartheta}{r^2}\]
    \[V = \frac{1}{4\pi\varepsilon_0}\frac{\vv p\cdot\vh r}{r^2}\]
    
    We can extend the superposition principle to a continuous distribution of charge by considering the potential \(dV\) due to charge element \(dq\):
    \[V = \int_\text{all charge}dV = \frac{1}{4\pi\varepsilon_0}\int\frac{1}{r}\,dq\]
    
    \example
    Consider the uniformly charged ring in figure \ref{fig:charged ring}.
    What is the potential at point \(P\)?
    The potential due to the charge of length element \(ds\) is given by
    \[dV = \frac{1}{4\pi\varepsilon_0}\frac{\lambda ds}{(R^2 + z^2)^{1/2}}\]
    \[V = \frac{1}{4\pi\varepsilon_0}\frac{\lambda}{(R^2 + z^2)^{1/2}}\int ds = \frac{\lambda R}{2\varepsilon_0(R^2 + z^2)^{1/2}}\]
    where we have used \(\int ds = 2\pi R\).
    The potential depends only on \(z\) so \(\vv E = \ve z\partial_z V\):
    \[\vv E = \pdv{V}{z}\ve z = \frac{\lambda Rz}{2\varepsilon_0(R^2 + z^2)^{1/2}}\ve z\].
    
    \section{Potential Energy}
    Since we have to do work against the electrostatic force and this force is conservative the work done must be stored as potential energy.
    Consider two charges \(q_1\) and \(q_2\). \(q_1\) is fixed and \(q_2\) is taken from infinity to a distance \(r_{12}\) from \(q_1\).
    The work done by an external agent to form this configuration is
    \[U = -\int_\infty^{r_{12}}\vv F\cdot d\vv r\]
    Since for point charges the field is radially symmetric it is possible to do this by a path where \(\vv F\) is parallel to \(d\vv r\) for the whole path.
    This gives us
    \[U = -\int_\infty^{r_{12}}F\,dr = \int_{r_{12}}^\infty F\,dr\]
    The electrostatic potential energy in a system of two charges as above is
    \[U_{12} = \frac{1}{4\pi\varepsilon_0}\int_{r_{12}}^\infty \frac{q_1q_2}{r^2}\,dr = \frac{1}{4\pi\varepsilon_0}\frac{q_1q_2}{r^2}\]
    We now assume that \(q_1\) and \(q_2\) are like charges so \(U_{12} > 0\). When they are released the total energy and total momentum of the system is conserved.
    When far apart we can consider all of the potential energy to have turned into kinetic energy.
    The charges will then have the kinetic energies
    \[K_1 = K_2 = \frac{1}{2}U_{12} = \frac{1}{8\pi\varepsilon_0}\frac{q_1q_2}{r_{12}}\]
    For small charges they will probably be fast enough that we need to use their relativistic kinetic energies \(K = (\gamma - 1)mc^2\) to calculate the velocity but it is possible to find the velocity.
    
    If we want to introduce a third charge \(q_3\) then we get
    \[U = U_{12} + U_{13} + U_{23} = \frac{1}{4\pi\varepsilon_0}\left[\frac{q_1q_2}{r_{12}} + \frac{q_1q_3}{r_{13}} + \frac{q_2q_3}{r_{23}}\right]\]
    In general for \(N\) charges we need to consider \(N(N - 1)/2\) terms since each of \(N\) charges has \(N - 1\) pairs and the factor of two accounts for the double counting of \(U_{ij}\) and \(U_{ji}\).
    For \(N\) charges \(q_i\) we get the total potential energy of the system as
    \[U = \sum_{i>j}^N U_{ij} = \frac{1}{4\pi\varepsilon_0}\sum_{i>j}^N \frac{q_iq_j}{r_{ij}}\]
    A charge \(q\) moved across a potential difference of \(\Delta V\) will have a potential energy change \(\Delta U = q\Delta V\).
    For energy to be conserved we require the kinetic energy change to be \(\Delta K = -\Delta U = -q\Delta V\).
    
    \subsection{Dipole in an External Electric Field}
    \begin{figure}[ht]
        \centering
        \begin{tikzpicture}
            %\draw[cyan] (0, 0) grid (5, 5);
            \draw[lightgray, ->] (0, 0) -- (5, 0);
            \draw[lightgray, ->] (0, 1) -- (5, 1);
            \draw[lightgray, ->] (0, 2) -- (5, 2);
            \draw[lightgray, ->] (0, 3) -- (5, 3);
            \draw[lightgray, ->] (0, 4) -- (5, 4);
            \draw[lightgray, ->] (0, 5) -- (5, 5);
            \node[right, lightgray] at (5, 5) {\(\vv E\)};
            \draw (1, 1) -- (4, 4);
            \draw[fill=blue, color=blue] (1, 1) circle (0.05);
            \draw[fill=red, color=red] (4, 4) circle (0.05);
            \node[above left] at (2.5, 2.5) {\(d\)};
            \node at (2.3, 2.1) {\tiny\(\vartheta\)};
            \node[above right] at (4, 4) {\(+q\)};
            \node[below left] at (1, 1) {\(-q\)};
        \end{tikzpicture}
        \caption{A dipole in an external electric field}
        \label{fig:dipole in an external electric field}
    \end{figure}
    Figure \ref{fig:dipole in an external electric field} shows a dipole in an external electric field.
    
    The positive charge experiences a force \(F_+ = qE\) to the right.
    
    The negative charge experiences a force \(F_- = qE\) to the left.
    
    The total force is \(F_+ + F_- = qE - qE = 0\), there is however still a torque \(\vv \tau = \vv r\times\vv F\).
    
    The torque on the positive charge is \(\tau_+ = qEd\sin(\vartheta)/2\) into the page.
    
    The torque on the negative charge is \(\tau_- = qEd\sin(\vartheta)/2\) into the page.
    
    The total torque is then \(\tau = qEd\sin\vartheta\) into the page.
    Writing this in terms of vectors we get \(\vv \tau = \vv p\times\vv E\).
    This acts to align \(\vv p\) and \(\vv E\).
    If they are in the same direction then we get a stable equilibrium and a small displacement will cause oscillations.
    If they are antiparallel then we get an unstable equilibrium and a small displacement will cause the dipole move rapidly towards the stable equilibrium point.
    To rotate away from the stable equilibrium we must do work so we must stork potential energy.
    \[U(\vartheta) = -W = -\int_0^\vartheta pE\sin\vartheta\,d\vartheta = -pE\cos\vartheta\]
    so the potential energy is
    \[U = -\vv p\cdot\vv E\]
    The potential energy change to go from parallel to antiparallel is \(2pE\).
    We have implicitly defined zero potential energy here as \(\vartheta = \pi/2\) which is fine since we can set it anywhere as we only deal with changes ins potential energy.
    
    \section{Capacitors}
    An electric field can store potential energy.
    To deal with the potential energy stored by a continuous distribution of charge we need to consider capacitance.
    A capacitor is a device for storing charge.
    If a capacitor has a voltage \(V\) applied to store charge \(Q\) then experimentally we see that \(Q\propto V\).
    This gives us
    \[Q = CV\]
    The constant of proportionality \(C\) is the capacitance of the capacitor and has units of \(\si{C.V^{-1}} = \si{F}\) which are called farads after Michael Faraday.
    Farads are a very large unit; most capacitors are measured in micro farads \si{\micro F}.
    Electronic devices typically have a capacitance measured in pico farads \(\si{\pico F} = \SI{e-12}{F}\).
    
    \subsection{Parallel Plate Capacitor}
    The simplest capacitor geometry is two parallel plates of area \(A\) separated by a distance \(d\), with a vacuum between.
    The potential difference between the two points is \(V\).
    From before we know that the electric field between the plates is
    \[E = \frac{\sigma}{\varepsilon_0} = \frac{Q}{\varepsilon_0 A}\]
    The potential difference is given by
    \[V = -\int_d^0 \vv e\cdot d\vv r\]
    since the charge must move through a distance of \(d\) from a plate of opposite charge to a plate of the same charge.
    Between the plates \(\vv E\) is parallel to \(d\vv r\) so \(\vv E\cdot d\vv r = Edr\).
    Furthermore the electric field strength is constant between the plates so
    \[V = -E\int_d^0 dr = E\int_0^d dr = Ed = \frac{Qd}{\varepsilon_0 A}\]
    From our definition of capacitance as \(C = V/Q\) we can see that the capacitance is
    \[C = \frac{\varepsilon_0 A}{d}\]
    This depends only on the geometry of the capacitor and not the circuit connecting the plates.
    This is because the electric field is conservative so it only depends on where the charge starts and ends up, not on the path that it takes.
    To have a large capacitance we want \(A\) to be as large as possible and \(d\) to be as small as possible.
    This gives us alternative units for \(varepsilon_0\) instead of \(\si{C^2.N^{-1}..m^{-2}}\) we can use \(\si{F.m^{-1}}\).
    \[\varepsilon_0 = \SI{8.85e-12}{F.m^{-1}}\]
    
    \subsection{Spherical Capacitor}
    Consider two concentric spheres with radii \(a\) and \(b\) where \(a < b\).
    If we give the two spheres opposite charges \(\pm Q\) then this will be a capacitor.
    Without loss of generality we can assume that the inner sphere is positive and the outer one is negative.
    A spherical Gaussian surface radius \(r > b\) will contain a net charge of \(Q - Q = 0\) so the electric field outside the larger sphere is zero.
    Similarly a spherical Gaussian surface radius \(r < a\) will contain no charge so the electric field inside the smaller sphere is zero.
    A spherical Gaussian surface radius \(a < r < b\) will contain charge \(Q\) so the electric field will be the same as for a charged sphere which is
    \[E = \frac{1}{4\pi\varepsilon_0}\frac{Q}{r^2}\]
    The potential difference between the two spheres is
    \[V = \int_a^b E\,dr = \frac{Q}{4\pi\varepsilon_0}\int_a^b\frac{1}{r^2}\,dr = \frac{Q}{4\pi\varepsilon_0}\frac{b - a}{ab}\]
    This gives a capacitance of
    \[C = 4\pi\varepsilon_0\frac{ab}{b - a}\]
    If we consider an isolated, conducting sphere, radius \(R\), like a Van De Graff generator then this is equivalent to making the outer sphere have an infinite radius.
    Setting \(a = R\) and letting \(b\to\infty\) we get the capacitance of an isolated sphere as
    \[C = 4\pi\varepsilon_0R\]
    
    \subsection{Cylindrical Capacitor}\label{sec:cylindrical capacitor}
    Consider two concentric cylinders, length \(L\), radii \(a\) and \(b\) where \(a < b\).
    Again give the two cylinders opposite charges \(\pm Q\) and without loss of generality assume the inner cylinder is positive.
    As with the spheres a cylindrical Gaussian surface with radius \(r\) outside of the range \(a < r < b\) will contain no net charge so the electric field strength is zero.
    For inside this range the enclosed charge is \(Q\) and the electric field is the same as from a long rod:
    \[E = \frac{\lambda}{2\pi\varepsilon_0 r} = \frac{Q}{2\pi\varepsilon_0 Lr}\]
    The potential difference is
    \[V = \int_a^b E\,dr = \frac{Q}{2\pi\varepsilon_0 L}\int_a^b \frac{1}{r}\,dr = \frac{Q}{2\pi\varepsilon_0 L}\ln\left(\frac{b}{a}\right)\]
    Which gives a capacitance
    \[C = 2\pi\varepsilon_0\frac{1}{\ln(b/a)}\]
    
    \subsection{Dielectrics}
    So far we have assumed that all charged objects are separated by a vacuum.
    Filling this space with an insulator (known in this case as a dielectric) will reduce the electric field between the charges, making it easier to add more charge at the same voltage so increasing the capacitance.
    For sufficiently nice insulators the capacitance increases by a dimensionless constant \(\kappa\) known as the dielectric constant fir that dielectric.
    This gives a capacitance for parallel plate capacitors of
    \[C = \frac{\kappa \varepsilon_0 A}{d}\]
    Note that \(\kappa\)  is also known as the relative permittivity of the dielectric and then has the symbol \(\varepsilon_r\).
    This is often used when dealing with fields not in a vacuum whereas \(\kappa\) is used when dealing with capacitors.
    It is also possible to define the permittivity as \(\varepsilon = \varepsilon_r\varepsilon_0 = \kappa\varepsilon_0\).
    \begin{table}[ht]
        \centering
        \begin{tabular}{c|c}\hline
            Material & \(\kappa\)\\\hline
            Vacuum & 1\\
            Air & 1.00054\\
            Paper & 3.5\\
            Transformer Oil & 4.5\\
            Pyrex & 4.7\\
            Mica & 5.4\\
            Porcelain & 6.5\\
            Water & \(\sim 80\)\\
            Titanic ceramic & 130\\
            Strontium titanate & 310\\\hline
        \end{tabular}
        \caption{Dielectric constant of common dielectrics}
    \end{table}
    The internal electric field of a dielectric is
    \[E = \frac{\sigma}{\kappa\varepsilon_0}\]
    it acts as to oppose the electric field created by the stored charge.
    
    \subsection{Polar Molecules}
    One of the main ways that the internal electric field of a dielectric is created is when there are polar molecules.
    Each molecule has a dipole moment \(\vv p\), for example water has a dipole moment of \(p \approx \SI{6.2e-30}{Cm}\).
    In an external electric field \(\vv E\) a torque \(\vv \tau = \vv p\times\vv E\) is applied to each molecule, aligning the dipoles.
    This forms the opposing electric field \(\vv E_m\).
    We can define the electric polarisation of a material as \(\vv P = N\vv p\) where \(N\) is the number of molecules.
    
    A parallel plate capacitor with a dielectric between the plates has an initial electric field \(E_0\) before the dielectric field is aligned.
    \[E_0 = \frac{Q}{\varepsilon_0 A}\]
    After the dipoles line up the resultant electric field \(E\) is
    \[E = E_0 - E_m = \frac{E_0}{\kappa}\]
    The \(E_m\) field can be thought of as coming from an induced charge \(Q'\) which is such that
    \[E_m = \frac{Q'}{\varepsilon_0 A}\]
    The plates with positive charge \(Q\) will have negative charge induced charge \(-Q'\) and vice versa.
    Substituting and cancelling constants we get
    \[Q - Q' = \frac{Q}{\kappa}\implies Q' = \left(1 - \frac{1}{\kappa}\right)Q\]
    A Gaussian surface around one plate will have a flux of
    \[\Phi_E = EA = \frac{Q - Q'}{\varepsilon_0} = \frac{Q}{\varepsilon_0\kappa}\]
    This gives us a modified version of Gauss's law for a field in a dielectric with dielectric constant \(\kappa\):
    \[\oint \vv E\cdot d\vv A = \frac{q}{\kappa\varepsilon_0}\]
    
    \subsection{Non-polar Molecules}
    A non-polar dielectric will have dipoles induced in it by an external electric field \(\vv E\).
    For linear (class A) dielectrics the induced dipole moment per unit volume is
    \[\vv P = \varepsilon_0\chi_e\vv E\]
    where \(\chi_e\) is the electric susceptibility of the material and is given by
    \[\kappa = 1 + \chi_e\]
    For non-linear dielectrics \(\chi_e\) is a tensor with the rank dependent on just how non-linear the dielectric is and things are much more complicated, but we don't need to know that yet.
    
    The resultant electric field opposes the initial field and the analysis above applies regardless of how the resultant field forms (from polar or non-polar molecules).
    
    \section{Circuits}
    Two capacitors \(C_1\) and \(C_2\) are connected in parallel.
    The potential across both is \(V\).
    The total charge stored is the sum of the charges stored on each:
    \[Q = Q_1 + Q_2 = C_1V + C_2V\]
    This is equivalent to one capacitor \(C_p\) with capacitance \(C_p = C_1 + C_2\).
    Expanding this to \(N\) capacitors, \(C_i\), in parallel this is equivalent to one capacitor of capacitance
    \[C_p = \sum_{i=1}^NC_i\]
    
    Two capacitors \(C_1\) and \(C_2\) are in series as shown in figure \ref{fig:capacitors in series}.
    The potential across each capacitor is \(V_1\) and \(V_2\) and the capacitances are \(C_1\) and \(C_2\).
    The section in the dashed box is isolated so must have a net charge of zero.
    This means that if the top capacitor has a charge \(+Q\) on its top plate then it must have a charge of \(-Q\) on its bottom plate so the charge of the top plate of the bottom capacitor must be \(+Q\).
    This means that both capacitors hold the same charge.
    The total potential difference across both capacitors is
    \[V = V_1 + V_2 = Q\left(\frac{1}{C_1} + \frac{1}{C_2}\right)\]
    This is equivalent to one capacitor of capacitance \(C_s\) where
    \[\frac{1}{C_s} = \frac{1}{C_1} + \frac{1}{C_2}\]
    Extending this to \(N\) capacitors, \(C_i\), in series this is equivalent to one capacitor of capacitance \(C_s\) where
    \[\frac{1}{C_s} = \sum_{i=1}^N\frac{1}{C_i}\]
    
    \begin{figure}[ht]
        \centering
        \begin{tikzpicture}
            %\draw[lightgray] (0, 0) grid (5, 5);
            \draw (2, 0) -- (2, 1);
            \draw (2, 1.5) -- (2, 3);
            \draw (2, 3.5) -- (2, 4.5);
            \draw (1, 1) -- (3, 1);
            \draw (1, 1.5) -- (3, 1.5);
            \draw (1, 3) -- (3, 3);
            \draw (1, 3.5) -- (3, 3.5);
            \node[above right] at (2, 1.5) {\tiny\(+Q\)};
            \node[below right] at (2, 1) {\tiny\(-Q\)};
            \node[below right] at (2, 3) {\tiny\(-Q\)};
            \node[above right] at (2, 3.5) {\tiny\(+Q\)};
            \draw[dashed] (0.5, 1.25) -- (3.5, 1.25) -- (3.5, 3.25) -- (0.5, 3.25) -- (0.5, 1.25);
        \end{tikzpicture}
        \caption{Capacitors in series}
        \label{fig:capacitors in series}
    \end{figure}
    
    \subsection{Energy Stored in a Capacitor}
    To find the energy stored in a capacitor we consider the work done by an external agent charging the capacitor to charge \(Q\).
    The work done \(dW\) to add a small charge \(dQ\) to a capacitor storing charge \(q\) is
    \[dW = Vdq = \frac{q}{C}dq\]
    If we start with no charge and end with charge \(Q\) then the total work done is
    \[W = \frac{1}{C}\int_0^Q q\,dq = \frac{Q^2}{2C}\]
    This work is stored in the electric field as potential energy
    \[U_E = \frac{Q^2}{2C} = \frac{1}{2}CV^2\]
    For a parallel plate capacitor the electric field strength and charge are related by
    \[E = \frac{\sigma}{\varepsilon_0} = \frac{Q}{\varepsilon_0 A} \implies Q = \varepsilon_0AE\]
    and the capacitance is
    \[C = \frac{\varepsilon_0 A}{d}\]
    This gives a potential energy of
    \[U_E = \frac{Q^2}{2C} = \frac{E^2A^2\varepsilon_0^2d}{2\varepsilon_0 A} = \frac{1}{2}\varepsilon_0 E^2Ad\]
    The energy per unit volume, also known as the energy density, is
    \[u_E = \frac{U_E}{V} = \frac{U_E}{Ad} = \frac{1}{2}\varepsilon_0E^2\]
    This is a general equation for the energy stored in an electric field.
    
    \subsection{Current}
    Current, \(I\), is defined as the rate of flow of charge:
    \[I = \dv{Q}{t}\]
    The units of current are amps \(\si{A} = \si{C.s^{-1}}\).
    If the current varies with time such that \(I = I(t)\) then the charge transferred in time \(t\) is
    \[Q = \int_0^t I(t)\,dt\]
    The convention is that current flows from positive to negative despite the fact that most current flow is actually flow of electrons in the opposite direction.
    
    We define the current density \(\vv J\) as the charge flow perpendicular to a surface per unit area per unit time.
    We can think of the current as the flux of the current density:
    \[I = \int \vv J\cdot d\vv A\]
    In the case where \(\vv J\) is constant and normal to the surface:
    \[I = \int\vv J\cdot d\vv A = \int J\,dA = JA\]
    so current density has units of amps per metre \(\si{A.m^{-2}}\).
    
    \subsection{Resistance}
    If a potential difference \(V\) across a component results in a current \(I\) then we define the resistance as
    \[R = \frac{V}{I}\]
    Resistance has units of ohms \(\si{\ohm} = \si{VA^{-1}}\).
    A material that follows Ohm's law has a constant value of \(R\) for all values of \(V\) (while other factors like temperature are held constant).
    
    \subsection{Resistivity}
    If an electric field of strength \(E\) results in a current density \(J\) then the resistivity \(\rho\) is defined as
    \[\rho = \frac{E}{J}\]
    The units of resistivity are ohm meters \si{\ohm.m}.
    Resistivity is a fundamental property of the material.
    It depends on the temperature and other factors but with these controlled it will be a constant.
    \begin{table}[ht]
        \centering
        \begin{tabular}{c|c}\hline
            Material & \(\rho\,(\si{\ohm.m})\)\\\hline
            Silver & \num{1.62e-8}\\
            Copper & \num{1.69e-8}\\
            Aluminium & \num{2.25e-8}\\
            Iron & \num{9.68e-8}\\
            Glass & \num{e10} to \num{e14}\\
            Fused Quartz & \(\sim\num{e16}\)\\\hline
        \end{tabular}
        \caption{Resistivity of a range of materials}
    \end{table}
    A section of material with resistivity \(\rho\) has constant cross sectional area \(A\) and length \(L\).
    A current \(I\) flows through the material due to a voltage \(V\).
    The electric field is given by
    \[E = \frac{V}{L}\implies V = EL\]
    The current density is given by
    \[J = \frac{I}{A} \implies I = JA\]
    Using Ohm's law we get
    \[R = \frac{V}{I} = \frac{EL}{JA} = \rho\frac{L}{A}\]
    
    \subsection{Charging a Capacitor}
    A capacitor \(C\) holding charge \(Q(t)\) and resistor \(R\) are connected with a switch.
    When the switch is closed the resistor and capacitor are in parallel so have the same voltage \(V(t)\) across them.
    The initial voltage is \(V(t=0) = V_0\).
    The voltage and current are given by
    \[V(t) = \frac{Q(t)}{C},\qquad I(t) = \frac{V(t)}{R}\]
    \[I(t) = -\dv{Q(t)}{t}\]
    The negative simply means that the current is away from the capacitor.
    Combining these three equations we get
    \[\dv{V(t)}{t} = \frac{1}{C}\dv{Q}{t} = -\frac{I(t)}{C} = -\frac{V(t)}{RC}\]
    We have the initial conditions that \(V(t=0) = 0\) and that as \(t\to\infty\) \(V(t)\to 0\).
    \[\frac{1}{V}\dv{V}{t} = -\frac{1}{RC}\]
    \[\int\frac{1}{V}\,dV = -\frac{1}{RC}\int dt\]
    \[\ln V = -\frac{1}{RC}t + \ln k\]
    \[V = ke^{-t/RC}\]
    \[V(t=0) = k = V_0\]
    \[V(t) = V_0e^{-t/RC} = V_0e^{-t/\tau}\]
    Where we have defined \(RC = \tau\) which we call the time constant of the circuit.
    It is the time taken for the voltage to fall to \(1/e\) of its initial value.
    
    \subsection{Coaxial Cable}
    A coaxial cable is a core with a protective shield.
    This is the same as the cylindrical capacitor we saw in section \ref{sec:cylindrical capacitor}.
    The core has a radius \(a\) the shield has a radius \(b\).
    The cross sectional area of the core is \(A = \pi a^2\).
    The length of the cable is \(L\).
    The resistance of the cable is
    \[R = \rho\frac{L}{A} = \rho\frac{L}{\pi a^2}\]
    The capacitance is, as before,
    \[C = 2\pi\varepsilon_0\frac{L}{\ln(b/a)}\]
    The time constant \(\tau = RC\) is
    \[\tau = RC = \rho\frac{L}{\pi a^2} 2\pi\varepsilon_0\frac{L}{\ln(b/a)} = 2\rho\varepsilon_0\frac{L^2}{a^2\ln(b/a)}\]
    To transmit high frequency signals we want to charge and discharge quickly so we want \(\tau\) to be a minimum.
    We assume that \(b\) is fixed.
    A small value of \(a\) would result in a large value of \(R\).
    A large value of \(a\) would result in a large value of \(C\).
    There must be some optimum value of \(a\).
    This value occurs when
    \[\dv{\tau}{a} = 0\]
    We find that the minimum value of \(\tau\) occurs at
    \[a = b^{-1/2} \approx 0.606b\]
    Giving a minimum time constant of
    \[\tau_\text{min}\approx 10.87\rho \varepsilon_0 \frac{L^2}{b^2}\]
    
    \section{Magnetic Fields}
    \subsection{Empirical Observations}
    Experimentally we see a magnetic field associated with certain objects, for example:
    \begin{itemize}
        \item Permanent magnets
        \item The Earth
        \item Loops of current carrying wire
    \end{itemize}
    The magnetic fields interact with each other.
    For bar magnets we see that like poles repel and unlike poles attract.
    There is no interaction with a static electric field but there is if it isn't stationary.
    The magnetic filed is a vector field from the north to south pole of a magnet.
    
    For a charge \(q\) at a velocity \(\vv v\) in a magnetic field \(\vv B\):
    \begin{itemize}
        \item If \(v = 0\) there is no force, from this we conclude that stationary magnetic and electric fields don't interact.
        \item If \(\vv v\) is parallel to \(\vv B\) then there is no force.
        \item If \(\vv v\) is perpendicular to \(\vv B\)  then the force \(F_B\propto qvB\)
    \end{itemize}
    Using these last two properties we can see that this behaviour could be described with a cross product:
    \[F_B = q\vv v\times\vv B\]
    where the magnetic field strength is defined as
    \[B = \frac{F_B}{|q|v}\]
    The units of \(B\) are \(\si{NC^{-1}m^{-1}} = \si{NA^{-1}m^{-1}} = \si{T}\) where the last unit is known as the tesla.
    \(\SI{1}{T}\) is quite a strong field strength.
    Most permanent magnets have field strength of the order of \(\si{mT}\).
    Since the velocity is perpendicular to the force for a point charge the magnetic force does no work.
    
    \subsection{Circular Motion in a Magnetic Field}
    In a uniform magnetic field a moving charged particle will do circular motion in the plane normal to \(\vv B\).
    We can balance the centripetal for ce \(F_c\) with the magnetic force \(F_B\):
    \[F_B qvB = \frac{mv^2}{r} = F_c\]
    Rearranging we get
    \[r = \frac{mv}{qB}\]
    No work is done so \(v\) is a constant as is the kinetic energy.
    The period of this rotation is
    \[T = \frac{2\pi r}{v} = \frac{2\pi m}{qB}\]
    The angular frequency is
    \[\omega = \frac{qB}{m}\]
    this is known as the cyclotron frequency.
    It only depends on the mass to charge ratio for a constant magnetic field strength.
    
    \subsection{Cyclotron}
    A cyclotron is formed of two D shaped electrodes which are connected to an oscillating voltage source \(V(t) = V_0\cos(\omega t)\).
    A permanent magnetic field \(\vv B\) is into these electrodes.
    A particle is injected and starts to perform circular motion.
    The electrodes switch their voltage at a rate such that the particle always sees a potential difference of \(-V_0\) across the gap between the electrodes.
    This ensures that the particle is accelerated every time it crosses this gap.
    Every cycle the kinetic energy increases by \(\Delta K = 2qV_0\).
    The radius must also increase.
    At some maximum radius \(R\) the particle is deflected out of the cyclotron, by an electric field, into a beam of particles.
    The total kinetic energy is
    \[K = \frac{q^2B^2R^2}{2m}\]
    assuming \(v\ll c\).
    Including relativistic effects
    \[\omega = \frac{qB}{\gamma m} = \frac{qB}{m + K/c^2}\]
    so \(\omega\) depends on \(K\) so we would have to change \(V(t)\) to account for this.
    In practice cyclotrons are only used for heavy particles at non-relativistic speeds.
    
    \subsection{Synchrotron}
    For higher energies we must use a different accelerator.
    A synchrotron has alternating magnetic fields, to curve the path of the particles, and potential differences, to accelerate the particles.
    The magnetic field strength, \(B\), can be varied to keep the radius, \(R\), constant.
    The required field strength is
    \begin{equation}\label{eqn:cyclotron magnetic field strength}
        B(K) = \left(\frac{2mK}{q^2R^2}\right)^\frac{1}{2}
    \end{equation}
    As \(K\) becomes larger relativistic effects become significant but the timing of acceleration pulses and \(\vv B\) can account for this.
    
    The largest synchrotron in existence is the LHC where \(K = \SI{13}{TeV}\) which gives \(\gamma \approx 13856\).
    We can modify equation \ref{eqn:cyclotron magnetic field strength} to account for this since relativistically the only change to the equation is that \(m\to\gamma m\).
    This gives
    \[B(\gamma) = \left(\frac{2m^2c^2\gamma(\gamma - 1)}{q^2R^2}\right)^\frac{1}{2} \approx \frac{\sqrt{2}mc\gamma}{qR}\]
    since \(\gamma - 1 \approx \gamma\) at sufficiently high speeds.
    At LHC speeds the error in this approximation is only \(\SI{0.007}{\%}\).
    In the LHC the maximum magnetic field strength is \(B = \SI{8.4}{T}\) which is very high.
    
    \section{Charges in a Magnetic Field}
    \subsection{Charged Particles in Magnetic Fields}
    The total force on a charged particle, \(q\), with velocity \(\vv v\) in an electric field, \(\vv E\), and magnetic field, \(\vv B\), is
    \[\vv F = \vv F_E + \vv F_B = q(\vv E + \vv v\times\vv B)\]
    This is known as the Lorentz force.
    A charged particle travelling perpendicular to orthogonal electric and magnetic fields will experience forces in opposite directions from both fields.
    It is possible to balance these forces so that the net force is zero.
    Then
    \[qE = qvB \implies v = \frac{E}{B}\]
    this allows us to measure \(v\) without knowing the charge or mass of the particle.
    
    If the magnetic field is then turned off we can measure how much the particle is deflected along the \(y\) axis as it passes through the electric field.
    The force \(\vv F_E = q\vv E\) so the vertical acceleration is
    \[a = \frac{qE}{m}\]
    In time \(t\) a distance of \(L\) perpendicular to the field is travelled so
    \[v = \frac{L}{t}\]
    The vertical displacement is then
    \[y = \frac{1}{2}at^2 = \frac{qEL^2}{2mv^2}\]
    substituting \(v = E/B\) we get
    \[\frac{m}{q} = \frac{B^2L^2}{2yE}\]
    This gives the mass to charge ratio in easily measurable terms.
    
    \subsection{Current in a Wire}
    Recall the definition of current:
    \[I = \dv{q}{t}\]
    From a microscopic perspective current is a flow of charge carriers.
    If there are \(n\) charge carriers per unit volume and they move at a drift velocity \(v_D\) then the number that pass through a plane, area \(A\), normal to the direction of current flow, in one second is given by
    \[N = nAv_D\]
    The current is then given by
    \[I = nqAv_D\]
    where \(q\) is the charge of one charge carrier.
    
    In a metal wire a charge carrier is an electron so \(q = -e\) and \(n\) is the number of electrons donated per atom times the number of atoms per unit volume.
    We account for the fact that electrons are negative but current is defined as positive by taking negative particles as travelling at speed \(-v_D\) such that it is the same as a positive particle with the same magnitude of charge and travelling at speed \(v_D\).
    The wire has length, \(L\), and is in a uniform magnetic field, \(B\).
    We define a unit vector \(\vh l\) in the direction of current flow.
    The force on the wire is
    \[\vv F_B = -eNv_D\vh l\times\vv B\]
    This follows from the force on one particle of charge \(q = -eN\) and velocity \(\vv v = v_D\vh l\).
    If the cross sectional area of the wire is \(A\) then \(N = nAL\) where \(n\) is the number density of electrons.
    We define \(\vv L = L\vh l\).
    The force is then
    \[\vv F_B = I\vv L\times\vv B\]
    If the wire isn't straight then it can be split into straight elements \(\dd \vv L\) which each experience a force
    \[\dd \vv F_B = I\dd\vv L\times\vv B\]
    \[\vv F_B = \int_\text{wire}\dd\vv F_B = I\int_\text{wire}\dd\vv L\times\vv B\]
    
    \subsection{Hall Effect}
    A metallic strip of width \(d\) and thickness \(l\) has a current \(I\) running through it.
    It is in an external magnetic field \(\vv B\) into the surface of the conductor.
    An electron at drift velocity \(v_D\) in the conductor experiences a force
    \[F_B = ev_DB\]
    perpendicular to the current flow.
    The electrons then build up on one side of the conductor.
    This charge separation results in an induced electric field, \(\vv E\).
    This in turn results in a force, \(F_E\), in the opposite direction to the force due to the magnetic field.
    If the system is allowed to reach equilibrium the sum of these forces will be zero so
    \[eE = ev_DB\implies E = v_DB\]
    This induced electric field results in a potential difference, \(V_H\), known as the Hall voltage:
    \[V_H = dv_DB\]
    If the current is \(I\) and the current density is \(J\) then
    \[v_D = \frac{J}{ne} = \frac{I}{neA}\]
    where \(A = dl\) is the cross sectional area of the conductor.
    The number density is
    \[n = \frac{BI}{V_Hle}\]
    This allows us to measure \(n\) easily.
    Also the magnetic field strength is
    \[B = \frac{neV_Hl}{I}\]
    This allows us to measure \(V_H\) to find \(B\) if we know enough about the conductor.
    This is how a Hall probe works.
    
    \subsection{Torque on a Current Loop}
    \begin{figure}[ht]
        \centering
        \begin{tikzpicture}
            %\draw[lightgray] (0, 0) grid (5, 5);
            \foreach \x in {0,...,5} {
                \draw[lightgray, ->] (0, \x) -- (5, \x);
            }
            \node[lightgray, right] at (5, 2.5) {\(\vv B\)};
            \draw (1, 3) -- (2, 4);
            \draw (3, 1) -- (4, 2);
            \draw (1, 3) -- (1.95, 2.05);
            \draw (3, 1) -- (2.05, 1.95);
            \draw (2, 4) -- (4, 2);
            \draw (2.05, 1.95) -- (1.05, 0.95);
            \draw (1.95, 2.05) -- (0.95, 1.05);
            \draw[->] (1.05, 0.95) -- (1.55, 1.45);
            \draw[->] (1.95, 2.05) -- (1.45, 1.55);
            \draw[->] (2.05, 1.95) -- (2.525, 1.475);
            \draw[->] (1, 3) -- (1.475, 2.525);
            \draw[->] (3, 1) -- (3.5, 1.5);
            \draw[->] (4, 2) -- (3, 3);
            \draw[->] (2, 4) -- (1.5, 3.5);
            \node[above right] at (3, 3) {\(b\)};
            \node[above left] at (1.5, 3.5) {\(a\)};
            \node[below right] at (1.5, 3.5) {1};
            \node[below left] at (3, 3) {4};
            \node[above left] at (3.5, 1.5) {3};
            \node[above right] at (2, 2) {2};
            
            \foreach \x in {0,...,5} {
                \draw[lightgray, ->] (6, \x) -- (11, \x);
            }
            \node[lightgray, right] at (11, 2.5) {\(\vv B\)};
            \draw (7, 4) -- (10, 1);
            \node at (7, 4) {\(\odot\)};
            \node at (10, 1) {\(\otimes\)};
            \draw[->] (7, 4) -- (7, 5);
            \draw[->] (10, 1) -- (10, 0);
            \node[above] at (7, 5) {\(\vv F_1\)};
            \node[below] at (10, 0) {\(\vv F_3\)};
            \draw[->] (8.5, 2.5) -- (9.5, 3.5);
            \node[above right] at (9.5, 3.5) {\(\vh n\)};
        \end{tikzpicture}
        \caption{A current loop in an external magnetic field}
        \label{fig:current loop}
    \end{figure}
    Figure \ref{fig:current loop} shows a loop of wire carrying a current.
    The forces on section 2 and 4 are equal in strength
    \[F_2 = F_4 = bIB\cos\vartheta\]
    These are, however, in opposite directions and act along the same line so cancel and do not create a net torque.
    The forces on section 1 and 3 are also equal in magnitude:
    \[F_1 = F_3 = aIB = F\]
    likewise these forces are in opposite directions but they are generally not collinear so they do cause a torque.
    In fact both act a distance \(b/2\) from the pivot and act clockwise so the torque created is
    \[\tau = 2F\frac{b}{2}\sin\vartheta = bF\sin\vartheta = abIB\sin\vartheta = AIB\sin\vartheta\]
    where \(A = ab\) is the area of the loop.
    The current direction can be defined by a unit vector normal to the plane of the loop such that when looking along the unit vector the current flows clockwise around it.
    This vector \(\vh n\) is at an angle of \(\vartheta\) the the magnetic field.
    The vector torque is then
    \[\vv\tau = IA\vh n\times\vv B\]
    This only depends on the area of the loop not its shape.
    This is actually true for any shape of current loop but showing this is too hard for this course.
    We define the magnetic dipole moment of the loop as \(\vv \mu = IA\vh n\) then as an analogy to a magnetic dipole
    \[\vv \tau = \vv\mu\times\vv B\]
    The loop is subject to a torque so it must store potential energy.
    We define zero potential energy at \(\vartheta = \pi/2\).
    At \(\vartheta = 0\) the torque \(\tau = 0\).
    The potential energy is the negative of the work done so can be calculated as
    \[U(\vartheta) = -W(\vartheta) = -\int_0^\vartheta \tau\,\dd\vartheta\]
    Note that the torque opposes motion away from \(\vartheta = 0\) so for positive \(\vartheta\) the torque is anticlockwise so \(\tau(\vartheta) = -AIB\sin\vartheta\).
    Hence
    \[U(\vartheta) = -\int_0^\vartheta-AIB\sin\vartheta\,\dd\vartheta = -AIB\cos\vartheta\]
    Note that we can find \(B\cos\vartheta = \vh n\cdot\vh B\).
    We can then see that the potential energy stored is given by
    \[U(\vartheta) = -\vv\mu\cdot\vv B\]
    
    \section{Biot--Savart Law}
    A current element \(I\dd\vv s\), where \(\dd\vv s\) is in the direction of current flow, will induce a magnetic field, \(\dd\vv B\), at a position \(\vv r\) relative to the current element.
    This magnetic field will be normal to the plane defined by \(\vv r\) and \(\dd\vv s\).
    Experimentally we see that the strength of this field is
    \[\dd B = \frac{\mu_0}{4\pi}\frac{I\dd s\sin\vartheta}{r^2}\]
    where \(\vartheta\) is the angle between \(\dd\vv s\) and \(\vv r\) and \(\mu_0 = \SI{4\pi e-7}{T.m.A^{-1}}\) is the permeability of free space.
    This can be written as a vector equation as
    \[\dd\vv B = \frac{\mu_0}{4\pi}\frac{I\dd\vv s\times\vh r}{r^2}\]
    This is the Biot--Savart law.
    The magnetic field at a point \(\vv r\) is
    \[\vv B = \frac{\mu_0}{4\pi}\int \frac{I\dd \vv s\times (\vv r - \vv s)}{|\vv r - \vv s|^3}\]
    where \(\vv s\) is the position vector of each current element and \(\vv r\) is the position vector of the point where the field is being calculated.
    
    \subsection{Magnetic Field at the Centre of a Current Loop}
    A loop of wire, radius \(R\), carries current \(I\) clockwise in the plane of the page.
    For every current element \(I\dd\vv s\) the magnetic field and relative position vector of the centre of the loop is perpendicular to \(\dd\vv s\) since \(\dd\vv B \propto \dd \vv s\times\vh r\) and the the radius is perpendicular to the tangent.
    This means that \(\vartheta = \pi/2\) for all current elements.
    Therefore \(|I\dd\vv s \times\vh r| = I\dd s\).
    Applying the Biot--Savart law we get
    \[\dd B = \frac{\mu_0}{4\pi}\frac{I\dd s}{R^2}\]
    \[B = \frac{\mu_0}{4\pi}\frac{I}{R^2}\int\dd s = \frac{\mu_0}{4\pi}\frac{I}{R^2}2\pi R = \frac{\mu_0I}{2R}\]
    The field will be into the page for a clockwise current.
    
    \subsection{Magnetic Field of a Wire}
    \begin{figure}[ht]
        \centering
        \begin{tikzpicture}
            \draw (0, 0) -- (0, 3);
            \draw (0.25, 0) -- (0.25, 3);
            \draw[->] (0.125, 3) -- (0.125, 3.5);
            \node[above] at (0.125, 3.5) {\(I\)};
            \draw (0.25, 0.5) -- (3, 0.5);
            \node[below] at (1.625, 0.5) {\(R\)};
            \draw (3, 0.5) -- (0.25, 2);
            \node[above right] at (1.625, 1.25) {\(r\)};
            \draw (0, 1.9) -- (0.25, 1.9);
            \draw (0, 2.1) -- (0.25, 2.1);
            \node[left] at (0, 2) {\(\dd s\)};
            \node at (3, 0.5) {\(\otimes\)};
            \draw[|-|] (-0.75, 0.5) -- (-0.75, 2);
            \node[left] at (-0.75, 1.25) {\(s\)};
        \end{tikzpicture}
        \caption{Current carrying wire}
        \label{fig:current carrying wire}
    \end{figure}
    An infinitely long wire carries a current \(I\) as shown in figure \ref{fig:current carrying wire}.
    For all current elements \(I\dd\vv s s\) both \(\dd\vv B\) and \(\vh r\) are perpendicular to \(\dd\vv s\).
    \(\dd\vv B\) is into the page.
    \(|I\dd\vv s\times\vh r| = I\dd s\sin\vartheta\) so
    \[\dd B = \frac{\mu_0}{4\pi}\frac{Ids\sin\vartheta}{r^2}\]
    \[B = \int_{-\infty}^{\infty}\dd B = 2\int_0^\infty\dd B = 2\int_0^\infty \frac{\mu_0}{4\pi}\frac{I\sin\vartheta}{r^2}\,\dd s = \frac{\mu_0 I}{2\pi}\int_0^\infty\frac{\sin\vartheta}{r^2}\,\dd s\]
    \[r = \sqrt{s^2 + R^2},\qquad \sin\vartheta = \frac{R}{\sqrt{s^2 + R^2}}\]
    \[B = \frac{\mu_0 I}{2\pi}\int_0^\infty\frac{R}{(s^2 + R^2)^{3/2}} = \frac{\mu_0 I}{2\pi}\left[\frac{sR}{R^2(s^2 + R^2)^{1/2}}\right]_0^\infty\]
    \[\lim_{s\to\infty}\frac{s}{(s^2 + R^2)^{1/2}} = \lim_{s\to\infty}\frac{1}{(1 + R^2/s^2)^{1/2}} = 1\]
    \[B = \frac{\mu_0 I}{2\pi R}\]
    The result is the magnetic fields lines are circles around the wire with \(B\propto 1/R\).
    
    \subsection{Parallel Wires}
    Two wires \(a\) and \(b\) run parallel carrying currents \(I_a\) and \(I_b\) respectively.
    The current \(I_a\) induces a magnetic field \(\vv B_a\) at the centre of wire \(b\).
    \(\vv B_a\) is down and has magnitude \(B_a = \mu_0 I_a/2\pi d\) where \(d\) is the distance between the wires.
    There is now a current and magnetic field so there is a force
    \[\vv F_{ba} = I_b\vv L\times\vv B_a\]
    \[F_{ba} = \frac{\mu_0}{2\pi}\frac{LI_aI_b}{d}\]
    and \(F_{ba}\) is towards wire \(a\).
    
    This set up was used to define the ampere as the current \(I_a = I_b = \SI{1}{A}\) for two parallel conductors \(\SI{1}{m}\) apart in a vacuum such that the force between them per metre is \(\SI{2e-7}{N.m^{-1}}\).
    The definition has changed recently to be in terms of fundamental constants, specifically the charge of an electron.
    
    \subsection{Current Ring}
    \begin{figure}[ht]
        \centering
        \begin{tikzpicture}
        %\draw[lightgray] (0, 0) grid (6, 6);
        \draw (3, 0) ellipse (3cm and 1.5cm);
        \draw (3, 0) ellipse (2.9cm and 1.4cm);
        \draw[fill=black] (3, 4) circle (0.05cm);
        \draw (3, 4) -- (6, 0);
        \draw (3, 4) -- (3, 0);
        \draw (3, 0) -- (6, 0);
        \node[left] at (3, 2) {\(z\)};
        \node at (4.6, 2.1) {\(\vv r\)};
        \draw[->] (6, 0) -- (4.5, 2);
        \node[below] at (4.5, 0) {\(R\)};
        \node[left] at (3, 4) {\(P\)};
        \node[above right] at (3, 4.3) {\(\alpha\)};
        \node[above left] at (5.7, 0) {\(\alpha\)};
        \draw[->] (3, 4) -- (4, 4);
        \draw[->] (3, 4) -- (3, 5);
        \draw[->] (3, 4) -- (4, 5);
        \node[right] at (4, 4) {\(\dd\vv B_\perp\)};
        \node[above] at (3, 5) {\(\dd\vv B_z\)};
        \node[above right] at (4, 5) {\(\dd\vv B\)};
        \draw[->] (6, 0) -- (6, 1);
        \node[above] at (6, 1) {\(\dd\vv s\)};
        \end{tikzpicture}
        \caption{Current carrying ring}
        \label{fig:current ring}
    \end{figure}
    Figure \ref{fig:current ring} shows a ring carrying a current \(I\) in the \(x{-}y\) plane.
    What is the magnetic field strength on the axis of the ring at a distance \(z\) from the plane containing the ring?
    The field strength due to a current element \(I\dd\vv s\) is
    \[\dd B = \frac{\mu_0}{4\pi}\frac{I\dd s}{r^2}\]
    This is normal to the \(\dd\vv s{-}\vv r\) plane.
    We can split \(\dd\vv B\) into two components.
    \(\dd B_z = \dd B\sin\alpha\) is in the \(z\) direction.
    \(\dd B_\perp = \dd B\cos\alpha\) is perpendicular to the horizontal.
    The symmetry of the situation means that the perpendicular components cancel.
    This leaves only the \(\dd B_z\) components.
    \[r = \sqrt{R^2 + z^2},\qquad\cos\alpha = \frac{R}{r} = \frac{R}{\sqrt{R^2 + z^2}}\]
    \[\dd B_z = \frac{\mu_0 IR}{4\pi(R^2 + z^2)^{3/2}}\dd s\]
    \[\int \dd s = 2\pi R\]
    \[B(z) = \int\dd B_z = \frac{\mu_0IR^2}{2(R^2 + z^2)^{3/2}}\]
    For \(z\gg R\)
    \[B(z)\approx \frac{\mu_0 IR^2}{2z^3}\]
    The area of the loop is \(A = \pi R^2\).
    We define the magnetic dipole moment as \(\vv \mu = IA\vh z\)
    \[\vv B = \frac{\mu_0}{2\pi}\frac{\vv\mu}{z^3}\]
    for \(z\gg R\).
    This is analogous to the far field approximation for an electric dipole which also falls away with \(1/z^3\).
    
    \subsection{Vector Potential}
    Like an electric field we can define a potential.
    For a magnetic field we need a vector potential, \(\vv A\).
    This is defined as being such that \(\vv B = \curl\vv A\).
    It can be shown that
    \[\vv A = \frac{\mu_0}{4\pi}\int_C \frac{I}{r}\dd\vv s\]
    where \(C\) is a path along a current loop defined by \(\dd\vv s\) and \(r\) is the distance from a current element \(I\dd\vv s\).
    
    \subsection{Electron Orbits}
    Electrons orbiting around an atom can be thought of as a current.
    We define the orbit to be in the \(x{-}y\) plane.
    The time period is
    \[T = \frac{2\pi r}{v}\]
    where \(r\) is the radius of the orbit and \(v\) is the velocity of the electron.
    The current created by this electron movement is
    \[I = \frac{-e}{T} = -\frac{ev}{2\pi r}\]
    The magnetic dipole created by this orbit is
    \[\vv\mu_\text{orb} = IA\vh z = -\frac{evr}{2}\vh z\]
    since \(A = \pi r^2\).
    The angular momentum of the associated orbit is
    \[\vv L_\text{orb} = m(\vv r \times \vv v) = mvr\vh z\]
    Equating the two we get
    \[-\frac{e}{2m}\vv L_\text{orb} = \vv\mu_\text{orb}\]
    We can't measure \(\vv L_\text{orb}\) only the components.
    For example the \(z\) component
    \[L_\text{orb}^z = m_l\hbar\]
    where \(m_l = 0,\pm 1,\pm 2,\dotsc\).
    From this we see
    \[\mu_\text{orb}^z = -m_l\frac{e\hbar}{2m}\]
    We define a quantised unit of magnetic dipole moment as
    \[\mu_B = \frac{e\hbar}{2m} = \SI{9.274e-24}{JT^{-1}}\]
    \(\mu_B\) is known as the Bohr magneton and allows us to write
    \[\mu_\text{orb}^z = -m_l\mu_B\]
    The electrons spin is an intrinsic angular momentum that it also has.
    The spin dipole moment is
    \[\vv\mu_s = -\frac{e}{m}\vv S\]
    where \(\vv S\) is the spin vector.
    Again spin can only be measured along one axis at a time.
    Along the \(z\) axis the spin is
    \[S_z = \pm\frac{\hbar}{2}\]
    \[\mu_s^z = \pm\mu_B\]
    
    \section{Ampere's Law}
    Ampere's law is that for any closed curve, \(C\), (henceforth known as an Amperian loop) with a current, \(I_\text{enc}\), passing through the surface enclosed by the loop then the relationship
    \[\oint_C\vv B\cdot\dd\vv s = \mu_0I_\text{enc}\]
    will hold.
    
    Note that \(I_\text{enc}\) is the current flow in one direction and hence if there current in one direction is \(I_1\) and the current in the other direction is \(I_2\) then the current \(I_\text{enc} = I_1 - I_2\).
    There is ambiguity about which way around the closed curve the line integral should be evaluated and this ambiguity fits the ambiguity in the direction we choose to measure the current.
    
    Like Gauss's law any current not flowing through the surface can be ignored.
    
    \subsection{Magnetic Field From a Wire}
    A wire of radius \(a\) carries a current \(I\).
    We construct a circular Amperian loop of radius \(R\).
    For the case when \(R > a\) we know from the Biot--Savart law that \(\dd\vv s\) is parallel to \(\vv B\) at all points outside the wire.
    Hence
    \[\oint\vv B\cdot\dd\vv s = B\oint\dd s = 2\pi RB = \mu_0 I\]
    \[B = \frac{\mu_0 }{2\pi R}\]
    This is the same result we derived with the Biot--Savart law but with much less work.
    We now know that this is true for a thick wire whereas before we assumed negligible thickness.
    We now consider the case where \(R < a\).
    We also assume that current density is uniform across the wire.
    The current density is then
    \[J = \frac{I}{\pi a^2}\]
    \[I_\text{enc} = \pi R^2 J = \frac{R^2}{a^2}I\]
    \[\oint\vv B\cdot\dd\vv s = B\oint\dd s = 2\pi RB\]
    \[B = \frac{\mu_0 I}{2\pi a}\]
    Evaluating both of these at \(R = a\) we see that the magnetic field strength is continuous at the boundary.
    
    \subsection{Solenoid}
    A solenoid is a current carrying wire wrapped around.
    Solenoids commonly come in two shapes cylindrical or toroidal.
    
    \begin{figure}[ht]
        \centering
        \begin{tikzpicture}
            \foreach \x in {0,...,5} {
                \node at (\x, 0) {\(\otimes\)};
                \node at (\x, 2) {\(\odot\)};
            }
            \foreach \x in {0.5,1,1.5}{
                \draw[lightgray, ->] (0, \x) -- (5, \x);
            }
            \node[lightgray, right] at (5, 1) {\(\vv B\)};
            \draw (0.5, 1.25) rectangle (4.5, 2.75);
            \node[below left] at (0.5, 1.25) {\(a\)};
            \node[below right] at (4.5, 1.25) {\(b\)};
            \node[above right] at (4.5, 2.75) {\(c\)};
            \node[above left] at (0.5, 2.75) {\(d\)};
            \draw[->] (4.5, 2.75) -- (2.5, 2.75);
            \node[above] at (2.5, 2.75) {\(\dd\vv s\)};
            \node[left] at (-0.1, 2) {\(I\)};
        \end{tikzpicture}
        \caption{Cross section of a solenoid}
        \label{fig:solenoid}
    \end{figure}
    A solenoid is formed from a wire wrapped around a cylinder.
    The wire carries a current, \(I\).
    There are \(n\) turns per unit length.
    An Amperian loop is constructed as shown in figure \ref{fig:solenoid}.
    We assume an infinitely long solenoid so there are no edge effects.
    The magnetic field strength outside the solenoid will be zero, this can be seen by considering an Amperian loop containing the whole solenoid.
    The net current flow through this loop is zero so \(B = 0\) outside the solenoid.
    We, however, are interested in the field inside the solenoid.
    For the Amperian loop shown we have
    \[\oint_\text{loop}\vv B\cdot\dd\vv s = \mu_0Inh\]
    where \(h\) is the length of the line from \(a\) to \(b\).
    We can split the integral into sections and we get
    \[\oint_\text{loop}\vv B\cdot\dd\vv s = \int_a^b\vv B\cdot\dd\vv s + \int_b^c\vv B\cdot\dd\vv s + \int_c^d\vv B\cdot\dd\vv s + \int_d^a\vv B\cdot\vv s\]
    Between \(a\) and \(b\) \(\vv B\) and \(\dd\vv s\) are parallel.
    This means that \(\vv B\cdot\dd\vv s = B\dd s\) so
    \[\int_a^b\vv B\cdot\dd\vv s = B\int_a^b\dd s = Bh\]
    From \(b\) to \(c\) and \(d\) to \(a\) where there is a magnetic field it is perpendicular to \(\dd\vv s\) so \(\vv B\cdot\dd\vv s = 0\) so
    \[\int_b^c\vv B\cdot\dd\vv s = \int_d^a\vv B\cdot\dd\vv s = 0\]
    From \(c\) to \(d\) \(B = 0\) so
    \[\int_c^d\vv B\cdot\dd\vv s = 0\]
    Hence
    \[\int_\text{loop}\vv B\cdot\dd\vv s = Bh = \mu_0In\]
    \[B = \mu_0 In\]
    This is constant and independent of the radius of the solenoid or where in the solenoid it is measured.
    
    A toroidal solenoid, of inner radius \(r_1\) and outer radius \(r_2\), is formed from a wire wrapped around \(N\) times.
    Consider an Amperian loop of radius \(R\in[r_1, r_2]\).
    The magnetic field goes around inside the torus.
    \[\oint\vv B\cdot\dd\vv s = B\int_0^{2\pi R}\dd s = 2\pi RB\]
    The total current is \(IN\) so the field strength is
    \[B = \frac{\mu_0}{2\pi}\frac{NI}{R}\]
    inside the torus.
    
    \section{Induction}
    \subsection{Experimental Observations}
    A magnet is moved through a loop of wire and the current in the wire is measured.
    It is observed that:
    \begin{itemize}
        \item Faster movement of the magnet results in a higher current
        \item Stationary magnets do not cause a current
        \item If the north pole is moved towards the loop the current is anticlockwise
        \item If the north pole is moved away from the loop the current is clockwise
        \item The current direction reverses if considering the south pole
    \end{itemize}
    Consider two loops of wire.
    One is connected to an ammeter.
    The other is connected in series to a voltage source, switch and resistor of resistance \(R\).
    The two loops are close together and the planes of each loop are parallel.
    When the switch is closed a current flows in the loop with the voltage source.
    Momentarily a current also flows in the other loop.
    When the switch is opened the current flow in the powered loop stops and in the other loop there is a current pulse in the opposite direction to before.
    Reducing the resistance \(R\) decreases the length of the pulse current and increases the peak current in the other loop.
    We deduce that:
    \begin{itemize}
        \item Induced current is a result of a time varying magnetic field
        \item The larger the time variance (\(\partial_t B\)) the larger the induced current
    \end{itemize}

    \subsection{Magnetic Flux}
    The flux of the magnetic field, \(\Phi_B\), through a surface \(S\) is defined as
    \[\Phi_B = \int_S\dd\Phi_B = \int_S\vv B\cdot\dd\vv A\]
    The units of magnetic flux are \([\Phi_B] = \si{T.m^2} = \si{Wb}\) where the last unit is the weber.
    
    \subsection{Faraday's Law}
    The induced electromotive force (emf), \(\emf\), is given by Faraday's law:
    \[\emf = -\dv{\Phi_B}{t} = -\dv{t}\int_S\vv B\cdot\dd\vv A\]
    The units of emf is \([\emf] = \si{V}\).
    If the loop has \(N\) turns then the total emf is
    \[\emf_T = -N\dv{\Phi_B}{t}\]
    If the resistance of the loop is \(R\) then the induced current is
    \[I = \frac{\emf_T}{R} = -\frac{N}{R}\dv{\Phi_B}{t}\]
    
    \subsection{Lenz's law}
    Lenz's law is the observation
    \begin{quote}
        The induced current has a direction such that the magnetic field due to this current opposes the change in the magnetic field that caused it.
    \end{quote}
    This is where the minus sign in Faraday's law comes from.
    \begin{figure}[ht]
        \centering
        \begin{tikzpicture}
            %\draw[lightgray] (0, 0) grid (6, 6);
            \foreach \x in {0,...,4} {
                \foreach \y in {0,...,4} {
                    \node[lightgray] at (\x, \y) {\(\otimes\)};
                }
            }
            \node[above left, lightgray] at (0, 4) {\(\vv B\)};
            \draw[lightgray] (0, 0) rectangle (4, 4);
            \draw (0.5, 0.5) rectangle (5.5, 3.5);
            \draw[->] (5.5, 2) -- (6.5, 2);
            \node[right] at (6.5, 2) {\(\vv v\)};
            \draw[<->] (4.5, 0.5) -- (4.5, 3.5);
            \node[left] at (4.5, 2) {\(L\)};
            \draw[<->] (0.5, 1.5) -- (4, 1.5);
            \node[below] at (2.5, 1.5) {\(x\)};
            \draw[->] (0.5, 2) -- (-0.5, 2);
            \node[left] at (-0.5, 2) {\(\vv F_1\)};
            \draw[->] (2.5, 0.5) -- (2.5, -0.5);
            \node[below] at (2.5, -0.5) {\(\vv F_2\)};
            \draw[->] (2.5, 3.5) -- (2.5, 4.5);
            \node[above] at (2.5, 4.5) {\(\vv F_3\)};
        \end{tikzpicture}
        \caption{Conductive loop being removed from a magnetic field}
        \label{fig:loop removed from magnetic field}
    \end{figure}

    Figure \ref{fig:loop removed from magnetic field} shows a conductive loop being removed from a magnetic field at a velocity \(v\)
    The flux through the loop is
    \[\Phi_B = \int\vv B\cdot\dd\vv A = B\int\dd A = BxL\]
    The emf is given by
    \[\emf = -\dv{\Phi_B}{t} = \dv{t}BLx = BLv\]
    Notice that the negative cancels with the fact that the flux is decreasing.
    If the resistance of the loop is \(R\) then the induced current is
    \[I = \frac{\emf}{R} = \frac{BLv}{R}\]
    There is a current in the loop so we expect a force in the magnetic field given by
    \[\vv F = I\vv L\times\vv B\]
    For any direction of current flow we have \(\vv F_2 = -\vv F_3\) so these forces cancel and the net force is \(\vv F_1\).
    \[F_1 = ILB = \frac{B^2L^2v}{R}\]
    and by Lenz's law this must act to the left to oppose the decreasing of the flux by trying to increase the area in the field.
    If \(v\) is constant then the power needed to remove the loop is
    \[P = Fv = \frac{B^2L^2v^2}{R}\]
    The electric power dissipated is
    \[P = I^2R = \frac{B^2L^2v^2}{R}\]
    The fact that both two powers are the same shows that Lenz's law is just conservation of energy rather than a specific physical law.
    
    \section{Electric and Magnetic Field Interaction}
    If a time varying magnetic field induces a current it must induce an electric field.
    In fact it must induce an electric field regardless of whether or not there is a loop to induce current in.
    
    If a charge \(q_0\) is moved by an induced current then the work done is
    \[W = \emf q_0\]
    If the force on the charge due to the electric field is \(\vv F_E\) then the fork done is
    \[W = \oint\vv F_E\cdot\dd\vv s = q_0\oint\vv E\cdot\dd\vv s = \emf q_0\]
    From this we get
    \[\emf = \oint\vv E\cdot\vv s\]
    hence
    \[\oint\vv E\cdot\dd\vv s = -\dv{\Phi_B}{t}\]
    This is actually a more general form of Faraday's law that is true in general, not just with a conducting loop.
    
    A time varying magnetic field induces an electric field.
    The reverse is also true, a time varying electric field induces a magnetic field.
    
    The same logic applies as to the induction of an electric field and it can be shown that
    \[\oint\vv B\cdot\dd\vv s = \mu_0\varepsilon_0\dv{\Phi_E}{t}\]
    this is Maxwell's law.
    
    Maxwell's law and Faraday's law form a symmetric link between electricity and magnetism.
    
    Recall Ampere's law
    \[\oint \vv B\cdot\dd\vv s = \mu_o I_e\]
    where \(I_e\) is the enclosed current.
    If there is also a time varying electric field present then this gives rise to a magnetic field following
    \[\oint \vv B\cdot\dd\vv s = \mu_0\varepsilon_0\dv{\Phi_E}{t}\]
    By the superposition of magnetic fields the total magnetic field must be such that
    \[\oint\vv B\cdot\dd\vv s = \mu_0 I e + \mu_0 I_d\]
    where
    \[I_d = \varepsilon_0\dv{\Phi_E}{t}\]
    \(I_d\) is known as the displacement current.
    It isn't a real current but has units of current.
    This equation for \(\vv B\) is the Ampere--Maxwell law.
    
    Consider a parallel plate capacitor with plate area \(A\) holding charges \(\pm Q\).
    If the electric field between the two plates is \(\vv E\) then the charge on the capacitor is
    \[Q = \varepsilon_0 AE\]
    Charging the capacitor with an external current
    \[I = \dv{Q}{t} = \varepsilon_0 A\dv{E}{t}\]
    \(\vv E\) is spatially constant so for any plane \(P\) of area \(A\) the flux through it is
    \[\Phi_E = AE\]
    Hence by the definition of displacement current
    \[I_d = \varepsilon_0\dv{\Phi_E}{t} = \varepsilon_0 A\dv{E}{t}\]
    so the induced displacement current is equal to the charging current.
    The displacement current induces a magnetic field.
    The displacement current density's magnitude is the current per unit area:
    \[J_d = \frac{I_d}{A} = \varepsilon_0\dv{E}{t}\]
    Hence the displacement current density is
    \[\vv J_d = \varepsilon_0\dv{\vv E}{t}\]
    so a time varying electric field acts like a current density and induces a magnetic field like any current density does.
    
    \section{Inductors}
    A solenoid has cross sectional area \(A\).
    We know that the magnetic field strength inside the solenoid is
    \[B = \frac{\mu_0NI}{l}\]
    where \(l\) is the length of the solenoid and \(N\) is the number of turns.
    The flux through the cross section of the solenoid is
    \[\Phi_B = BA = \frac{\mu_0NIA}{l}\]
    The inductance, \(L\), of the solenoid is defined as
    \[L = \frac{N\Phi_B}{I}\]
    \[L = \frac{\mu_0N^2A}{l}\]
    this depends only on the physical parameters of the solenoid, not the current it carries.
    This is also sometimes written as \(L = \mu_0n^2Al\) where \(n = N/l\) is the number of turns per unit length.
    The units of inductance are \(\si{T.m^2.A^{-1}} = \si{H}\) which is called the henry.
    
    Inductance is the magnetic version of capacitance and, like capacitance, can be increased by putting a core in the solenoid, analogous to a dielectric in a capacitor.
    In this case a ferromagnetic material is used.
    
    Using Faraday's law:
    \[\emf = -N\dv{\Phi_B}{t}\]
    and the definition of inductance \(N\Phi_B = LI\) we get the self induced emf, \(\emf_L\):
    \[\emf_L = -L\dv{I}{t}\]
    this shows that an emf is induced that opposes the current flow through a solenoid.
    
    \subsection{LR Circuit}
    Consider the simple series LR circuit shown in figure \ref{fig:LR circuit}
    When the switch closes the supply voltage, \(V_b\), the resistance, \(R\), and the self induced emf, \(\emf_L\), determine the current.
    The sum of emf sources must be equal to the voltage drop over the load, assuming an ideal inductor with no impedance this means
    \[\emf_b + \emf_L = IR\]
    \(\emf_b\) is the battery's emf:
    \[V_b - L\dv{I}{t} = IR\]
    \[\dv{I}{t} + \frac{R}{L}I = \frac{V_b}{L}\]
    This is a first order ODE in terms of current.
    It can be solved with an integrating factor \(e^{Rt/L}\).
    To find the initial conditions we consider the current flow at time \(t = 0\) when the switch first closes.
    The self induced emf at this point is equal to the voltage of the battery so there is no initial current flow.
    After a long time the circuit will reach a steady state and the self induced emf will be zero as \(I\) will be constant.
    When this happens the current will be \(V_b/R\).
    The solution to the differential equation is
    \[I(t) = \frac{V_b}{R}\left[1 - \exp\left(-\frac{R}{L}t\right)\right]\]
    The units of \(R/L\) are seconds.
    \(\tau = R/L\) is known as the time constant for this circuit.
    \begin{figure}[ht]
        \centering
        \begin{tikzpicture}
            \draw (0, 2) to[R=\(R\)] (2, 2);
            \draw (2, 2) to[nos] (4, 2);
            \draw (4, 2) to[cute inductor=\(L\)] (4, 0);
            \draw (4, 0) to[short, i=\(I\)] (0, 0);
            \draw (0, 2) to[battery1, l=\(V_b\)] (0, 0);
        \end{tikzpicture}
        \caption{Series LR circuit}
        \label{fig:LR circuit}
    \end{figure}
    
    \subsection{Multiple Inductors}
    If the inductor in figure \ref{fig:LR circuit} is replaced by two inductors, \(L_1\) and \(L_2\), in series then the initial equation is changed to
    \[V_b = IR + L_1\dv{I}{t} + L_2\dv{I}{t} = IR + (L_1 + L_2)\dv{I}{t}\]
    this is the same as before with the single inductor giving an equivalent inductance \(L = L_1 + L_2\).
    This shows that for inductor in series the inductances add.
    That is for \(N\) inductors \(L_i\) in series the equivalent inductance, provided that the inductors are sufficiently separated, is
    \[L = \sum_{i = 1}^N L_i\]
    
    If instead the inductor in figure \ref{fig:LR circuit} is replaced by two inductors, \(L_1\) and \(L_2\), in parallel then the current is shared between the inductors.
    If the current through \(L_1\) is \(I_1\) and the current through \(L_2\) is \(I_2\) then we know \(I = I_1 + I_2\).
    denoting the voltage across the inductors as \(V_L\) we get
    \[\dv{I_1}{t} = \frac{V_L}{L_1},\qquad \dv{I_2}{t} = \frac{V_L}{L_2}\]
    So the total rate of change of current is
    \[\dv{I}{t} = \dv{t}[I_1 + I_2] = \frac{V_L}{L_1} + \frac{V_L}{L_2} = V_L\left(\frac{1}{L_1} + \frac{1}{L_2}\right)\]
    This shows that inductors in parallel add as reciprocals.
    Provided that they are sufficiently separated \(N\) inductors, \(L_i\), in parallel add together to give equivalent inductance, \(L\), that satisfies
    \[\frac{1}{L} = \sum_{i = 1}^N \frac{1}{L_i}\]
    
    \subsection{Energy Stored}
    To establish a current in an inductor we have to do work against the self induced emf, this work establishes a magnetic field.
    The power used against the emf is
    \[P = \emf I = LI\dv{I}{t}\]
    The potential energy added in time \(\dd t\) is
    \[\dd U_B = LI\dv{I}{t}\dd t = LI\dd I\]
    Hence the potential energy stored to create a current \(I\) is
    \[U_B = \int_0^{I}\dd U_B(I) = L\int_0^I I\,\dd I = \frac{1}{2}LI^2\]
    This has the same form as the potential energy stored in a capacitor.
    For a solenoid
    \[U_B = \frac{1}{2}\frac{\mu_0N^2A}{l}I^2\]
    also
    \[B = \frac{\mu_0NI}{l}\implies I = \frac{lB}{\mu_0N}\]
    so the potential energy stored is
    \[U_B = \frac{B^2}{2\mu_0}Al\]
    \(Al\) is the volume of the solenoid so the energy density is
    \[u_B = \frac{B^2}{2\mu_0}\]
    This holds in general for a magnetic field not just for a solenoid.
    
    \subsection{Mutual Induction}
    If two inductors \(L_1\) and \(L_2\) are in a circuit and aren't sufficiently separated then they will induce a current in each other.
    We define the mutual induction, \(M_{12}\), of inductor 2 with respect to inductor 1 as
    \[M_{21} = \frac{N_2\Phi_{21}}{I_1}\]
    where \(I_1\) is the current in inductor 1, \(\Phi_{21}\) is the magnetic flux through inductor 2 due to inductor 1 and \(N_2\) is the number of coils in inductor 2.
    If we vary \(I_1\) as a function of time then \(\vv B_1\), the magnetic field due to inductor 1, varies and so does \(\Phi_{21}\).
    If we differentiate \(M_{21}\) we get
    \[M_{21}\dv{I_1}{t} = N_2\dv{\Phi_{21}}{t}\]
    hence the emf in coil 2 due to current in coil 1 is
    \[\emf_{21} = -M_{21}\dv{I_1}{t}\]
    The same can be done the other way round to find \(\emf_{12}\).
    We can show that \(M_{21} = M_{12} = M\).
    If both inductors are in series in place of the inductor in figure \ref{fig:LR circuit} then we get
    \begin{align*}
        V_b &= IR + \emf_1 + \emf_2 + \emf_{12} + \emf_{21}\\
        &= IR + L_1\dv{I}{t} + L_2\dv{I}{t} + M\dv{I}{t}  + M\dv{I}{t}\\
        &= IR + (L_1 + L_2 + 2M)\dv{I}{t}
    \end{align*}
    so the inductance of two inductors, in series, with mutual inductance \(M\) is
    \[L = L_1 + L_2 + 2M\]
    
    \section{Quantum Numbers}
    An electron orbiting a nucleus can be modelled as a current loop.
    As such it has a magnetic moment.
    The value of the magnetic moment depends on the quantum numbers of the electron.
    
    The quantum numbers for an electron are
    \begin{enumerate}
        \item Principle quantum number, \(n = 1, 2, 3\dotsc\)
        \item Orbital angular momentum quantum number, \(l = 0, 1, 2,\dotsc, n - 1\)
        \item Magnetic quantum number, \(m_l = 0, \pm 1, \pm 2, \dotsc, \pm l\)
        \item Spin quantum number, \(m_s = \pm 1/2\)
    \end{enumerate}
    The orbital angular momentum of the electron is \(\vv L\) and \(L = \hbar\sqrt{l(l + 1)}\) and each component is given by \(L_z = m_l\hbar\).
    The magnetic moment due to this orbital angular momentum is
    \[\mu = -m_l\mu_B,\qquad\text{where }\mu_B = \frac{eh}{4\pi m} = \frac{e\hbar}{2m}\]
    \(\mu_B\) is the Bohr magneton.
    The axial magnetic field at a large distance, \(z\), from the atom is
    \[B = \frac{\mu_0}{2\pi}\frac{\mu}{z^3}\]
    The electron also has angular momentum associated with its spin, \(\vv S\).
    Each component of the spin is \(S_z = \pm \hbar/2 = m_s\hbar\) and this gives rise to a magnetic moment
    \[\mu = \pm\mu_B\]
    The total angular momentum of the electron is
    \[\vv J = \vv L + \vv S\]
    it is important that this is a vector equation as the magnetic moments can cancel each other out.
    
    \begin{table}
        \centering
        \begin{tabular}{c|cccc|c}\hline
            Atom & \(n\) & \(l\) & \(m_l\) & \(m_s\) & In Spectroscopic Notation\\ \hline
            &&&&&\\[-0.25cm]
            \ce{_2^4He} & 1 & 0 & 0 & \(+\frac{1}{2}\) & \(1s^2\)\\[0.1cm]
            & 1 & 0 & 0 & \(-\frac{1}{2}\) & \\[0.1cm] \hline
            &&&&&\\[-0.25cm]
            \ce{_3^6Li} & 1 & 0 & 0 & \(+\frac{1}{2}\) & \(1s^22S^1\)\\[0.1cm]
            & 1 & 0 & 0 & \(-\frac{1}{2}\) & \\[0.1cm]
            & 2 & 0 & 0 & \(+\frac{1}{2}\) & \\[0.1cm] \hline
            &&&&&\\[-0.25cm]
            \ce{_4^9Be} & 1 & 0 & 0 & \(+\frac{1}{2}\) & \(1s^22S^2\)\\[0.1cm]
            & 1 & 0 & 0 & \(-\frac{1}{2}\) & \\[0.1cm]
            & 2 & 0 & 0 & \(+\frac{1}{2}\) & \\[0.1cm]
            & 2 & 0 & 0 & \(-\frac{1}{2}\) & \\[0.1cm] \hline
        \end{tabular}
        \caption{The quantum numbers for electrons in the ground states of \ce{^4_2He}, \ce{^6_3Li} and \ce{^9_4Be}}
        \label{tab:quantum numbers}
    \end{table}
    
    \section{Types of Magnetism}
    \subsection{Diamagnetism}
    Diamagnetism is the weakest form of magnetism.
    All materials are diamagnetic but the effect is so weak that it is swamped by any other magnetic effects.
    If a diamagnetic material is placed in an external field then the the orbits which were previously degenerate become distorted and no longer degenerate so magnetic moments which cancelled with each other no longer cancel.
    A magnetic field is induced.
    By Lenz's law the induced field opposes the external field so a repulsive force is felt by the diamagnetic object.
    Both \ce{^4_2He} and \ce{^9_4Be} shown in table \ref{tab:quantum numbers} are diamagnetic as the spin up and spin down electrons are degenerate and cancel if there is no external field.
    
    \subsection{Paramagnetism}
    Paramagnetism occurs when atoms or molecules have a permanent magnetic moment as with \ce{^6_3Li} shown in table \ref{tab:quantum numbers}, there is only one electron in the second shell and as such the magnetic moment doesn't cancel.
    When placed in an external magnetic field the magnetic moments of the atoms align and an internal magnetic field is created.
    The magnetic moments align so that the potential energy is minimised which means that the resulting force will be attractive.
    The magnetism, \(M\), of a sample is a measure of how well aligned the dipoles are.
    \[M = \frac{\mu}{V}\]
    where \(\mu\) is the measured dipole moment and \(V\) is the volume of the sample.
    The units of \(M\) are \(\si{A.m^{-1}}\).
    The maximum value of \(M\) occurs when all of the dipoles align.
    It is rare for all dipoles to align but usually enough align to have a noticeable effect.
    The potential energy per unit volume is \(u = -\vv\mu\cdot\vv B_\text{ext}\).
    The potential difference between antiparallel and parallel magnetic moments is \(\Delta u = 2\mu B_\text{ext}\).
    The kinetic energy is \(K = 3k_BT/2\).
    At normal temperatures \(K\gg \Delta u\) so collisions between atoms are enough to knock atoms out of alignment.
    The number of aligned atoms is proportional to \(\delta u/K\) which is in turn proportional to \(B_\text{ext}/T\).
    Curie's law is
    \[M = C\frac{B_\text{ext}}{T}\]
    where \(C\) is Curie's constant and depends on the material.
    Curie's law holds at high temperatures and for small \(B_\text{ext}\).
    At low temperatures or high values of \(B_\text{ext}\) \(M\) approaches a saturation point.
    Critically for a paramagnetic material removing the external field returns the magnetism to zero.
    
    \subsection{Ferromagnetism}
    To explain permanent magnets we need the concept of ferromagnetism.
    In ferromagnetism as well as the above effects the electron spin of one atom interacts with the spin of adjacent atoms resulting in a net magnetic field.
    This is known as exchange coupling and is a quantum mechanical effect.
    At higher temperatures atoms have enough kinetic energy to overcome this coupling.
    The temperature when this happens is known as the Curie temperature and is lower than the melting point.
    This coupling only extends so far and separate regions will couple together.
    In an external magnetic field these regions align their fields and then remain aligned after the external magnetic field is removed.
    This is the critical difference that allows for permanent magnets.
    
    \example
    An iron compass needle of mass \(m = \SI{0.2}{g}\) has a magnetic dipole moment of \(\mu = \SI{1.5e-2}{J.T^{-1}}\).
    What percentage of the iron atom in the needle are aligned?
    The magnetic moment of each iron atom is \(\mu_{\ce{Fe}} = \SI{2.1e-23}{J.T^{-1}}\) and the molar mass of iron is \(M_{\ce{Fe}} = \SI{55.8}{g.mol^{-1}}\).
    The total magnetic moment is \(\mu = N\mu_{\ce{Fe}}\) where \(N\) is the number of aligned magnetic dipole moments.
    Hence
    \[N = \frac{\mu}{\mu_{\ce{Fe}}}\]
    The fraction that are aligned is \(N/N_{\ce{Fe}}\) where \(N_{\ce{Fe}}\) is the total number of iron atoms.
    \(N_{\ce{Fe}} = N_An\) where \(n = m/M_{\ce{Fe}}\) is the number density of iron atoms.
    Hence the fraction of atoms that are aligned is
    \[\frac{N}{N_{\ce{Fe}}} = \frac{\mu/\mu_{\ce{Fe}}}{N_A m/M_{\ce{Fe}}} = \frac{\mu M_{\ce{Fe}}}{N_Am\mu_{\ce{Fe}}} = \frac{\SI{1.5e-2}{J.T^{-1}}\cdot \SI{55.8}{g.mol^{-1}}}{\SI{6.02e23}{mol^{-1}}\cdot\SI{2.1e-23}{J.T^{-1}}\cdot\SI{0.2}{g}} = 0.33 = \SI{33}{\%}\]
    If you hold the needle so it is pointing exactly the wrong way what is its potential energy?
    \(B_\text{Earth} = \SI{30}{\micro T} = \SI{30e-6}{T}\).
    \[u = -\vv\mu\cdot\vv B_\text{Earth} = -\mu B_\text{Earth}\cos\pi = \mu B_\text{Earth} = \SI{1.5e-2}{J.T^{-1}}\cdot\SI{30e-6}{T} = \SI{4.5e-7}{J}\]
    
    \section{LRC Circuits}
    \subsection{Ideal LC Circuit}
    Recall that the energy, \(U_E\), stored in a the electric field by a capacitor, \(C\), holding charge \(Q\) is
    \[U_E = \frac{1}{2}\frac{Q^2}{C}\]
    and that the energy, \(U_B\), stored in the magnetic field by an inductor, \(L\), carrying current \(I\) is
    \[U_B = \frac{1}{2}LI^2\]
    The total energy, \(U_T\), stored in a circuit of a capacitor and inductor (an LC circuit) is then
    \[U_T = U_E + U_B = \frac{1}{2}\frac{Q^2}{C} + \frac{1}{2}LI^2\]
    If the system is isolated and lossless then the total potential energy is conserved but the ratio of potential energy split between the electric and magnetic fields can change.
    We can start by considering specific cases for an LC circuit with a capacitor holding charge \(q\) and with an inductor carrying charge \(i\),
    \begin{itemize}
        \item When all charge starts on the capacitor so \(q = Q\) and \(i = 0\) the total energy is
        \[U_T = U_E = \frac{1}{2}\frac{Q^2}{C}\]
        \item When the capacitor starts with no charge then the peak current is when \(U_T = U_B\).
        This gives peak current
        \[I = \frac{Q}{\sqrt{LC}}\]
    \end{itemize}
    There are four boundary conditions that are useful to consider for an LC circuit:
    \begin{itemize}
        \item All charge, \(+Q\), on the capacitor and \(i = 0\)
        \item No charge on the capacitor and \(i = I\)
        \item All charge, \(-Q\), on the capacitor and \(i = 0\)
        \item No charge on the capacitor and \(i = -I\)
    \end{itemize}
    Given these conditions we expect the system to oscillate between them.
    Assuming a lossless system (ie no resistance) we have
    \begin{align*}
        0 &= \dv{U_T}{t}\\
        &= \frac{1}{2C}\dv{t}q^2 + \frac{1}{2}L\dv{t}i^2\\
        &= \frac{1}{C}q\dv{q}{t} + Li\dv{i}{t}\\
        \intertext{Substituting in \(i = \dot q\) we get}
        &= \frac{1}{C}q\dv{q}{t} + L\dv{q}{t}\dv[2]{q}{t}\\
        &= \frac{q}{C} + L\dv[2]{q}{t}
    \end{align*}
    This gives us
    \[\dv[2]{q}{t} = -\frac{1}{LC}q\]
    This is simple harmonic motion with a solution
    \[q(t) = a\cos(\omega_0 t + \varphi),\qquad\text{where }\omega_0 = \frac{1}{\sqrt{LC}}\]
    where \(a\) and \(\varphi\) are constants determined by the initial conditions.
    
    \example
    If at \(t = 0\) the capacitor has charge \(q(0) = Q\) then \(q(t) = Q\cos(\omega_0 t)\) and \(i(t) = \dot q(t) = -q\omega_0\sin(\omega_0 t)\).
    The stored potential energies are then
    \[U_E = \frac{Q^2}{2C}\cos^2(\omega_0 t) = \frac{Q^2}{4C}(\cos(2\omega_0 t) + 1)\]
    \[U_B = \frac{Q^2}{2C}\sin^2(\omega_0 t) = \frac{Q^2}{4C}(1 - \cos(2\omega_0 t))\]
    so the potential energy oscillates between the electric and magnetic fields at a frequency of \(2\omega_0\).
    
    \subsection{Damping LRC circuits}
    In reality all circuits have resistance and hence losses.
    We model the total resistance with an external series resistor, \(R\).
    The resistance acts to damp the oscillations.
    The power dissipated by the resistor is \(P = i^2R\) so
    \[\dv{U_T}{t} = -i^2R\]
    where the negative sign signifies that energy is lost due to the resistor.
    Hence we now have
    \begin{align*}
        -i^2 R &= \dv{U_T}{t}\\
        -i^2 R &= Li\dv{i}{t} + \frac{q}{C}\dv{q}{t}\\
        -i^2 R &= Li\dv[2]{q}{t} + \frac{q}{C}i\\
        -iR &= L\dv[2]{q}{t} + \frac{q}{C}\\
    \end{align*}
    This gives
    \[0 = \dv[2]{q}{t} + \frac{R}{L}\dv{q}{t} + \frac{1}{LC}q\]
    Applying the boundary conditions that at \(t = 0\) the charge on the capacitor is \(q(0) = Q\) and \(i(0) = 0\) we get
    \[q(t) = Q\exp\left(-\frac{R}{2L}t\right)\cos(\omega t),\qquad\text{where }\omega = \sqrt{\omega_0^2 - \left(\frac{R}{2L}\right)^2}\]
    assuming under damping meaning that \(\omega\in\bb R\) so \(\omega_0 > R/2L\).
    The current is then
    \[i = \dv{q}{t} = -Q\exp\left(-\frac{R}{2L}t\right)\left[\frac{R}{2L}\cos(\omega t) + \omega\sin(\omega t)\right]\]
    We consider now the case of light damping where \(R/2L\ll\omega_0\implies\omega\approx\omega_0\), the charge, \(q(t)\), is unchanged but the current is now
    \[i(t)\approx -Q\omega\exp\left(-\frac{R}{2L}t\right)\sin(\omega t)\]
    This is the same as the undamped (LC) case but with a decaying factor \(\exp(-Rt/2L)\) and a slightly lower frequency.
    The potential energies in this case become
    \[U_E(t) = \frac{Q^2}{2C}\exp\left(-\frac{R}{2L}t\right)\cos^2(\omega t) = \frac{Q^2}{4C}\exp\left(-\frac{R}{2L}t\right)(\cos(2\omega t) + 1)\]
    \[U_B(t) = \frac{Q^2}{2C}\exp\left(-\frac{R}{2L}t\right)\sin^2(\omega t) = \frac{Q^2}{4C}\exp\left(-\frac{R}{2L}t\right)(1 - \cos(2\omega t))\]
    \[U_T(t = \frac{Q^2}{2C}\exp\left(-\frac{R}{2L}t\right)\]
    so the total potential energy decays away as \(\exp(-Rt/2L)\).
    There are two time constants associated with this system:
    \begin{itemize}
        \item \(t_e = L/R\) which is the time taken for \(U_T\) to decay to \(e^{-1}\) times its original value:
        \[U_T(t_e) = \frac{1}{e}U_T(0)\]
        \item \(T - 2\pi/\omega\) which is the time for one oscillation
    \end{itemize}
    The system goes through \(N = t_e/T\) oscillations before the potential energy decays to \(e^{-1}\) of its original value.
    The quality factor of the system is defined as
    \[Q_f = 2\pi N = 2\pi\frac{t_e}{T} = \omega\frac{L}{R} = \frac{1}{R}\sqrt{\frac{L}{C}}\]
    where in the last step we have assumed light damping so \(\omega \approx \omega_0 = (LC)^{-1/2}\).
    
    \section{Alternating Current}
    So far we have considered only constant supply voltages (ie DC current).
    This isn't the case for AC.
    For a voltage source alternating at a driving frequency of \(\omega_d\) the emf is given by
    \[\emf = \emf_m\cos(\omega_d t)\]
    where \(\emf_m\) is the peak emf.
    The resulting current is
    \[i = I\cos(\omega_d t - \varphi)\]
    where \(I\) is the peak current and \(-\varphi\) is the phase difference from the emf.
    By convention \(\varphi\) is negative and we have defined \(\emf\) using \(\cos\).
    It would be equally valid to take \(\varphi\) as positive and/or define \(\emf\) using \(\sin\).
    
    \subsection{Resistive Load}
    Consider an AC voltage source providing emf, \(\emf\), connected to a resistor, \(R\), in series.
    The current in the resistor is \(i_R\) and the voltage across the resistor is \(v_R\).
    At all times the sum of voltages is zero so \(\emf - v_R = 0\), hence \(v_R = \emf_m\cos(\omega_d t)\).
    We define the peak voltage to be \(V_R\) so that
    \[v_R = V_R\cos(\omega_d t)\]
    Since for this system the peak voltage across the resistor is equal to the peak emf.
    From the definition of resistance we can find the current:
    \begin{equation}\label{eqn:iR}
        i_R = \frac{v_R}{R} = \frac{V_R}{R}\cos(\omega_d t) = I_R\cos(\omega_d t)
    \end{equation}
    where \(I_R\) is the peak current.
    From this we can see that the resistive load is \(V_R = I_RR\) and that the phase difference is zero.
    
    \subsection{Capacitive Load}
    Consider the same circuit but with the resistor replaced with a capacitor, \(C\).
    The current through the capacitor is \(i_C\) and the voltage across it is \(v_C\).
    As before
    \[v_c = V_C\cos(\omega_d t)\]
    where \(V_C = \emf_m\) is the peak voltage across the capacitor.
    The charge stored in the capacitor is
    \[q_C = v_CC = V_CC\cos(\omega_d t)\]
    The current is
    \[i_C = \dv{q_C}{t} = -V_CC\omega_d\sin(\omega_d t) = V_CC\omega_d\cos\left(\omega_d t - \frac{\pi}{2}\right)\]
    Compare this with equation \ref{eqn:iR} and we can equate the prefactors to define
    \[X_C = \frac{1}{\omega_dC}\]
    which is called the capacitive reactance.
    \[i_C = \frac{V_C}{X_C}\cos\left(\omega_d t + \frac{\pi}{2}\right)\]
    From this we get \(V_C = I_CX_C\) is the capacitive load and the current is \(-\pi/2\) out of phase with the voltage.
    
    \subsection{Inductive Load}
    Consider the same circuit but with the capacitor replaced with an inductor, \(L\).
    The current through the inductor is \(i_L\) and the voltage across it is \(v_L\).
    As before
    \[v_L = V_L\cos(\omega_d t)\]
    where \(V_L = \emf_m\) is the peak voltage across the inductor.
    The self induced emf is
    \[\emf_L = -L\dv{I}{t}\implies v_L = L\dv{i_L}{t}\implies \dv{i_L}{t} = \frac{V_L}{L}\cos(\omega_d t)\]
    \[i_L = \frac{V_L}{L}\int_{0}^{t}\cos(\omega_d t)\,\dd t = \frac{V_L}{\omega_d L}\sin(\omega_d t) = \frac{V_L}{\omega_d L}\cos\left(\omega_d t - \frac{\pi}{t}\right)\]
    Compare this with equation \ref{eqn:iR} and we can equate the prefactors to define
    \[X_L = \omega_d L\]
    which is called the inductive reactance.
    \[i_L = \frac{V_L}{X_L}\cos\left(\omega_d t - \frac{\pi}{2}\right)\]
    From this we get \(V_L = I_LX_L\) is the inductive load and the current is \(+\pi/2\) out of phase with the voltage.
    
    \subsection{RL Circuit}
    Consider a resistor, \(R\), and an inductor, \(L\), in series with an alternating voltage source, \(\emf\), with driving frequency \(\omega_d\).
    The voltages across the resistor and inductor are \(v_R\) and \(v_L\) respectively.
    Since all components are in series the current in each component is the same and equal to \(i\).
    Since all components have the same current but the current and voltage experience different phase differences around the circuit we apply the phase difference to the voltage instead of the current noting that a negative phase difference in the current is the same as a positive phase difference in the voltage.
    The emf is
    \[\emf = \emf_m\exp(j\omega_d t)\]
    where \(\emf_m\) is the peak emf and \(j^2 = -1\).
    The current is then
    \[i(t) = I\exp(j(\omega_d t - \varphi))\]
    where \(I\) is the peak current and \(\varphi\) is the phase difference between the emf and current.
    
    Strictly speaking \(\emf\) is the complex emf and the measured emf is \(\Re(\emf) = \emf_m\cos(\omega_d t)\).
    Similarly \(i\) is the complex current and the measured current is \(\Re(i) = I\cos(\omega_d t - \varphi)\).
    In general at the end of a problem take the real parts of all values to find the measured value.
    
    At all times the voltages add to \(\emf\) so
    \[\emf = v_R + v_L\]
    We know that there is no phase shift for a resistor and that there is a phase shift of \(\pi/2\) for an inductor so
    \[v_R = V_R\exp(j\omega_d t)\]
    \[V_L\exp\left(j\left(\omega_d t - \frac{\pi}{2}\right)\right) = V_L\exp\left(-j\frac{\pi}{2}\right)\exp(j\omega_d t) = -jV_L\exp(j\omega_d t)\]
    Therefore we get the emf is
    \[\emf = \emf_m\exp(j\omega_d t) = V_R\exp(j\omega_d t) - jV_L\exp(j\omega_d t)\]
    Taking the square modulus we get
    \[\emf_m^2 = V_R^2 + V_L^2 = (IR)^2 + (IX_L)^2\]
    \[I = \frac{\emf_m}{\sqrt{R^2 + X_L^2}}\]
    Compare this with Ohm's law \(I = \emf_m/Z\) where \(Z = \sqrt{R^2 + X_L^2}\) is the impedance.
    Finally substituting for \(X_L\) we get
    \[I = \frac{\emf_m}{\sqrt{R^2 + (\omega_d L)^2}}\]
    At \(\omega_d = 0\) (DC) we recover Ohm's law, as \(\omega_d\to\infty\) \(I\to 0\).
    So an inductor acts as a low pass filter (it doesn't let high frequencies through).
    To get the phase consider the instantaneous current \(i = \emf/Z\).
    We get
    \[I\exp(-j\omega_d t)\exp(-j\varphi) = \frac{1}{Z}[V_R\exp(j\omega_d t) - jV_L\exp(j\omega_d)]\]
    \[\exp(-j\varphi) = \frac{1}{IZ}[V_R - jV_L]\]
    Note that since \(I, Z\in\bb R\) that the phase is just the ratio of the imaginary part to the real part we get
    \[\tan(\varphi) = \frac{V_L}{V_R} = \frac{IX_L}{IR} = \frac{X_L}{R} = \frac{\omega_d L}{R}\]
    the negative in front of the imaginary part disappears since \(\varphi\) is negative.
    
    \subsection{RC Circuits}
    Consider the same circuit as before but with the inductor replaced with a capacitor, \(C\).
    As before we have
    \[\emf = v_R + v_C\]
    there is a phase difference of \(-\pi/2\) for the 
    \[v_C = V_C\exp\left(j\left(\omega_d + \frac{\pi}{2}\right)\right) = V_C\exp\left(j\frac{\pi}{2}\right)\exp(j\omega_d t) = jV_C\exp(j\omega_d t)\]
    The magnitude squared of the emf is then
    \[\emf_m^2 = V_R^2 + V_C^2 = (IF)^2 + (IX_C)^2\]
    \[I = \frac{\emf_m}{\sqrt{R^2 + X_C^2}} = \frac{\emf_m}{Z}\]
    \[Z = \sqrt{R^2 + X_C^2} = \sqrt{R^2 + \frac{1}{(\omega_d C)^2}}\]
    When \(\omega_d = 0\) (DC) we get \(I = 0\) so a capacitor acts as a high pass filter (it doesn't let low frequencies through).
    As \(\omega_d\to\infty\) we recover Ohm's law.
    To get the phase consider the instantaneous current \(i = \emf/Z\)
    \[I = \exp(-j\omega_dt)\exp(-j\varphi) = \frac{1}{Z}[V_R\exp(j\omega_d t) + jV_C\exp(j\omega_d t)]\]
    \[\exp(-j\varphi) = \frac{1}{IZ}[V_R + jV_C]\]
    Since \(I, Z\in\bb R\) the phase is
    \[\tan(\varphi) = \frac{-V_C}{V_R} = -\frac{IX_C}{IR} = -\frac{X_C}{R} = -\frac{1}{R\omega_dC}\]
    Note the appearance of a negative sign as we define the phase as negative.
    
    \subsection{LRC Circuits}
    Consider a circuit made of an emf source, \(\emf\), in series with a resistor, \(R\), a capacitor, \(C\), and an inductor, \(L\).
    Since they are in parallel the current in all must be the same.
    Importantly it must have the same phase but the current and voltages in each component have a different phase relationship.
    We fix this by applying the phase difference to the voltage instead of the current.
    Note that a positive phase difference applied to the current is the same as a negative phase difference applied to the voltage and vice versa.
    The total emf around the circuit is
    \[\emf = v_R + v_C + v_L\]
    \[v_R = V_R\exp(j\omega_d t)\]
    \[v_C = V_C\exp\left(j\left(\omega_d t + \frac{\pi}{2}\right)\right) = jV_C\exp(j\omega_d t)\]
    \[v_L = V_L\exp\left(j\left(\omega_d t - \frac{\pi}{2}\right)\right) = -jV_L\exp(j\omega_d t)\]
    The magnitude squared of the emf is then
    \[\emf_m^2 = V_R^2 + (V_C - V_L)^2 = (IR)^2 + (IX_C - IX_L)^2\]
    \[I = \frac{\emf_m}{\sqrt{R^2 + (X_C - X_L)^2}} = \frac{I}{Z}\]
    \[Z = \sqrt{R^2 + (X_C - X_L)^2} = \sqrt{R^2 + \left(\frac{1}{\omega_d C} - \omega_d L\right)^2}\]
    To get the phase difference consider the instantaneous current \(i = \emf/Z\):
    \[I\exp(j\omega_d t)\exp(-j\varphi) = \frac{1}{Z}[V_R\exp(j\omega_d t) + j(V_C - V_L)\exp(j\omega_d t)]\]
    \[\exp(-j\varphi) = \frac{1}{IZ}[V_R + j(V_C - V_L)]\]
    \[\tan(\varphi) = \frac{-(V_C - V_L)}{V_R} = \frac{V_L - V_C}{V_R} = \frac{IX_L - IX_C}{IR} = \frac{X_L - X_C}{R}\]
    The maximum current occurs when the impedance is lowest.
    This occurs when \(X_L = X_C\) which give the resonant impedance \(Z_R = R\).
    This occurs when
    \[\omega_d L = \frac{1}{\omega_d C}\implies \omega_d = \frac{1}{\sqrt{LC}}\]
    which was the natural frequency for an undamped LC circuit.
    At resonance
    \[I_R = \frac{\emf_m}{R},\qquad \varphi = 0\]
    so the circuit acts as if it was purely resistive.
    We can write the impedance as
    \[Z = \sqrt{R^2 + \frac{L}{C}\left(\frac{\omega_d}{\omega_0} - \frac{\omega_0}{\omega_d}\right)^2}\]
    We also have the quality factor
    \[Q_f = \frac{1}{R}\sqrt{\frac{L}{C}}\]
    so the impedance is
    \[Z = R\sqrt{1 + Q_f^2\left(\frac{\omega_d}{\omega_0} - \frac{\omega_0}{\omega_d}\right)^2}\]
    And the current is
    \[I = \frac{\emf}{Z}\]
    which has a maximum at \(\omega_d = \omega_0\).
    Small values of \(Q_f\) gives a wide resonance whereas large values give a narrow resonance.
    The phase is
    \[\tan(\varphi) = Q_f\left(\frac{\omega_d}{\omega_0} - \frac{\omega_0}{\omega_d}\right)\]
    At resonance this is zero and along way from resonance it is \(\pm\pi/2\) depending on whether we are above or below the resonance frequency.
    The larger \(Q_f\) is the more \(\varphi\) changes about the point of resonance.
    
    \section{Maxwell's Laws}
    \subsection{Gauss's Law for Magnetic Fields}
    The magnetic flux, \(\Phi_B\), through a Gaussian surface is
    \[\Phi_B = \oint\dd\Phi_B = \oint\vv B\cdot\dd\vv A\]
    Since magnetic field lines are closed loops this value is necessarily zero as the flux in is equal to the flux out.
    This is Gauss's law for magnetic fields:
    \[\Phi_B = \oint\vv B\cdot\dd\vv A = 0\]
    Another way to phrase this is that the simplest magnetic structure that can exist is a magnetic dipole, or that we aren't able to isolate a magnetic monopole, or that there is no magnetic equivalent of electric charge.
    
    \subsection{History of Maxwell's Equations}
    James Clerk Maxwell formulated his original equations before vector calculus was widely known.
    He had 20 original equations which were reduced to 8 equations by Oliver Heaviside and Willard Gibbs using vector calculus.
    Of these 8 equations 4 are fundamental and apply for free space and the other 4 adjust for a medium and can be derived from the 4 free space equations.
    These 4 equations are Maxwell's equations.
    With these equations Maxwell linked the previously separate concepts of electricity, magnetism and light.
    There are multiple ways to write them, through this course we have been using the integral form but they are more commonly quoted in the equivalent differential form.
    
    \subsection{Maxwell's Equations in Free Space}
    \subsection{Integral Form}
    Let \(\vv E\) be the electric field, \(\vv B\) be the magnetic field, \(Q_\text{enc}\) be the charge enclosed by some Gaussian surface \(\partial V\) of volume \(V\) with surface normal \(\dd\vv S\), \(\Phi_E\) and \(\Phi_B\) be the electric and magnetic fluxes through a surface \(S\) and \(I_\text{enc}\) be the current through this surface, the boundary of \(S\) is \(\partial S\) with line element \(\dd\vv r\).
    Maxwell's equations in free space in integral form are:
    \begin{align*}
        \oint_{\partial V}\vv E\cdot\dd\vv S &= \frac{Q_\text{enc}}{\varepsilon_0}\\
        \oint_{\partial V}\vv B\cdot\dd\vv S &= 0\\
        \oint_{\partial S}\vv E\cdot\dd\vv r &= -\dv{\Phi_B}{t}\\
        \oint_{\partial S}\vv B\cdot\dd\vv r &= \mu_0I_\text{enc} + \mu_0\varepsilon_0\dv{\Phi_E}{t}
    \end{align*}
    These are, in order, Gauss's laws for electric and magnetic fields, Faraday's law of electromagnetic induction and the Ampere--Maxwell law of magnetic fields.
    The last two form coupled equations which have non--zero solutions even in a vacuum with no charge or current.
    
    \subsection{Differential Form}
    Maxwell's equations are relatively easy to turn into differential form using the divergence theorem:
    \[\int_V\div\vv a\,\dd V = \oint_{\partial V}\vv a\cdot\dd\vv S\]
    and Stokes' theorem:
    \[\int_S(\curl\vv a)\cdot\dd\vv S = \oint_{\partial S}\vv a\cdot\dd\vv S\]
    First we need a few more definitions.
    The charge density \(\rho(\vv r)\) is a scalar field such that
    \[\int_V\rho(\vv r)\,\dd V = Q_\text{enc}\]
    The current density \(\vv J(\vv r)\) is a vector field such that
    \[\int_S\vv J\cdot\dd\vv S = I_\text{enc}\]
    We also need the definitions of flux for each field
    \[\Phi_E = \int_S\vv E\cdot\dd\vv S,\qquad \Phi_B = \int_S\vv B\cdot\dd\vv S\]
    We will start with Gauss's law for electric fields:
    \[\oint_{\partial V}\vv E\cdot\dd\vv S = \frac{Q_\text{enc}}{\varepsilon_0}\]
    Using the divergence theorem on the left and the definition of \(\rho\) on the right we get
    \[\int_V\div\vv E\,\dd V = \frac{1}{\varepsilon_0}\int_V\rho\,\dd V\]
    For this to be true for arbitrary volume, \(V\), we require that the integrands are equal
    \[\div\vv E = \frac{\rho}{\varepsilon_0}\]
    Next we use Gauss's law for magnetic fields
    \[\oint_{\partial V}\vv B\cdot\dd\vv S = 0\]
    Applying the divergence theorem to the left hand side we get
    \[\int_V\div\vv B\,\dd V = 0\]
    For this to be true for arbitrary volume, \(V\), we require that the integrand is zero
    \[\div\vv B = 0\]
    Next we use Faraday's law of electromagnetic induction
    \[\oint_{\partial S}\vv E\cdot\dd\vv r = -\dv{\Phi_B}{t}\]
    Applying Stokes' theorem on the left and the definition of \(\Phi_B\) on the right we get
    \[\int_S(\curl\vv E)\cdot\dd\vv S = -\dv{t}\int_S\vv B\cdot\dd S\]
    For a time independent surface (one that doesn't change with time) we can bring the derivative inside the integral as a partial derivative
    \[\int_S(\curl\vv E)\cdot\dd\vv S = -\int_S\partial_t\vv B\cdot\dd S\]
    For this to be true for arbitrary surface, \(S\), we require the integrands to be equal
    \[\curl\vv E = -\partial_t \vv B\]
    Finally using the Ampere--Maxwell law
    \[\oint_{\partial S}\vv B\cdot\dd\vv r = \mu_0I_\text{enc} + \mu_0\varepsilon_0\dv{\Phi_E}{t}\]
    Applying Stokes' theorem on the left and the definitions of \(I_\text{enc}\) and \(\Phi_E\) on the right we get
    \[\int_S(\curl\vv B)\cdot\dd\vv S = \mu_0\int_S\vv J\cdot\dd\vv S + \mu_0\varepsilon_0\dv{t}\int_S\vv E\cdot\dd\vv S\]
    For a time independent surface we can bring the derivative inside the integral as a partial derivative
    \[\int_S(\curl\vv B)\cdot\dd\vv S = \mu_0\int_S\vv J\cdot\dd\vv S + \mu_0\varepsilon_0\int_S\partial_t\vv E\cdot\dd\vv S\]
    Combining the integrals on the right using the linearity of integration we get
    \[\int_S(\curl\vv B)\cdot\dd\vv S = \int_S\left[\mu_0\vv J + \mu_0\varepsilon_0\partial_t\vv E\right]\cdot\dd\vv S\]
    For this to be true for an arbitrary surface, \(S\), we require the integrands to be equal
    \[\curl\vv B = \mu_0\vv J = \mu_0\varepsilon_0\partial_t\vv E\]
    Thus we have derived Maxwell's laws in differential form.
    In the same order as the integral form they are
    \begin{align*}
        \div\vv E &= \frac{\rho}{\varepsilon_0}\\
        \div\vv B &= 0\\
        \curl\vv E &= -\partial_t\vv B\\
        \curl\vv B &= \mu_0\vv J + \mu_0\varepsilon_0\partial_t\vv E
    \end{align*}
    
    \subsection{Maxwell's Equations Inside Materials}
    As previously mentioned Maxwell's equations can be extended to work in a material, to do this we define two new fields.
    If the electric field is \(\vv E\) and the dipole moment per unit volume of the material is \(\vv P\) the the electric displacement field is
    \[\vv D = \varepsilon_0\vv E + \vv P\]
    This is true for any material although in this course we have only considered linear effects for \(\vv P\).
    For linear materials \(\vv P = \varepsilon_0\chi_e\vv E\) so \(\vv D = \varepsilon_0(1 + \chi_e)\vv E = \varepsilon_0\varepsilon_r\vv E = k\vv E = \varepsilon\vv E\) where \(\chi_e\) is the linear electric susceptibility of the material, \(k\) is the dielectric constant, \(\varepsilon_r\) is the relative permittivity and \(\varepsilon\)  is the absolute permittivity.
    
    Similarly if the magnetic field is \(\vv B\) and the magnetisation per unit volume due to a combination of dia-, para- and ferromagnetism is \(\vv M\) then we define the magnetising field as
    \[\vv H = \frac{1}{\mu_0}\vv B - \vv M\]
    These fields also have the nice effect of removing the constants \(\varepsilon_0\) and \(\mu_0\) by absorbing them into the definitions
    
    Using these two fields we can write Maxwell's equations inside a material, in the same order as before, they are
    \begin{align*}
        \div\vv D &= \rho\\
        \div\vv B &= 0\\
        \curl\vv E &= -\partial_t\vv B\\
        \curl\vv H &= \vv J + \partial_t\vv D
    \end{align*}
    
    \subsection{Maxwell's Equations In a Vacuum}
    While it may now seem like a backwards step the case of Maxwell's equations in a vacuum gives rise to perhaps the most important result of the course.
    In a vacuum there is no charge or current so Maxwell's equations, in the same order as before, are
    \begin{align*}
        \div\vv E &= 0\\
        \div\vv B &= 0\\
        \curl\vv E &= -\partial_t\vv B\\
        \curl\vv B &= \mu_0\varepsilon_0\partial_t\vv E
    \end{align*}
    We need the identity that for a vector field \(\vv a\)
    \[\curl(\curl\vv a) = \grad(\div\vv a) - \laplacian\vv a\]
    Applying this to \(\vv E\) we can calculate \(\curl(\curl\vv E)\) directly as
    \begin{align*}
        \curl(\curl\vv E) &= \curl(-\partial_t\vv B)\\
        &= -\partial_t(\curl\vv B)\\
        &= -\partial_t(\mu_0\varepsilon_0\partial_t\vv E)\\
        &= -\mu_0\varepsilon_0\partial_t^2\vv E
    \end{align*}
    where we have used to commutativity of partial derivates to commute \(\partial_t\) and \(\curl\).
    We can also calculate the right hand side of the identity to get
    \begin{align*}
        \curl(\curl\vv E) &= \grad(\div\vv E) - \laplacian\vv E\\
        &= \grad(0) - \laplacian\vv E\\
        &= -\laplacian\vv E
    \end{align*}
    Hence
    \[\laplacian\vv E = \mu_0\varepsilon_0\pdv[2]{\vv E}{t}\]
    We can do the same for the magnetic field, directly calculating \(\curl(\curl\vv B)\) we get
    \begin{align*}
        \curl(\curl\vv B) &= \curl(\mu_0\varepsilon_0\partial_t\vv E)\\
        &= \mu_0\varepsilon_0\partial_t(\curl\vv E)\\
        &= \mu_0\varepsilon_0\partial_t(-\partial_t\vv B)\\
        &= -\mu_0\varepsilon_0\partial_t^2\vv B
    \end{align*}
    We can also calculate the right hand side of the identity to get
    \begin{align*}
        \curl(\curl\vv B) &= \grad(\div\vv B) - \laplacian\vv B\\
        &= \grad(0) - \laplacian\vv B\\
        &= -\laplacian\vv B
    \end{align*}
    Hence
    \[\laplacian\vv B = \mu_0\varepsilon_0\pdv[2]{\vv B}{t}\]
    The wave equation in one dimension is
    \[\pdv[2]{u}{x} = \frac{1}{v^2}\pdv[2]{u}{t}\]
    where \(u\) is the displacement caused by the wave and \(v\) is the speed of the wave
    This extends into three dimensions as
    \[\laplacian\vv u = \frac{1}{v^2}\pdv[2]{\vv u}{t}\]
    Comparing this with the equations for the Laplacian of the electric and magnetic fields we see that they have the same form if we take the speed of the wave to be
    \[c = \frac{1}{\sqrt{\mu_0\varepsilon_0}}\]
    This means that electromagnetic waves have speed \(c\).
    Noting that this is independent of any basis means that \(c\) is the same in all frames!
    This key insight and the assumption that physics is the same in all inertial frames leads to special relativity.
\end{document}
