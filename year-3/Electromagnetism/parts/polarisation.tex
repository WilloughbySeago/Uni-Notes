\part{Polarisation}
    \section{Polarisation}
    So far we have considered superposition of waves travelling along the \(z\)-axis with \(\vv{E}\) aligned and \(\vv{B}\) aligned an perpendicular to \(\vv{E}\).
    This isn't always the case and to generalise we need to look at polarised light.
    
    \subsection{Linear Polarisation}
    Consider the two following waves
    \[\vv{E_1} = \ve{x}E_{0x}\cos(kz - \omega t - \Phi_1), \qquad\text{and}\qquad \vv{E_2} = \ve{y}E_{0y}\cos(kz - \omega t - \Phi_2).\]
    The first of these is \(x\)-polarised as the electric field oscillates only along the \(x\)-axis, the second is \(y\)-polarised.
    We can write these as complex exponentials discarding the imaginary parts for the final values:
    \[\vv{E_1} = \ve{x}E_{0x}\exp[i(kz - \omega t - \Phi_1)], \qquad\text{and}\qquad \vv{E_2} = \ve{y}E_{0y}\exp[i(kz - \omega t - \Phi_2)].\]
    The superposition is then given by the sum of these two vectors:
    \[\vv{E} = \vv{E_1} + \vv{E_2}= \ve{x}E_{0x}\exp[i(kz - \omega t - \Phi_1)] + \ve{y}E_{0y}\exp[i(kz - \omega t - \Phi_2)].\]
    We can write these in terms of column vectors with the first component giving the \(x\) component and the second the \(y\) component:
    \[
        \vv{E} =
        \begin{pmatrix}
            E_{0x}\exp[i(kz - \omega t - \Phi_1)]\\
            E_{0y}\exp[i(kz - \omega t - \Phi_2)]
        \end{pmatrix}
        =
        \begin{pmatrix}
            E_{0x}e^{-i\Phi_1}\\
            E_{0y}e^{-i\Phi_2}
        \end{pmatrix}
        \exp[i(kz - \omega t)].
    \]
    Notice that we separate the oscillatory term into the exponential outside the vector and the vector is constant in time.
    This vector is called the \define{Jones vector} for \(\vv{E}\).
    The Jones vector is simply the complex amplitudes of the components of the wave in the \(x\) and \(y\) directions.
    
    \subsubsection{No Phase Shift}
    Suppose there is no relative phase shift between the two components.
    That is \(\Phi_1 = \Phi_2 = \Phi\), then
    \[
        \vv{E} = 
        \begin{pmatrix}
            E_{0x}\\ E_{0y}
        \end{pmatrix}
        e^{-i\Phi}\exp[i(kz - \omega t)].
    \]
    We can then identify \(\vv{E}\) as another linearly polarised wave with amplitude
    \[E_{0} = \sqrt{E_{0x}^2 + E_{0y}^2}\]
    which is polarised at angle
    \[\vartheta = \arctan\left( \frac{E_{0y}}{E_{0x}} \right)\]
    to the \(x\)-axis.
    
    \subsubsection{Circularly Polarised Light}
    Suppose the two waves have a constant phase difference of \(\pi/2\) so \(\Phi_1 - \Phi_2 = \pi/2\).
    Suppose also that \(E_{0x} = E_{0y} = E_{01}\).
    Then
    \[
        \vv{E} = 
        \begin{pmatrix}
            E_{01}e^{-i(\Phi_2 + \pi/2)}\\
            E_{01}e^{-i\Phi_2}
        \end{pmatrix}
        \exp[i(kz - \omega t)] = 
        \begin{pmatrix}
            -i\\ 1
        \end{pmatrix}
        e^{-i\Phi_2}\exp[i(kz - \omega t)].
    \]
    It is fairly easy to show that as \(z\) increases \((E_x(z), E_y(z))\) maps out a clockwise circle completing one circle each time \(z\) increases by one wavelength.
    Similarly \((E_x(t), E_y(t))\) maps out an anticlockwise circle with angular frequency \(\omega\).
    This is a circularly polarised state.
    We call this \define{left hand circularly polarised}, or \(\mathcal{L}\).
    Similarly if \(\Phi_1 - \Phi_2 = -\pi/2\) then we have Jones vector
    \[
        \begin{pmatrix}
            i\\ 1
        \end{pmatrix}
    \]
    and the circles are in the opposite direction.
    We call this \define{right hand circularly polarised}, or \(\mathcal{R}\).
    
    \subsubsection{Elliptically Polarised}
    The most general case has no relationship between amplitudes or phases.
    Then \(\vv{E}\) both rotates and changes magnitude in the \((x, y)\)-plane tracing out an ellipse as it does so competing one rotation every wave length with angular frequency \(\omega\).
    We call this \define{elliptically polarised light}.
    
    \subsection{Normalisation}
    As is often the case with vectors we wish to work with orthonormal vectors as a basis.
    For this we need a well defined inner product.
    Since the polarisation vectors are two dimensional and contain complex values we can identify them with \(\complex^2\) and the correct inner product is then
    \[
        \vv{a} \cdot \vv{b} = \vv{a}\hermit\vv{b} =
        \begin{pmatrix}
            a_1\\ a_2
        \end{pmatrix}
        \hermit
        \begin{pmatrix}
            b_1\\ b_2
        \end{pmatrix}
        =
        \begin{pmatrix}
            a_1^* & a_2^*
        \end{pmatrix}
        \begin{pmatrix}
            b_1\\ b_2
        \end{pmatrix}
        = a_1^*b_1 + a_2^*b_2.
    \]
    In particular we say that \(\{\ve{i}\}\) are orthonormal if \(\ve{i}\cdot\ve{j} = \delta_{ij}\).
    
    Clearly the vectors denoting \(x\) and \(y\) polarised light,
    \[
        \ve{x} = 
        \begin{pmatrix}
            1\\ 0
        \end{pmatrix}
        , \qquad\text{and}\qquad \ve{y} = 
        \begin{pmatrix}
            0\\ 1
        \end{pmatrix}
    \]
    are orthonormal.
    The vectors that we found for describing circularly polarised light are orthogonal and can be normalised by a factor of \(1/\sqrt{2}\):
    \begin{equation}\label{eqn:circular basis}
        \ve{\mathcal{R}} = \frac{\sqrt{2}}{2}
        \begin{pmatrix}
            i\\ 1
        \end{pmatrix}
        , \qquad\text{and}\qquad \ve{\mathcal{L}} = \frac{\sqrt{2}}{2}
        \begin{pmatrix}
            i\\ -1
        \end{pmatrix}
        .
    \end{equation}
    In order to be a proper basis a set must span the space while being linearly independent.
    This is known to be true for \(\{\ve{x}, \ve{y}\}\) and is easy to show for \(\{\ve{\mathcal{R}}, \ve{\mathcal{L}}\}\) since \(\ve{\mathcal{R}} + \ve{\mathcal{L}} = i\sqrt{2}\ve{x}\) and \(\ve{\mathcal{R}} - \ve{\mathcal{L}} = i\sqrt{2}\ve{y}\) so as long as \(\{\ve{x}, \ve{y}\}\) spans the space \(\{\ve{\mathcal{R}}, \ve{\mathcal{L}}\}\) spans the space too meaning both are orthonormal bases.
    Which basis we use depends on the types of polarisation present.
    
    \subsection{Intensity}
    The electric field can be written as
    \[
        \vv{E} = 
        \begin{pmatrix}
            E_{0x}e^{-i\Phi_1}\\
            E_{0y}e^{-i\Phi_2}
        \end{pmatrix}
        \exp[i(kz - \omega t)] = \vv{E_0} \exp[i(kz - \omega t)].
    \]
    The intensity is given by the square of the magnitude of the electric field.
    Since we deal with the complex amplitude here this is
    \[I = \frac{1}{2}c\varepsilon_0E_0 \abs{E_0}^2 = \frac{1}{2}c\varepsilon_0\vv{E_0}\cdot\vv{E_0} = \frac{1}{2}c\varepsilon_0E_0^*E_0 = \frac{1}{2}c\varepsilon_0(E_{0x}^2 + E_{0y}^2).\]
    
    \section{Polarisers}
    Any individual wave has a polarisation however most light is formed of many different waves which are incoherent and, on average, have no net polarisation.
    For this reason we need a way to produce polarised light.
    
    \subsection{Linear Polarisers}
    A \define{linear polariser} (often just called a polariser) will only allow light polarised along a fixed axis to pass through.
    Natural (unpolarised) light has the form
    \[
        \begin{pmatrix}
            E_x(z, t)\\
            E_y(z, t)
        \end{pmatrix}
        =
        \begin{pmatrix}
            E_{0x}(t)\cos[kz - \omega t - \Phi_x(t)]\\
            E_{0y}(t)\cos[kz - \omega t - \Phi_y(t)]
        \end{pmatrix}
        .
    \]
    In general the two components are only coherent over short time scales, less than the coherence time, \(\Delta t\).
    Over time scales larger than this the ratio of the amplitudes, \(E_{0x}/E_{0y}\), and the relative phase, \(\Phi_y - \Phi_x\), will be unpredictable.
    If this light passes through a polariser then this forces a correlation between the amplitudes and phases and therefore the resulting light will be coherent and the output from the polarised behaves like an ideal harmonic wave.
    For this reason it is common to use a polariser as the first step in an optics experiment.
    
    \subsection{Malus's Law}
    Consider light from one polariser passing through a second polariser.
    If the axes of polarisation are aligned then we expect that all light from the first polariser will pass through the second.
    If the axes of polarisation are perpendicular then we expect the second polariser to block all of the light.
    What happens if the axes are not parallel or perpendicular?
    
    Suppose that the first polariser produces light, \(\vv{E_{\mathrm{in}}}\), and that the axis of the second polariser, \(\vh{p}\), is at angle \(\vartheta\) to the polarisation of the incoming light.
    The light transmitted through the second polariser, \(\vv{E_{\mathrm{out}}}\), is the component of \(\vv{E_{\mathrm{in}}}\) in the direction of \(\vh{p}\), and is polarised in the direction \(\vh{p}\).
    That is
    \[\vv{E_{\mathrm{out}}} = (\vv{E_{\mathrm{in}}}\cdot\vh{p})\vh{p} = E_{\mathrm{in}}\cos(\vartheta)\vh{p}.\]
    The intensity of the light coming from the second polariser is then
    \[I_{\mathrm{out}}(\vartheta) = \frac{1}{2}c\varepsilon_0E_{\mathrm{in}}^2\cos^2\vartheta = I_{\mathrm{in}}\cos^2\vartheta.\]
    So as the second polariser is rotated we expect the intensity of the output to go as \(\cos^2\vartheta\).
    This is \define{Malus's law}.
    Notice that in a whole rotation there are two maxima, \(\vartheta = 0, \pi\), which correspond to the polarisers being aligned and anti-aligned, and two minima, \(\vartheta = \pi/2, 3\pi/2\), which correspond to the two times the polarisers are perpendicular.
    
    \subsection{Jones Matrices}
    We managed to find the intensity in the last section by a logical argument.
    However there is a much more powerful mathematical framework that allows us to work with arbitrary combinations of polarisers.
    This framework, called Jones matrices, assigns to each polariser a matrix which then acts on the Jones vector of the incoming light.
    A linear polariser in the \(x\) direction has the Jones matrix
    \[
        \begin{pmatrix}
            1 & 0\\
            0 & 1
        \end{pmatrix}
    \]
    since
    \[
        \begin{pmatrix}
            1 & 0\\
            0 & 1
        \end{pmatrix}
        \begin{pmatrix}
            E_{x}\\ E_{y}
        \end{pmatrix}
        =
        \begin{pmatrix}
            E_x\\ 0
        \end{pmatrix}
        ,
    \]
    so we are left with only the \(x\) component.
    Similarly a linear polariser in the \(y\) direction has Jones matrix
    \[
        \begin{pmatrix}
            0 & 0\\
            0 & 1
        \end{pmatrix}
        .
    \]
    From here we can construct the matrix for a polariser at arbitrary angle \(\vartheta\) by realising that this is really just a rotated version of an \(x\) aligned polariser.
    This is best demonstrated by considering the same example of incoming polarised light, which without loss of generality we assume to be \(x\) polarised, and a polariser at angle \(\vartheta\).
    Then
    \begin{align*}
        \vv{E_{\mathrm{out}}} &= 
        \begin{pmatrix}
            1 & 0\\
            0 & 1
        \end{pmatrix}
        R(\vartheta)
        \begin{pmatrix}
            1\\ 0
        \end{pmatrix}
        \\
        &= 
        \begin{pmatrix}
            1 & 0\\
            0 & 1
        \end{pmatrix}
        \begin{pmatrix}
            \cos\vartheta & -\sin\vartheta\\
            \sin\vartheta & \cos\vartheta
        \end{pmatrix}
        \begin{pmatrix}
            1\\ 0
        \end{pmatrix}
        \\
        &= 
        \begin{pmatrix}
            1 & 0\\
            0 & 1
        \end{pmatrix}
        R(\vartheta)
        \begin{pmatrix}
            \cos\vartheta\\ \sin\vartheta
        \end{pmatrix}
        \\
        &=
        \begin{pmatrix}
            \cos\vartheta\\ 0
        \end{pmatrix}
        .
    \end{align*}
    From this we easily recover Malus's law for the intensity.
    
    This worked well for this simple case.
    For a more complicated case it would be better to have each polariser have a single matrix and not have to worry about rotating as we go.
    To do this we define a frame of reference for each individual polariser in which the \(x\) direction aligns with the polarisation axis.
    In the case of incoming polarised light have the un-primed frame have \(x\) be aligned with the initial polarisation.
    In this frame the input light has the Jones vector \(\left( \begin{smallmatrix} 1\\ 0 \end{smallmatrix} \right)\).
    We can be more general and assume that the incoming light has Jones vector \(\vv{u}\) and the output is light with Jones vector \(\vv{v}\).
    Let \(J\) be the Jones matrix of the polariser.
    Then we have \(\vv{v} = J\vv{u}\).
    This must also hold if we express all terms in the frame of the polariser, which is simply a rotated frame.
    Denote by a prime the same vectors and matrices but in this rotated frame.
    Then we have \(\vv{v'} = J'\vv{u'}\).
    We can convert between the two frames with a rotation matrix so \(\vv{u} = R(\vartheta)\vv{u'}\) and \(\vv{v} = R(\vartheta)\vv{v'}\).
    Noting that the inverse of a rotation is simply a rotation by the same amount in the opposite direction we have
    \[\vv{v'} = \ident \vv{v'} = R(-\vartheta)R(\vartheta)\vv{v'} = R(-\vartheta)\vv{v} = R(-\vartheta)J\vv{u} = R(-\vartheta)JR(\vartheta)\vv{u'} = J'\vv{u'}\]
    so we identify
    \[R(-\vartheta)JR(\vartheta) = J'.\]
    This is just the normal rule for transforming matrices between bases.
    
    We can repeat our calculation of the output but now in the primed frame:
    \begin{align*}
        \vv{E_{\mathrm{out}}'} &= R(-\vartheta)
        \begin{pmatrix}
            1 & 0\\
            0 & 1
        \end{pmatrix}
        R(\vartheta)
        \begin{pmatrix}
            1\\ 0
        \end{pmatrix}
        \\
        &= 
        \begin{pmatrix}
            \cos^2\vartheta & -\cos\vartheta\sin\vartheta\\
            -\cos\vartheta\sin\vartheta & \sin^2\vartheta\\
        \end{pmatrix}
        \begin{pmatrix}
            1\\ 0
        \end{pmatrix}
        \refstepcounter{equation}\tag{\theequation}\label{eqn:rotate matrix}\\
        &=
        \begin{pmatrix}
            \cos^2\vartheta\\
            -\cos\vartheta\sin\vartheta
        \end{pmatrix}
        .
    \end{align*}
    This gives the same \(\cos^2\vartheta\) relationship for the output intensity.
    
    The transformations can be easily remembered with a commutative diagram\footnote{the `commutative' part of this diagram is simply that any path between the same two points gives the same result. Matrix multiplication is non-commutative so the order arrows are traversed is important.}:
    \tikzexternaldisable
    \[
        \begin{tikzcd}[column sep=1cm, row sep=1cm]
            \vv{u}  \arrow{r}{J}  & \vv{v}\\
            \vv{u'} \arrow{r}[swap]{J'} \arrow{u}{R} & \vv{v'} \arrow{u}[swap]{R}
        \end{tikzcd}
    \]
    \tikzexternalenable
    To read this diagram simply pick the starting point and end point and find a path between them.
    Then traversing this path every arrow gives multiplication by the corresponding matrix if going in the direction of the arrow or the inverse of that matrix if going against the arrow.
    
    \section{More Polarisers}
    In a vacuum there is nothing to distinguish between different polarisations.
    Therefore anything that causes polarisation must have some level of anisotropy, it must preferentially select for one direction over the others.
    We have seen that result of light incident at an oblique angle (i.e. non-normal incidence) depends on the polarisation state and this can be used to create polarised light.
    This happens even if the medium is isotropic, it is the geometry of the experiment that introduces the anisotropy.
    We have been considering \gls{lih} materials so far.
    In this section we relax this and consider materials with some anisotropy (LH materials).
    
    \subsection{Wire Grid Polariser}
    The simplest polariser is a collection of wires running parallel with only a small amount of spacing between wires (small as usual compared to the wavelength).
    The gap can be on a visible scale if we are interested in microwaves which have wavelengths of a few centimetres.
    Oscillations of the electric field along the wires will lead to electrons moving along the wire.
    In a real metal this movement is damped by the resistance reducing the intensity of the field oscillating along that direction.
    Thus the field oscillating perpendicular to the wires is the field that passes through.
    This is somewhat counter-intuitive if we consider the slots between wires to be like holes one may expect the field oscillating across the wires to be blocked but this isn't the case.
    The property of strongly absorbing one polarisation and allowing another to pass is called \define{dichroism} and the effect in this case is to produce linearly polarised light so these are a form of \define{dichroic linear polaiser}.
    
    \subsection{Polaroid}
    Polaroid works similarly to a wire grid polariser but on an atomic scale.
    They are made by creating sheets of polymer and then stretching in one direction introducing anisotropy as the long chained molecules are aligned.
    The material is then treated with iodine which binds to the molecules and fixes them in place.
    Polaroids can be made to work well across the entire visible spectrum.
    
    The quality of a polariser is given by its \define{extinction ratio}.
    This is defined as the ratio \(R = I_0 / I_L\) where \(I_0\) is the intensity of light incident on polariser orthogonally polarised to the transmission axis and \(I_L\) is the intensity of light that it lets through which should have been blocked.
    For polaroid a typical value is \(R = 100\).
    For research a much higher value is needed.
    
    \subsection{Polarisation by Reflection}
    If a glass block is illuminated at Brewster's angle with unpolarised light then there is no reflection of P-polarised light and so the reflected light will be entirely S-polarised.
    For technical reasons it is often more convenient to work with the transmitted beam which is not entirely polarised but has far more P-polarised light than the incident beam.
    
    \subsection{Polarisation by Scattering}
    Recall that when a dipole absorbs light it re-emits it at the same frequency and the intensity is not spherically symmetrical.
    No light is emitted along the dipole vector and the highest intensity is in a plane normal to the dipole axis.
    Thus light which is scattered in this plane at a right angle to the incoming light will be scattered with oscillation only in the plane as oscillation perpendicular to the plane and along the dipole is not allowed.
    
    The intensity of such scattered light is
    \[I = \frac{\omega^4 p_0^2\sin^2\vartheta}{32\pi^2\varepsilon_0r^2c^3}.\]
    The factor of \(\omega^4\) means that blue light is scattered far more than red light.
    This is the reason that the sky is blue.
    Light from the sun, which is approximately white, is scattered when it passes through the atmosphere with blue being scattered far from the sun (so into the rest of the sky) and red light passing straight through without much scattering making the sun appear slightly red/yellow.
    This is called \define{Rayleigh scattering}.
    
    \section{Birefringence}
    \define{Birefringence} is the property by which the refractive index depends on the polarisation.
    A birefringent material is necessarily anisotropic.
    In this case the permittivity is a tensor but some simple cases can still be considered without needing tensors which is what we will do here.
    
    \subsection{Uniaxial Birefringence}
    A material is uniaxially birefringent if there exists a basis in which the permittivity tensor, \(\varepsilon\), is diagonal and \(\varepsilon_x = \varepsilon_y \ne \varepsilon_z\).
    Thus a ray travelling along the optical axis (that is the \(z\)-axis) will behave the same for any polarisation.
    If \(n_x\) (resp. \(n_y\)) is the refractive index for light polarised in the \(x\) direction (resp. \(y\) direction) then for uniaxial birefringence we have \(n_x = n_y\).
    
    If we consider Huygen's principle then each wavelet that sets out travels in all directions.
    Since the refractive index depends on the direction of polarisation (for travel not along the \(z\)-axis) this means that instead of spherical wave fronts we get ellipsoids but the net wave front still progresses forwards as normal.
    
    For propagation perpendicular to the optical axis we instead get two wave fronts, one for light that with polarisation along the optical axis and one for light with polarisation perpendicular to the optical axis.
    Let \(n_o\) be the refractive index for light polarised perpendicular the optical axis and \(n_e\) be the refractive index for light with polarisation parallel to the optical axis.
    The \(o\) stands for ordinary as this wave propagates as we would expect for any material and the \(e\) stands for extraordinary.
    Let \(\Delta n = n_e - n_o\).
    If \(\Delta n > 0\) then we say that the material is \define{positive uniaxial} and if \(\Delta n < 0\) we say it is \define{negative uniaxial}.
    The extraordinary thing about waves polarised parallel to the optical axis is that the wave vector, \(\vv{k}\), which gives the direction of propagation of the wave, doesn't necessarily align with the ray direction which is given by the Poynting vector, \(\vv{S}\).
    
    \subsection{Biaxial Birefringence}
    A biaxial crystal has a different refractive index along each axis and is therefore characterised by three separate refractive indices, \(n_\alpha\), \(n_\beta\), and \(n_\gamma\).
    The biaxial part here is due to the fact that such crystals actually have two optical axis along which the refractive index is independent of polarisation (as opposed to the single axis of a uniaxial crystal).
    
    \subsection{Birefringence from the Oscillator Model}
    The oscillator model assumes that only the magnitude of the displacement of the electron cloud is important for the response of the system.
    We can picture this as the electron cloud being held in place by six springs (two along each axis) all of which are the same strength.
    If instead we take some of the springs to be of different strengths then the direction of displacement matters.
    This means that the natural frequency of oscillations will depend on the direction of displacement and therefore the polarisation of the electric field.
    This also explains dichroism as the different springs sill absorb energy at different rates and therefore the energy absorbed also depends on the polarisation.
    
    \subsection{Retarders}
    Consider linearly polarised light incident on a uniaxial material but such that the optical axis is not necessarily parallel or perpendicular to the incident polarisation.
    Suppose also that this light is monochromatic with frequency \(\omega\) and wavelength in air, \(\lambda_{\mathrm{air}}\).
    Further assume that the material is transparent in the relevant region of the \gls{em} spectrum and we can ignore the small fraction of light that is reflected.
    
    When the light enters the medium the component parallel to the optical axis experiences refractive index \(n_e\) and travels at speed \(c/n_e\) whereas the component perpendicular to the optical axis experiences refractive index \(n_o\) and travels at speed \(c/n_o\).
    One component will therefore travel faster than the other.
    Which component depends on the material since \(\Delta n = n_e - n_o\) can be positive or negative so from now on we will refer to a fast axis and a slow axis.
    
    Suppose that the medium is a slab with the incident face and back parallel.
    Then the light exits at normal incidence also.
    Since \(\lambda = \lambda_{\mathrm{air}}/n\) the number of wavelengths that the light will complete while in the material will be different.
    This means that if the two components are in phase when they enter the material they will be out of phase when they leave.
    This introduces a time delay to the slow ray which, when expressed in units of phase angle, is called the \define{retardation} of the slow wave, \(\delta\).
    As such uniaxial material arranged in slabs like this are called retarders or retardation plates.
    
    The effect of a retarder is that the slow wave is retarded in time but advanced in space (in that the same point on the fast and slow waves will occur further away for the slow wave).
    The effect is that the the slow wave picks up a phase factor of \(e^{i\delta}\).
    If we take the \(x\)-axis to be aligned with the fast axis then the Jones matrix of a retarder is
    \[
        J = 
        \begin{pmatrix}
            e^{i\delta_{\text{fast}}} & 0\\
            0 & e^{i\delta_{\text{slow}}}
        \end{pmatrix}
    \]
    since both waves are retarded somewhat.
    If the slab has physical thickness \(d\) then it has optical thickness \(nd\) and phase thickness \(2\pi nd/\lambda_{\text{air}}\) and we have \(\delta_{\text{fast}} = 2\pi n_{\text{fast}}d/\lambda_{\text{air}}\) and \(\delta_{\text{slow}} = 2\pi n_{\text{slow}}d/\lambda_{\text{air}}\).
    It is often possible to use the freedom in deciding the global phase to put all of the effect in the slow direction and so
    \[
        J = 
        \begin{pmatrix}
            1 & 0\\
            0 & e^{i\delta}
        \end{pmatrix}
    \]
    where
    \[\delta = \delta_{\text{slow}} - \delta_{\text{fast}} = \frac{2\pi d}{\lambda_{\text{air}}}(n_{\text{slow}} - n_{\text{fast}}) > 0.\]
    So \(\delta\) is the retardation of the slow wave relative to the fast wave.
    
    \subsubsection{Real Retarders}
    Suppose we want a retarder that results in a phase difference of \(\pi\), that is half a cycle, then we need the physical thickness to be
    \[d = \frac{\lambda_{\mathrm{air}}/2}{n_{\text{slow}} - n_{\text{fast}}}.\]
    Unfortunately this is often prohibitively thin.
    Fortunately if its only the phase difference we care about then there is no difference between a phase difference of \(\pi\) or \(3\pi\), or indeed \((2n + 1)\pi\) for any \(n \in \integers\).
    So we can simply add any integer number of wavelengths to the optical path difference and it results in the same phase difference.
    
    Another problem is that in general refractive index depends on frequency, and this is still the case with retarders.
    In fact \(\Delta n\) depends on frequency which means that the phase difference achieved by a retarder depends on the frequency of the incident light.
    We will come back to this later.
    
    \subsection{Retarder Uses}
    \subsubsection{Half-Wave Plate}
    A \gls{hwp} is a retarder that results in a phase difference of \(\pi\).
    If the fast axis is aligned with the \(x\)-axis then the Jones matrix is
    \[
        J_{\lambda/2} = 
        \begin{pmatrix}
            1 & 0\\
            0 & e^{i\pi}
        \end{pmatrix}
        =
        \begin{pmatrix}
            1 & 0\\
            0 & -1
        \end{pmatrix}
        .
    \]
    The result of this is to reverse the sign of the \(E_y\) component.
    For elliptically polarised light the \gls{hwp} reverses its handedness and rotates the ellipse so that if its major axis was at angle \(\varphi\) before the \gls{hwp} it will be at \(-\varphi\) afterwards (so a rotation of \(-2\varphi\)).
    
    \subsubsection{Quarter-Wave Plate}
    Similarly a \gls{qwp} results in a phase difference of \(\pi/2\) which means it has the Jones matrix
    \[
        J_{\lambda/4} =
        \begin{pmatrix}
            1 & 0\\
            0 & e^{-i\pi}
        \end{pmatrix}
        =
        \begin{pmatrix}
            1 & 0\\
            0 & i
        \end{pmatrix}
        .
    \]
    For the special case where \(E_x = \pm E_y\) if the beam is at an angle of \(\pm \pi/4\) to the fast axis then the result is circularly polarised light.
    
    \subsubsection{Crystal Polarisers}
    Birefringent materials can be used to create a polariser.
    These all work off of the basic principle that different polarisations are refracted a different amount.
    This means that incident light is split into two beams when it enters the material and the two beams have orthogonal polarisations.
    To create a polarised source simply arrange the set up such that upon reaching the other side of the medium one of the beams is at an angle greater than the critical angle and so is reflected back into the material whereas the other beam passes out of the material for use in the experiment.
    Note that the critical angle also depends on the polarisation since it depends on the refractive index.
    
    
    \section{More Jones Algebra}
    One of the useful things about using Jones matrices to describe the effect of a polarisers and retarders is that the effect of multiple polarisers and retarders can be combined into one Jones matrix by simply multiplying all relevant Jones matrices and possibly changing basis if the polarisation/fast axes aren't aligned.
    
    \begin{example}
        Consider three polarisers aligned such that in the lab frame the first has its polarisation axis aligned along the \(x\)-axis, the second has its polarisation axis at an angle, \(\vartheta\), to the first, and the third is aligned along the \(y\)-axis.
        The Jones matrices for the first and third polarisers are simple:
        \[
            J_1 = 
            \begin{pmatrix}
                1 & 0\\
                0 & 0
            \end{pmatrix}
            ,\qquad\text{and}\qquad J_3 =
            \begin{pmatrix}
                0 & 0\\
                0 & 1
            \end{pmatrix}
            .
        \]
        The Jones matrix for the second polariser is found by rotating the Jones matrix to align with the others which is done with the matrix in equation~\ref{eqn:rotate matrix} but reversing the sign of \(\vartheta\) since in that equation we are changing the basis and in this we wish to rotate the matrix (passive vs. active transform):
        \[
            J_2 = 
            \begin{pmatrix}
                \cos^2\vartheta & \cos\vartheta\sin\vartheta\\
                \cos\vartheta\sin\vartheta & \sin^2\vartheta
            \end{pmatrix}
        \]
        Therefore the matrix describing the effects of these three polarisers is
        \[
            J = J_3J_2J_1 =
            \begin{pmatrix}
                0 & 0\\
                0 & 1
            \end{pmatrix}
            \begin{pmatrix}
                \cos^2\vartheta & \cos\vartheta\sin\vartheta\\
                \cos\vartheta\sin\vartheta & \sin^2\vartheta
            \end{pmatrix}
            \begin{pmatrix}
                1 & 0\\
                0 & 0
            \end{pmatrix}
            =
            \begin{pmatrix}
                0 & 0\\
                0 & 1
            \end{pmatrix}
            \begin{pmatrix}
                \cos^2\vartheta & 0\\
                \cos\vartheta\sin\vartheta & 0
            \end{pmatrix}
            =
            \begin{pmatrix}
                0 & 0\\
                \cos\vartheta\sin\vartheta & 0
            \end{pmatrix}
            .
        \]
        So for unpolarised incident light the polarisation after these three polarisers is
        \[
            \begin{pmatrix}
                0 & 0\\
                \cos\vartheta\sin\vartheta & 0
            \end{pmatrix}
            \begin{pmatrix}
                E_x\\
                E_y
            \end{pmatrix}
            =
            \begin{pmatrix}
                0\\
                E_x\cos\vartheta\sin\vartheta
            \end{pmatrix}
            = \frac{1}{2}
            \begin{pmatrix}
                0\\ E_x\sin(2\vartheta)
            \end{pmatrix}
            .
        \]
        So the result is, \(y\)-polarised light, which it must be given the last polariser is aligned with the \(y\)-axis.
        The intensity of said light is
        \[I(\vartheta) = \frac{I_0}{4}\sin^2(2\vartheta)\]
        where \(I_0\) is the intensity of the incident light.
        The minimum transmission is 0 which occurs when any two neighbouring polarisers are orthogonal.
        The maximum transmission is \SI{25}{\percent} which occurs when the middle polariser is directly between the other two, so \(\vartheta = \pi/4, 3\pi/4, 5\pi/4, 7\pi/4\).
    \end{example}
    \begin{example}
        Consider the same set up as the previous example but replace the middle polariser with a retarder with its fast axis at angle \(\vartheta\).
        The Jones matrix for this retarder is
        \begin{align*}
            J_2 &= 
            \begin{pmatrix}
                \cos\vartheta & -\sin\vartheta\\
                \sin\vartheta & \cos\vartheta
            \end{pmatrix}
            \begin{pmatrix}
                1 & 0\\
                0 & e^{i\delta}
            \end{pmatrix}
            \begin{pmatrix}
                \cos\vartheta & -\sin\vartheta\\
                \sin\vartheta & \cos\vartheta
            \end{pmatrix}
            \\
            &=
            \begin{pmatrix}
                \cos\vartheta & -\sin\vartheta\\
                \sin\vartheta & \cos\vartheta
            \end{pmatrix}
            \begin{pmatrix}
                \cos\vartheta & \sin\vartheta\\
                -e^{i\delta}\sin\vartheta & e^{i\delta}\cos\vartheta
            \end{pmatrix}
            \\
            &= 
            \begin{pmatrix}
                \cos^2\vartheta + e^{i\delta}\sin^2\vartheta & \cos\vartheta\sin\vartheta(1 - e^{i\delta})\\
                \cos\vartheta\sin\vartheta(1 - e^{i\delta}) & e^{i\delta}\cos^2\sin\vartheta + \sin^2\vartheta
            \end{pmatrix}
        \end{align*}
        We could now find the Jones matrix representing the three different polarisers/retarders but instead we will make use of the fact that the first is an \(x\) aligned polariser so that the Jones vector after the first polariser is
        \[
            \vv{E} = 
            \begin{pmatrix}
                \sqrt{I_0/2}\\ 0
            \end{pmatrix}
        \]
        where we have used the fact that the light is unpolarised so \(E_x = E_y\) and \(E^2 = E_x^2 + E_y^2 = 2E_x^2 = I_0\).
        Hence the electric field that we get out the other side of the set up is
        \begin{align*}
            J_3J_2
            \begin{pmatrix}
                1\\ 0
            \end{pmatrix}
            \sqrt{I_0/2} &= 
            \begin{pmatrix}
                0 & 0\\
                0 & 1
            \end{pmatrix}
            \begin{pmatrix}
                \cos^2\vartheta + e^{i\delta}\sin^2\vartheta & \cos\vartheta\sin\vartheta(1 - e^{i\delta})\\
                \cos\vartheta\sin\vartheta(1 - e^{i\delta}) & e^{i\delta}\cos^2\sin\vartheta + \sin^2\vartheta
            \end{pmatrix}
            \begin{pmatrix}
                1\\ 0
            \end{pmatrix}
            \sqrt{I_{0}/2}\\
            &= 
            \begin{pmatrix}
                0 & 0\\
                0 & 1
            \end{pmatrix}
            \begin{pmatrix}
                \cos^2\vartheta + e^{i\delta}\sin^2\vartheta\\
                \cos\vartheta\sin\vartheta(1 - e^{i\delta})
            \end{pmatrix}
            \sqrt{I_0/2}\\
            &=
            \begin{pmatrix}
                0\\
                \cos\vartheta\sin\vartheta(1 - e^{i\delta})
            \end{pmatrix}
            \sqrt{I_0/2}\\
            &= \frac{\sqrt{2}}{4}I_0
            \begin{pmatrix}
                0\\
                \sin(2\vartheta)(1 - e^{i\delta})
            \end{pmatrix}
            .
        \end{align*}
        So this time the result is \(y\)-polarised light with intensity
        \begin{align*}
            I(\vartheta) &= \frac{1}{8}I_0\sin^2(2\vartheta)(1 - e^{i\delta})(1 - e^{-i\delta})\\
            &= \frac{1}{8}I_0\sin^2(2\vartheta)(1 - e^{i\delta} - e^{-i\delta} + 1)\\
            &= \frac{1}{8}I_0\sin^2(2\vartheta)(2 - 2\cos\delta)\\
            &= \frac{1}{4}I_0\sin^2(2\vartheta)(1 - \cos\delta).
        \end{align*}
        To see the effect of changing the retarder lets fix \(\vartheta = \pi/4\) and consider what happens as \(\delta\) changes.
        Suppose that \(d = \SI{1}{\milli\metre}\) and \(\Delta n = \num{1.25e-3}\).
        If the system is illuminated with white light then the light that we get out will be blue-green.
        This is because for \(\lambda \approx \SI{500}{\nano\meter}\) we have \(\delta\approx \pi\) and so for blue-green light the retarder acts as a \gls{hwp}.
        The effect of this is to rotate the polarisation by \(\pi/2\) which means that this light passes through the final polariser unimpeded.
    \end{example}
    The effect described in the previous example by which crossed polarisers can result in colour can be used to analyse transparent media for stresses since internal stresses can rotate polarisations (as stresses are anisotropic).
    
    \subsection{Eigenpolarisations}
    As the name suggests the \define{eigenpolarisations} of a device are the polarisations which are not rotated by the polariser.
    That is if the Jones matrix of the polariser is \(J\) and \(\vv{u}\) is an eigenpolarisation then
    \[J\vv{u} = \lambda\vv{u}\]
    where \(\lambda\) is some constant\footnote{\(\lambda\) is not the wavelength here.}.
    Finding eigenpolarisations is as simple as finding the eigenvectors of a \(2\times 2\) matrix.
    \begin{example}
        \textit{Find the eigenpolarisations of a polariser aligned with the \(x\)-axis.}
        
        The Jones matrix for this polariser is
        \[
            J =
            \begin{pmatrix}
                1 & 0\\
                0 & 0
            \end{pmatrix}
            .
        \]
        First we need to find the eigenvalues by solving for the roots of the characteristic polynomial:
        \[
            0 = 
            \begin{vmatrix}
                1 - \lambda & 0\\
                0 & -\lambda
            \end{vmatrix}
            = (1 - \lambda)(-\lambda) \implies \lambda_{1, 2} = 0, 1.
        \]
        From here its a simple case of substituting these into the eigenvalue equation and solving for \(\vv{u}\):
        \[
            \vv{u_1} = 
            \begin{pmatrix}
                0\\ 1
            \end{pmatrix}
            ,\qquad\text{and}\qquad \vv{u_2} =
            \begin{pmatrix}
                1\\ 0
            \end{pmatrix}
            .
        \]
        Notice that these are simply the two orthogonal linear polarisations that we use as a basis.
    \end{example}
    \begin{example}
        \textit{Find the eigenpolarisations of a retarder.}
        
        The Jones matrix for a retarder is
        \[
            J =
            \begin{pmatrix}
                1 & 0\\
                0 & e^{i\delta}
            \end{pmatrix}
            .
        \]
        First we need to find the eigenvalues by solving for the roots of the characteristic polynomial:
        \[
            0 = 
            \begin{vmatrix}
                1 - \lambda & 0\\
                0 & e^{i\delta} - \lambda
            \end{vmatrix}
            = (1 - \lambda)(e^{i\delta} - \lambda) \implies \lambda_{1, 2} = e^{i\delta}, 1.
        \]
        From here its a simple case of substituting these into the eigenvalue equation and solving for \(\vv{u}\):
        \[
            \vv{u_1} = 
            \begin{pmatrix}
                0\\ 1
            \end{pmatrix}
            ,\qquad\text{and}\qquad \vv{u_2} =
            \begin{pmatrix}
                1\\ 0
            \end{pmatrix}
            .
        \]
        So the eigenpolarisations are the same as for the polariser but with different eigenvalues.
    \end{example}
    \subsubsection{Diagonalisation}
    We can diagonalise the Jones matrix once we have found its eigenvectors and eigenvalues.
    To do this we use a matrix, \(T\), which has the eigenvectors of \(J\) as its columns and the result is that \(J' = T^{-1}JT\) is a diagonal matrix with the eigenvalues of \(J\) along its diagonal in the same order that the corresponding eigenvectors appear in \(T\).
    
    \subsection{Circular Systems}
    Chiral molecules are ones which have exactly the same constituents but are mirror images of each other.
    A solution of these molecules treats all linear polarisations the same as a solution is necessarily isotropic.
    However the handedness of these molecules means that they don't treat circular polarisations equally.
    
    We can consider a circular polariser which has eigenvectors \(\ve{\mathcal{L}}\) and \(\ve{\mathcal{R}}\).
    Using the circular basis, which we consider to be the primed basis, the Jones matrix of this device is of the form
    \[
        J' =
        \begin{pmatrix}
            t_{\mathcal{L}} & 0\\
            0 & t_{\mathcal{R}}
        \end{pmatrix}
        .
    \]
    In order to combine the effects of this circular polariser with linear polarisers and retarders we need to express this in the linear polariser basis.
    To do this we need to find the transformation matrix \(T\) which has as its columns the vectors representing \(\ve{\mathcal{L}}\) and \(\ve{\mathcal{R}}\) in the linear polariser basis.
    These vectors are given in equation~\ref{eqn:circular basis} and so we have
    \[
        T = \frac{\sqrt{2}}{2}
        \begin{pmatrix}
            1 & 1\\
            i & -i
        \end{pmatrix}
        .
    \]
    It is then trivial to show that
    \[
        T^{-1} = \frac{\sqrt{2}}{2}
        \begin{pmatrix}
            1 & -i\\
            1 & i
        \end{pmatrix}
    \]
    Hence in the linear polariser basis
    \begin{align*}
        J &= TJ'T^{-1}\\
        &= \frac{1}{2}
        \begin{pmatrix}
            1 & 1\\
            i & -i
        \end{pmatrix}
        \begin{pmatrix}
            t_{\mathcal{L}} & 0\\
            0 & t_{\mathcal{R}}
        \end{pmatrix}
        \begin{pmatrix}
            1 & -i\\
            1 & i
        \end{pmatrix}
        \\
        &=
        \begin{pmatrix}
            \bar{t} & i\Delta/2\\
            -i\Delta/2 & \bar{t}
        \end{pmatrix}
    \end{align*}
    where \(\bar{t} = (t_{\mathcal{L}} + t_{\mathcal{R}})/2\) and \(\Delta = t_{\mathcal{R}} - t_{\mathcal{L}}\).
    Notice here that we use \(J = TJ'T^{-1}\) since we are inverting the transform \(J' = T^{-1}JT\).
    
    \section{Reflections and Other Polarisation Effects}
    So far we have considered optical devices which transmit normally incident light.
    We can also consider non-normal incidence and reflective devices with the same Jones algebra.
    One complication is that we then need two different coordinate systems, one for the incident beam and one for the reflected beam.
    We use the convention that for any beam we define the \(x\) direction as the direction of oscillation of P-polarised light and the \(y\)-direction of oscillation of S-polarised light.
    
    The simplest case of a reflection occurs at normal incidence.
    At normal incidence the polarisation is not important and \(\abs{\fresnelCoeff{r}{S}} = \abs{\fresnelCoeff{r}{P}} = \abs{r}\).
    The Jones matrix describing a reflection at normal incidence is then
    \[
        J = \abs{r}
        \begin{pmatrix}
            1 & 0\\
            0 & -1
        \end{pmatrix}
        .
    \]
    Notice the factor of \(-1\) for \(y\)-polarised light (S-polarised light).
    This is because of the \(\pi\) phase shift that S-polarised light undergoes upon reflection.
    Ignoring the change in direction and assuming \(\abs{r}\approx 1\) we see that the effect of a reflection is essentially the same as the effect of a \gls{hwp}.
    Notice that both of these reverse the handedness of the incident light, which is what we would expect from a mirror.
    
    One interesting use of this is to construct an \emph{anti}-reflection device.
    First polarise the light at \(\pi/4\) to the \(x\)-axis.
    Then pass the light through a \gls{qwp} which results in \(\mathcal{L}\)-circularly polarised light.
    Then reflect this light which results in \(\mathcal{R}\)-circularly polarised light.
    This then passes back through the \gls{qwp} resulting in light polarised at \(-\pi/4\) to the \(x\) axis which is then completely blocked by the polariser.
    
    \subsection{Polarimetry}
    \define{Polarimetry} is the general term describing experiments to determine the polarisation of a beam of light.
    For simplicity here we will assume that this light is indeed polarised and discus a few ways to find out what this polarisation is.
    It is possible to extend our work so far to partially polarised light but we then need a four dimensional analogue of Jones algebra known as Stokes--Mueller calculus.
    
    Suppose that we have a polariser, a \gls{qwp}, and an intensity detector and we wish to determine the polarisation of a beam of monochromatic polarised light.
    Passing the beam through the polariser and measuring the transmitted intensity as the polariser is rotated there are three possible outcomes:
    \begin{itemize}
        \item Two maxima are observed with the full intensity of the incident beam and two minima are observed with intensity 0.
        This is the behaviour predicted by Malus' law for linearly polarised light polarised such at the same angle at which the maxima occur.
        
        \item The intensity is half the incident intensity and doesn't vary with the polariser angle.
        This is due to circularly polarised light.
        This can be confirmed by inserting the \gls{qwp} before the polariser which will turn the circularly polarised light into linearly polarised light which we can then check using the first bullet point.
        
        \item Two maxima and two minima are observed but the intensity is never zero.
        This corresponds to elliptically polarised light and is simply the combination of the two previous cases.
        To find the size and orientation of the major axis of the ellipse pass the beam through the \gls{qwp} and polariser.
        There will be a particular angle of the \gls{qwp} at which the light produced is linearly polarised.
        We can use the polariser to check for this with Malus' law.
        From here we can then compute the orientation and ellipticity of the polarisation.
    \end{itemize}