\section{Size of a Union}
        Suppose that \(A_i\) are sets for some \(i\in {1, \dotsc, n}\).
        Let
        \[\bigunion_{i = 1}^{n} A_i = A_1 \union A_2 \union \dotsb \union A_n = \{a\st a\in A_1\logicalOr A\in A_2\logicalOr\dotsb\logicalOr a\in A_n\}.\]
        It is useful to know what the size of this union is as it often occurs in probability.
        For example
        \[P(A_1\logicalOr A_2\logicalOr \dotsb \logicalOr A_n) = \frac{1}{|S|}\abs{\bigunion_{i = 1}^n A_i}.\]
        We have already seen that in the case \(n = 2\) we have \(|A_1\union A_2| = |A| + |B| - |A\intersection B|\).
        We arrived at this result by counting the elements in each set and then subtracting the elements that we double counted due to them being counted in both sets.
        We can actually be a little bit more rigorous than this with the following theorem\cite[liebeck].
        \begin{theorem}
            Let \(A\) and \(B\) be finite sets.
            Then
            \[|A \union B| = |A| + |B| - |A\intersection B|\]
        \end{theorem}
        \begin{proof}
            Let \(|A\intersection B| = k\).
            Then \(A\union B = \{x_1, \dotsc, x_k\}\).
            These elements, and no others, are in both \(A\) and \(B\) so we have
            \[A = \{x_1, \dotsc, x_k, a_1, \dotsc, a_l\},\qquad\text{and}\qquad B = \{x_1, \dotsc, x_k, b_1, \dotsc, b_m\}\]
            where \(a_i\) and \(b_i\) are all distinct from each other and \(a_i\notin B\) and \(b_i\notin A\) for all \(i\).
            Then clearly \(|A| = k + l\) and \(|B| = k + m\).
            Now we compute the union
            \[A \union B = \{x_1, \dotsc, x_k, a_1, \dotsc, a_l, b_1, \dotsc, b_m\}.\]
            All of these elements are different so
            \begin{align*}
                |A\union B| &= k + l + m\\
                &= (k + l) + (k + m) - k\\
                &= |A| + |B| - |A\intersection B|.
            \end{align*}
        \end{proof}
        That's all well and good for two sets but what about a finite number of finite sets, \(A_i\)?
        We can use another theorem\cite{liebeck}.
        \begin{theorem}
            Let \(n\) be a positive integer, and  let \(A_1, \dotsc, A_n\) be finite sets.
            Then
            \[\abs{\bigunion_{i = 1}^n A_i} = |A_1 \union \dotsb \union A_n| = c_1 - c_2 + c_3 - \dotsb + (-1)^nc_n\]
            where for \(1 \le i \le n\) the number \(c_i\) is the sum of the sizes of the intersections of the sets taken \(i\) at a time.
            For example if \(n = 3\) then
            \begin{align*}
                c_1 &= |A_1| + |A_2| + |A_3|\\
                c_2 &= |A_1 \intersection A_2| + |A_1 \intersection A_3| + |A_2 \intersection A_3|\\
                c_3 &= |A_1 \intersection A_2 \intersection A_3|.
            \end{align*}
        \end{theorem}
        \begin{proof}
            Let \(x \in \bigunion_{i = 1}^n A_i.\)
            We will show that \(x\) contributes exactly once to the expression we have for the size of this union.
            Suppose that \(x\) belongs to precisely \(k\) of the sets \(A_1, \dotsc, A_n\) where \(1 \le k \le n\).
            Then \(x\) contributes \(k\) to the sum 
            \[c_1 = \sum_{1\le i\le n}|A_i|.\]
            \(x\) also contributes to each term of
            \[c_2 = \sum_{1\le i<j \le n}|A_i\intersection A_j|\]
            in which \(A_i\) and \(A_j\) are among the \(k\) sets which contain \(x\).
            There are \({k\choose 2}\) such terms.
            Similarly \(x\) contributes \({k\choose 3}\) to the sum \(c_3\).
            In general \(x\) contributes \({k\choose i}\) to the sum \(c_i\).
            Therefore the total contribution of \(x\) to our expression is
            \[k - {k\choose 2} + {k\choose 3} - \dotsb + (-1)^{k-1}{k\choose k}.\]
            This happens to be 1.
            Hence each element \(x \in A_1\union\dotsb\union A_n\) contributes exactly once to the expression so we find that
            \[\abs{\bigunion_{i = 1}^n A_i} = |A_1 \union \dotsb \union A_n| = c_1 - c_2 + c_3 - \dotsb + (-1)^nc_n.\]
        \end{proof}