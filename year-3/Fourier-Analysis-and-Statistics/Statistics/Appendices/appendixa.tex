        \section{Sets and Logic}
        \subsection{Sets}
        A set is a collection of objects, say the objects \(\square\), \(\dagger\), and \(\star\).
        To denote a set containing all of these objects and nothing else we write \(\mathcal{A} = \{\square, \dagger, \star\}\), here `\(\mathcal{A}\)' is just a name that we give the set, in the same way we may say \(x = 2\).
        Sets may contain an infinite number of objects, for example the rational numbers, \(\rationals\), is a set of all fractions of the form \(a/b\) with \(a\) and \(b \ne 0\) being integers.
        The objects in a set are called its elements.
        It is also possible for a set to have no members, in which case we denote this set \(\emptyset\).
        We denote an element being in a set with \(\in\).
        For example we might write \(\square \in \mathcal{A}\) which we read as `\(\square\) is in the set \(\mathcal{A}\)', or \(\pi \in \reals\) which is read as '\(\pi\) is in the set of real numbers' or more commonly as `\(\pi\) is a real number'.
        We denote an element \emph{not} being in a set with \(\notin\), for example \(\diamondsuit \notin \mathcal{A}\) or \(e\notin\rationals\).
        
        Two sets are equal if and only if every element in one set is in the other set and vice versa.
        Note that their is no mention of the order of the elements or how many times an element is included as long as it is included at least one in each set.
        Therefore the following is a true statement:
        \[\mathcal{A} = \{\square, \dagger, \star\} = \{\dagger, \star, \square\} = \{\dagger, \square, \star, \square, \star, \star\}.\]
        Note also that by this definition of equality there is only one empty set, \(\emptyset\).
        It doesn't matter what is not in the set, equality is defined based on what is in the set so if there is nothing in the set then the set is the empty set.
        
        Set builder notation is a way of defining sets where simply listing all of the elements in the set is impractical.
        A set in set builder notation looks like
        \[\mathcal{B} = \{\text{a} \in \mathcal{A} \st \text{condition on }a\}.\]
        Here \(\mathcal{A}\) is some underlying set.
        \(\mathcal{B}\) is composed of elements in \(\mathcal{A}\) such that the given condition on \(a\) is true.
        For example the even numbers could be written as
        \[\{n\in\integers \st n\equiv 0\pmod{2}\}\]
        where \(n \equiv 0 \pmod{2}\) means that \(n/2\) has a remainder of 0.
        
        Two commonly defined operations between set are unions and intersections.
        The union between two sets, \(\mathcal{A}\) and \(\mathcal{B}\) is
        \[\mathcal{A} \union \mathcal{B} = \{a\st a\in\mathcal{A}\;\text{or}\;a\in\mathcal{B}\}.\]
        That is the set of all elements, \(a\), that are in either one of the two sets \(\mathcal{A}\) or \(\mathcal{B}\) (or both sets).
        The intersection between two sets, \(\mathcal{A}\) and \(\mathcal{B}\) is
        \[\mathcal{A} \intersection \mathcal{B} = \{a\st a\in \mathcal{A}\;\text{and}\;\mathcal{B}\}.\]
        That is the set of all elements that are in both of the sets, \(\mathcal{A}\) and \(\mathcal{B}\).
        
        The use of sets in probability is as collections of possible outcomes.
        For example if we are rolling a normal six sided die then the set of all possible outcomes is
        \[\mathcal{S} = {1, 2, 3, 4, 5, 6}\]
        This is the sample space, it is the set of all possible outcomes.
        Suppose we ask the question `what is the chance of rolling an odd prime number'?
        There are two conditions here `odd' and `prime'.
        The set of all odd numbers that are in \(\mathcal{S}\) is
        \[\mathcal{O} = {1, 3, 5}.\]
        The set of all prime numbers in \(\mathcal{S}\) is
        \[\mathcal{P} = {2, 3, 5}.\]
        The set of all odd prime numbers in \(\mathcal{S}\) is then
        \[\mathcal{O} \intersection \mathcal{P} = \{1, 3, 5\} \intersection \{2, 3, 5\} = \{3, 5\}.\]
        Thus we see there are 2 odd primes out of 6 possibilities so the probability of rolling an odd prime is \(2/6 = 1/3\).
        We could ask the similar question `what is the probability of rolling a number that is either odd, prime, or both'?
        The set of all numbers that are odd, prime, or both is
        \[\mathcal{O} \union \mathcal{P} = \{1, 3, 5\} \union \{2, 3, 5\} = \{1, 2, 3, 5\}.\]
        Thus there are 4 numbers that are odd or prime and the chance that rolling the die results in one of these numbers is \(4/6 = 2/3\).
        
        With a frequentist view of probability we see that `and' questions reduce to intersections of sets of possible values whereas `or' questions reduce to unions of sets of possible values.
        This reflects the `and' and `or' in the definitions of intersections and unions.
        
        \subsection{Logic}
        In boolean logic a statement is something that takes one of two values, true, denoted 1, or false, denoted 0.
        We can define various operations on these statements.
        Let \(\mathscr{A}\) and \(\mathscr{B}\) be statements.
        Then we define conjunction, \(\conjunction\), by the following table:
        \[
            \begin{array}{cc|c}
                \mathscr{A} & \mathscr{B} & \mathscr{A} \conjunction \mathscr{B}\\\hline
                1 & 1 & 1\\
                1 & 0 & 0\\
                0 & 1 & 0\\
                0 & 0 & 0
            \end{array}
        \]
        That is \(\mathscr{A}\conjunction\mathscr{B}\) is true only if both \(\mathscr{A}\) and \(\mathscr{B}\) are true.
        We define disjunction, \(\disjunction\), by the following table:
        \[
            \begin{array}{cc|c}
                \mathscr{A} & \mathscr{B} & \mathscr{A} \disjunction \mathscr{B}\\\hline
                1 & 1 & 1\\
                1 & 0 & 1\\
                0 & 1 & 1\\
                0 & 0 & 0
            \end{array}
        \]
        That is \(\mathscr{A}\disjunction\mathscr{B}\) is true so long as at least one of \(\mathscr{A}\) or \(\mathscr{B}\) is true.
        
        We return to rolling a 6 sided die.
        We now use a Bayesian view of probability where we look at confidence in particular statements being true.
        If \(\mathscr{O}\) is the statement `is odd' and \(\mathscr{P}\) is the statement `is prime' then the statement `is odd and prime' is \(\mathscr{O}\conjunction\mathscr{P}\).
        Similarly the statement `is odd or prime' is \(\mathscr{O}\disjunction\mathscr{P}\).
        
        \subsection{Sets and Logic}
        Note the similarity between this notation and the set notation, in particular compare \(\union\)/\(\intersection\) too \(\disjunction\)/\(\conjunction\).
        Also note how the statements \(\mathcal{O}\union\mathcal{P}\)/\(\mathcal{O}\intersection\mathcal{P}\) are the answers too the same questions as \(\mathscr{O}\disjunction\mathscr{P}\)/\(\mathscr{O}\conjunction\mathscr{P}\).
        The reason for this similarity in notation is that (at least when it comes to probability) these two ideas are really equivalent but viewed from a frequentist and Bayesian perspective respectively.
        
        In this work we take both a frequentist and Bayesian view of probability, using whichever better suits our needs.
        For this reason we choose notation that is distinct from both of these.
        If \(A\) and \(B\) are outcomes of some trial then we combine them as
        \[A\logicalAnd B = A, B \iff \mathcal{A} \intersection \mathcal{B} \iff \mathscr{A} \conjunction \mathscr{B},\]
        and
        \[A \logicalOr B \iff \mathcal{A} \union \mathcal{B} \iff \mathscr{A} \disjunction \mathscr{B}.\]
        Here \(\mathcal{A}/\mathcal{B}\) and \(\mathscr{A}/\mathscr{B}\) are the frequentist and Bayesian versions respectively of \(A/B\).