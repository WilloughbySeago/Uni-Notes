\section{Clebsch--Gordan Coefficients}
    The Clebsch--Gordan coefficients are, using the notation of section~\ref{sec:addition of angular momenta}, the coefficients
    \[\braket{j_1, m_1, j_2, m_2}{j, m; j_1, j_2}\]
    in the transformation from the uncoupled to coupled basis:
    \[\ket{j, m; j_1, j_2} = \sum_{m_1 = -j_1}^{j_1}\sum_{m_2 = -j_2}^{j_2} \braket{j_1, m_1, j_2, m_2}{j, m; j_1, j_2} \ket{j_1, m_1, j_2, m_2}.\]
    There is an explicit solution to this, namely
    \begin{align*}
        \braket{j_1, m_1, j_2, m_2}{j, m; j_1, j_2} &= \delta_{m,m_1+m_2} \sqrt{\frac{(2j + 1)(j + j_1 - j_2)!(j - j_1 + j_2)!(j_1 + j_2 - j)!}{(j_1 + j_2 + j + 1)!}} \times\\
        &\qquad \sqrt{(j + m)!(j - m)!(j_1 - m_1)!(j_1 + m_1)!(j_2 - m_2)!(j_2 + m_2)!} \times\\
        &\qquad \sum_k \left[\sqrt{\frac{(-1)^k}{k!(j_1 + j_2 - j - k)!(j_1 - m_1 - k)!(j_2 + m_2 - k)!}}\right.\times\\
        &\qquad\qquad \left.\sqrt{\frac{1}{(j - j_2 + m_1 + k)!(j - j_1 - m_2 + k)!}}\,\,\right].
    \end{align*}
    The summation is over all \(k\in\integers\) such that the argument of every factorial is non-negative.
    Solutions with \(m < 0\) or \(j_1 < j_2\) have been omitted.
    Solutions with \(m < 0\) can be found using the relation
    \[\braket{j_1, m_1, j_2, m_2}{j, m; j_1, j_2} = (-1)^{j - j_1 - j_2}\braket{j_1, m_2, j_2, m_2}{j, -m, j_1, j_2}\]
    and solutions with \(j_1 < j_2\) can be found using the relation
    \[\braket{j_1, m_1, j_2, m_2}{j, m; j_1, j_2} = (-1)^{j - j_1 - j_2}\braket{j_2, m_2, j_1, m_2}{j, m; j_2, j_1}.\]
    
    Clearly this is impractical to use every time.
    The common thing to do is to pre-calculate values or use a computer.
    Some pre-calculated values are given here.
    \subsection{Pre-Calculated Clebsch--Gordan Coefficients}
    When \(j_2 = 0\)  the Clebsch--Gordan coefficients are given by \(\delta_{jj_1}\delta_{mm_1}\).
    
    \begin{table}[ht]
        \centering
        \begin{subtable}{0.25\textwidth}
            \centering
            \begin{tabular}{|l|l|}\hline
                \backslashbox{\(m_1, m_2\)}{\(j\)} & 1\\ \hline
                \(1/2, 1/2\) & 1\\ \hline
            \end{tabular}
            \subcaption{\(m = 1\)}
        \end{subtable}
        \begin{subtable}{0.25\textwidth}
            \centering
            \begin{tabular}{|l|l|}\hline
                \backslashbox{\(m_1, m_2\)}{\(j\)} & 1\\ \hline
                \(-1/2, -1/2\) & 1\\ \hline
            \end{tabular}
            \subcaption{\(m = -1\)}
        \end{subtable}
        \begin{subtable}{0.4\textwidth}
            \centering
            \begin{tabular}{|l|l|l|}\hline
                \backslashbox{\(m_1, m_2\)}{\(j\)} & 1 & 0\\ \hline
                &&\\[-1em]
                \(1/2, -1/2\) & \(\sqrt{1/2}\) & \(\sqrt{1/2}\)\\ \hline
                &&\\[-1em]
                \(-1/2, 1/2\) & \(\sqrt{1/2}\) & -\(\sqrt{1/2}\)\\ \hline
            \end{tabular}
            \subcaption{\(m = 0\)}
        \end{subtable}
        
        \caption{Clebsch--Gordan coefficients for \(j_1 = 1/2\) and \(j_2 = 1/2\)}
    \end{table}
    
    \begin{table}[ht]
        \centering
        \begin{subtable}{0.25\textwidth}
            \centering
            \begin{tabular}{|l|l|}\hline
                \backslashbox{\(m_1, m_2\)}{\(j\)} & 3/2\\\hline
                1, 1/2 & 1\\\hline
            \end{tabular}
            \caption{\(m = 3/2\)}
        \end{subtable}
        \begin{subtable}{0.4\textwidth}
            \centering
            \begin{tabular}{|l|l|l|}\hline
                \backslashbox{\(m_1, m_2\)}{\(j\)} & 3/2 & 1/2\\\hline
                &&\\[-1em]
                1, 1/2 & \(\sqrt{1/3}\) & \(\sqrt{2/3}\)\\\hline
                &&\\[-1em]
                0, 1/2 & \(\sqrt{2/3}\) & \(-\sqrt{1/3}\)\\\hline
            \end{tabular}
            \caption{\(m = 1/2\)}
        \end{subtable}
        \caption{Clebsch--Gordan coefficients for \(j_1 = 1\) and \(j_2 = 1/2\)}
    \end{table}
    
    \begin{table}[ht]
        \centering
        \begin{subtable}{0.25\textwidth}
            \centering
            \begin{tabular}{|l|l|}\hline
                \backslashbox{\(m_1, m_2\)}{\(j\)} & 2\\\hline
                1, 1 & 1\\\hline
            \end{tabular}
            \caption{\(m = 2\)}
        \end{subtable}
        \begin{subtable}{0.4\textwidth}
            \centering
            \begin{tabular}{|l|l|l|}\hline
                \backslashbox{\(m_1, m_2\)}{\(j\)} & 2 & 1\\\hline
                &&\\[-1em]
                1, 0 & \(\sqrt{1/2}\) & \(\sqrt{1/2}\)\\\hline
                &&\\[-1em]
                0, 1 & \(\sqrt{1/2}\) & \(-\sqrt{1/2}\)\\\hline
            \end{tabular}
            \caption{\(m = 1\)}
        \end{subtable}
        \begin{subtable}{0.5\textwidth}
            \centering
            \begin{tabular}{|l|l|l|l|}\hline
                \backslashbox{\(m_1, m_2\)}{\(j\)} & 2 & 1 & 0\\\hline
                &&&\\[-1em]
                1, -1 & \(\sqrt{1/6}\) & \(\sqrt{1/2}\) & \(\sqrt{1/3}\)\\\hline
                &&&\\[-1em]
                0, 0 & \(\sqrt{2/3}\) & 0 & \(-\sqrt{1/3}\)\\\hline
                &&&\\[-1em]
                -1, 1 & \(\sqrt{1/6}\) & \(-\sqrt{1/2}\) & \(\sqrt{1/3}\)\\\hline
            \end{tabular}
            \caption{\(m = 0\)}
        \end{subtable}
        \caption{Clebsch--Gordan coefficients for \(j_1 = 1\) and \(j_2 = 1\)}
    \end{table}

    \begin{table}[ht]
        \centering
        \begin{subtable}{0.25\textwidth}
            \centering
            \begin{tabular}{|l|l|}\hline
                \backslashbox{\(m_1, m_2\)}{\(j\)} & 2\\\hline
                3/2, 1/2 & 1\\\hline
            \end{tabular}
            \caption{\(m = 2\)}
        \end{subtable}
        \begin{subtable}{0.4\textwidth}
            \centering
            \begin{tabular}{|l|l|l|}\hline
                \backslashbox{\(m_1, m_2\)}{\(j\)} & 2 & 1\\\hline
                &&\\[-1em]
                3/2, -1/2 & 1/2 & \(\sqrt{3/4}\)\\\hline
                &&\\[-1em]
                1/2, 1/2 & \(\sqrt{3/4}\) & \(-1/2\)\\\hline
            \end{tabular}
            \caption{\(m = 1\)}
        \end{subtable}
        \begin{subtable}{0.4\textwidth}
            \centering
            \begin{tabular}{|l|l|l|}\hline
                \backslashbox{\(m_1, m_2\)}{\(j\)} & 2 & 1\\\hline
                &&\\[-1em]
                1/2, -1/2 & \(\sqrt{1/2}\) & \(\sqrt{1/2}\)\\\hline
                &&\\[-1em]
                -1/2, 1/2 & \(\sqrt{1/2}\) & \(-\sqrt{1/2}\)\\\hline
            \end{tabular}
            \caption{\(m = 0\)}
        \end{subtable}
        
        \caption{Clebsch--Gordan coefficients for \(j_1 = 3/2\) and \(j_2 = 1/2\)}
    \end{table}
    
    \begin{table}[ht]
        \centering
        \begin{subtable}{0.25\textwidth}
            \centering
            \begin{tabular}{|l|l|}\hline
                \backslashbox{\(m_1, m_2\)}{\(j\)} & 5/2\\\hline
                3/2, 1 & 1\\\hline
            \end{tabular}
            \caption{\(m = 5/2\)}
        \end{subtable}
        \begin{subtable}{0.4\textwidth}
            \centering
            \begin{tabular}{|l|l|l|}\hline
                \backslashbox{\(m_1, m_2\)}{\(j\)} & 5/2 & 3/2\\\hline
                &&\\[-1em]
                3/2, 0 & \(\sqrt{2/5}\) & \(\sqrt{3/5}\)\\\hline
                &&\\[-1em]
                1/2, 1 & \(\sqrt{3/5}\) & \(-\sqrt{2/5}\)\\\hline
            \end{tabular}
            \caption{\(m = 3/2\)}
        \end{subtable}
        \begin{subtable}{0.5\textwidth}
            \centering
            \begin{tabular}{|l|l|l|l|}\hline
                \backslashbox{\(m_1, m_2\)}{\(j\)} & 5/2 & 3/2 & 1/2\\\hline
                &&&\\[-1em]
                3/2, -1 & \(\sqrt{1/10}\) & \(\sqrt{2/5}\) & \(\sqrt{1/2}\)\\\hline
                &&&\\[-1em]
                1/2, 0 & \(\sqrt{3/5}\) & \(\sqrt{1/15}\) & \(-\sqrt{1/3}\)\\\hline
                &&&\\[-1em]
                -1/2, 1 & \(\sqrt{3/10}\) & \(-\sqrt{8/15}\) & \(\sqrt{1/6}\)\\\hline
            \end{tabular}
            \caption{\(m = 0\)}
        \end{subtable}
        \caption{Clebsch--Gordan coefficients for \(j_1 = 3/2\) and \(j_2 = 1\)}
    \end{table}
