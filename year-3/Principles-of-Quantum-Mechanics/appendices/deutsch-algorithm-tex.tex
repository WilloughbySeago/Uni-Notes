%% AMS-LaTeX Created with the Wolfram Language for Students - Personal Use Only : www.wolfram.com

\newcounter{mathematicapage}

\section{Deutsch Algorithm}
I wrote a Mathematica note book to apply the Deutsch algorithm.

Define the operators:

\begin{doublespace}
\noindent\(\pmb{\text{id} = \text{DiagonalMatrix}[\{1, 1\}]}\)
\end{doublespace}

\begin{doublespace}
\noindent\(\{\{1,0\},\{0,1\}\}\)
\end{doublespace}

\begin{doublespace}
\noindent\(\pmb{X = \text{PauliMatrix}[1]}\)
\end{doublespace}

\begin{doublespace}
\noindent\(\{\{0,1\},\{1,0\}\}\)
\end{doublespace}

\begin{doublespace}
\noindent\(\pmb{\text{CNOT} = \{\{1, 0, 0, 0\}, \{0, 1, 0, 0\}, \{0, 0, 0, 1\}, \{0, 0, 1, 0\}\}}\)
\end{doublespace}

\begin{doublespace}
\noindent\(\{\{1,0,0,0\},\{0,1,0,0\},\{0,0,0,1\},\{0,0,1,0\}\}\)
\end{doublespace}

\begin{doublespace}
\noindent\(\pmb{\text{Hd} = \frac{1}{\text{Sqrt}[2]}*\{\{1, 1\}, \{1, -1\}\}}\)
\end{doublespace}

\begin{doublespace}
\noindent\(\left\{\left\{\frac{1}{\sqrt{2}},\frac{1}{\sqrt{2}}\right\},\left\{\frac{1}{\sqrt{2}},-\frac{1}{\sqrt{2}}\right\}\right\}\)
\end{doublespace}

\begin{doublespace}
\noindent\(\pmb{\text{Of1} = \text{DiagonalMatrix}[\{1, 1, 1, 1\}]}\)
\end{doublespace}

\begin{doublespace}
\noindent\(\{\{1,0,0,0\},\{0,1,0,0\},\{0,0,1,0\},\{0,0,0,1\}\}\)
\end{doublespace}

\begin{doublespace}
\noindent\(\pmb{\text{Of2} = \text{CNOT}}\)
\end{doublespace}

\begin{doublespace}
\noindent\(\{\{1,0,0,0\},\{0,1,0,0\},\{0,0,0,1\},\{0,0,1,0\}\}\)
\end{doublespace}

\begin{doublespace}
\noindent\(\pmb{\text{Of3} = \text{CNOT}.\text{KroneckerProduct}[\text{id}, X]}\)
\end{doublespace}

\begin{doublespace}
\noindent\(\{\{0,1,0,0\},\{1,0,0,0\},\{0,0,1,0\},\{0,0,0,1\}\}\)
\end{doublespace}

\begin{doublespace}
\noindent\(\pmb{\text{Of4} = \text{KroneckerProduct}[\text{id}, X]}\)
\end{doublespace}

\begin{doublespace}
\noindent\(\{\{0,1,0,0\},\{1,0,0,0\},\{0,0,0,1\},\{0,0,1,0\}\}\)
\end{doublespace}

\begin{doublespace}
\noindent\(\pmb{\text{operators} = \{\{\text{id}, \text{{``}id{''}}\}, \{X, \text{X}\}, \{\text{CNOT}, \text{{``}CNOT{''}}\}, \{\text{Hd}, \text{{``}Hd{''}}\},}\)\\
\noindent\(\pmb{\{\text{Of1}, \text{{``}Of1{''}}\}, \{\text{Of2}, \text{{``}Of2{''}}\}, \{\text{Of3}, \text{{``}Of3{''}}\}, \{\text{Of4}, \text{{``}Of4{''}}\}\};}\)
\end{doublespace}

\begin{doublespace}
\noindent\(\pmb{\text{Do}[\text{Print}[\text{operators}[[i]][[2]], \text{{``} = {''}}, \text{MatrixForm}[\text{operators}[[i]][[1]]]], \{i, 8\}]}\)
\end{doublespace}

\noindent\(\text{id}\text{ = }\left(
\begin{array}{cc}
 1 & 0 \\
 0 & 1 \\
\end{array}
\right)\)

\noindent\(\text{X}\text{ = }\left(
\begin{array}{cc}
 0 & 1 \\
 1 & 0 \\
\end{array}
\right)\)

\noindent\(\text{CNOT}\text{ = }\left(
\begin{array}{cccc}
 1 & 0 & 0 & 0 \\
 0 & 1 & 0 & 0 \\
 0 & 0 & 0 & 1 \\
 0 & 0 & 1 & 0 \\
\end{array}
\right)\)

\noindent\(\text{Hd}\text{ = }\left(
\begin{array}{cc}
 \frac{1}{\sqrt{2}} & \frac{1}{\sqrt{2}} \\
 \frac{1}{\sqrt{2}} & -\frac{1}{\sqrt{2}} \\
\end{array}
\right)\)

\noindent\(\text{Of1}\text{ = }\left(
\begin{array}{cccc}
 1 & 0 & 0 & 0 \\
 0 & 1 & 0 & 0 \\
 0 & 0 & 1 & 0 \\
 0 & 0 & 0 & 1 \\
\end{array}
\right)\)

\noindent\(\text{Of2}\text{ = }\left(
\begin{array}{cccc}
 1 & 0 & 0 & 0 \\
 0 & 1 & 0 & 0 \\
 0 & 0 & 0 & 1 \\
 0 & 0 & 1 & 0 \\
\end{array}
\right)\)

\noindent\(\text{Of3}\text{ = }\left(
\begin{array}{cccc}
 0 & 1 & 0 & 0 \\
 1 & 0 & 0 & 0 \\
 0 & 0 & 1 & 0 \\
 0 & 0 & 0 & 1 \\
\end{array}
\right)\)

\noindent\(\text{Of4}\text{ = }\left(
\begin{array}{cccc}
 0 & 1 & 0 & 0 \\
 1 & 0 & 0 & 0 \\
 0 & 0 & 0 & 1 \\
 0 & 0 & 1 & 0 \\
\end{array}
\right)\)

The first step of the algorithm is to apply \((\hat{H}_d \otimes \hat{H}_d)(\hat{X} \otimes \hat{X})\) to the state \(|0\rangle|0\rangle\), so define this operator and this state and compute the
product:

\begin{doublespace}
\noindent\(\pmb{\text{initOp} = \text{KroneckerProduct}[\text{Hd}, \text{Hd}].\text{KroneckerProduct}[X,X]}\)
\end{doublespace}

\begin{doublespace}
\noindent\(\left\{\left\{\frac{1}{2},\frac{1}{2},\frac{1}{2},\frac{1}{2}\right\},\left\{-\frac{1}{2},\frac{1}{2},-\frac{1}{2},\frac{1}{2}\right\},\left\{-\frac{1}{2},-\frac{1}{2},\frac{1}{2},\frac{1}{2}\right\},\left\{\frac{1}{2},-\frac{1}{2},-\frac{1}{2},\frac{1}{2}\right\}\right\}\)
\end{doublespace}

\begin{doublespace}
\noindent\(\pmb{\text{initState} = \{1, 0, 0, 0\}}\)
\end{doublespace}

\begin{doublespace}
\noindent\(\{1,0,0,0\}\)
\end{doublespace}

\begin{doublespace}
\noindent\(\pmb{\text{step1}=\text{initOp}.\text{initState}}\)
\end{doublespace}

\begin{doublespace}
\noindent\(\left\{\frac{1}{2},-\frac{1}{2},-\frac{1}{2},\frac{1}{2}\right\}\)
\end{doublespace}

Next we apply the oracle:

\begin{doublespace}
\noindent\(\pmb{\text{step2} = \{\text{Of1}.\text{step1}, \text{Of2}.\text{step1}, \text{Of3}.\text{step1}, \text{Of4}.\text{step1}\}}\)
\end{doublespace}

\begin{doublespace}
\noindent\(\left\{\left\{\frac{1}{2},-\frac{1}{2},-\frac{1}{2},\frac{1}{2}\right\},\left\{\frac{1}{2},-\frac{1}{2},\frac{1}{2},-\frac{1}{2}\right\},\left\{-\frac{1}{2},\frac{1}{2},-\frac{1}{2},\frac{1}{2}\right\},\left\{-\frac{1}{2},\frac{1}{2},\frac{1}{2},-\frac{1}{2}\right\}\right\}\)
\end{doublespace}

Then we apply \((\operator{H}_d \tensorProd \operator{I})\):

\begin{doublespace}
\noindent\(\pmb{\text{finalOp}=\text{KroneckerProduct}[\text{Hd}, \text{id}]}\)
\end{doublespace}

\begin{doublespace}
\noindent\(\left\{\left\{\frac{1}{\sqrt{2}},0,\frac{1}{\sqrt{2}},0\right\},\left\{0,\frac{1}{\sqrt{2}},0,\frac{1}{\sqrt{2}}\right\},\left\{\frac{1}{\sqrt{2}},0,-\frac{1}{\sqrt{2}},0\right\},\left\{0,\frac{1}{\sqrt{2}},0,-\frac{1}{\sqrt{2}}\right\}\right\}\)
\end{doublespace}

\begin{doublespace}
\noindent\(\pmb{\text{finalStates} = \{\text{finalOp}.\text{step2}[[1]], \text{finalOp}.\text{step2}[[2]],\text{finalOp}.\text{step2}[[3]], \text{finalOp}.\text{step2}[[4]]\}}\)
\end{doublespace}

\begin{doublespace}
\noindent\(\left\{\left\{0,0,\frac{1}{\sqrt{2}},-\frac{1}{\sqrt{2}}\right\},\left\{\frac{1}{\sqrt{2}},-\frac{1}{\sqrt{2}},0,0\right\},\left\{-\frac{1}{\sqrt{2}},\frac{1}{\sqrt{2}},0,0\right\},\left\{0,0,-\frac{1}{\sqrt{2}},\frac{1}{\sqrt{2}}\right\}\right\}\)
\end{doublespace}

\begin{doublespace}
\noindent\(\pmb{\text{Do}[\text{Print}[\text{{``}Result with oracle {''}},i,\text{{``} = {''}}, \text{MatrixForm}[\text{finalStates}[[i]]]], \{i,
4\}]}\)
\end{doublespace}

\noindent\(\text{Result with oracle }1\text{ = }\left(
\begin{array}{c}
 0 \\
 0 \\
 \frac{1}{\sqrt{2}} \\
 -\frac{1}{\sqrt{2}} \\
\end{array}
\right)\)

\noindent\(\text{Result with oracle }2\text{ = }\left(
\begin{array}{c}
 \frac{1}{\sqrt{2}} \\
 -\frac{1}{\sqrt{2}} \\
 0 \\
 0 \\
\end{array}
\right)\)

\noindent\(\text{Result with oracle }3\text{ = }\left(
\begin{array}{c}
 -\frac{1}{\sqrt{2}} \\
 \frac{1}{\sqrt{2}} \\
 0 \\
 0 \\
\end{array}
\right)\)

\noindent\(\text{Result with oracle }4\text{ = }\left(
\begin{array}{c}
 0 \\
 0 \\
 -\frac{1}{\sqrt{2}} \\
 \frac{1}{\sqrt{2}} \\
\end{array}
\right)\)

If we were to measure the register of these states we would get 0 for the second and third oracles (meaning \(f(0) \ne f(1)\)) and we would get 1 for
the first and fourth oracles (meaning \(f(0) = f(1)\)).
