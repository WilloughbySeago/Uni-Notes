\section{Algebraic Structures}
    In section~\ref{sec:vector spaces} we discussed vector spaces.
    We defined a vector space, \(\hilbert\), over a field, \(\mathbb{F}\).
    But what is a field?
    Intuitively a field is a set with addition, multiplication, subtraction, and division defined as we would expect.
    To be more rigorous we need a couple of definitions first.
    
    \subsection{Groups}
    A \define{group}, \((G, \circ)\), is a set, \(G\), along with a binary operation 
    \begin{align*}
        \circ\colon G\times G &\to G\\
        x, y &\mapsto x\circ y = z\in G.
    \end{align*}
    This binary operation satisfies the following:
    \begin{itemize}
        \item Associativity -- For all \(x, y, z\in G\) we have
        \[x \circ (y \circ z) = (x \circ y) \circ z.\]
        \item Identity -- There exists \(e\in G\) such that for all \(x\in G\) we have
        \[e \circ x = x \circ e = x.\]
        \(e\) is called the identity.
        \item Inverse -- For all \(x\in G\) there exists \(x^{-1}\in G\) such that
        \[x \circ x^{-1} = x^{-1} \circ x = e\]
        where \(e\) is the identity.
    \end{itemize}
    An example of a group is the group of all permutations, \(\sigma\colon \{1, \dotsc, n\} \to \{1, \dotsc, n\}\) such that \(\sigma\) is a bijection.
    Here the group operation is function composition.
    
    If \((G, \circ)\) is a group and for all \(x, y\in G\) we have
    \[x\circ y = y\circ x\]
    then we say \((G, \circ)\) is an \define{abelian group}.
    An example of an abelian group is \(\integers_2 = \{0, 1\}\) under addition modulo 2.
    
    \subsection{Rings}
    A \define{ring}, \((R, +, \cdot)\), is a set, \(R\), along with two binary operations:
    \begin{align*}
        +\colon R\times R &\to R\\
        x, y &\mapsto x + y = z\in R,
    \end{align*}
    which we call addition, and
    \begin{align*}
        \cdot\colon R\times R &\to R\\
        x, y &\mapsto xy = z'\in R,
    \end{align*}
    which we call multiplication.
    These two operations satisfy the following:
    \begin{itemize}
        \item \((R, +)\) is an abelian group.
        \item Associativity of multiplication -- For all \(x, y, z\in R\)
        \[x(yz) = (xy)z.\]
        \item Distributivity of multiplication over addition -- For all \(x, y, z\in R\)
        \[x(y + z) = xy + xz,\qquad\text{and}\qquad (x + y)z = xz + yz.\]
    \end{itemize}
    We denote the identity of \((R, +)\) by \(0\) and inverses of addition as \(-x\) (as opposed to \(x^{-1}\)).
    An example of a ring is \(M_2(2\integers)\), which is the set of \(2\times 2\) matrices with entries from \(2\integers = \{0, \pm 2, \pm 4, \pm 6, \dotsc\}\).
    Here the operations are matrix addition and matrix multiplication.
    
    
    If \(R\) is a ring (we drop the \(+\) and \(\cdot\) for brevity) and there exists \(1\in R\) such that \(1x = x1 = x\) for all \(x\in R\) then \(R\) is a \define{ring with unity}.
    An example of a ring with unity is \(M_2(\integers)\) which is the set of all \(2\times 2\) matrices with entries from \(\integers\), in this case the identity matrix plays the role of the multiplicative identity.
    
    If \(R\) is a ring with unity and for all \(x\in R\) such that \(x\ne 0\) there exists \(x^{-1}\in R\) such that \(xx^{-1} = x^{-1}x = 1\) then \(R\) is a \define{division ring}.
    An example of a ring with unity is the quaternions, \(\quaternions\), with quaternion addition and multiplication.
    
    If \(R\) is a ring and for all \(x, y\in R\) we have \(xy = yx\) then \(R\) is a \define{commutative ring}.
    An example of a commutative ring is \(2\integers\) with normal integer addition and multiplication.
    
    
    If \(R\) is a commutative ring and for all \(a, b\in R\) we have \(ab = 0\) if and only if \(a = 0\) and/or \(b = 0\) then \(R\) is an \define{integral domain}.
    An example of an integral domain is \(\integers\) with normal integer addition and multiplication.
    
    If \(R\) is a commutative division ring, and \(1\ne 0\), then it is a \define{field}.
    Examples of fields includes \(\rationals\), \(\reals\), and \(\complex\), with their respective addition and multiplication operations.
    As well as these \(\integers_p = \{0, \dotsc, p-1\}\) for prime \(p\) under addition and multiplication modulo \(p\) is a field.
